\lecture{Expressions and Assignments}{14:00}{15/04/24}{Jiacheng Tan}

\section*{Expressions}

An expression is a combination of values, variables, operators and/ or function calls. Expressions can be used to evaluate
 mathematical values, move data around and more.

The main types of expressions are
\begin{itemize}
  \item Arithmetic
  \item Relational
  \item Boolean
  \item Assignment
\end{itemize}

These expressions can then be used and chained together to create various algorithms and programs. To be able to correctly
 evaluate expressions, you must know the rules of precedence and the rules of associativity of operators.

\subsection*{Rules of Precedence}

The rules of precedence determine the order in which operations should be evaluated in an expression. The typical order
 is along the lines of
\begin{enumerate}
  \item Parentheses
  \item The unary operators \verb`++` and \verb`--`
  \item The unary operators \verb`+` and \verb `-`
  \item The binary operators \verb`*` and \verb`/`
  \item the binary operators \verb`+` and \verb`-`
\end{enumerate}

\subsection*{Rules of Associativity}

The rules of associativity determines how operators with the same precedence are grouped together, if they have been
 left in an ambiguous state with no parentheses. For example, the expression \verb`3 / 5 * 0` is ambiguous as \verb`/`
 and \verb`*` have the same precedence. Therefore, the order of operations is determined using the associativity of each
 operator.

Any given operator may be
\begin{itemize}
  \item Associative - The operations can be grouped in any order, e.g. \verb`a * b * c` could be \verb`(a * b) * c`
   or \verb`a * (b * c)`
  \item Left-associative - The operators must be grouped together from left-to-right
  \item Right-associative - The operators must be grouped together from right-to-left
\end{itemize}

Most mathematical operators inherently have associativity, such as subtraction and division being left-associative and
 addition and multiplication being associative.

Many programming languages include a table of operator precedence and associativity in their documentation, but in
 general, most operators are left-associative apart from the assignment operator, which is necessarily right-associative.

\subsection*{Operator Overloading}

When a language uses the same operator for more than one purpose, it is \textit{overloading} the operator. For example,
 in Dart \verb`+` is used for adding \verb`int`s and \verb`float`s, but is also the operator for string concatenation.
 This means that the function of the operator is determined by the semantics of the language, as they determine which
 variant of the operator should be used. Some languages, such as C++ and C\# allow user-defined operator overloads, which
 can be used to improve the readability of user-defined classes and data types.

\subsection*{Side Effects}

An expression is said to have a \textit{side effect} if as well as returning a value, it also modifies a variable or
 changes the flow control of the program. The four main causes of side effects are
\begin{itemize}
  \item Assignment operators
  \item Increment/ Decrement operators
  \item Function calls
  \item Method invocation
\end{itemize}

\subsection*{Assignment as an Expression}

In some languages, the assignment operator (\verb`=`) is treated the same as any other binary operator, and as such
 still returns a value. However, it also has the side effect of changing the value of the left operand. For example, the
 expression \verb`x = y + 1` both returns the value \verb`y + 1`, and changes the value of \verb`x` to be \verb`y + 1`.

This has the advantage of allowing you to do things like \verb`x = y = z + 1`, but also has issues. For example, if you
 were to use an if statement with the condition \verb`x=y`, you might be trying to use it as an assignment and be treating
 the returned integer as a boolean value, which is perfectly valid, or you might have missed the second \verb`=` to use
 the relational operator \verb`==`. This mistake is impossible to detect by the compiler, but may occasionally be detected
 by the IDE or Linter as a logic error.

\subsection*{Compound Assignment Operators}

A compound assignment operator is one which combines an assignment with some form of arithmetic operator. This includes
 operators such as \verb`+=` and \verb`-=`, which add and subtract the right operand from the current value of the left
 operand, but also assign the new value to the left operand.

\subsection*{Unary Assignment Operators}

Some languages also include unary assignment operators, which both perform some arithmetic and assign that value to the
 only operand. This is operators such as \verb`++` and \verb`--`, which increment and decrement the value respectively.
 The order of assignment and increment is important, as \verb`++count` and \verb`count++` have two different meanings.
 In C, \verb`sum = ++count` would increment \verb`count` by one, then assign it to \verb`sum` (pre-increment), but
 \verb`sum = count++` would assign \verb`count` to \verb`sum` and then increment \verb`count` by one (post-increment).

\subsection*{Type Conversions}

Different languages follow different approaches when it comes to the compatibility of different types, when used in an
 expression. For example, some languages such as Dart require you to \textit{explicitly} convert the types using a built-
in function, but some languages such as JavaScript allow you to \textit{cast} values between types. There are also some
 languages which allow casting for some types, but require explicit type conversion for others.

Casting is type conversion which is explicitly requested by the code, but which does not make use of a function. For
 example, in C you can cast an unsigned integer to a signed integer using
\begin{verbatim}
  uint a = 10;
  int b;
  b = (int)a;
\end{verbatim}

Type coercion is when casting is done implicitly by the compiler. This means that the compiler checks the compatibility
 of the types of the left and right operand, and automatically casts them if they are compatible. This is used heavily
 in scripting languages such as JavaScript and Python.

\subsection*{Arithmetic Type Conversions}

There are two types of type conversion - widening and narrowing. A widening conversion is one in which an object is
 converted to another type which can store all values of the original type, and more. One example of this is casting
 an integer to a long integer (32-bit integer to 64-bit integer). This type of conversion is completely safe and used
 all the time. A narrowing conversion is one in which the new type can store only a subset of the values the original
 type could store. An example of this is converting an integer to an unsigned integer. This type of conversion is unsafe
 since an overflow or other issue could occur if the programmer is not careful. Typically, compilers will only coerce
 types when it is safe to do so, e.g. a widening conversion.

\section*{Notations}

There are several different ways of writing any given expression, which are known as different notations. The notation
 used in mathematics and most programming languages is known as \textit{Infix Notation}, but the issue with it is that
 it's inherently ambiguous, at least when not used in conjunction with the rules of associativity and precedence.
 Given the inherent issues with infix notation, several other notations have been developed.

\subsection*{Prefix Notation}

With prefix notation, operators appear before the operands. For example, the expression \verb`a + b - c * d` would be
 written as \verb`-+ab*cd`. It is evaluated from left-to-right by combining the operator with the two operands in front
 of it, meaning that is inherently unambiguous. When evaluating the previous expression, you would evaluate the operators
 in the following order
\begin{verbatim}
  -+ab*cd
  -(+ab)*cd
  -(+ab)(*cd)
  (-(+ab)(*cd))
\end{verbatim}
Below is the same expression, but using numbers to make it easier to see the order. (Numbers are shown in brackets for
 ease of reading)
\begin{verbatim}
  -+(7)(9)*(2)(5)
  -(16)*(2)(5)
  -(16)(10)
  (6)
\end{verbatim}
This is the same expression, but with \verb`a = 7`, \verb`b = 9`, \verb`c = 2` and \verb`d = 5`

\subsection*{Postfix Notation}

With postfix notation, operators appear after their operands. For example, the expression \verb`a + b - c * d` would be
 written as either \verb`ab+cd*-` \textit{or} \verb`abcd*-+`. (I'm not sure about that second one so I'll just use the
 first here on out). It is also evaluated left-to-right, but by combining two operands with the operator after them.
 This notation is also inherently unambiguous, but is harder to convert to since there are several valid options. To
 evaluate the previous expression, you would do the following
\begin{verbatim}
  ab+cd*-
  (ab+)cd*-
  (ab+)(cd*)-
  ((ab+)(cd*)-)
\end{verbatim}
Once again, below is the same expression but using the numbers \verb`a = 7`, \verb`b = 9`, \verb`c = 2` and \verb`d = 5`
\begin{verbatim}
  (7)(9)+(2)(5)*-
  (16)(2)(5)*-
  (16)(10)-
  (6)
\end{verbatim}

\subsection*{Cambridge Prefix Notation}

Cambridge prefix notation (or just Cambridge notation) is a variant of Prefix notation which introduces parentheses. This
 has the advantage of allowing operators like \verb`+` and \verb`-` to be n-ary, e.g. \verb`a + b + c` would be written
 as \verb`++abc` in standard prefix notation, but as \verb`(+abc)` in Cambridge notation.

