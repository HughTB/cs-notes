\lecture{Elementary Data Types}{14:00}{15/03/24}{Jiacheng Tan}

An elementary data type is one which has a fixed size in memory. Typically, languages have a few elementary data types,
 such as numeric types (Integer, Floating Points, etc), Boolean, Character, etc. There are also usually ``enumerated''
 types, which can take any of a set of values, which are usually represented internally as Integers.

Most imperative languages also provide built-in operators for computing using these elementary data types. These
 typically include arithmetic, relational and boolean operations, such as\\\verb`+ - * / == < <= > >= || &&`, etc.

\section*{Enumerated Types}

Enumerated types are \textit{ordinal} types, in which the possible values can be associated with the set of positive
 integers. This means that they can be used to represent order, and that you can effectively `index' the values using
 an integer.

Some built-in types are enumerated, but they can also be user-defined. This is useful for storing information such as
 the day of the week, month of the year, or season. For example, you could represent the seasons in C++ as follows
\begin{verbatim}
enum Season = {
  spring,
  summer,
  autumn,
  winter,
};
\end{verbatim}
Then, a variable which has the type \verb`Season` could store any of the four seasons.

Most languages also automatically define operators for enumerated types. These include basic functionality such as the
 equality operator, but more complex functions like relational operators, which allow you to do things like checking if
 the season is `greater' than \verb`summer`. You can usually also interact with the real data type behind the enumerated
 values, which is typically an integer. This allows you to increment or decrement, as well as performing arithmetic with
 your enumerated values.

\section*{Pointer Types}

Pointer (or reference) variables store a location in memory as their value. This allows them to point to a piece of data
 in memory. Pointers are meant to work with memory locations, allowing for more efficient use of limited memory space.
 As this is less of an issue nowadays, fewer and fewer languages allow you to access the raw memory locations, and is
 typically limited to lower-level languages like C and C++.

Memory is often dynamically allocated from the heap, and are known as heap-dynamic variables. Typically, they do not have
 an identifier, and can therefore only be accessed by using a pointer to the memory address.

Pointer \textit{dereferencing} is the act of accessing the value stored in the memory pointed to by a pointer. With the
 dereferenced value, you can read the current value or overwrite a new one. This is useful, as it allows you to pass a
 pointer to a memory location, and modify it in a function, without having to pass a value in and return the modified
 value. This is especially useful in low-level languages which don't support compound data types.

There is also a special \verb`null` pointer, which is usually stored as \verb`0`. This is used to signify that a pointer
 does not point to any location, and is typically the value which pointers are initialised to.

\subsection*{Dynamic Memory Allocation}

Since a pointer is a reference to a location in memory, you can also use them to dynamically allocate space in memory for
 values at runtime. This allows you to both store as many values as you need to, but also to store arbitrarily large
 values in memory. For example, if you wanted to store a string, you might not know how long the string will be at
 compile time. Therefore, you can dynamically allocate a space in memory which is large enough to store the string, and
 then access it using a pointer.

\subsection*{Memory Deallocation}

Memory addresses need to be \textit{deallocated} when they are no longer needed, otherwise the program's memory space
 would fill up over time, with unused values being stored unnecessarily. Deallocation can be done either implicitly or
 explicitly. Assigning \verb`null` to a pointer removes the only way of accessing the allocated memory, which still
 contains the last value which was assigned to it. The runtime environment can then `garbage collect' the memory address
 and deallocate it so it can be used again. To deallocate memory explicitly, you call the \verb`free` function, with a
 pointer to the address you wish to free. This will mark it as available for reuse. It is good practice to also set the
 value of the pointer to \verb`null`, so that you don't have a \textit{dangling pointer}.

\subsection*{Array Indices}

In languages such as C, arrays are actually just pointers to a block of memory. For example, an array of 10 integers
 would be stored in memory as a contiguous block of memory which is 10 times the size of an integer. Then, when indexing
 the array, the specified index acts like an offset, adding to the pointer to the first element to get the address of
 the value you're interested in. Since they are accessed via a pointer, it is also possible to dynamically allocate an
 array of arbitrary length, useful for storing a list of input values.

\subsection*{Dangling Pointers}

A dangling pointer is one which points to a memory address which has already been deallocated. This is dangerous and liable
 to cause the program to crash, as it may attempt to read from or write to the memory of another program. On Unix systems,
 accessing a dangling pointer guarantees a crash as the kernel will notice the program attempting to read outside of
 its assigned memory and kill it, giving a \verb`SEGMENTATION FAULT` error. On Windows systems, this could still cause a
 crash as the program may write into another programs memory, or read an invalid value placed there by a different
 program. Because of this, it's important to destroy the pointer when it is deallocated, typically by setting its value
 to a \verb`null` pointer.