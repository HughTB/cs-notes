\lecture{Functions and Parameter Passing}{14:00}{26/04/24}{Jiacheng Tan}

A \textit{subprogram} is a fundamental part of program flow control, and are present in almost every language. Decomposing
 a problem in to sub-problems makes it much easier to handle, as the complexity of any given part is much lower.
 Subprograms go by many names, but are usually called a \textit{function}, \textit{procedure} or \textit{subprogram}. It
 is any piece of code which is identified by a name, and has its own local reference space. There is usually some
 facility to exchange data with the other code using parameters. Procedures and functions are both types of subprogram,
 where a function returns a value, and a procedure does not.

\section*{Abstraction}

Subprograms provide one of the two fundamental abstractions in programming languages
\begin{itemize}
  \item Process control abstraction -- Allows details of procedures to be hidden, and only exposes the interface and not
   the implementation
  \item Data abstraction -- Allows the use of sophisticated data types without needing to know their implementation, only
   the interface
\end{itemize}

\section*{Function Declarations \& Definitions}

The \textit{definition} of a function refers to the actual implementation of the function. This includes both the header
 or interface, as well as the body of the function, the actual code that runs. The header consists of the name, return
 type, and parameter profile of the function. The parameter profile of a function is the number, order and types of
 parameters, as well as their names in languages which support named parameters.
 
The \textit{declaration} of a function (sometimes known as the prototype) is the header of the function on its own. This
 is used in languages like C and C++ to expose functions to other files in the program. If a function needs to be used
 in another file, it's declaration is placed in a header file, which is then included in other files. This allows the
 header of the function to be exposed, without exposing the implementation as well.

\subsection*{Parameters and Return Values}

A \textit{formal parameter} is a dummy variable in the function header, which can then be referred to by the implementation
 of the function. The \textit{actual parameters} are then filled in at runtime, and is usually either the value or
 address which the caller used. Functions are able to report results by returning a value, typically of a type which was
 specified in the declaration of the function. When a return statement is called, the value is passed back to the caller,
 and execution resumes at the caller.

\subsection*{Parameter Binding}

Actual parameters can be bound to the formal parameters in one of two ways -- by position or by keyword/ name. When bound
 by position, the values are bound in the order that they appear in the function call, which is safe and easy. When bound
 by keyword, the names of the formal parameters are specified manually in the function call, and the order they appear
 is entirely irrelevant. This means that it is impossible for a programmer to accidentally put the wrong values into the
 parameters, but it does mean that they have to remember the names of the formal parameters.

\section*{Parameter Passing}

There are several ways in which a parameter can be passed into a function. These determine how the values are sent to the
 function, and if it is possible for the function to modify them. The relationship between the actual and formal
 parameters can be one of three models
\begin{itemize}
  \item In (Put) Mode -- Formal parameters receive data from the actual parameters, aka pass by value
  \item Out (Put) Mode -- Formal parameters transmit data to the actual parameters, aka pass by result
  \item Inout Mode -- Both occur, can be implemented either using both pass by value and result, or by passing the reference
\end{itemize}

The vast majority of languages support passing values by value, result or both and default to passing by value. Only
 some languages support passing values by reference, and most are low level, such as C or C++.

\subsection*{Pass by Value}

The value of the actual parameter is used to initialise the corresponding formal parameter, at the time of function call.
 During the execution of the function, the local variable uses the value passed to it, and can be modified, but is then
 destroyed when the function returns.

This is very simple and fast for sending elementary data types, and is the default in many languages, such as C, C++ \&
 Java. It does have the downside of needing to use more memory, as the value is stored twice, and needing to move data
 in memory, which can be expensive, depending upon the type of the variable.

\subsection*{Pass by Result}

The value is transmitted to the variable when control returns to the caller, and no value is sent from the caller to The
 function. During execution of the function, the formal parameter acts as a local variable within the function, and
 their values are copied to the real parameters when control returns to the caller.

This is also simple, but does have quite a few limitations and downsides. Each of the parameters must be a variable, and
 the order which the values are copied back to the actual parameters does matter. This is a bigger issue in languages
 which have multiple implementations, such as C\#.

\subsection*{Pass by Value and Result}

Sometimes known as pass-by-copy, this is a combination of the two previous methods. When the function is called, the
 values of the actual parameters are stored in the formal parameters using pass-by-value, and at the end of the function
 call, the values of the formal parameters are copied back to the actual variables.

This facilitates bi-directional data exchange, but does have the disadvantage of needing to copy the data twice, and
 store two copies of it the whole time.

\subsection*{Pass by Reference}

Rather than passing the values, a pointer to the address in memory is used instead. This is sometimes known as pass-by-
sharing. The formal parameters and actual parameters are simply treated as two pointers to the same address in memory,
 which can be written to by both the caller and the function.
 
This also allows bi-directional data exchange, but without having to make multiple copies of the data. This makes it much
 more efficient, and more convenient than passing by value and result, but could be seen as insecure, as it gives the
 function direct access to the memory space.

\section*{Local Reference Space}

When a function is called, a new \textit{local reference space} is created, which contains all of the local variables
 and actual parameters. In most languages, local variables are created on the stack, as stack-dynamic variables, but if
 the variable is labelled as static, it must be assigned statically.