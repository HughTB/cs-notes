\lecture{Regular Expressions}{14:00}{05/02/24}{Jiacheng Tan}

The full definition of a language includes definitions of it's lexical structures, syntax and semantics. The lexical
 structures of a language are the form and structure of the individual symbols, such as keywords, identifiers, etc. The
 syntax determines the structure of the language, such as how a statement is defined, how to structure an expression,
 and so on. The semantics of a language determine how you can use each operator, what types they support, checking for
 type consistency in strongly typed languages, etc. The semantics of a language also define it's ``grammar'', which is
 how the compiler enforces the semantics.

\section*{Language Analysis}

The implementations of a language must analyse the lexical and syntactic structure of the source code to determine
 if it is valid or not. This is usually implemented using two separate systems, the lexical analyser and syntax
 analyser. If the analyser is implemented using regex, it is a finite automaton, based on a regular grammar (that of
 the language)

\section*{Lexical Analysis}

A lexical analyser reads the source code one character at a time and outputs a list of tokens to the next stage of the
 compiler. These tokens are made up of smaller substrings of source code, known as lexemes. Each lexeme matches a
 character pattern from the language's grammar.

The lexical analyser can be implemented in several ways, but the most common are by using regular expressions (Regex),
 or a deterministic finite automata (DFA).

\subsection*{Definitions}

\begin{itemize}
  \item The Alphabet
  \begin{itemize}
    \item Each language has it's own alphabet, which is the set of all characters which could be used in a lexeme
    \item An alphabet is usually represented using $\Sigma$
  \end{itemize}
  \item String or Word
  \begin{itemize}
    \item A string or word \textit{over} an alphabet is a finite string of symbols from the alphabet
    \item The length of a string is the number of symbols which make up the string
    \item An empty or null string is denoted by $\varepsilon$, and so $\mid \varepsilon \mid = 0$
    \item The set of all strings over $\Sigma$ is denoted by $\Sigma^*$.
    \item For a symbol or string $x$, $x^n$ represents a string of that symbol, $n$ times, e.g. $a^4 = \mathrm{aaaa}$ 
  \end{itemize}
\end{itemize}

\section*{Regular Expressions}

Regular expressions specify patterns which can be used to match strings of symbols. A regular expression, $r$ matches
 or is matched by a set of strings if the strings conform to $r$'s pattern. The set of strings matched by $r$ is
 denoted by $L(r) \subseteq \Sigma^*$, i.e. all strings which are over the alphabet $\Sigma$. This is known as the
 language generated by $r$.

$\emptyset$ is in and of itself a regular expression, but does not match any strings at all, and is only very rarely
 useful. $\varepsilon$ is also a regular expression, which matches only the empty string $\varepsilon$.

Since $\varepsilon$ is an empty string, it can be used as the identity element for concatenation, and as such,
 $\varepsilon + s = s + \varepsilon = s$.

For each symbol where $c \in \Sigma$, $c$ is a regular expression over $\Sigma$. In this case, the expression only
 matches a single instance of the symbol. 

If $r$ and $s$ are both regular expressions, then $r \mid s$ is also a regular expression. $a \mid b$ would match a
 single instance of either $a$ or $b$. $a \mid \varepsilon$ would match a single instance of $a$ or $\varepsilon$.

If $r$ and $s$ are both regular expressions, then $rs$ is also a regular expression. This would match a single instance
 of the string $rs$, as the string would have to match both the regular expression $r$ \textbf{and} $s$. As with
 arithmetic expressions, brackets can be used to make the meaning of a regular expression clearer. e.g. $(a \mid b)a$
 matches the strings $aa$ and $ba$

If $r$ is a regular expression, then $r*$ would match any number of $r$s in a row. Specifically, it means zero or more
 instances of $r$. $r+$ would match one or more instances of $r$, which could also be written as $rr*$.

As with arithmetic expressions, there is a specific order of operations which the symbols must follow. The order is as
 follows: (), * or +, concatenation, |. This is similar to arithmetic as anything inside parentheses must be processed
 before everything else.

A regular definition is a named regular expression, which can be used to make up more complex regular expressions,
 without re-writing the same expression several times. For example, you might define number as
 $\mathrm{number} = 0 \mid \dots \mid 9$

\section*{Regular Expressions for Lexical Analysis}

Regular expressions provide a method to describe the patterns which make up the lexical structure of a language, as
 well as restricting the alphabet which can be used to write source code. In most cases, languages use a standard
 alphabet, such as ASCII or UTF-8. An example of a regular expression used in a typical language could be $if$ for the
 token of IF, $;$ for a semicolon, $(0 \mid \dots \mid 9)+$ for a number, etc.

Languages are sets of strings chosen from some alphabet $\Sigma$. More formally, a language $L$ over an alphabet
 $\Sigma$, $L \subseteq \Sigma^*$.

Given a language $L$ over some alphabet $\Sigma$, it is necessary to be able to write an algorithm which takes any
 input string $w \in \Sigma^*$, and outputs True if $w \in L$ and False if $w \notin L$. This algorithm is known as a
 decision procedure for $L$. A decision procedure can be written using either a Deterministic Finite Automaton (DFA) or
 a Non-deterministic Finite Automaton (NFA). Any language which can be denoted by a regular expression is known as a
 regular language.