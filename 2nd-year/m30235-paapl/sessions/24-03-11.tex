\lecture{Scopes and Memory Allocation}{14:00}{11/03/24}{Jiacheng Tan}

\section*{Variables}

A variable is a place-holder for a run-time value. Each variable has multiple run-time attributes. These attributes
 include:
\begin{itemize}
  \item Name - The name which is used in code to refer to the variable
  \item Address - The memory address which stores the value of the variable
  \item Value - The contents of the memory which corresponds to the variable
  \item Type - The range of values the variable can store, and the operations which can be performed upon it
  \item Lifetime - The time for which the variable is bound to the specific memory location
  \item Scope - The area in code which the variable is accessible
\end{itemize}

\subsection*{Implicit vs Explicit Declaration}

A variable is introduced into the scope using a declaration. This can be done implicitly or explicitly. An explicit
 declaration is a statement in the source code which defines the type and name of a new variable, e.g.
\begin{verbatim}
C - int i;
Pascal - i : integer
\end{verbatim}

An implicit declaration is a mechanism by which variables are automatically assigned a type based upon conventions of
 the language, rather than declaring it manually. For example, in Fortran the type of a variable is defined by the first
 character of the variable name. If the first character is I, J, K, L, M, or N then it is an integer, otherwise it is
 real.
 
Another form of implicit declaration is type inference. For example, if you define a \verb`late` variable in Dart, it
 infers the type of the variable from the first value which is assigned to it, e.g.
\begin{verbatim}
late final variable;

variable = "This is a string";
\end{verbatim}
In this case, the variable would be assigned the \verb`String` type.

\subsection*{Binding}

A binding is an association between an entity and attribute, such as between a variable and it's type, or a symbol and
 the operation it corresponds to. Binding can take place at many different times, but is always referred to as the
 \textit{binding time}. An example of this are the following C statements
\begin{verbatim}
1. int x;
2. x = 1;
3. x = x + 1;
\end{verbatim}
In this case,
\begin{itemize}
  \item The type of \verb`x` is bound at compile time, as C is a statically typed language
  \item The range of values which \verb`x` can take is bound when the compiler is designed, as this is when they pick
   the number of bits, and therefore maximum value that an integer can be
  \item The operation of the \verb`+` operator is bound at compile time, as this is when the types of it's operands are
   known
  \item The value of \verb`x` is bound at run-time, since that is when the statement is executed
\end{itemize}
Binding can also take place at load time (when the variable is bound to a memory location) or link time (the variable
 in one module is bound to another module)

\subsection*{Static vs Dynamic Type Binding}

A type binding can be either static or dynamic. It is static if it occurs before run-time, e.g. at compile time, and
 remains unchanged throughout the execution of a program. Declarations always have type information, and so binding is
 done at compile time. Languages with static type declaration are known as statically typed as, once set, the type cannot
 change. This gives the advantage of type errors being detected at compile time, and using less memory for each variable,
 but it is not very flexible and can lead to issues when user input is involved.

A dynamic type binding occurs when the type is bound during execution, or if it is able to change at run-time. This is
 the case in languages such as Python, JavaScript and PHP. These languages are known as dynamically typed, since the
 type of any given variable is unknown until run-time and can change at any point. This has the advantage of being more
 flexible (useful when user input is necessary) and allowing the developer to not need to know the type of a variable
 when the program is written, but has a much higher overhead due to dynamic type checking, and makes compile-time type
 error detection impossible.

\subsection*{Strongly and Weakly Typed Languages}

A language is known as strongly typed if \textbf{all} type errors can be detected by the compiler. In this case, a
 language is also type safe, as it is impossible for the program to crash due to an invalid type. Some examples are
 Java, Haskell and Ada. Some languages may seem to be strongly typed languages, such as Fortran or C, but actually aren't.
 A strongly typed language can be either statically or dynamically typed.

A weakly typed language is any language which is not strongly typed, such as JavaScript. Once again, a weakly typed
 language can be either statically or dynamically typed.

\section*{Scopes}

A block is a section of code, which defines the local environment for any given statement. Blocks are usually denoted by
 a start and end marker, such as in C, which uses \verb`{` and \verb`}`, but can also be denoted by indentation, if you
 are a masochist and/ or sadist (cough cough Python). A block can contain variables local to said block, and has it's
 own reference environment (the names and identifiers which are accessible to statements in the block).

Most languages allow blocks to be nested within each other, but some may restrict the type of blocks which can nest. In
 C and Java, methods can be nested within classes and other blocks, but not within other methods. In Python and Pascal,
 it is possible to nest a method within a method.

Since blocks contain variable definitions, the local reference environment for each statement must be determined, which
 also allows you to determine the \textbf{scope} of identifiers. The scope determines where each identifier can be used,
 and if an error should be thrown if there are multiple identifiers with the same name. In most languages, the scope of
 an identifier is the block in which it is declared and, by extension, any nested blocks. There is also a scope known as
 the global scope, which allows identifiers within it to be accessed anywhere within the program. These are known as
 global identifiers, and are typically only used when they are absolutely necessary, to reduce the chance of overlapping
 identifiers.

\subsection*{Duplicated Identifiers}

Most languages have a rule or set thereof to determine which declaration of an identifier takes precedence. In languages
 such as Pascal, the local variable \textit{hides} any variables with a larger scope. In this case, it is said that
 there is a hole in the scope of the hidden variable, as it's scope is everywhere aside from any blocks in which it is
 hidden.

Some languages just flat out refuse to allow any duplicated identifiers, as this reduces the confusion since it will
 cause an error at compile time, rather than a mysterious run-time error.

\subsection*{Static and Dynamic Scoping}

In the previous example, the scope of variables is determined by their visibility, and therefore the scope depends upon
 the lexical structure of the program, and the compiler can therefore determine the scope of all variables. This is
 known as static or lexical scoping. On the other hand, with dynamic scoping, the reference environment depends upon the
 sequence of sub-programs, rather than the layout of the nested blocks. This means that the scope of any given identifier
 can only be determined at run-time.

Dynamic scoping is typically less used than static scoping, as it is less reliable due to the unknown order of execution
 and results in poor code readability. It is however, easier to implement using stacks and resolving a variable does not
 require tracing the structure of the entire program.

\subsection*{Lifetimes}

The lifetime of a variable is the period of time for which the variable exists, and has a value that is only known at
 the time of execution. The lifetime of a local variable is the execution of the method. Each recursive execution of a
 method has its own copy of any local variables. The lifetime of a global variable is the execution of the entire program.

This is related to the way that memory is allocated and deallocated. There are several basic mechanisms which a programming
 language may use to allocate memory
\begin{itemize}
  \item Static allocation is when a fixed memory address is retained throughout the variables lifetime
  \item Stack-based allocation is done on a last-in first-out basis, and is used for function calls, as the lifetime of
   the variable is that of the function
  \item Heap-based allocation is used for variables that are dynamically allocated. They often have no identifier, and
   are therefore known as anonymous variables, and can only be referenced by pointers
\end{itemize}