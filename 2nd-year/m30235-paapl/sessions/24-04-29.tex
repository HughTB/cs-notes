\lecture{ADTs and Concurrency}{14:00}{26/04/24}{Jiacheng Tan}

\section*{Abstraction}

\textit{Abstraction} means separating the interface of a thing from the details of its implementation, such that the
 programmer only needs to know how to use it, but not how it does its job. This improves readability, maintainability,
 re-usability and security of software and libraries.

Modern languages typically provide two types of abstraction-- data abstraction and process/ control abstraction.

\subsection*{Data Abstraction}

\textit{Data Abstraction} enforces a clear separation between the abstract properties of a data type, and the concrete
 details of its implementation. The abstract properties are visible to \textit{client code} (any code which makes use of
 the abstracted item) which makes use of the data type. This includes information about methods which can be used to
 manipulate the data, and how items are arranged on a stack. The concrete details of the implementation are kept entirely
 private, and can change version-to-version without affecting any client code.

Data abstraction is achieved in most languages by defining an Abstract Data Type (ADT)-- user defined data types which
 consist of a set of data, and a set of operations which can be performed upon the data. ADTs are often used to implement
 a data structure, which acts as a representation of the data and provides implementations for the operations which are
 allowed on the data. This includes some structures included in languages, such as lists, dictionaries, and more.

\subsection*{ADT Interfaces}

An \textit{interface} is a method of interaction between two parties-- the implementer of the ADT, and the user of the ADT.
 The implementer is responsible for the code which performs the operations in a correct and efficient way, and the user
 is responsible only for the code which interacts with the interface of the ADT. The interface defines the representation
 of the ADT which the user will actually interact with, but abstracts away the implementation details.

When implementing a new ADT, you must select a number of \textit{core operations} which users will need to interact with
 your data type. The standard set of operations which almost all ADTs need are: a method to add an item; remove an item;
 find or retrieve an item. There are also some operations which some ADTs will not need, but are typically implemented
 anyway, such as checking empty/ fullness, retrieving a subset, etc.

Another thing to consider when implementing a new ADT is how you will actually store the data. For example, you need to
 select an \textit{internal storage container}, which is used to actually hold the items in the data structure. Users
 should not need to know or be able to interfere with the internal representation of the data. If users actually need to
 interact with the data, \textit{accessors} and \textit{mutators} should be used.

{PAGE 13}