\lecture{Introduction to Programming Languages}{14:00}{22/01/24}{Jiacheng Tan}

Since there are many different types of application, there are also many types of programming language.
 The main programming domains are as follows
\begin{itemize}
  \item Scientific (e.g. ForTran)
  \item Business (e.g. COBOL)
  \item AI (e.g. LISP)
  \item Systems Programming (e.g. C, C++)
  \item Web Software (e.g. HTML, JavaScript)
\end{itemize}

\section*{Language Categories}

There are several ways to categorise programming languages, such as by uses, paradigms, abstraction level, etc

\subsection*{Machine Languages}

\begin{itemize}
  \item Machine languages directly run on the hardware, using the instruction set of the processor
  \item Machine code is usually written in hexadecimal as this is a more efficient way of displaying the binary which
  represent the instructions
  \item It is very hard for programmers to directly write machine code, as it is not easy to remember instructions
  and it lacks features such as jump targets, subroutines, etc
\end{itemize}

\subsection*{Assembly Languages}

\begin{itemize}
  \item A slight abstraction over machine languages
  \item Each instruction is replaced with an alphanumeric symbol which is easier for programmers to remember and
   understand
  \item They also include features such as subroutines, jump targets, etc which make it much easier to create complex
  programs
\end{itemize}

\subsection*{System Programming Languages}

\begin{itemize}
  \item More abstracted from machine languages, but you are still concerned with low-level functions such as memory
   management
  \item Used to create operating systems, and for embedded applications where low system requirements do not allow
  the use of high-level languages
\end{itemize}

\subsection*{High Level Languages}

\begin{itemize}
  \item Languages that are machine-independant (are not written directly in machine code, and are therefore portable
  between CPU architectures)
  \item Need to be compiled or otherwise translated from text to machine code before they can be run
\end{itemize}

\subsection*{Scripting Languages}

\begin{itemize}
  \item Used to create programs which perform a single, simple task
  \item Thses are used for system administration
  \item Usually interpreted languages
  \item More akin to pseudocode than other programming languages
\end{itemize}

\subsection*{Domain-Specific Languages}

\begin{itemize}
  \item Some languages are designed to perform a specific task much more efficiently
  \item The specific purpose could be just about anything, but are specific to that task and either cannot be used
   otherwise
  or are not well suited for it
\end{itemize}

\section*{Programming Paradigms}

There are several different paradigms which are used in programming
\begin{itemize}
  \item Procedural
  \begin{itemize}
    \item Most programming languages are procedural
    \item A program is made up of one or more routines which are run in a specific order
  \end{itemize}
  \item Functional
  \begin{itemize}
    \item Applies mathematical functions to inputs to get a result
    \item Useful for data processing applications such as data analysis or big data
  \end{itemize}
  \item Logical
\end{itemize}

There are also two major types of programming languages, which are designed for different purposes
\begin{itemize}
  \item Imperative Languages
  \begin{itemize}
    \item Programs are defined as a sequence of commands for the computer to perform
    \item Like a recipe for exactly how to get the desired output
  \end{itemize}
  \item Declarative Languages
  \begin{itemize}
    \item Programs describe the desired results without actually specifiying how the program should complete the task
    \item Functional and logical programming langauges are examples of this
  \end{itemize}
\end{itemize}