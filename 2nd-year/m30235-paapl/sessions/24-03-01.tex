\lecture{Bottom-Up Parsing}{14:00}{01/03/24}{Jiacheng Tan}

Bottom-up parsing works (shockingly enough) in the opposite direction as top-down parsing. A bottom-up parser starts
 with the string of terminals and works backwards to the start symbol, applying the productions in reverse as it goes.

With the grammar
\begin{itemize}
  \item $S \rightarrow AB$
  \item $A \rightarrow aA \mid \varepsilon$
  \item $B \rightarrow b \mid bB$
\end{itemize}
A bottom-up parse of the string $aaab$ would look like
\begin{itemize}
  \item Starting with the right-most symbol, we can apply the production $B \rightarrow b \mid bB$ in reverse, to end up
   with the string $aaaB$
  \item Since neither $a$ nor $aB$ are the right-hand-side of a production, insert $\varepsilon$ to give the string
   $aaa{\varepsilon}B$.
  \item Replace $\varepsilon$ with $A$ to get $aaaAB$
  \item Replace $aA$ with $A$ to get $aaAB$
  \item Replace $aA$ with $A$ to get $aAB$
  \item Replace $aA$ with $A$ to get $AB$
  \item Replace $AB$ with $S$ to get the start symbol, and therefore a valid sentence
\end{itemize}

\section*{vs Top-Down}

Bottom-up parsers are typically more powerful than top-down parsers. There are excellent generator tools, such as
 \verb`yacc` that can build a parser from an input specification, as \verb`lex` does for scanners.

\section*{Shift-Reduce Parsing}

A shift-reduce parser takes a stream of tokens as an input and creates the list of productions used to build a parse
 tree. It uses a stack to track the position in the parsing process and a parse table to determine the correct production
 to use. Shift-reduce parsing is typically the most common and most powerful method of bottom-up parsing.

One type of shift-reduce parser is an LR parser (Scans input from left-to-right, using a reversed right-most derivation)

\section*{The Shift-Reduce Process}

When parsing a string of tokens, $v$, the input is initialised to $v$ and the stack is empty. At each step, the parser
 can take one of four actions at each step - shift, reduce, accept or error. The first step of the process is always
 to shift the first token to the top of the stack.

For this example, the grammar will be as follows
\begin{itemize}
  \item $S \rightarrow E$
  \item $E \rightarrow T \mid E+T$
  \item $T \rightarrow \mathrm{id} \mid (E)$
\end{itemize}

\subsection*{Shift}

The token at the start of the input string is \textit{shifted} onto the top of the stack.

\subsection*{Reduce}

Suppose that the contents of the stack are $qw$ where $w$ is a string of terminal and non-terminal symbols, and $q$ may
 be an empty string. If there is a production such that $A \rightarrow w$, the stack can be \textit{reduced} to $qA$.
 This means that the production for $A$ is applied backwards, such that we replace the right-hand-side ($w$) with the
 left-hand-side non-terminal ($A$). In this case, $w$ is known as a handle.

\subsection*{Accept}

If the entire contents of the stack has been reduced to the start symbol ($S$), and there are no remaining input tokens,
 the input is a valid sentence in the parsers grammar. This means that the parser has ended in an acceptance state.

\subsection*{Error}

If it is not possible to Shift, Reduce or Accept, then the parser must Error. In this case, the sequence on the stack
 cannot be reduced to the left-hand-side of any production, and any further shifting would be pointless as the input
 cannot form a valid sentence in the parsers grammar.