\lecture{Compound Data Types}{14:00}{22/03/24}{Jiacheng Tan}

A compound (or structured) data type is one which is made up of simpler types. This includes types such as arrays,
 strings, records, structs and maps.

\section*{Arrays}

Arrays or lists are the most common compound data type and are found in most programming languages. In general, an array
 has several attributes, namely
\begin{itemize}
  \item The type of its elements (This is also the compound type)
  \item The type of its indices (This is usually integers, but it is possible to use an enumerated type in some languages)
  \item The number of elements in the array
\end{itemize}

Different languages define each of these attributes in different ways. For example, some languages like Dart have
 dynamically sized and allocated arrays, which allows you to increase or decrease the size of the array at runtime, but
 others, such as C or C++, have statically sized and allocated arrays. This means that the size of the array is determined
 at compile time, and cannot change at runtime. Different languages might also use a different starting index for their
 arrays. For example, C uses 0-indexed arrays, but higher level languages such as Lua use 1-indexed arrays.

\subsection*{Dynamic Arrays}

There are two methods of creating dynamic arrays - stack-dynamic and heap-dynamic arrays. A stack-dynamic array has its
 size dynamically set, and the storage is allocated at runtime. They live on the stack, and the size of the array is
 fixed after it's created. Heap-dynamic arrays live on the heap and are able to dynamically change size at runtime.

\subsection*{Heterogeneous Arrays}

A heterogeneous array is one in which the type of the elements is not necessarily the same. This is only supported by a
 few high-level languages: Perl, Python and JavaScript. A heterogeneous array can also be the element of another
 heterogeneous array, which allows for some very funky stuff.

\subsection*{Rectangular \& Jagged Arrays}

A rectangular array is a multi-dimensional array in which each row has the same number of elements and each column has
 the same number of elements.

A jagged array is a multi-dimensional array in which each row has a varying number of elements, columns don't really
 exist as they would be impossible to align meaningfully. This is simply an array of arrays. Supported by C, C++ and Java.

\section*{Strings}

In the vast majority of languages, a string is an array or list of characters. More literally in some languages, such as
 C/C++ and Haskell, but in higher-level languages, it is usually a built-in data type which hides its true type behind
 a `fake' type. These languages also tend to have a set of logical operations such as finding the length of a string,
 and functions such as concatenation.

\section*{Records}

A record is a compound data type which is composed of a number of named elements. These are useful for storing information
 about an object, such as a Person or Student, while encapsulating them into one pseudo-type.

\subsection*{Variant Records}

Records are typically very inflexible, and may be memory-inefficient since all of the values of a record have a fixed
 size and set of fields. For example, if you wanted to store a student record, you would need to store a name, registration
 number, information about their course, and a year of entry and graduation. This means that even if the student has not
 graduated, you still need to store a null value for their year of graduation. Pascal has a \textit{variant record},
 which allows you to store a value for one element, if and only if another element has a specific value. In the case of
 a student record, you could store a boolean value for their graduation status, and then only store a graduation date if
 the status is true.

\subsection*{Unions}

C \& C++ have a type known as a Union rather than variant records. Unions are designed to store data of multiple types
 in the same memory space. When you define a union, you specify the types that should be stored for that value, but then
 you have to set the type and reference it as such elsewhere in the code. This allows you to store values for different
 purposes without wasting memory. You can have only one of the values stored at a time, since they would use the same
 space in memory. The main difference between this and a variant record is that the two values stored in a union are not
 related in any way, and it is possible to ignore the precedent set elsewhere in the code, which can cause issues.

\subsection*{Structs}

Languages such as C, C++, C\# and Rust all have a similar concept to a record, known as a \textit{struct}. These perform
 a similar function to records, but can also include member functions. A member function does not directly access the
 data in the struct (unlike a class), but is used to group a function with the data type it operates on.

\subsection*{Classes}

Structurally, classes and structs are very similar, but a class has a constructor (a function which is called when the
 class is instantiated), and methods rather than member functions. A method is able to directly access the data stored
 in the class, and is therefore able to act as a getter/setter. Another difference is that in most languages, member
 functions in structs are public by default, but methods in classes are private by default.