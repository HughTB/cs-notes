\lecture{Syntax Analysis and Parsing}{14:00}{19/02/24}{Jiacheng Tan}

\section*{Ambiguity}

Some grammars are ambiguous, such that there are multiple valid derivations of any given sentence. This means that the
 parse trees would be different, and therefore could produce different results. For exmaple, using the same grammar as
 the previous lecture, a left-most derivation of the sentence \verb`x + y * z` could be either
\begin{verbatim}
<exp> => <exp> + <exp>
      => x + <exp>
      => x + <exp> * <exp>
      => x + y * <exp>
      => x + y * z
\end{verbatim}
which would give you a parse tree equivalent to \verb`x + (y * z)`, or it could be
\begin{verbatim}
<exp> => <exp> * <exp>
      => <exp> + <exp> * <exp>
      => x + <exp> * <exp>
      => x + y * <exp>
      => x + y * z
\end{verbatim}
which would give you a parse tree equivalent to \verb`(x + y) * z`, which gives you a completely different value.

For almost any language, it is possible to completely remove the ambiguity by introducing new or extra non-terminals
 and rules. For example, if you were to add a new rule that forces the \verb`+` operation to appear higher in parse
 trees than \verb`*`. E.g.
\begin{verbatim}
<exp> -> <exp> + <term> | <term>
<term> -> <term> * <factor> | <factor>
<factor> -> x | y | z
\end{verbatim}
where \verb`term` and \verb`factor` are new non-terminals which have been added to remove the ambiguity.

\section*{The Limits of Context-Free Grammars}

Some programming languages cannot be fully described using only CFGs. For example, if a variable must be defined before
 it is referenced, the context is required to determine whether the reference or decleration comes first. These are 
 known as Context-sensitive properties, and must be resolved by the semantic analyser rather than the syntax analyser.

\section*{Syntax Analysis}

Given some input source code, the goal of syntax analysis is to: find all syntax errors and produce a descriptive error
 for the user; and produce the parse tree for the program to be used in code generation. This process is completed by a
 syntax analyser, sometimes known as the parser. There are several algorithms which can be used for parsing, which fall
 into two categories.

\subsection*{Top-Down Parsers}

Starting at the root (the start symbol of the grammar), each node of the parse tree is visited before it's branches.
 The branches are visited from left-to-right, giving a left-most derivation. When manually performing the derivation,
 you start by replacing the start symbol with the right-hand-side of it's production. Then you replace the left-most
 non-terminal symbol with the right-hand-side of (one of) it's production(s). You repeat this process until the string
 consists only of terminal symbols.

\textbf{INSERT EXAMPLE HERE}

\subsection*{Bottom-Up Parsers}

Starting at the leaves of the parse tree, and progressing towards the root, giving a right-most derivation.

\textbf{INSERT EXAMPLE HERE}

\section*{More on Top-Down Parsers}

Different top-down parsers may use different information or rules to determine which production should be selected to
 replace a non-terminal symbol. Most compare the next input token with the first symbol of each production, these
 parsers are known as predictive parsers. These work using only the next input symbol and the current non-terminal.

\subsection*{Recursive Descent Parsers (RDP)}

A recursive descent parser is an implementation of a parser based upon the BNF of a grammar. An RDP consists of a
 collection of functions (or subprograms), many of which are recursive. Each non-terminal symbol corresponds to a
 single function, which handles parsing that particular non-terminal symbol in the grammar. For example, if you wanted
 to implement the following grammar
\begin{verbatim}
<exp> -> <exp> + <term> | <exp> - <term> | <term>
<term> -> <term> * <factor> | <term> / <factor> | <factor>
<factor> -> integer | (<exp>)
\end{verbatim}
with an RDP, you would need to implement 3 functions - \verb`exp()`, \verb`term()` and \verb`factor()`. Assuming that
 there is another function, \verb`lex()` which updates the variable \verb`nextToken` to be the next token in the
 sentence, each function will need to
\begin{itemize}
  \item Check if the symbol is terminal, in which case make a call to \verb`lex()`
  \item Check if the symbol is non-terminal within the current production, in which case make a call to the corresponding
   function
  \item If it is neither, then there is a syntax error and it should be raised with a helpful message for the user
\end{itemize}

\subsection*{Rules With Multiple Productions}

When parsing a rule with more than one production, it is necessary to select which of the productions should be parsed.
 This can be done in several ways, but in the case of a predictive parser, the production should be selected based upon
 the next input token. The next input token is compared with the first token of each production until either a match is
 found, or all options are expended. If the token does not match any of the productions, there is a syntax error and an
 error should be raised with a helpful message for the user.