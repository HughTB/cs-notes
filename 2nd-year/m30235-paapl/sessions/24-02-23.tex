\lecture{LL(k) Parsers}{14:00}{23/02/24}{Jiacheng Tan}

An LL(k) parser is a top-down, predictive parser. It's name means that it parses from (L)eft-to-right, (L)eft-most
 derivation, with (k) tokens of look-ahead. They are also known as a table-driven predictive parser, since they use a
 stack and a parsing table.

An LL(1) parser parses the input left-to-right, and always using a left-most derivation. In this case, it uses one token
 of lookahead to predict which production should be used. It also uses a stack to store the symbols of the right-hand-side
 of productions, in right-to-left order, as that way the left-most symbol is always at the top of the stack. A parsing
 table is also used to store the rules which the parser should use based upon the input token and which value is at the
 top of the stack.

\subsection*{Parse Tables}

With the grammar
\begin{verbatim}
E  -> TE'
E' -> +TE' | Epsilon
T  -> FT'
T' -> *FT' | Epsilon
F  -> (E) | int
\end{verbatim}
the parse table might look as below
\begin{table}[h]
  \centering
  \begin{tabular}{ c | c | c | c | c | c | c }
    Top of Stack Input & \verb`int` & \verb`+` & \verb`*` & \verb`(` & \verb`)` & \verb`$` \\
    \hline
    $E$ & $E \rightarrow TE'$ & & & $E \rightarrow TE'$ & & \\
    $E'$ & & $E' \rightarrow +TE'$ & & & $E' \rightarrow \epsilon$ & $E' \rightarrow \epsilon$ \\
    $T$ & $T \rightarrow FT'$ & & & $T \rightarrow FT'$ & & \\
    $T'$ & & $T' \rightarrow \epsilon$ & $T' \rightarrow *FT'$ & & $T' \rightarrow \epsilon $ & $T' \rightarrow \epsilon$ \\
    $F$ & $F \rightarrow$ \verb`int` & & & $F \rightarrow (E)$ & & \\
  \end{tabular}
\end{table}

With this parse table, it is quite easy to parse a sentence, as it is a matter of simply picking the rule from the
 table, according to the current non-terminal (on the top of the stack) and the current input symbol and pushing the
 right-hand side of the production back onto the stack. \verb`$` is selected if the end of the input is reached. The
 parsing process begins with the start symbol ($E$) and it's right-hand side ($TE'$) is pushed onto the stack, and so
 $T$ is the top of the stack. If the next token is \verb`int`, we would pick the rule $T \rightarrow FT'$, and so $FT'$
 is pushed onto the stack. As such, the top of the stack is now $F$.

This process is more generally written as
\begin{itemize}
  \item If $X$ and $w$ are both the end symbol, \verb`$`, stop and accept the input
  \item if $X$ is a terminal, if $X = w$, pop $X$ off the stack and get the next token, otherwise halt and give a
   descriptive error to the user
  \item If $X$ is a non-terminal, if there is a production at position $[X, w]$, push the right-hand side onto the stack,
   otherwise halt and give a descriptive error to the user
\end{itemize}

\subsection*{Parse Table Construction}

It is easy to perform an LL(1) parse if the parse table is already available. To construct the table, you must compute
 the first and follow sets of the non-terminals from the grammar. These are sets of terminal symbols. If these sets are
 available, the construction of the table is a simple procedure which can be performed automatically.

\subsection*{First Sets}

For the grammar
\begin{itemize}
  \item $E \rightarrow TE'$
  \item $E' \rightarrow +TE' \mid \epsilon$
  \item $T \rightarrow FT'$
  \item $T' \rightarrow *FT' \mid \epsilon$
  \item $F \rightarrow (E) \mid \mathrm{int}$
\end{itemize}
The first sets of its non-terminals are
\begin{itemize}
  \item First($E$) $=$ First($T$) $=$ First($F$) (Rules 1\&3) $= \{(, \mathrm{int})\}$ (Rule 5)
  \item First($T'$) $= \{*, \epsilon\}$ (Rule 4)
  \item First($E'$) $= \{+, \epsilon\}$ (Rule 2)
\end{itemize}

The first set of a non-terminal symbol, $A$, is the set of terminals which start the sequences of symbols which can be
 derived from $A$. (Written as First($A$)). To calculate First($A$) where $A$ is in the form
 $A \rightarrow X_1, X_2, \dots, x_n$, you must follow the process below
\begin{itemize}
  \item If $X_1$ is a terminal, add $X_1$ to First($A$), and that's it.
  \item Otherwise, add First($X_1$) to First($A$), as any symbol which starts $X_1$ also starts $A$
  \item If $X_1$ can go to $\epsilon$, e.g. it is nullable, add First($X_2$) to First($A$). Repeat this for $X_2$ until
   you find the first non-nullable symbol.
  \item If all of these are nullable, add $\epsilon$ to the first set.
\end{itemize}

\subsection*{Follow Sets}

% Come back to this nonsense