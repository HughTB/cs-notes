\lecture{Implementation and Compilation}{14:00}{02/02/24}{Jiacheng Tan}

There are 3 main methods of implementing a language:
\begin{itemize}
  \item Compilation - Programs are translated into machine language, either before (Compilation) or during (JIT)
   execution
  \item Pure Interpretation - Programs are interpreted by another program, known as an interpreter
  \item Hybrid Interpretation - A compromise between the two, code is compiled into an intermediary language, which is
   then interpreted with a Language Virtual Machine
\end{itemize}

\section*{Compilation}

\begin{itemize}
  \item High-level code is translated into machine code for a specific platform
  \item This results in slow translation, but much faster execution
  \item The compilation process has multiple stages
  \begin{itemize}
    \item Lexical Analysis - Converts characters in the source into lexical units
    \item Syntax Analysis - Transforms lexical units into parse trees which represent the syntactic structure of the
     program
    \item Semantics Analysis - Generate intermediary code
    \item Code Generation - Intermediary code is translated into platform-specific machine code
  \end{itemize}
  \item The program which completes this process is known as the Compiler
  \item During this process, the compiler uses a "Symbol Table", which each stage interacts with
\end{itemize}

\subsection*{Lexical Analysis}

The scanner reads the source code one character at a time and returns a sequence of tokens which are sent to the next
 phase. Tokens are symbolic names for elements of the source language. An example of a token in C++ is the keyword
 `void`, which is a type definition, another example is `;`, which delimits the end of a statement. Each token is also
 stored in the symbol table, along with it's attributes.

\subsection*{Symbol Table}

The symbol table stores all of the identifiers of a source program, along with their attributes. These attributes
 include information such as the type of a variable, the size or length of a string or array, the arguments to be used
 with a function and the types of each argument, etc.

\subsection*{Syntax Analysis or Parsing}

The parser analysis the structure of the source code. The parser takes the output of the lexical analyser as a sequence
 of tokens. It attempts to apply the syntactic rules (or grammar) of the language to the sequence of tokens. The parser
 uses the language's grammar to derive a parse tree for each statement. Parsers usually construct Abstract Syntax Trees
 (ASTs), which are slightly simpler and easier to represent with a computer, but which still represent the same syntax.
 If the syntax tree is invalid for the language's grammar, a syntax error is generated and the compilation process stops

\subsection*{Semantic Analysis}

The semantic analyser catches any other issues that are still valid syntax. For example, if you attempt to add a string
 to a float, it could still be syntactically correct, but semantically makes no sense and is not possible to compute.
 It is also able to find issues with the variable types of function arguments, such as attempting to use a string in the
 place of an integer or float.

\subsection*{Code Generation and Optimisation}

The code optimiser attempts to improve the time and space efficiency of the program. It can do this in several ways,
 such as simplifying constants (e.g. replacing 10 * 10 with 100), removing unreachable code, optimising the flow of
 code, etc.

The final task of the compiler is to generate the final output code. This could be in the form of platform-specific
 machine code, or intermediary code for use with a virtual machine. This stage also deals with scheduling and assigning
 registers for use during execution

\section*{Pure Interpretation}

\begin{itemize}
  \item High level code is directly executed by another program known as the interpreter
  \item There is no syntax or semantics analysis, and there is no optimisation
  \item Only really suitable for small, non-real-time applications
  \item It also often requires more space as it needs to store the symbol table during execution
  \item Very few modern languages use interpreters, other than Python, JavaScript and PHP
\end{itemize}

\section*{Hybrid Interpretation}

\begin{itemize}
  \item A compromise between compilers and pure interpreters
  \item High-level languages are translated or compiled into an intermediary language, using the same compilation steps
   as before
  \item The intermediary code is then run by a platform-specific virtual machine, which interprets the code into machine
   language
\end{itemize}

\section*{Just-in-Time}

\begin{itemize}
  \item Programs are initially translated into an intermediary language
  \item This is then loaded into memory and segments of the program are then translated into machine code just before
   execution
  \item The machine code is then kept in case the function is called again somewhere else in the program
  \item This drastically improves the execution speed as compared to pure interpretation, but is still slower and
   typically less space and memory efficient than a compiled program
\end{itemize}