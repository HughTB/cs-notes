\lecture{Web Application Attacks}{09:00}{15/04/24}{Tobi Fajana}

\section*{Web Applications}

A web server is either software or hardware whose sole purpose is to respond to HTTP requests with requested content,
 for example HTML web pages or JSON API content. Some web servers also support server-side scripting languages, such as
 PHP.
 
There are multiple options for web server software, but the most common two on the internet are Apache and Nginx. Both
 are still updated, and support most modern features, but by nature of being significantly older, Apache is much more
 prone to security issues. It is very important to know what server you are trying to exploit, as they are very different
 and have different vulnerabilities.

\subsection*{Attacks}

Attacks can occur either on the client-side or server-side, which can cause different issues for the server's owner. For
 example, a client-side attack may steal credentials or other sensitive information from the client, without the server
 even knowing, but a server-side attack is usually more devastating since it would most likely have direct access to the
 database which the application uses to store all of it's information.

\subsection*{Finding Attack Vectors}

There are several methods for finding attack vectors, namely
\begin{itemize}
  \item Fuzzing
  \item Encoding
  \item Encryption
  \item etc
\end{itemize}

\section*{SQL Injection}

One example of a server-side attack is an SQL injection attack, which works by tricking the database into executing
 unintended commands, or accessing unintended data using a normal query. The usual attack vector for an SQL injection is
 anywhere that user-input is used as part of a query, for example a search box or registration form.

\subsection*{Types of SQL Injection}

There are three main types of SQL injection, namely
\begin{itemize}
  \item In-Band SQL Injection - Data is returned through the same channel (page, input box, etc) that is used to inject
   the SQL code. This type of injection usually exploits issues with very simple queries, or by exploiting unions or
   errors in the way the query is written
  \item Inferential SQL Injection - Data is not returned at all through the web application, and so it is not possible
   to see the results of the attack'
  \item Out-of-Band SQL Injection - Data is returned through a different channel to the injection channel, for example
   through an email or notification
\end{itemize}

Another method of SQL injection is with a Union-based injection, which uses the \verb`UNION` keyword to append another
 query onto the end of the intended query. This is quite easy to defend against, but can be very powerful if it is
 possible to exploit.

\subsection*{Time-blind SQL Injection}

One method of exploiting a blind injection point is to use a time-delay or sleep function in the query, such that if the
 data you are expecting is in the database, the server will take longer to respond. For example, if you want to check if
 a user with the name `administrator' exists in the database, we could use the following query
 \verb`IF (SELECT user FROM users WHERE username='admin' AND SLEEP(10)) --`. In this case, if there is a user with the
 username `administrator', the server will delay by 10 seconds before responding to the request.

\section*{Defending Against Injection}

There are several methods to prevent injection attacks, for example
\begin{itemize}
  \item Validate, filter and sanitize every input before executing
  \item Using parametrized queries to prevent escaping quotations
  \item Using server-side input validation
  \item Using permissions to limit the scope of each users access to the database
  \item Web Application Firewalls
\end{itemize}