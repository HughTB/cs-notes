\lecture{Windows AD \& Password Attacks}{09:00}{29/04/24}{Tobi Fajana}

\section*{Active Directory}

Active Directory (AD) is a hierarchical structure which stores information about objects on the network. Each object
 could be a user, computer, printer, file share, group of users, etc. It provides centralised management for users and
 computers, which makes it easier for administrators to manage, and users to use as they can log in on any computer.

An Active Directory domain is made up of several services
\begin{itemize}
  \item Domain Services
  \item Lightweight Domain Services (\& Lightweight Directory Access Protocol)
  \item Certificate Services
  \item Active Directory Federation Services
  \item Rights Management Services
\end{itemize}

\subsection*{Security Identifiers}

A security identifier (SID) is a unique value which is used to identify a so-called `security principal' (e.g. security
 group) within Windows. The last 4 digits of an SID are the `Relative ID', and the most important is \verb`0500`, as
 this is the relative ID of an administrator account.

When attacking an AD network, the first thing you should do is \textit{enumerate} all of the devices on the network, and
 gather their SIDs. This allows you to find the Domain Controller on the network, and if you can gain control of that,
 you have effective control over the entire AD network.

\subsection*{Domain Services}

Active Directory Domain Services (ADDS) uses a hierarchical structure consisting of domains, trees and forests. A domain
 is a group of objects, such as users and computers, which share the same AD Database. A tree is a group of one or more
 domains, and a forest is a group of multiple trees. A forest consists of shared catalogues, schemas, application
 information and domain configurations. A schema defines an object's class and attributes within a forest.

\subsection*{Domain Controller}

The domain controller of an AD network is simply the server running ADDS. There can be many domain controllers on a
 network, but only one can be the primary controller at a time. This allows you to have complete fail-over if one of the
 servers were to fail. It also allows some basic load balancing in larger domains, but since one of the controllers is
 always primary, any secondary servers must be in constant communication with the primary.

\section*{Password Attacks}

There are several methods of obtaining passwords, but the easiest will always be social engineering, to get people to
 willingly hand over their passwords. Second to that, you can brute-force or crack a password.

\subsection*{Password Guessing}

While this is the simplest method, it is not fool-proof. This is because while it is a very simple method, it has many
 downsides--
\begin{itemize}
  \item Easy to detect
  \item Takes a long time
  \item Uses lots of bandwidth
  \item Could be blocked at either end
\end{itemize}

\subsection*{Sniffing}

It is possible to `sniff' packets as they move across a network. This typically only works if the passwords are sent in
 plaintext, so not for any websites which use HTTPS, which is basically all of them at this point, since browsers now
 enforce secure connections. It also requires direct access to the network, which may be easy if the target is using
 unprotected wifi, but may be difficult if it is a physical network in a secure building.

\subsection*{Malware}

If malware could be installed to the targets device, it is possible to use a \textit{keylogger}, which records all of
 the keypresses made on a computer, which may include them inputting passwords.