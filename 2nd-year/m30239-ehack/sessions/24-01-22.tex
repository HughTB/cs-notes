\lecture{Introduction to Penetration Testing}{09:00}{22/01/24}{Tobi Fajana}

\section*{Weekly Teaching Materials}

\begin{itemize}
  \item 1 hour lecture
  \item 2 hour practical
  \item Instructional video (Guide for Labs)
  \item Compulsory Moodle quiz
\end{itemize}

\section*{Assessments}

\begin{itemize}
  \item Practical Exam, 2 hours (50\%, 20/03/2024)
  \begin{itemize}
    \item 2 Devices to exploit
    \item 5 Vulnerabilities expected, give the name, software, risk rating, a brief description, and corrective actions for each exploit
  \end{itemize}
  \item Muliple Choice Exam, 1 hour (50\%, May/June)
\end{itemize}

\section*{CIA}

\begin{itemize}
  \item The three main properties which are protected by cyber security
  \item Confidentiality - Protecting information from being disclosed to unintended parties
  \item Integrity - Protecting information from being modified, intentionally or otherwise
  \item Availability - Ensuring the information is available to access when it is needed
\end{itemize}

\section*{Penetration Testing}

\begin{itemize}
  \item Black Box - No information about the target
  \item Grey Box - Some information about the target, but not all
  \item White Box - All information given, including source code, etc
  \item Timeframe - There is usually a fixed timeframe for the test
  \item Penetration Testing is similar to vulnerability assessment, but actual exploits the vulnerabilities
\end{itemize}

\section*{Port Scanning}

\begin{itemize}
  \item Scan every port on a server to check which ports are open
  \item Attempt to ping the server on each port
  \item If a response is given from a port, it must be open
\end{itemize}