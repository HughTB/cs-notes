\lecture{Binary Exploitation}{09:00}{22/04/24}{Tobi Fajana}

If you have permission to read the binary of a program, it is usually possible to extract some of the text which was
 compiled in. This is because \verb`cat` will simply write out the binary to the terminal, and any strings would still
 be encoded as they were before the program was compiled. This means that any code that compares user input to a string
 could accidentally be insecure.

When you \verb`cat` a binary, it will often also output the binary information added to the file by the compiler. This
 often includes information such as which compiler was used, what OS it was compiled by, etc.

While the program is running, it is also sometimes possible to view the stack and heap of the application, which will
 often include sensitive information, such as a string that is being compared against, or the result of some arithmetic
 being done in the background.

\section*{Buffer Overflow}

With the techniques discussed previously, it may be possible to determine the length of a buffer used by the program to
 store user-input. This allows you to determine how long your input needs to be to overflow the buffer, which would then
 crash the program. On some platforms, this can also be used to overwrite the program code and cause unintended code to
 run in place of the program. This typically affects availability, but depending upon the memory layout of the program,
 it could possibly affect the integrity or confidentiality of some data.

\subsection*{Defending Against Buffer Overflow}

There are several methods for protecting against buffer overflows, such as
\begin{itemize}
  \item Address Space Layout Randomization (ASLR)
  \item Data Execution Protection
  \item Structured Exception Handling
  \item Patch devices/ software
  \item Validate Data
\end{itemize}