\lecture{Workshop 15 and 16 Async}{14:00}{12/01/24 (Redone)}{Dalin Zhou}

\section*{Graphical Data Structures}

\begin{itemize}
  \item A graph consists of nodes which have many predecessors and successors
  \item Graphs are used for a wide variety of applications, such as storing a map of a rail network, social media connections, etc
\end{itemize}

\subsection*{Terminology}

\begin{itemize}
  \item Nodes/Vertices
  \begin{itemize}
    \item The actual items, could be locations, people, or items
  \end{itemize}
  \item Edges
  \begin{itemize}
    \item Connections between nodes/vertices
  \end{itemize}
  \item Undirected Graph
  \begin{itemize}
    \item Each edge is just a link, with no particular direction
  \end{itemize}
  \item Directed Graph
  \begin{itemize}
    \item Each edge has a direction (explicit predecessor and successor) and can only be traversed in that direction
  \end{itemize}
  \item Adjacent Vertices
  \begin{itemize}
    \item Vertices connected by a single edge
  \end{itemize}
  \item Path
  \begin{itemize}
    \item A sequence of edges which can be used to traverse from one vertex to another
  \end{itemize}
  \item Degree
  \begin{itemize}
    \item The number of edges or adjacent vertices of a vertex
  \end{itemize}
  \item In/Out-Degree
  \begin{itemize}
    \item Sum of entering/leaving edges respectively
  \end{itemize}
  \item Weighted Graph
  \begin{itemize}
    \item Each edge has an assigned weight
    \item The weight can mean anything (distance, capacity, etc)
  \end{itemize}
\end{itemize}

\subsection*{Graph Representation}

\begin{itemize}
  \item Adjacency Matrix
  \begin{itemize}
    \item An adjacency matrix is a matrix with a position for each possible connection between two vertices
    \item It is an $n$-by-$n$ matrix, where $n$ is the number of vertices in the graph
    \item Each position stores a boolean value, representing whether there is a connection between the two vertices
    \item For a weighted graph, the value is either the weight of the edge, or infinity
    \item For a directed graph, the value remains 0 or 1, but is a 1 for the row->column
    \item The size of the graph must be known in advance, and duplicate edges cannot be stored
    \item If a graph is sparse, a significant part of the matrix will be empty
    \item $O(1)$ efficiency for all edge operations, $O(n^2)$ to insert or delete a new vertex
  \end{itemize}
  \item Adjacency List
  \begin{itemize}
    \item A set of SLLs, one for each vertex
    \item Each SLL contains all vertices which are adjacent to the vertex, stored at the head
    \item One SLL for each vertex in an undirected graph, or 2 for a directed graph
    \item The size of the graph can dynamically change, allows for duplicated edges
    \item For an undirected graph, each edge is stored twice
    \item Space efficient for a sparse graph, less so for a dense graph
    \item $O(n)$ efficieny to search for or delete an edge, $O(1)$ to insert an edge at the head, $O(1)$ to insert a new vertex, $O(n)$ to delete a vertex
  \end{itemize}
\end{itemize}

\subsection*{Traversal}

\begin{itemize}
  \item Depth First Search (DFS)
  \begin{itemize}
    \item Traverse as deep as possible until a dead-end is found
    \item Backtrack to find another dead-end
    \item Repeat until all paths are found
    \item Pesudocode implementation (each vertex starts as being marked "unvisited")
  \begin{verbatim}
push vertex onto stack
mark as visited
while stack not empty {
  pop vertex off stack
  process the vertex
  for each adjacent unvisited vertex {
    push vertex onto stack
    mark as visited
  }
}
  \end{verbatim}
  \end{itemize}
  \item Breadth First Search (BFS)
  \begin{itemize}
    \item Find all adjacent vertices (fan out as much as possible)
    \item Fan out on all adjacent vertices until every vertex is visited
    \item Pesudocode implementation (each vertex starts as being marked "unvisited")
  \begin{verbatim}
enqueue vertex onto queue
mark as visited
while queue not empty {
  dequeue vertex off queue
  process the vertex
  for each adjacent unvisited vertex {
    enqueue vertex onto queue
    mark as visited
  }
}
  \end{verbatim}
  \end{itemize}
\end{itemize}

\section*{Applications of Graphical Data Structures}

\subsection*{Greedy Algorithms}

\begin{itemize}
  \item A decision is made at each stage which seems good at the time, with no consideration for it's consequences
  \item Once this decision is made, it is never reconsidered
  \item It is hoped that a series of relative good decisions leads to globally optimal solution
  \item Greedy algorithms do not consistently find the optimal solution, as they are unable to operate on all of the data, only that which they considered when making local decisions
\end{itemize}

\subsection*{Shortest Path Trees}

\begin{itemize}
  \item A shortest path tree is a tree (subgraph) that contains all of the vertices and a subset of edges, where the path from the root to any other vertex has the shortest distance
  \item One vertex is chosen as the root, which could be strategic, or arbitrary
  \item This is useful for route planning algorithms for road maps, airlines, etc
\end{itemize}

\subsection*{Dijkstra's Algorithm}

\begin{itemize}
  \item Finds all of the shortest paths from a given vertex of a graph to all other vertices at once
  \item Sometimes known as the 'single-source shortest paths algorithm'
  \item The method goes as follows
  \begin{itemize}
    \item Place the source vertex, $S$ at the root of the shortest path tree
    \item For each vertex, $V_i$ that has a single edge from $S$, the distance is stored and $V_i$ is added to a priority queue
    \item All other vertices are given a distance of infinity
    \item Perform this loop until the priority queue is empty
    \begin{itemize}
      \item The vertex in the priority queue with the shortest distance from $S$, $V_s$ is selected and added to the shortest path tree
      \item For each vertex $V_i$ adjacent to $V_s$ that has a single edge from $V_s$ to $V_i$, the distance from $S$ to $V_i$ is calculated
      \item If this new value is less than the value already stored, update the value and add the vertex $V_i$ to the priority queue
    \end{itemize}
  \end{itemize}
\end{itemize}

\subsection*{Spanning Trees}

\begin{itemize}
  \item A graph can contain multiple paths between vertices
  \item A spanning tree is a tree (subgraph) which contains all of the vertices with only one path between each pair of vertices
  \item Any one graph can have many spanning trees
  \begin{itemize}
    \item BFS of a graph gives a breadth-first spanning tree
    \item DFS of a graph gives a depth-first spanning tree
  \end{itemize}
  \item This is useful for planning routes to use the minimum resources, such as a telephone network between towns, connections between terminals on a circuit board, etc
\end{itemize}

\subsection*{Minimum Spanning Tree}

\begin{itemize}
  \item There can be several spanning trees for any given graph, but in a weighted graph you can have more and less efficient spanning trees
  \item The minimum spanning tree is a specific spanning tree which has the lowest total weight
  \item To find the minimum spanning tree, every other spanning tree must also be known and compared
  \item This is useful for planning a new network, with the weight of each edge representing it's cost
\end{itemize}

\subsection*{Prim's Algorithm}

\begin{itemize}
  \item Builds upon a single partial minimum spanning tree
  \item At each step, add an edge connecting the nearest vertex which is not already in the spanning tree
  \item Pseudocode implementation
\begin{verbatim}
while not all vertices in tree {
  examine all vertices in the graph with an endpoint in the tree and the other not
  find the shortest edge and add to the tree
}
\end{verbatim}
\end{itemize}

\subsection*{Kruskal's Algorithm}

\begin{itemize}
  \item Combines sets of partial spanning trees
  \item At each step, add the shortest edge in the graph whose vertices are in different partial minimum spanning trees
  \item Pseudocode implementation
\begin{verbatim}
while tree is incomplete and there are edges to add {
  find the shortest edge which hasn't been considered yet
  if it's endpoints are in different trees {
    join the trees
  }
}
\end{verbatim}
\end{itemize}