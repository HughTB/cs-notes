\lecture{Workshop 5 Async}{15:00}{12/10/23}{Dalin Zhou}

\section*{Sorting Algorithms}

\begin{itemize}
  \item Merge Sort
  \begin{itemize}
    \item Recursively divide the list into two equal halves until there is only one element in each group
    \item Merge the smaller lists into several larger ones, sorting as you go
    \begin{itemize}
      \item Start by checking the first item of each list, move the smaller one to the combined list
      \item Continue comparing the first item from each list until all items have been moved into the combined list
    \end{itemize}
    \item Repeat the merging process until you are left with only one sorted list
    \item On average, the Big-O is $O(n \log_2 n)$
  \end{itemize}
  \item Quick Sort
  \begin{itemize}
    \item Select an item from the array to be the pivot (this can be selected arbitrarily, or randomly)
    \item Any elements which are smaller than the pivot are placed to it's left, and all larger elements to it's right
    \item The pivot will always be placed in it's correct, final position
    \item Then, recursively perform a quicksort on the two halves of the array (everything to the left and everything to the right of the pivot)
    \item The best-case Big-O is $O(n \log_2 n)$, worst-case is $O(n^{2})$. On average, the Big-O is approximately $O(n \log_2 n)$
  \end{itemize}
\end{itemize}

\section*{Backtracking Algorithms}

\begin{itemize}
  \item A backtracking algorithm is one which attempts to search for a solution by constructing partial solutions, and checking if they're consistent with the requirements and limitations of the problem
  \item The algorithm takes partial solutions one step at a time, and if said step violates the requirements, it backtracks a step and tries again
  \item If nothing it can do from there works, it backtracks again until either it can find a suitable solution, or it has tested every option and deems the problem unsolvable
  \item Backtracking algorithms work especially well for problems which have a large solution space (lots of possible solutions) and very strict requirements, as many possible solutions can be eliminated quickly
\end{itemize}