\lecture{Workshop 8 and 9 Async}{11:00}{12/11/23}{Dalin Zhou}

\section*{Self-Balancing Trees}

\begin{itemize}
  \item A self-balancing tree is one which automatically reduces it's height as much as possible
  \item This improves the searching performance within the tree, and since insertion and deletion also require searching, the entire tree becomes more efficient
  \item Rotations are performed on the tree, which reduces it's height
  \item A rotation changes the structure of the tree, but without chaning the order, and as such it remains a BST
  \item There are two popular methods for creating a self-balancing tree
  \begin{itemize}
    \item AVL trees
    \item Red-Black trees
  \end{itemize}
\end{itemize}

\section*{Tree Rotation}

\begin{itemize}
  \item A tree rotation is used to change the structure of the tree, without changing the order of the elements
  \item This is useful for reducing the height of a BST, while keeping it a BST
  \item A psuedocode implementation would look like this
  \item \begin{verbatim}
temp = root.left
root.left = root.left.right
temp.right = root
root = temp
  \end{verbatim}
  \item This would cause a right rotation, and could be used on a left-sided tree. The same operation could be performed on a right-sided tree simply by flipping left and right in the example
\end{itemize}

\section*{AVL Trees}

\begin{itemize}
  \item The heights of the left and right tree differ by no more than 1
  \item Left and right subtrees are also AVLs in and of themselves
  \item Each node of an AVL tree includes an additional variable known as the balance factor
  \begin{itemize}
    \item The balance factor is the height of the left subtree minus the height of the right subtree
    \item A node with a balance factor anywhere from -1 to 1 is considered balanced
    \item A balance factor of -2 or 2 is considered unbalanced, and any other value should be impossible as the tree should've been rebalanced before it was possible for it to become so unbalanced
  \end{itemize}
  \item To rebalance the tree, a rotation is performed on either the entire tree, or a subtree depending upon which node is unbalanced
\end{itemize}

\subsection*{Creating an AVL tree}

\begin{itemize}
  \item Data items are inserted using the normal BST insertion rules
  \item The balance factor of each node from the root to the newly inserted node may change, and so need to be recalculated
  \item If the balance factor of any node is too high, the tree becomes unbalanced
  \item This means that the tree needs to be rebalanced using a rotation
  \item The rotation will take place about the unbalanced node closest to the newly inserted node
  \item To reduce the complexity of rotations, there are only 4 possible rotations, split into two groups
  \item Single Rotations
  \begin{itemize}
    \item Left-Left Rotation (LL)
    \begin{itemize}
      \item The unbalanced node and it's left subtree are left-heavy
      \item The balance factor of the unbalanced node is +2 and it's left subtree has a balance factor of +1
      \item The tree is rebalanced by completing a single rotation to the right
    \end{itemize}
    \item Right-Right Rotation (RR)
    \begin{itemize}
      \item The unbalanced node and it's right subtree are right-heavy
      \item The balance factor of the unbalanced node is -2 and it's right subtree has a balance factor of -1
      \item The tree is rebalanced by completing a single rotation to the left
    \end{itemize}
  \end{itemize}
  \item Double Rotations
  \begin{itemize}
    \item Left-Right Rotation (LR)
    \begin{itemize}
      \item The unbalanced node is left-heavy and it's left subtree is right-heavy
      \item The balance factor of the unbalanced node is +2 and it's left subtree has a balance factor of -1
      \item The tree cannot be rebalanced by a single rotation, so a left rotation around the child followed by a right rotation around the unbalanced node is required
    \end{itemize}
    \item Right-Left Rotation (RL)
    \begin{itemize}
      \item The unbalanced node is right-heavy and it's right subtree is left-heavy
      \item The balance factor of the unbalanced node is -2 and it's left subtree has a balance factor of +1
      \item The tree cannot be rebalanced by a single rotation, so a right rotation around the child followed by a left rotation around the unbalanced node is required
    \end{itemize}
  \end{itemize}
\end{itemize}

\section*{Self-Organising Trees}

\begin{itemize}
  \item A Splay Tree is a type of self-organising tree which ensures that recently accessed nodes are always quick to access again
  \item This is based on the idea that you are likely to need to access the same data again soon
  \item It keeps the most commonly used data near the top of the tree where it can be accessed quickly
\end{itemize}

\subsection*{Splay Trees}

\begin{itemize}
  \item A Splay Tree adjusts itself after every search, insertion or deletion operation
  \item It has excellent performance in cases where some data is accessed more frequently than others
  \item Does not have the minimum height available
  \item Still follows all of the normal rules for a BST
\end{itemize}

\subsection*{Creating a Splay Tree}

\begin{itemize}
  \item Data items are inserted using the normal BST insertion rules
  \item The inserted item is then moved to thr root of the tree using a series of rotations, in a process known as splaying
  \item This means that the most recently inserted item is at the root of the tree, and as such will be the fastest item to access
  \item Splaying always moves a lower item to the top
  \item It uses a series of double rotations until the node is either
  \begin{itemize}
    \item The root of the tree
    \item A child of the root node, in which case a single rotation is used
  \end{itemize}
  \item There are more possible rotations with a Splay tree, but 4 of them are the same as those used in an AVL tree
  \begin{itemize}
    \item A Zig rotation is a sinle rotation and is equivalent to an LL or RR rotation
    \item A ZigZag rotation is a double rotation, equivalent to an LR or RL rotation
  \end{itemize}
  \item The additional type of rotation is known as a ZigZig rotation
  \begin{itemize}
    \item A ZigZig rotation does not reduce the height of the splay tree, but instead essentially reverses the left and right subtree
    \item (If all subtrees are on the right, they are re-ordered and placed on the left of the previously lowest leaf node)
  \end{itemize}
\end{itemize}

\subsection*{Searching a Splay Tree}

\begin{itemize}
  \item Use the normal searching rules for a BST
  \item If the item is successfully found, splay the found item into the root of the tree
  \item If the item is not found, splay the last checked item into the root of the tree
  \item This makes it faster to locate an existing item
  \item It also makes it faster to determine that the item is not in the tree
\end{itemize}

\subsection*{Deleting from a Splay Tree}

\begin{itemize}
  \item Search for the item as usual
  \item If the item is located, either
  \begin{itemize}
    \item Splay the item to be deleted into the root of the tree
    \item Remove the item and replace it in the normal way for a BST (rightmost item in left subtree, or rightmost item in left subtree)
  \end{itemize}
  \item or
  \begin{itemize}
    \item Remove the item and replace it in the normal way for a BST
    \item Splay the parent of the deleted node into the root of the tree
  \end{itemize}
  \item If the item was not located, splay the last checked item into the root of the tree
\end{itemize}

\subsection*{Performance}

\begin{itemize}
  \item No operation is guaranteed to be more efficient than a BST
  \item The best case Big-O is still $O(1)$
  \item The worst case Big-O is still $O(n)$
  \item However, over a series of operations, the average Big-O trends towards $O(\log_2 n)$
\end{itemize}