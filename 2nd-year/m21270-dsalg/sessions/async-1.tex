\lecture{Workshop 1 and 2 Async}{16:00}{27/09/23}{Dalin Zhou}

\section*{Data Structures}

\begin{itemize}
  \item A data structure is a way to store and organise data
  \begin{itemize}
    \item A collection of elements
    \item A set of relations between the elements
  \end{itemize}
  \item Classification of data structures
  \begin{itemize}
    \item Linear
    \begin{itemize}
      \item Unique predecessor and successor
      \item e.g. Stack, Queue, etc
    \end{itemize}
    \item Hierarchical
    \begin{itemize}
      \item Unique predecessor, many successors
      \item e.g. Family tree, management structre, etc
    \end{itemize}
    \item Graph
    \begin{itemize}
      \item Many predecessors, many successors
      \item e.g. Railway or road map, social network
    \end{itemize}
    \item Set
    \begin{itemize}
      \item No predecessor or successor
      \item e.g. A class of students
    \end{itemize}
  \end{itemize}
  \item Static vs Dynamic
  \begin{itemize}
    \item A static data structure has a fixed size, which cannot be exceeded and must be allocated in it's entirety when created
    \item A dynamic data structure has a dynamic size, which can change at runtime and only uses as much memory as is needed for it's contents
  \end{itemize}
  \item Most data structures follow CRUD(S) for their basic operations;
  \begin{itemize}
    \item Create - Add a new element
    \item Read - Read an existing element
    \item Update - Modify an existing element
    \item Destroy (or Delete) - Remove an existing element
    \item (Search) - Search for an element matching certain search conditions
  \end{itemize}
  \item Each program would need a different data structre - there is no one-size-fits-all
\end{itemize}

\subsection*{Abstract Data Types}

\begin{itemize}
  \item An ADT is a collection of data and associated methods
  \item The data in the ADT cannot be directly accessed, only by using the methods defined in the ADT
\end{itemize}

\begin{itemize}
  \item Stacks
  \begin{itemize}
    \item A stack is a collection of objects in which only the top-most element can be modified at any time
    \item This means it is a LIFO (Last-in first-out) structure
    \item A stack must implement the following methods:
    \begin{itemize}
      \item Push - Add an item to the top of the stack
      \item Pop - Remove the item from the top of the stack
      \item Peek - Examine the item at the top of the stack
      \item Empty - Determine if the stack is empty
      \item Full - Determine if the stack is full
      \item A stack is typically implemented using an array, which is hidden behind the methods of the ADT
    \end{itemize}
  \end{itemize}
  \item Queues
  \begin{itemize}
    \item A queue is a collection of objects in which the object which has been in the queue for the longest time is removed first
    \item This means it is a FIFO (First-in first-out) structure
    \item A queue must implement the following methods:
    \begin{itemize}
      \item Enqueue - Add a new item to the tail of the queue
      \item Dequeue - Remove the item at the head of the queue
    \end{itemize}
    \item A queue is typically implemented using an array, which is hidden behind the methods of the ADT
    \item There are two methods of implementing a queue
    \begin{itemize}
      \item Fixed head
      \begin{itemize}
        \item The head of the queue is always at the 0th position in the array, and elements are shifted as they are dequeued
      \end{itemize}
      \item Dynamic head
      \begin{itemize}
        \item The head of the queue changes to the index of the current head of the queue, and elements are not shifted
        \item Since this leaves empty spaces before the head, space may be wasted. To solve this, a circular array could be used
      \end{itemize}
    \end{itemize}
  \end{itemize}
\end{itemize}

\section*{Algorithms}

\begin{itemize}
  \item An algorithm is a procedure that takes a value or set there of as input and produces a value or set there of as an output
  \item A sequence of computational steps that transforms the input into the output
  \item Ideally, we would always use the most efficient algorithm that is available
  \item Classification of algorithms
  \begin{itemize}
    \item Brute-force
    \item Divide and conquer
    \item Backtracking
    \item Greedy
    \item etc
  \end{itemize}
\end{itemize}

\section*{Big-O Notation}

\begin{itemize}
  \item To determine the Big-O of an algorithm
  \begin{itemize}
    \item Count how many basic operations (assignment, addition, multiplication and division) there are for the worst-case scenario (e.g. 10 items in a stack of length 10)
    \item Ignore the less dominant terms of this equation
    \item Ignore any constant coefficients
    \item The remaining terms become the Big-O complexity of the algorithm
  \end{itemize}
  \item Example 1:
  \begin{itemize}
    \item There are $2n + 1$ operations for an algorithm
    \item We can ignore the $+ 1$ as this has less impact, leaving $2n$
    \item We can ignore the constant coefficient $2$, leaving $n$
    \item So the Big-O notation would be $O(n)$
  \end{itemize}
  \item Example 2:
  \begin{itemize}
    \item There are $2n^2 + n + 1$ operations for an algorithm
    \item We can ignore the $+ 1$, leaving $2n^2 + n$
    \item We can ignore the $+ n$, leaving $2n^2$
    \item We can ignore the $2$, leaving $n^2$
    \item So the Big-O notation would be $O(n^2)$
  \end{itemize}
  \item Order of dominance:
  \begin{itemize}
    \item Constant < Linear < Logarithmic < Quadratic < Cubic < ...
    \item $O(1)$ < $O(n)$ < $O(n\log_2 n)$ < $O(n^2)$ < $O(n^3)$ < ...
  \end{itemize}
\end{itemize}