\lecture{Workshop 12 Async}{11:00}{05/12/23}{Dalin Zhou}

\section*{Hash Tables}

\begin{itemize}
  \item Hashing involves using an array for the efficient storage and retrieval of data
  \item Ideally, both the insertion and retrieval of data would have an efficiency of $O(n)$
  \item A hash table allows this, as far as possible with a finite array
  \item A perfect hash table has an infinite length, and has a hashing function which uniquely maps keys to the data they relate to
  \item It is not possible to do either of these things, and so some tradeoffs must be made
\end{itemize}

\subsection*{Hashing Functions}

\begin{itemize}
  \item A good hashing function must
  \begin{itemize}
    \item Be quick to compute
    \item Minimise collisions
    \item Evenly distribute keys
  \end{itemize}
  \item There are a few types of hashing function
  \item Truncation
  \begin{itemize}
    \item Ignore parts of the value and use the remaining parts as the hash value
    \item This is very fast but does nothing to distribute keys evenly, or avoid collisions
  \end{itemize}
  \item Folding
  \begin{itemize}
    \item Split the value into several parts and then combine them in a different way to obtain the hash value
    \item Usually results in better distribution than truncation
  \end{itemize}
  \item Modulo Arithmetic
  \begin{itemize}
    \item Divide the value by the table size and use the remainder as the hash value
    \item To achieve good distribution, the size of the table should be a prime number
  \end{itemize}
\end{itemize}

\subsection*{Collisions}

\begin{itemize}
  \item A collision occurs when more than one data value results in the same key value
  \item In this situation, it must be decided how to handle the collision and there are multiple strategies
  \item Chaining
  \begin{itemize}
    \item Singly linked lists (SLLs) are used to resolve collisions
    \item Elements which collide are stored in an SLL with it's first element stored in the hash table
    \item This is an easy method but is not fast, as SLLs have an efficiency of $O(n)$
  \end{itemize}
  \item Linear probing
  \begin{itemize}
    \item If an element collides, search the hash table linearly from the initial hash position and store the value in the first available slot
    \item This makes both insertion and retrieval slower as there is a chance that the entire table must be searched linearly to check the item is not there
  \end{itemize}
  \item Quadratic probing
  \begin{itemize}
    \item Rather than linearly checking the next slot, check them quadratically
    \item i.e. If a collision occurs, check $h + i^2$, so $h + 1$, $h + 4$, $h + 9$, etc
    \item This works well if the size of the table is prime
  \end{itemize}
  \item Double hashing
  \begin{itemize}
    \item Two hashing functions, can either be the same or different functions, are used
    \item The hash value of the first function is added to that of the second function repeatedly until a slot is found
    \item i.e. If a collision occurs, check $h + g$, $h + 2g$, $h + 3g$, etc
    \item This works well if the size of the table is prime
  \end{itemize}
  \item Random rehashing
  \begin{itemize}
    \item A pesudo-random number generator is used to generate the offsets
    \item Since a seed is used for the random number generator, it is possible to recreate the sequence of random numbers when attempting to retrieve data
    \item This method is excellent at avoiding collisions but much slower than the other available methods
  \end{itemize}
\end{itemize}

\subsection*{Deletion}

\begin{itemize}
  \item When deleting an item from a hash table, you cannot leave that space blank
  \item This is because it would break the following keys in the group which hashed to the same key
  \item Instead, a special value or flag can be used to mark that the space is free for insertion, but searches should continue looking and not stop there
  \item When inserting, we can check if a space is free, or has this special value in it
  \item When searching, we can still check if there is an item in the space, as it is taken up by the flag
\end{itemize}