\lecture{Workshop 10 and 11 Async}{11:00}{18/11/23}{Dalin Zhou}

\section*{Two-Three Trees}

\begin{itemize}
  \item A two-three tree is a perfectly balanced tree in which
  \begin{itemize}
    \item Each node contains one or two keys (a 2-node and 3-node respectively)
    \item Every internal node has either two (for a 2-node) or three (for a 3-node) children
    \item All leaves are on the same level, hence the tree is always balanced
  \end{itemize}
  \item For each individual node,
  \begin{itemize}
    \item The keys of all children in the left subtree are less than the first key
    \item The keys of all children in the centre subtree are greater than the first key
    \item If a right tree exists (making it a 3-node), the keys of all children in the centre subtree are less than the second key
    \item The keys of all children in the right subtree are greater than the second key
  \end{itemize}
\end{itemize}

\subsection*{Searching}

\begin{itemize}
  \item A similar searching method is used to that of a BST
  \item Start at the root node, if the key it not in the root node, move to the only subtree it could be in and recurse
  \item You must compare with both keys of the node before moving to a subtree
  \item Since there can be 3 subtrees for each node, it is effectively a ternary search, rather than a binary search
\end{itemize}

\subsection*{Inserting}

\begin{itemize}
  \item A similar insertion method is used to that of a BST
  \item Unlike a BST insertion, a new node is not created
  \item Use the same searching strategy to find where the key should be inserted, and then depending upon if the node is a 2-node or a 3-node, a different method is needed
  \item If the leaf node contains one key, the new key is inserted into the leaf and the insertion is finished
  \item If the leaf node contains two keys, a new space must be created
  \begin{itemize}
    \item The two existing keys and the new key are of equal importance
    \item The existing 3-node is split into two nodes, in this case node1 and node2
    \item The smallest key is placed in node1 and the largest in node2
    \item The remaining key (the middle key) is promoted to be the parent node
    \item If the existing parent node is a 2-node, then the promoted key is inserted into the existing parent, and the references are set to node1 and node2
    \item If the existing parent node is a 3-node, then the split and promote process is repeated until a new place is found for the promoted key. This may result in the creation of a new root node
  \end{itemize}
\end{itemize}

\subsection*{Deleting}

\begin{itemize}
  \item A similar deletion method is used to that of a BST
  \item The key to be deleted is either
  \begin{itemize}
    \item In a leaf node
    \begin{itemize}
      \item Which is removed, as for a normal BST
      \item This leaves either a 2-node leaf or a hole in the tree
    \end{itemize}
    \item In an internal node
    \begin{itemize}
      \item Which is replaced by it's In Order successor or predecessor, which will be in a leaf node, which is once again normal for a BST
      \item This leaves either a 2-node leaf or a hole in the tree
    \end{itemize}
  \end{itemize}
  \item In either case, we end up with a 2-node leaf or a hole in the tree
  \begin{itemize}
    \item A 2-node leaf can be left as is
    \item A hole in the tree needs to be removed
  \end{itemize}
  \item To remove the hole, you
  \begin{itemize}
    \item Traverse the 2-3 tree upwards towards the root
    \item The hole is propagated through the tree until it can be eliminated
  \end{itemize}
  \item The hole is eliminated by either
  \begin{itemize}
    \item Being absorbed into the tree
    \item Being propagated to the root of the tree, which results in the node being removed and the height of the tree being reduced by one
  \end{itemize}
\end{itemize}

\section*{B Trees}

\begin{itemize}
  \item A B tree is a multi-way search tree which is optimised for reading directly from a disk
  \begin{itemize}
    \item Each node represents a block or page of secondary storage
    \item Accessing a node means reading the keys from secondary storage, which is expensive and slow. This means that we want to reduce the number of nodes as much as possible
  \end{itemize}
  \item A B tree of order $n$ has the following properties
  \begin{itemize}
    \item The root node has either no children, or between 2 and $n$ children
    \item All internal nodes have between $\frac{n}{2}$ and $n$ children
    \item All leaves are on the same level
    \item Each node can store 1 less key than it has children
  \end{itemize}
  \item This also means that a 2-3 tree is a B tree of order 3, and a BST a B tree of order 2
  \item For any given tree of order $n$ and height $h$, we can store $n^h - 1$ keys
  \begin{itemize}
    \item e.g. a tree of order 10 and height 3 can store $10^3 - 1 = 1000 - 1 = 999$ keys
  \end{itemize}
\end{itemize}

\subsection*{Searching}

\begin{itemize}
  \item Start at the root node, and perform a binary search on the keys in the root noed
  \item If the key is found, we are finished
  \item If the node we are currently in is a leaft node, and the key is not found, then the key is not in the tree
  \item Otherwise, follow the required branch to the next node an repeat the process
\end{itemize}

\subsection*{Inserting}

\begin{itemize}
  \item The tree is searched to find the leaf node in which the key should be inserted
  \item If the node is not full
  \begin{itemize}
    \item Insert the key into the node in it's correct order
  \end{itemize}
  \item If the node is full
  \begin{itemize}
    \item The node is split in two and the middle key is promoted upwards to the parent
    \item If the parent node is full, then the split and promote process is repeated until space is found, or a new root node is created
  \end{itemize}
\end{itemize}

\subsection*{Deleting}

\begin{itemize}
  \item Similar to the deletion method for a 2-3 tree
  \item If the key to be deleted is not in a leaf node, then the immediate predecessor or successor will be in a leaf node, which will replace the key to be deleted
  \item Therefore, you need to check if there will be enough items in the leaf node
  \item If more than the minimum number of keys will be left, you can delete the key
  \item If not, you cannot delete the item directly, and so must
  \begin{itemize}
    \item Check adjacent leaf nodes and if possible move items between nodes
    \item If there are no spare items in adjacent nodes, then nodes must be combined, possibly reducing the height of the tree
  \end{itemize}
\end{itemize}

\section*{B* Trees}

\begin{itemize}
  \item In a B* tree, all nodes except the root must be at least two thirds full, as opposed to a B tree where all other nodes must be half full
  \item The frequency of splitting a node is reduced by delaying a split
  \item When a node overflows
  \begin{itemize}
    \item A split is delayed by redistributing the keys between the node and it's sibling
    \item When a split is made, two nodes are split into three
  \end{itemize}
  \item Items are redistributed by
  \begin{itemize}
    \item Moving the median key into the root
    \item Moving the remaining keys into the children of the root, half in each
    \item This leaves room in both nodes for at least 1 additional key
  \end{itemize}
  \item If both the node and it's sibling is full, a split occurs and a new node is created
  \item keys from the split node, it's sibling and the parent are distributed evenly between the now three nodes
  \item The two separating keys are put into the parent node so that it can have enough children
  \item This always leaves the child nodes two thirds full
\end{itemize}

\section*{B** \& Bn Trees}

\begin{itemize}
  \item These forms of B trees allow the "fill factor" to be increased
  \item Some DBMSes will allow the user to select a fill factor anywhere between 0.5 and 1.0
  \item A B tree with a fill factor of 75% or $\frac{3}{4}$ is called a B** tree
  \item In general, a Bn tree is a B tree whose nodes are required to be $\frac{n + 1}{n + 2}$ full
  \item B* trees are much less used than B+ trees
\end{itemize}

\section*{B+ Trees}

\begin{itemize}
  \item What if we need all of the data in a B tree to be presented in ascending order?
  \begin{itemize}
    \item We can use an In Order traversal, but this requires accessing each page of data to be accessed
    \item Or we can use a B+ tree
  \end{itemize}
  \item In a B+ tree, only the leaf nodes contain the data associated with each key or index
  \item Leaf nodes
  \begin{itemize}
    \item Contain a set of keys and their associated data
    \item Are linked together in sequence
    \item Stored in secondary storage
  \end{itemize}
  \item Internal nodes and the root are stored in main memory, and
  \begin{itemize}
    \item Contain only keys
    \item Are used as an index to the leaf nodes
    \item Are known as an index set
    \item Are implemented as a B tree
  \end{itemize}
  \item B+ trees are often used as an alternative to indexed-sequential files
\end{itemize}

\subsection*{Inserting}

\begin{itemize}
  \item A new record is inserted into a leaf node following the same rules as a regular B tree
  \item Scan the index set to located the relevant leaf node
  \item If the leaf node has space, the record is inserted and the index set is unchanged
  \item If the leaf node is full
  \begin{itemize}
    \item The leaf is split
    \item The index set is updated to show the new leaf node
    \item Records are distributed evenly between the old and new leaves
    \item The first key of the second node is copied to the parent node, without it's data
    \item If the parent node is not full, the key is inserted as usual
    \item If the parent node is full, then the splitting process is performed as usual for a B tree
  \end{itemize}
\end{itemize}

\subsection*{Deleting}

\begin{itemize}
  \item If the deletion of the record does not cause an underflow, then delete the record
  \begin{itemize}
    \item No changes are made to the index set, even if the record is also in the index set, as the key is still required to search through the B+ tree
  \end{itemize}
  \item If the deletion of the record causes an underflow, delete the record and then either
  \begin{itemize}
    \item Redistribute the records in the leaf and it's sibling(s), and update the index set
    \item Delete the leaf and move the remaining records to it's sibling(s), and update the index set
  \end{itemize}
\end{itemize}

\subsection*{B+ Trees vs Indexed-Sequential Files}

\begin{itemize}
  \item Advantages of B+ trees
  \begin{itemize}
    \item Automatically reorganises itself during insertions and deletions
    \item Only small, local updates are required
    \item The entire file does not need to be reorganised to maintain performance
  \end{itemize}
  \item Disadvantages of B+ trees
  \begin{itemize}
    \item Extra overhead for insertion and deletion operations
    \item Space overhead for the index set
  \end{itemize}
  \item Disadvantages of Indexed-Sequential files
  \begin{itemize}
    \item Performance degrades as the file grows, as overflow blocks are required
    \item Periodic reorganisation of the entire file is required to maintain performance
  \end{itemize}
\end{itemize}