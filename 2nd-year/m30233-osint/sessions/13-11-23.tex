\lecture{Routing Information Protocol (RIP)}{09:00}{13/11/23}{Athanasios Paraskelidis}

\begin{itemize}
  \item A very old protocol from the 80s
  \item Has support for classless networks
  \item It uses a distance vector algorithm to determine the best path, specifically a variant of the Bellman-Ford algorithm
  \item Therefore, the main metric it uses is hop count
  \item It also means that RIP does not consider bandwidth, load or reliability of each hop
\end{itemize}

\section*{Versions of RIP}

\subsection*{Version 1}

\begin{itemize}
  \item Class-based routing
  \item Limited support for classless routing
  \item Shouldn't be in use any more, as it is very outdated and insecure
\end{itemize}

\subsection*{Version 2}

\begin{itemize}
  \item Auto-summarisation (route aggregation or CIDR) is on by default
  \item Multicast is used rather than broadcast when communicating with other routers
  \begin{itemize}
    \item Routers listen on 224.0.0.9 for updates from other routers, reducing overhead on the network for devices uninterested in RIP updates
  \end{itemize}
  \item Supports simple password authentication. Not very secure but better than nothing
\end{itemize}

\section*{Advertising Routes}

\begin{itemize}
  \item When the router is booted, a new routing table is created and populated with only directly connected networks, and any statically assigned routes
  \item After initialisation, the router sends it's routing table to every immediate neighbour on every interface
  \item It continues to send it's routing table to every neighbour every 30 seconds, even if there is no updated information
  \item An intentional random delay can be added so that each router sends updates at slightly different times, as to not overwhelm the network
  \item It is possible to configure a RIP router to only send updates when necessary
\end{itemize}

\section*{Learning Routes}

\begin{itemize}
  \item Once a new route is discovered by recieving updates from other routers, it is added to the routing table
  \item The router advertises it's new route to all other routers
  \item Every router learns every path to every network
  \item Changes or failures are also advertised to neighbouring routers
  \item The time it takes for this information to spread to all routers is known as convergence time
  \item Once the information is spread to all routers, the network is considered to be converged
  \item Every router continues to broadcast their routing protocol every 30 seconds
\end{itemize}

\section*{The Routing Table}

\begin{itemize}
  \item The routing table contains lots of information about different routes that are known by the router
  \item The first column contains a letter
  \begin{itemize}
    \item A C means that the route is directly connected to this router
    \item An R means that the route was learned using RIP updates
  \end{itemize}
  \item The second column contains the destination network or host
  \item The third column contains the network mask of the destination
  \item The fourth column contains the administritive cost (120 for RIP and 110 for OSPF) and routing cost (in the case of RIP, the number of hops)
  \item The final column contains the next hop, including it's address, the port to be used, and when it was last updated
\end{itemize}

\section*{Routing Timers}

\begin{itemize}
  \item RIP has 3 timers it uses to update it's routing tables
  \item Update Timer
  \begin{itemize}
    \item Determines how often to broadcast the routing table
    \item Defaults to 30 seconds unless configured otherwise
  \end{itemize}
  \item Invalid Timer
  \begin{itemize}
    \item If a route is not in any updated tables within this time, the route is assumed to be no longer available
    \item Defaults to 180 seconds unless configured otherwise
  \end{itemize}
  \item Flush Timer
  \begin{itemize}
    \item If a route is still not heard in any updated tables within this time, after the invalid timer, neighbours are informed that the route should be flushed (removed)
    \item Defaults to 90 seconds unless configured otherwise
  \end{itemize}
  \item The invalid and flush timer are there to reduce the frequency that a route is removed and then added again when an interface goes down for a short time
\end{itemize}

\section*{Preventing Routing Loops}

\begin{itemize}
  \item Only maintain the best routes with the lowest metric within your routing table
  \item Timeout directly connected routes immediately upon failure
  \item Other methods include
  \begin{itemize}
    \item Route Poisoning
    \item Split Horizon
    \item Triggered Updates
    \item Poison Reverse
    \item Maximum of 15 hops
  \end{itemize}
\end{itemize}

\section*{Advantages}

\begin{itemize}
  \item Widely supported on a variety of hardware
  \item Still usable in smaller networks
  \item Reasonable ease of configuration
  \item Good for networks with few to no redundant paths
  \item Good for networks where each link has similar speed and latency
\end{itemize}

\section*{Disadvantages}

\begin{itemize}
  \item The maximum hop count is 15, and so it is unsuitable for large networks
  \begin{itemize}
    \item If the cost for a route is 16, it means the destination is unreachable via RIP
  \end{itemize}
  \item Prone to routing loops due to slow convergence
  \item Uses a significant portion of bandwidth for itself (Known as a chatty protocol) as it broacasts the entire routing protocol over the network periodically
  \item Shortest path is not necessarily the fastest
\end{itemize}

\section*{Preventing Routing Loops}

\subsection*{Split Horizon}

\begin{itemize}
  \item Used in routing protocols other than RIP
  \item Never advertise a route to the router it was learned from
  \item This helps to prevent loops caused by a dead link that was learned from immediate neighbours
\end{itemize}

\subsection*{Hold-Down Timer}

\begin{itemize}
  \item The default value for the hold down-timer is 180 seconds
  \item Once an entry in the routing table is updated, any updates should be ignored until the hold-down timer expires
  \item This helps to improve the routing stability, as the routing table won't be constantly changing
  \item In the event of an intermittent failure of one router, it is possible that the network traffic would be significantly disrupted and cause massive overhead
  \item This is known as route flapping, which can be reduced by the hold-down timer, as it means that the router won't be re-added to the routing table for at least 180 seconds
\end{itemize}

\subsection*{Triggered Updates}

\begin{itemize}
  \item A trigger update is a complementary rule to route poisoning
  \item A router that knows of a failed route will not wait for the update timer to inform neighbours of the failure
  \item This update only includes information about the poisoned routes
  \item This increases the speed at which poisoned routes converge
\end{itemize}