\lecture{Network Services}{09:00}{25/09/23}{Athanasios Paraskelidis}

\section*{DHCP}

\begin{itemize}
  \item DHCP stands for Dynamic Host Configuration Protocol
  \item The main motivation of DHCP was to provide a set of configuration parameters to automatically configure new devices as they are added to the network. The main parameters are as follows:
  \begin{itemize}
    \item IP address
    \item Gateway address
    \item Subnet mask
    \item DNS server address
  \end{itemize}
  \item Before DHCP, either devices were configured manually, or the bootstrap protocol (BOOTP) was used
  \item The main improvements over BOOTP are:
  \begin{itemize}
    \item Support for temporary allocation of IP addresses
    \item DHCP clients can automatically discover the local DHCP server
    \item Once the server is setup, there is almost 0 human interaction unless something goes wrong
    \item Still compatible with BOOTP clients
  \end{itemize}
  \item A lease is the length of time that an allocated IP address can be used before either a new address is needed, or a request to continue using the current address needs to be approved
  \item An IP address can be released by the client if it is nolonger needed, e.g. the device shuts down or nolonger needs to communicate on the network
  \item Advantages
  \begin{itemize}
    \item Saves manually configuring every single device
    \item Ability to move to a different network without having to reset any network settings on the device
    \item Allows more efficient use of the IP space, as inactive devices do not need to keep a lease on an address
  \end{itemize}
  \item Disadvantages
  \begin{itemize}
    \item DHCP uses UDP to configure devices, so the communication is unreliable and insecure
    \item Possibly allows unauthorised clients, but this can be avoided using MAC address white/blacklisting
    \item Potential for malicious DHCP servers setup on a network that provide incorrect network settings
  \end{itemize}
\end{itemize}

\section*{DNS}

\begin{itemize}
  \item DNS stands for Domain Name System
  \item This is the system which devices use to convert the human-readable domain names into IP addresses which computers can use to communicate to the server
  \item DNS is a globally distributed mapping database between domain names and IP addresses
  \item There are 3 main components
  \begin{itemize}
    \item A \textbf{name space}
    \item \textbf{Servers} that make the name space available
    \item \textbf{Resolvers} which make the request from clients to servers
  \end{itemize}
  \item As DNS is globally distributed, some data is maintained locally, but also retrievable globally
  \begin{itemize}
    \item No single server holds all DNS records
  \end{itemize}
  \item A DNS lookup can be performed my any device
  \begin{itemize}
    \item Remote DNS data is usually cached locally to improve performance
  \end{itemize}
  \item DNS has 'loose coherency'
  \begin{itemize}
    \item The database is always internally consistent
    \item Each version of the database has a serial number, which is incremented every time the database changes
    \item Changes to the master copy of the database are replicated to secondary servers regularly, depending upon the timing set by the zone administrator
    \item Cached data expires depending upon timing set by the zone administrator
  \end{itemize}
  \item There is no limit to the size of the database
  \begin{itemize}
    \item Having a very large number of records on one server would decrease performance
    \item Therefore, the database is spread across many servers around the internet
  \end{itemize}
  \item There is no limit to the number of queries which can be made at any time, or by a single user
  \item Queries are usually distributed between multiple DNS servers as well as local caches
  \begin{itemize}
    \item e.g. nameserver1 and 2
  \end{itemize}
  \item Clients can query and use the data from any server, primary or secondary
  \item Clients will typically have their own local cache of more frequently accessed records
  \item DNS uses both TCP and UDP
  \begin{itemize}
    \item TCP is used for communication between servers, for example when replicating records from a primary to secondary server
    \item UDP is used for communication between clients and servers
  \end{itemize}
  \item The database can be updated dynamically
  \begin{itemize}
    \item Add, delete or modify any record on the server
    \item These only need to be performed on the main server, as the secondary servers will replicate the changes over time
  \end{itemize}
  \item There are two main types of servers
  \begin{itemize}
    \item Authoritative
    \begin{itemize}
      \item Primary server - Where data is added and modified
      \item Secondary server - Servers which replicate the primary server to share the load
    \end{itemize}
    \item Non-authoritative
    \begin{itemize}
      \item Caching servers - temporarily retain records from authoritative servers to improve resolving performance and reduce load on authoritative servers
    \end{itemize}
  \end{itemize}
  \item Domains can be resolved either Iteratively or Recursively
  \begin{itemize}
    \item Iteratively
    \begin{itemize}
      \item The client's domain name resolver starts by querying the root nameserver
      \item The root nameserver responds with the address of the nameserver on the next level down
      \item The domain name resolver then queries this nameserver, and so on until the nameserver with the full domain is found
    \end{itemize}
    \item Recursively
    \begin{itemize}
      \item The client's domain name resolver queries the root nameserver
      \item The root nameserver itself queries the nameserver on the next level down
      \item This process repeats until the nameserver with the full address is found
      \item The IP address of the domain is then passed back up the chain and to the requester
    \end{itemize}
  \end{itemize}
\end{itemize}

\section*{Domain Names}

\begin{itemize}
  \item A domain name is the sequence of labels from a node to the root, separated by dots
  \begin{itemize}
    \item e.g. port.ac.uk has 3 labels
    \begin{itemize}
      \item port
      \item ac
      \item uk
    \end{itemize}
    \item There can be up to 127 labels in a domain name
    \item But there can only be 255 characters in the domain overall
  \end{itemize}
  \item The root domain or Top Level Domain (TLD) of a domain is the final label, e.g. the TLD of port.ac.uk would be uk
  \item A subdomain is any domain which resides below the TLD. In the case of port.ac.uk,
  \begin{itemize}
    \item uk is the TLD
    \item ac is the Second Level Domain
    \item port is the actual domain name
  \end{itemize}
  \item Name servers store information about domains in units called zones
  \begin{itemize}
    \item Each zone usually corresponds to a subdomain, for example the .uk TLD has many sub-zones, such as .ac.uk, .gov.uk and .co.uk
  \end{itemize}
\end{itemize}