\lecture{IP Addresses and Subnetting I}{09:00}{02/10/23}{Athanasios Paraskelidis}

\begin{itemize}
\item Layer 3 networking is the layer that handles routing data between networks, such as a local and wide area network
\item Layer 3 functionality is spread over the network - in routers as well as client devices
\end{itemize}

\section*{Internet Protocol}

\begin{itemize}
  \item Connectionless
  \item Best effort delivery
  \begin{itemize}
    \item Unreliable
    \item Packets may not arrive at all, or in the wrong order
    \item There is no built-in error handling other than error checking
  \end{itemize}
  \item When transmitting
  \begin{itemize}
    \item Encapsulates data from the transport layer in datagrams
    \item Adds header data to the datagrams (source and destination addresses, time to live, etc)
    \item Apply the routing algorithm once on the internet
    \item Will continue to route the packet until either it reaches the target network card, or exceedes it's time to live
  \end{itemize}
  \item When receiving
  \begin{itemize}
    \item Check the validity of all datagrams
    \item Reads the header and checks if forwarding is needed
    \begin{itemize}
      \item If the destination address is on the LAN, send the datagram to the destination
      \item If it isn't, relay it to the next router on it's route
    \end{itemize}
  \end{itemize}
\end{itemize}

\subsection*{IPv4 \& Subnetting}

\begin{itemize}
  \item 32-bit string
  \begin{itemize}
    \item 4*8 bit integers (1 byte, or 1 octet)
    \item Each byte can be 0-255
    \item Intended to be human-readable
    \item Allows for $2^{32}$ or $4,294,967,296$ theoretical addresses, less in practice
  \end{itemize}
  \item The IP address is split into two fields
  \begin{itemize}
    \item The Network Prefix
    \item The Host ID
    \item Any devices with the same network prefix are on one network
    \item Two devices cannot have the same Host ID
  \end{itemize}
  \item The network prefix can be 8, 16 or 24 bits in length, leaving 24, 16 or 8 bits for Host IDs
  \item There are 3 main classes of IP addresses
  \begin{itemize}
    \item Class A: Network prefix always starts with 0, followed by 7 bits, and 24 bits for the Host ID (1.X.X.X - 126.X.X.X)
    \item Class B: Network prefix always starts with 10, followed by 14 bits, and 16 bits for the Host ID (128.0.X.X.X - 191.255.X.X)
    \item Class C: Network prefix always starts with 110, followed by 21 bits, and 8 bits for the Host ID (192.0.0.X - 223.255.255.X)
  \end{itemize}
  \item There are also 2 more reserved classes
  \begin{itemize}
    \item Class D: Network prefix starts with 1110, followed by a 28 bit multicast ID (224.X.X.X - 239.X.X.X)
    \item Class E: Network prefix starts with 1111, followed by 28 bits reserved for experimentation (240.X.X.X - 255.X.X.X)
  \end{itemize}
  \item The Class A network 127.X.X.X is reserved for loopback addresses, every computer has 127.0.0.1 as a loopback address, some computers may have more than 1
  \item Each class of network has 2 less usable host addresses than would be expected, e.g.
  \begin{itemize}
    \item A class A network provides $2^{24} - 2$ host addresses ($16,777,214$)
    \item A class B network provides $2^{16} - 2$ host addresses ($65,634$)
    \item A class C network provides $2^8 - 2$ host addresses ($254$)
  \end{itemize}
  \item This is because X.X.X.0 is the network address, which cannot be assigned to a computer, and X.X.X.255 is used as the network's broadcast address
  \item There are 3 ranges of IP addresses that can be used by anyone on a local network, one in class A, B and C
  \begin{itemize}
    \item Class A: 10.0.0.0 - 10.255.255.255
    \item Class B: 172.16.0.0 - 172.32.255.255
    \item Class C: 192.168.0.0 - 192.168.255.255
  \end{itemize}
  \item In order to connect to devices on the internet, a NAT (Network Address Translation) service is needed on the network (usually on the router)
\end{itemize}

\subsection*{Subnet Masks}

\begin{itemize}
  \item A subnet mask tells the computer which parts of the IP address are reserved for the network prefix
  \item The default subnet mask for each class is as follows:
  \begin{itemize}
    \item Class A: 255.0.0.0
    \item Class B: 255.255.0.0
    \item Class C: 255.255.255.0
  \end{itemize}
  \item The subnet mask also determines how many devices can be on the network at a time, as it sets how many bits are left for the Host ID
  \item A network address is the network prefix, with all host bits set to 0, e.g.
  \begin{itemize}
    \item The network address of 10.128.47.87 is 10.0.0.0
    \item The network address of 192.168.10.104 is 192.168.10.0
  \end{itemize}
  \item But why do we need subnets?
  \begin{itemize}
    \item If you are assigned a Class B network address (e.g. 128.147.0.0), you have up to 65534 host addresses available
    \item It's not possible to effectively manage that many clients on a single physical network, and would be a waste of most of those IP addresses
    \item Therefore, we should split the network into subnets which each have a much more managable number of devices
    \item e.g. we could use the 3rd byte of the address as a subnet ID and have 256 subnets, each with 254 hosts
    \item Since the subnet ID is configurable, we could also use 4 bits for the subnet ID, allowing 16 subnets with 4094 hosts each
  \end{itemize}
  \item There is an alternative way to write subnet masks, which is called the slash notation
  \begin{itemize}
    \item This is noted by adding the number of bits reserved for the network address after a slash at the end of the IP
    \item For example, you could say that you have a Class B network with the ID 148.197.0.0 and therefore subnet mask 255.255.0.0, or you could simply write it as 148.197.0.0/16
  \end{itemize}
  \item This allows us to use custom length subnet masks when creating subnets
  \item For example, you may have a Class B address and want to have 32 subnets. To do this, you would need an additional 5 bits for a subnet ID, and therefore should use the subnet mask /21. This allows for 32 subnets with $2^{11}-2$ or $2,046$ hosts each
\end{itemize}