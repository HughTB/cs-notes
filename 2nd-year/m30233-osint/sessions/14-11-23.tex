\lecture{File Systems}{13:00}{14/11/23}{Tamer Elboghdadly}

\begin{itemize}
  \item There are many methods of physically storing a file
  \item Operating Systems reduce the complexity of programs by abstracting the way that each file is accessed
  \item Files typically have some content, and metadata
  \item The main functions that can be performed with a file are
  \begin{itemize}
    \item Creating a new file
    \item Rename an existing file
    \item Deleting an existing file
    \item Opening an existing file into memory
    \item Closing a file in memory
    \item Read data from a file in memory
    \item Write data to a file in memory
    \item Seek to a specific position in a file in memory
    \item Get information from the metadata
    \item Set information in the metadata
  \end{itemize}
  \item Metadata usually contains information such as 
  \begin{itemize}
    \item File Type
    \item File Size
    \item Ownership and Permissions
    \item Time and Date of last modification
    \item Location of the file
  \end{itemize}
\end{itemize}

\section*{Directories}

\begin{itemize}
  \item A directory is a special type of file which contains a list of references to other files
  \item Entries in a directory always point to another file, which can either be a regular file or a directory
  \item Directories contained within another directory are known as child directories
  \item Referenced files or directories are considered to be contained within the directory
\end{itemize}

\subsection*{The Root Directory}

\begin{itemize}
  \item Every file system has a root directory
  \begin{itemize}
    \item In UNIX-based systems, this is known as "/"
    \item In Windows, this is known as "\textbackslash", but also depends upon the drive letter, so the root of the boot drive would be "C:\textbackslash"
  \end{itemize}
  \item Neither files nor directories contain an absolute path to themselves or their children
  \item This means that you have to traverse the file system from the root to get to a file
  \item To locate the file "/home/hugh/cs-notes/README.md"
  \begin{itemize}
    \item Go to the root directory, /, and find the directory "home"
    \item Go to "home" and find the directory "hugh"
    \item Go to "hugh" and find the directory "cs-notes"
    \item Go to "cs-notes" and find the file "README.md"
  \end{itemize}
\end{itemize}

\section*{Blocks and Clusters}

\begin{itemize}
  \item Most file systems store file contents in the unit of storage of the drive it will be stored on
  \item In the case of a HDD, a unit may consist of one or more consecutive sectors
  \begin{itemize}
    \item In UNIX-based systems, these units are known as blocks
    \item In Windwos, these units are known as clusters
  \end{itemize}
  \item Either way, a the size of a block is usually some binary multiple of the physical sector size of the disk (e.g. 1KB, 2KB, 4KB, etc)
  \item Each file is allocated a whole number of blocks to store it's content
\end{itemize}

\section*{Types of File System}

\begin{itemize}
  \item On the user level, most file systems look very similar
  \item This is because the operating system abstracts most of the complexity of a file system, and usually supports more than one file system
  \item Different file systems exist, and the one which should be used depends on many factors, such as
  \begin{itemize}
    \item The type and characteristics of the physical storage
    \item The operating system in use
    \item The version of the operating system
    \item The preference of the person who setup the file system
  \end{itemize}
  \item A few examples are
  \begin{itemize}
    \item FAT (File Allocation Table) - Legacy DOS file system, but still widely used in it's 32-bit variant for smaller devices
    \item NTFS (New Technology File System) - Default boot partition format for modern Windows versions, much more reliable and configurable than FAT
    \item Ext (Extended file system) Family - Default format for most modern Linux OSes
    \item HFS+ (Hierarchical File System Plus) - Default format in modern versions of macOS
  \end{itemize}
\end{itemize}

\section*{File Systems in Detail}

\subsection*{FAT (File Allocation Table)}

\begin{itemize}
  \item The original file system used for MS-DOS, circa 1980
  \item Designed by Bill Gates and used in Microsoft Basic
  \item Updated versions of FAT were still used by Windows up until Windows Millenium Edition
  \item Windows ME replaced FAT with NTFS
  \item Updated versions, such as FAT32, are still used on smaller storage devices as it is usable in all modern operating systems, so allows easy file transfer between them
  \item A FAT32 directory is a 32 byte string and contains the file name, file metadata and a reference to the first cluster of data
  \item Since the directory only contains a reference to the first cluster, all remaining clusters of a file are located in a separate data structure, the File Allocation Table
  \item This uses a linked list, where the links are stored on their own in a dedicated area of the disk
  \item The FAT entries contain only a reference to a single cluster in the data area
  \item They act as a data structure known as a cluster chain, but for all intents and purposes, it's just a linked list
\end{itemize}

\subsection*{Ext (Extended File System) Family}

\begin{itemize}
  \item Used by various UNIX-based systems, such as Linux
  \item The current default format for Linux systems is Ext4
  \item Everything in UNIX is a file, including physical devices such as USB peripherals, DVD drives, and even displays
  \item Allocation follows an inode (index node) approach
  \begin{itemize}
    \item This means that any block can be in an allocated or unallocated state
    \item Every file or directory has exactly one inode
    \item The inode is a small (typically 128 byte) data structure containing metadata and block pointers for the contents of the file
    \item Within the file system, the inode number is the main method of refering to a file or directory
  \end{itemize}
  \item Since directories are also files, they have their own inode
  \item This inode contains a list of names and inode numbers for the file and directories inside the directory, and basically nothing else
  \item Every directory also contains two files, "." and ".." which are a reference to the directory itself, and the previous directory respectively
\end{itemize}

\subsection*{NTFS (New Technology File System)}

\begin{itemize}
  \item Introduced and used for versions of Windows after NT (New Technology)
  \item This includes all modern versions of Windows, such as 10 and 11
  \item A much more complex file system than FAT, which has native support for
  \begin{itemize}
    \item Long unicode file names
    \item Ownership and Permission structures
    \item Encryption
    \item Journaling
    \item Etc
  \end{itemize}
  \item The primary metadata storage is in the Master File Table (MFT)
  \item It contains at least one entry, or file record, for every file and directory
  \item A file record is similar to an inode from Ext
  \item Every file record in the MFT has a fixed size, which is configurable in the boot sector but is usually left at the default of 1KB
  \item The value of any attribute can be resient or non-resident
  \begin{itemize}
    \item Attributes with a short, fixed length will usually be resident, and therefore stored entirely within the file record
    \item Attributes with large values (including \$DATA attributes which store the actual file content) will usually be non-resident, and therefore stored outside the file record with only references to the actual location
  \end{itemize}
  \item Unlike Ext where inodes are stored in a separate space, the entire NTFS file system is able to store clusters
  \item This means that the MFT is paradoxically stored within a file itself
  \item It is also not stored in any specific physical location, and can be spread around the disk
\end{itemize}