\lecture{Concurrency}{13:00}{03/10/23}{Tamer Elboghdadly}

\begin{itemize}
  \item A sequential system is one where multiple threads are exectued to completion, one after the other
  \item A concurrent system is one where multiple threads are exectued at the same time
  \item A parallel system is one where multiple threads are executed in parallel, usually on multiple hardware threads
\end{itemize}

\begin{itemize}
  \item There are several issues that may arise from concurrency, especially in cases where threads of the same process are run on multiple processors in parallel
  \begin{itemize}
    \item A thread is a sequence of instructions defined by a program or part there of
    \item A process has one or more threads, each of which may be run in parallel
    \item Threads within a single process can share resources with each other, but have their own thread of execution
  \end{itemize}
\end{itemize}