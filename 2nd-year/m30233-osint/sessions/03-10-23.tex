\lecture{Concurrency}{13:00}{03/10/23}{Tamer Elboghdadly}

\begin{itemize}
  \item A sequential system is one where multiple threads are exectued to completion, one after the other
  \item A concurrent system is one where multiple threads are exectued at the same time
  \item A parallel system is one where multiple threads are executed in parallel, usually on multiple hardware threads
\end{itemize}

\begin{itemize}
  \item There are several issues that may arise from concurrency, especially in cases where threads of the same process are run on multiple processors in parallel
  \begin{itemize}
    \item A thread is a sequence of instructions defined by a program or part there of
    \item A process has one or more threads, each of which may be run in parallel
    \item Threads within a single process can share resources with each other, but have their own thread of execution
  \end{itemize}
  \item When multiple threads are run concurrently, the result is usually non-deterministic
  \begin{itemize}
    \item This means that it is impossible to predict the order in which the instructions will run
    \item This can lead to a program giving completely different results depending upon which order the instructions run
  \end{itemize}
  \item Non-determinism can also lead to race conditions if multiple threads access the same variable
  \begin{itemize}
    \item A race condition can lead to counting errors, data corruption or a complete deadlock of the program, depending upon which order the instructions execute
    \item A specific type of race condition is interference, which is when multiple threads operate on the same variable, and interfere with each others results, such as two threads adding to a value, and the second thread overwriting the value of the first thread
  \end{itemize}
  \item One method of avoiding non-determinism is to avoid threads sharing any variables at all. This works perfectly for some applications, but is very restrictive and can greatly reduce the efficiency of certain algorithms
  \item Any part of a program which requires using shared variables is known as the critical section, which must not be run at the same time as the critical section of any other thread
\end{itemize}