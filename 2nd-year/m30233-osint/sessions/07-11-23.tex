\lecture{Interprocess Communication}{13:00}{07/11/23}{Tamer Elboghdadly}

\begin{itemize}
  \item Threads can pass information between them using shared variables, as they use the same address space
  \item Since each process has it's own address space, they can't share variables with other processes
  \item This means that a specific mechanism to allow communication between processes, which is known as Interprocess Communication or IPC
  \item There are many different methods and implementations of IPC, which are designed for different purposes, such as communicating between threads on different computers
  \begin{itemize}
    \item Pipes and Sockets (Stream oriented)
    \item Shared Memory (Memory oriented)
    \item Message Passing (Message oriented)
    \item Remote Procedure Call (Procedure oriented)
  \end{itemize}
\end{itemize}

\section*{Single System IPC}

\begin{itemize}
  \item Communication between two or more processes within the same computer
\end{itemize}

\subsection*{Pipes}

\begin{itemize}
  \item One of the simplest forms of IPC
  \item Commonly used in Unix-based systems, such as Liunx
  \item A pipe has an input and an output
  \begin{itemize}
    \item A stream of bytes from one process is written to the output
    \item The same stream of bytes can then be read from the input to the second process
  \end{itemize}
  \item The pipe is a one-way file, which can be read only by one process and written by the other
  \item The kernel buffers data between the two threads, specifically using a bounded buffer (typically 64KB in size) with blocking (when a process is reading or writing, the other process must wait)
  \item The pipe is created by a parent process, and can be used to communicate one-way between two child processes
  \begin{itemize}
    \item The parent process is usually the user's shell, with which they can pipe information from one command into another
  \end{itemize}
\end{itemize}

\subsection*{Shared Files}

\begin{itemize}
  \item Multiple process access one file to communicate
  \item The file requires a lock to ensure only one process is accessing it at a time
\end{itemize}

\section*{Multi-System IPC}

\begin{itemize}
  \item Communication between two or more processes on two or more computers
\end{itemize}

\subsection*{Message Passing}

\begin{itemize}
  \item Similar to a pipe, but the data is split into messages of a certain size
  \item It uses connectionless messaging
  \item This can cause issues as there is no guarantee that a message was recieved, or that they were recieved in the correct order
  \item This can be solved using a buffer on the sender side
  \begin{itemize}
    \item Zero-capacity buffer - The Sender will always wait for the reciever to acknowledge the message
    \item Finite-capacity buffer - The Sender will continue to send messages until the buffer is full, messages are removed from the buffer when the reciever acknowledges them
    \item Infinite-capacity buffer - The Sender will continue to send messages forever, regardless of acknowledgement
  \end{itemize}
  \item There are a few implementations available, but the standard is the Message Passing Interface, which is used by open source projects, as well as hardware vendors
\end{itemize}

\subsection*{Sockets}

\begin{itemize}
  \item Sockets are a form of message passing more similar to pipes
  \item They can connect processes on different computers, including different operating systems or over the internet
  \item They are more similar to pipes as they can be used in UDP mode, allowing for stream communication rather than the message-based communication of TCP
  \item The reciever creates a socket with a certain port, and the sender opens (connects to) the socket based upon it's IP address and port
  \item If the connection is accepted, the sender can start to send a stream of data over the network
  \item This means that it is a connection-based protocol
\end{itemize}

\subsection*{Remote Procedure Call}

\begin{itemize}
  \item A server exposes a set of procedures (functions) which other computers are able to call
  \item A client is then able to call a remote procedure on a server, and recieve the results of the procedure
  \item This allows for low-powered clients to run high-power functions on a remote server
  \item It also makes it easier to write client/server based programs, as the programmer only needs to write the application itself, and not the protocol that the application will use for communication
  \item There are some issues with RPC
  \begin{itemize}
    \item If the two computers use a different architecture or operating system, they may need to convert results between different formats
    \item This requires External Data Representation (XDR), which is a format both computers are able to use, which is converted to/from on both ends
  \end{itemize}
  \item There are several implementations
  \begin{itemize}
    \item CORBA, Common Object Request Broker Architecture
    \item RMI, Java Remote Method Invokation
    \item SOAP, Simple Object Access Protocol
    \item WCF, Windows Communication Foundataion
  \end{itemize}
\end{itemize}