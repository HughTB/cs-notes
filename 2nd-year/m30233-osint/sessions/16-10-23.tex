\lecture{Variable Length Subnet Masks}{09:00}{16/10/23}{Athanasios Paraskelidis}

\section*{Subnetting with VLSM}

\begin{itemize}
  \item Using classful subnetting wastes a lot of IP addresses, so a new approach was needed - VLSM
  \item VLSM allows for more than one subnet mask within the same network, allowing for different subnets to be different sizes, based on their needs
  \item This also allows for much more efficient used of the IP address space
  \item Also allows route aggregation, meaning that less routing information is needed
  \item Links between routers are in and of themselves networks - although they only need enough IP addresses for the routers, so a direct link only needs 2 usable addresses
  \item The smallest subnet you can have is a /30 subnet, which has 4 IP addresses, with only 2 usable
  \begin{itemize}
    \item This is usually used only for point-to-point links, such as a connection between 2 routers
  \end{itemize}
\end{itemize}

\begin{itemize}
  \item The routing protocol needs to support VLSM, by carrying extended network prefixes
  \begin{itemize}
    \item Some routing protocols that support VLSM are: RIPv2, OSPF, EIGRP, BGP
  \end{itemize}
  \item When using VLSM, we can completely ignore the standard classes of IP addresses, as long as we have been assigned the full address range
  \item When assigning subnets with VLSM, you should consider the largest subnet first, and then use the remaining space to create any additional subnets
  \item If needed, you can create sub-subnets within larger subnets to allow creating more, smaller subnets
\end{itemize}

\section*{Supernetting}

\begin{itemize}
  \item Supernetting is combining multiple small (usually class C) networks into one larger one
  \item This creates a single address space with more addresses
  \item You can only use supernetting if the networks you wish to combine are contiguous
  \item Convert the addresses into binary, and count from left to right the number of identical bits. This gives you the supernet mask for the network
  \item The supernet address will always be the network address of the first of the contiguous networks
  \item You need to make sure that the supernet mask does not allow you to address IPs outside of your contiguous networks
\end{itemize}