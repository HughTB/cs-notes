\lecture{Supernetting \& Classless Inter-Domain Routing}{09:00}{30/10/23}{Athanasios Paraskelidis}

\section*{CIDR}

\begin{itemize}
  \item Also known as Supernetting
  \item Considered as a fundamental solution to reducing the size of routing tables
  \item Also used as a temporary solution to the lack of internet addresses in IPv4
  \item A Class C address is too small for most companies, so most requested a class B address, but in the process wasted lots of addresses, as they did not need as many addresses as it provided
  \item CIDR is similar to VLSM, but applies to the entire internet rather than one specific network
  \item For CIDR to work, the network prefix must always be specified
  \item Routers on the internet nolonger use classes, and only the network prefix is important
  \item Since the network prefix is important for routers to know, the only requirement for CIDR to work is for the routing protocols to forward network prefixes
  \item CIDR will only provide any routing advantages if topologically significant addresses are used (addresses which all reside within one routing area)
\end{itemize}

\section*{Supernetting}

\begin{itemize}
  \item Supernetting is combining multiple small (usually Class C) networks into a larger one to create a larger range of addresses
  \item The IP addresses you wish to supernet must be contiguous
  \item The exact opposite of subnetting, and reserves part of the network address to use as host addresses
  \item You must have control over enough networks to make up an entire bit (if you wanted to reserve 2 bits, you would need 4 contiguous addresses)
  \item Your supernet address becomes the first IP address, followed by the new subnet prefix
\end{itemize}