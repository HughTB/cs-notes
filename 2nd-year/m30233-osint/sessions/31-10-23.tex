\lecture{Processes and Scheduling}{13:00}{31/10/23}{Tamer Elboghdadly}

\section*{Processes}

\begin{itemize}
  \item A process is a collection of threads which belong to a program
  \item Several copies of the same program can be running at once, using separate processes
  \item A process has a space in memory which is split into multiple sections
  \begin{itemize}
    \item Program code - The actual code for the program
    \item Data section - Global and static variables
    \item Heap - Memory which can be dynamically allocated during runtime
    \item Stack - Memory used temporarily during runtime when functions are called
  \end{itemize}
  \item Part of this memory is known as the execution state, which includes the program counter, stack and data section
\end{itemize}

\section*{Process Control Blocks}

\begin{itemize}
  \item The operating system maintains a process table
  \item Within this process table, each process has a Process Control Block
  \item For processes which are not currently running, the PCB contains
  \begin{itemize}
    \item The contents of all registers at the time the process was last running
    \item Pointers associated with memory management for the process
    \item Any other information which may be needed to restore the process to a running state, exactly as it was before
    \item It does not contain variables, as these are assumed to be in the processes' main memory block
  \end{itemize}
\end{itemize}

\section*{Dispatcher}

\begin{itemize}
  \item The dispatcher actually gives control of the processor over to the process selected by the scheduler
  \item This involves
  \begin{itemize}
    \item Stopping the currently running process
    \item Storing it's information in the PCB associated with it
    \item Re-loading the information in the new process' PCB
    \item Switching between Kernel and User mode
    \item Jumping to the correct point in memory to resume execution
  \end{itemize}
  \item This set of tasks is collectively known as context switching
\end{itemize}

\section*{Scheduling}

\begin{itemize}
  \item The scheduler is responsible for selecting the next process to run on the processor
  \item To do this, it uses a scheduling algorithm, which takes into account the following:
  \begin{itemize}
    \item When should processes be swapped?
    \item Under what circumstance should the process be swapped? (Time used, if a process stalls, etc)
    \item How long should threads be given the CPU?
    \item Which order should processes run in?
  \end{itemize}
  \item Processes which are ready to be run, but waiting for processor time are placed into the ready queue
  \item The scheduling algorithm is responsible for deciding when to swap the currently running process and the one at the start of the ready queue
  \item Typically, when a process is stopped, it is placed at the rear of the ready queue, unless it was stopped to wait for a resource
  \item There is usually also a waiting queue, where processes which are not yet ready to continue go to wait for whatever is stalling them
\end{itemize}

\section*{System Calls}

\begin{itemize}
  \item Since user processes are run in user mode, they do not have direct access to hardware or other resources
  \item Therefore, the must user so-called system calls to request that the operating system performs a function on their behalf
  \item System calls are usually implemented using software interrupts (sometimes called traps) so that the running process is immediately suspended while the operating system runs the function
  \item System calls include the following functions:
  \begin{itemize}
    \item Creating a new process
    \item Waiting for another process to exit
    \item Exiting the current process
    \item Creating or opening a file
    \item Closing a file
    \item Deleting an existing file
    \item Reading from or writing to a file
    \item Creating or removing a directory
    \item Setting the working directory
    \item etc
  \end{itemize}
\end{itemize}