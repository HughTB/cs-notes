\lecture{Development Process Models}{11:00}{06/10/23}{Claudia Iacob}

\section*{System Requirements}

\begin{itemize}
  \item Functional Requirements
  \begin{itemize}
    \item What does the system need to do?
    \item What features does it need?
  \end{itemize}
  \item Non-functional Requirements
  \begin{itemize}
    \item Overall constraints of the system
    \item What hardware does it need to run on?
    \item How quickly does it need to run?
    \item How fast should each endpoint of an API be?
  \end{itemize}
  \item Requirements should be as specific as possible, to avoid confusion between teams
  \item They should also take into account what data is needed for the system to run, as well as it's computational requirements
\end{itemize}

\section*{Software Design}

\begin{itemize}
  \item More than just the design of the UI
  \item Also includes
  \begin{itemize}
    \item Algorithms to be used for data processing
    \item Use case modelling
    \item Architectural design - How the application will be structured overall (frontend and backend and how they will be connected)
    \item Database design - How will the data be stored in the database. What data needs to be stored?
    \item Behaviour modeling - How will the parts of the system interact? How will errors be handled?
  \end{itemize}
\end{itemize}

\section*{Software Implementation}

\begin{itemize}
  \item Coding and debugging
  \item Writing Documentation
  \item Continuous integration - deploy new versions of code automatically after testing functionality
  \item Version control - manage versions of code
\end{itemize}
  
\subsection*{Software Testing and Evaluation}

\begin{itemize}
  \item Evalutation - Are the non-fucntional requirements met?
  \item Validation - Is this the right system for the problem?
  \item Verification - Are the functional requirements met?
  \item Acceptance - Does the client agree that requirements are met, and will they accept and use it?
\end{itemize}

\section*{Software Development Lifecycle Models}

\begin{itemize}
  \item Iterative
  \item Incremental
  \item Aglie
  \item Reuse
  \item Waterfall
\end{itemize}

\subsection*{Iterative}

\begin{itemize}
  \item Iterate over the Specification, Design, Implementation and Testing
  \item The first iteration is usually a basic mockup to check that the client requirements are understood
  \item Gather feedback after each iteration, and keep improving until the final product is finished
\end{itemize}

\subsection*{Incremental}

\begin{itemize}
  \item Specify multiple increments based upon the specification
  \item Each increment could be as simple as one feature, or could itself need to be broken down more in a very complex system
  \item Each increment should be designed and implemented on their own, and then integrated and tested with all other increments, until the final product is reached
\end{itemize}

\subsection*{Reuse}

\begin{itemize}
  \item Create the complete specification for the system
  \item Discover all available existing software and evaluate how they could be integrated
  \item Adapt existing software and write new systems to interface with it
  \item Integrate all reused and new software into a final system matching the specification
\end{itemize}

\subsection*{Waterfall}

\begin{itemize}
  \item Create the specification
  \item Create the complete design of the overall system
  \item Convert the design into code and integrate systems with each other
  \item Test the system as a whole
  \item Deploy and maintain the system. If there are any issues with the software, go back to the specification stage
\end{itemize}