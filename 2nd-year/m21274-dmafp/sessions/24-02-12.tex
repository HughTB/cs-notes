\lecture{Tuples, Lists and Strings}{12:00}{12/02/24}{Matthew Poole}

\section*{Characters and Strings}

Haskell includes both \verb`Char` and \verb`String` data types. The \verb`Data.Char` module includes some useful
 functions for working with characters, such as \verb`toUpper` and \verb`toLower`, which do as they suggest. There is
 also the function \verb`isDigit`, which is useful for checking if a string or character can be parsed into a number.
 As with most programming languages, Strings are defined as a list of characters.

\section*{Tuples}

Tuples are used to combine several pieces of data into a single value which can be passed between functions more easily.
 For example, if we wanted a function to return someone's name, and the score they got in a test, it could return that
 using a tuple on the form \verb`("Thomas", 68)`. The specific type of that tuple is \verb`(String, Int)`, a singular
 data type. This allows you to use it as an input or return type for any function, for example, the following function
 takes two student's scores and returns the name of whichever student got the higher score
\begin{verbatim}
betterScore :: (String, Int) -> (String, Int) -> String
betterScore (name1, score1) (name2, score2)
  | score1 >= score2  = name1
  | otherwise         = name2
\end{verbatim}

When using tuples or other composite data types in Haskell, it is a good idea to define a type synonym, such as
 \verb`:type StudentMark = (String, Int)`. This can then be used in the code rather than writing the full type definition
 each time, for example, \verb`betterScore` can be re-written as follows
\begin{verbatim}
betterScore :: StudenMark -> StudentMark -> String
betterScore (name1, score1) (name2, score2)
  | score1 >= score2  = name1
  | otherwise         = name2
\end{verbatim}

\section*{Polymorphic Functions}

A polymorphic function is one which has multiple definitions with different input types. For example, the length
 function defined in the prelude works on any type of list, and always returns the number of items in the list. The
 length function actually uses a type variable, which can take an arbitrary type which can be referenced in the function.
 The type definition of length is \verb`length :: [a] -> Int`, which is known as the function's \textbf{most general
 type}. You can define a polymorphic function in several ways, but if you don't include a type definition, and just
 defined the function itself, e.g. \verb`square n = n * n`, then Haskell attempts to infer the most general type by
 analysing the structure of the function. In this case, it is inferred as \verb`square :: Num a => a -> a`, which means
 that a can be any numeric data type.

\section*{Lists}

Lists are used to store any number of values of the same type, as in any other language. In Haskell, they are the main
 data structure, and are used extensively in actual programs. Lists are defined as in most other languages, e.g.
 \verb`[1, 2, 3, 4, 5]`. The data type of a list can be defined in a function definition by surrounding the data type
 with square brackets, e.g. \verb`stringList :: [String]`. The empty list (\verb`[]`) can be of any data type. Strings
 in Haskell, as with most languages, are defined as a list of characters, quite literally defined as
 \verb`:type String = [Char]` in the prelude. This means that any operation working on lists will also work on strings,
 such as concatenating.

When creating a list, you can also use a range format to populate a list, for example \verb`[1 .. 5]` is the same as
 writing out \verb`[1, 2, 3, 4, 5]`. This also works with floats and characters and therefore strings, as
 \verb`['a' .. 'z']` gives a string containing the entire alphabet.

\subsection*{List Comprehension}

A list comprehension is effectively a method of mapping one list onto another. For example, if we have the list
 \verb`a = [1, 2, 3, 4, 5]`, then the comprehension \verb`[ 2*i | i <- a ]` would give the value \verb`[2, 4, 6, 8, 10]`.
 The data type of the output list does not have to be the same as the input list, which allows you to check through a
 list and return a list of boolean values all at once.

You can also add a test at the end of the generator (\verb`i <- a`) which will only add a value to the output list if
 the input value passes the test. For example, if we modified the previous comprehension to be \verb`[ 2*i | i <- a, i < 5]`
 it would only return the values \verb`[2, 4, 6]`, as only the input values \verb`1, 2 & 3` pass the test.

Rather than a single variable, you can use a pattern on the left side of the \verb`<-` to extract multiple values. For
 example, if we have a list of tuples and wish to add the two values together, we could use the comprehension
 \verb`[ i+j | (i,j) <- b ]`.

You can also use comprehensions within functions. If you wanted to define a function to do that pair addition, you can
 write it as follows
\begin{verbatim}
addPairs :: [(Int, Int)] -> [Int]
addPairs pairs = [ i+j | (i,j) <- pairs ]
\end{verbatim}

\subsection*{List Functions}

Every list function is polymorphic, and can be used on any type of list, as long as both input lists are of the same
 type. More specifically, they all have the same type definition of \verb`[a] -> [a] -> [a]`

The list function \verb`:` is probably one of the most used list functions, and adds an element to the front of a list,
 e.g. \verb`3:[5, 7, 2]` returns \verb`[3, 5, 7, 2]`.

The \verb`++` operator joins two lists together, e.g. \verb`[1, 2, 3] ++ [4, 5, 6]` returns \verb`[1, 2, 3, 4, 5, 6]`.
 Since strings are lists of characters, this is also how you concatenate strings together, e.g. \verb`"hello " ++ "there"`
 returns \verb`"hello there"`.

The \verb`!!` operator returns the element at a given position in a list, e.g. \verb`["one", "two", "three"] !! 2`
 returns `"three"`, since Haskell uses 0-indexed lists. This is not used very often in Haskell, but is still useful to
 know just in case it's ever needed.

Finally, the \verb`null` function checks if a list is empty, e.g. \verb`null [1, 2]` returns \verb`False` and
 \verb`null []` returns \verb`True`