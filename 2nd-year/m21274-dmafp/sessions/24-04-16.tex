\lecture{Trees}{17:00}{16/04/24}{Janka Chlebikova}

A tree is a subclass of graph which is connected, but has no cycles (known as acyclic). Alternatively, you could say
 that there is exactly one path between any two nodes in the graph. One basic property of such trees is that if the tree
 contains more than one vertex, it must contain a vertex of degree one. Said vertex is known as a leaf, or terminal vertex.

It is very easy to identify if a graph is a tree, even without drawing it. This is because a graph with $n$ vertices is
 a tree if and only if it has $n - 1$ edges. If the graph is unconnected, you can automatically discount it from being a
 tree, since it has to be a connected graph.

\begin{minipage}[c]{0.45\linewidth}
  \begin{figure}[H]
    \centering
    \usetikzlibrary{graphs}
    \begin{tikzpicture}
      \tikzset{
        node distance=1.5cm,
      }

      \node[vertex] (ve) {$e$};
      \node[vertex, left of=ve, above of=ve] (vb) {$b$};
      \node[vertex, right of=ve, above of=ve] (vc) {$c$};
      \node[vertex, left of=ve, below of=ve] (vf) {$f$};
      \node[vertex, right of=ve, below of=ve] (vg) {$g$};
      \node[vertex, left of=vb, above of=vb, xshift=-1cm] (va) {$a$};
      \node[vertex, left of=vb, below of=vb, xshift=-1cm] (vd) {$d$};

      \draw (va) edge[] (vb)
            (va) edge[] (vd)
            (vb) edge[] (vd)
            (vb) edge[] (vc)
            (vd) edge[] (vf)
            (ve) edge[] (vb)
            (ve) edge[] (vc)
            (ve) edge[] (vd)
            (ve) edge[] (vf)
            (ve) edge[] (vg)
            (vf) edge[] (vg);

    \end{tikzpicture}
    \caption{A graph, $G$}
  \end{figure}
\end{minipage}\hfill
\begin{minipage}[c]{0.45\linewidth}
  \begin{figure}[H]
    \centering
    \usetikzlibrary{graphs}
    \begin{tikzpicture}
      \tikzset{
        node distance=1.5cm,
      }

      \node[vertex] (ve) {$e$};
      \node[vertex, left of=ve, above of=ve] (vb) {$b$};
      \node[vertex, right of=ve, above of=ve] (vc) {$c$};
      \node[vertex, left of=ve, below of=ve] (vf) {$f$};
      \node[vertex, left of=vb, above of=vb, xshift=-1cm] (va) {$a$};
      \node[vertex, left of=vb, below of=vb, xshift=-1cm] (vd) {$d$};

      \draw (va) edge[] (vb)
            (va) edge[] (vd)
            (vb) edge[] (vc)
            (vd) edge[] (vf)
            (ve) edge[] (vb);
    \end{tikzpicture}
    \caption{A tree, $T$ of graph $G$}
  \end{figure}
\end{minipage}

\section*{Spanning Trees}

A spanning tree $H$, of the graph $G$, is a subgraph of $G$ which contains every vertex in $G$, and is a tree. Any given
 graph can have multiple spanning trees, each of which is considered different if they are made up of different edges
 of $G$.

\subsection*{Finding a Spanning Tree}

It is very easy to find a spanning tree of any given graph, $G$, by using the following method:
\begin{enumerate}[start=0]
  \item Check if $G$ contains any cycles. If not, $G$ is itself a spanning tree of itself
  \item\label{item:spt1} Delete any single edge in $G$ that forms part of a cycle
  \item Repeat step \ref*{item:spt1} until there are no edges which form part of a cycle. This means that you have found
   a connected subgraph of $G$ with no cycles, and therefore a spanning tree of $G$
\end{enumerate}

Another method is to use a depth-first search, which is useful if you need to implement this in code. The method is as
 follows:
\begin{enumerate}[start=0]
  \item Start at any vertex, and give it a label
  \item\label{item:dfs1} Move to any unlabelled adjacent vertex
  \item\label{item:dfs2} Label the vertex
  \item Repeat steps \ref*{item:dfs1}-\ref*{item:dfs2} until there are no adjacent unlabelled vertices
  \item Backtrack to the first vertex with any adjacent unlabelled vertices, then go back to step \ref*{item:dfs1}. If
   you reach the starting node without finding any adjacent unlabelled vertices, you have found a spanning tree
\end{enumerate}
If you also keep track of the edges you used to traverse the graph, you have a list of vertices and edges which make up
 a spanning tree of the graph.

\begin{figure}[H]
  \centering
  \usetikzlibrary{graphs}
  \begin{tikzpicture}
    \tikzset{
      node distance=1.5cm,
    }

    \node[vertex] (ve) {$e$};
    \node[vertex, left of=ve, above of=ve] (vb) {$b$};
    \node[vertex, right of=ve, above of=ve] (vc) {$c$};
    \node[vertex, left of=ve, below of=ve] (vf) {$f$};
    \node[vertex, right of=ve, below of=ve] (vg) {$g$};
    \node[vertex, left of=vb, above of=vb, xshift=-1cm] (va) {$a$};
    \node[vertex, left of=vb, below of=vb, xshift=-1cm] (vd) {$d$};

    \draw[line width=0.15cm, red!50]
          (va) edge[] (vb)
          (vb) edge[] (vd)
          (vb) edge[] (vc)
          (vb) edge[] (ve)
          (vd) edge[] (vf)
          (ve) edge[] (vg);

    \draw (va) edge[] (vb)
          (va) edge[] (vd)
          (vb) edge[] (vd)
          (vb) edge[] (vc)
          (vd) edge[] (vf)
          (ve) edge[] (vb)
          (ve) edge[] (vc)
          (ve) edge[] (vd)
          (ve) edge[] (vf)
          (ve) edge[] (vg)
          (vf) edge[] (vg);

  \end{tikzpicture}
  \caption{$G$, with a spanning tree, $T$, overlaid in red}
\end{figure}

\section*{Minimum Spanning Trees}

If you now consider a weighted graph, it is obvious that some spanning trees have a lower total weight for all of their
 edges. The tree with the lowest of these is known as the minimum spanning tree. This is useful for optimising problems
 such as finding the most efficient path between locations, or the lowest cost way to connect users to a telephone
 system. While it is possible to use the previous methods to create a spanning tree in a weighted graph, it almost
 certainly will not be the minimum spanning tree.

For this purpose, there are two known efficient algorithms for finding (one of) the minimum spanning tree(s), namely
 Kruskal's Algorithm and Prim's Algorithm.

\subsection*{Kruskal's Algorithm}

For a graph with $n$ vertices,
\begin{enumerate}
  \item `Fill in' the edge with the lowest weight
  \item\label{item:kru1} Add the next lowest weight edge, if adding this won't create a circuit
  \item Repeat step \ref*{item:kru1} until you have $n - 1$ edges
\end{enumerate}
This will result in (one of) the minimum spanning tree(s)

Kruskal's algorithm is known as a `Greedy Algorithm' since it makes decisions based upon what is best at that step, with
 no regard for the future implications.

\subsection*{Prim's Algorithm}

To construct a tree, $T$ of graph $G$,
\begin{enumerate}[start=0]
  \item Start with any vertex
  \item\label{item:pri1} Add the edge of lowest weight which is adjacent to $T$ and connects vertices in $T$ with those
   not in $T$
  \item Repeat step \ref*{item:pri1} until all vertices are in $T$
\end{enumerate}

\begin{minipage}[t][][t]{0.45\linewidth}
  \begin{figure}[H]
    \centering
    \usetikzlibrary{graphs}
    \begin{tikzpicture}
      \tikzset{
        node distance=1.5cm,
      }
  
      \node[vertex] (ve) {$e$};
      \node[vertex, left of=ve, above of=ve] (vb) {$b$};
      \node[vertex, right of=ve, above of=ve] (vc) {$c$};
      \node[vertex, left of=ve, below of=ve] (vf) {$f$};
      \node[vertex, right of=ve, below of=ve] (vg) {$g$};
      \node[vertex, left of=vb, above of=vb, xshift=-1cm] (va) {$a$};
      \node[vertex, left of=vb, below of=vb, xshift=-1cm] (vd) {$d$};
  
      \draw (va) edge[] node[above] {$7$} (vb)
            (va) edge[] node[left] {$5$} (vd)
            (vb) edge[] node[below] {$9$} (vd)
            (vb) edge[] node[below] {$8$} (vc)
            (vd) edge[] node[below] {$6$} (vf)
            (ve) edge[] node[right] {$7$} (vb)
            (ve) edge[] node[right] {$5$} (vc)
            (ve) edge[] node[below] {$7$} (vd)
            (ve) edge[] node[below] {$8$} (vf)
            (ve) edge[] node[right] {$9$} (vg)
            (vf) edge[] node[below] {$11$} (vg);
  
    \end{tikzpicture}
    \caption{The weighted graph, $G_2$\\\ }
  \end{figure}
\end{minipage}\hfill
\begin{minipage}[t][][t]{0.45\linewidth}
  \begin{figure}[H]
    \centering
    \usetikzlibrary{graphs}
    \begin{tikzpicture}
      \tikzset{
        node distance=1.5cm,
      }
  
      \node[vertex] (ve) {$e$};
      \node[vertex, left of=ve, above of=ve] (vb) {$b$};
      \node[vertex, right of=ve, above of=ve] (vc) {$c$};
      \node[vertex, left of=ve, below of=ve] (vf) {$f$};
      \node[vertex, right of=ve, below of=ve] (vg) {$g$};
      \node[vertex, left of=vb, above of=vb, xshift=-1cm] (va) {$a$};
      \node[vertex, left of=vb, below of=vb, xshift=-1cm] (vd) {$d$};
  
      \draw[line width=0.15cm, red!50]
            (va) edge[] (vd)
            (vc) edge[] (ve)
            (va) edge[] (vb)
            (vb) edge[] (ve)
            (vd) edge[] (vf)
            (ve) edge[] (vg);
  
      \draw (va) edge[] node[above] {$7$} (vb)
            (va) edge[] node[left] {$5$} (vd)
            (vb) edge[] node[below] {$9$} (vd)
            (vb) edge[] node[below] {$8$} (vc)
            (vd) edge[] node[below] {$6$} (vf)
            (ve) edge[] node[right] {$7$} (vb)
            (ve) edge[] node[right] {$5$} (vc)
            (ve) edge[] node[below] {$7$} (vd)
            (ve) edge[] node[below] {$8$} (vf)
            (ve) edge[] node[right] {$9$} (vg)
            (vf) edge[] node[below] {$11$} (vg);
  
    \end{tikzpicture}
    \caption{$G_2$, with the minimum spanning tree generated by Kruskal's Algorithm, $T_2$, total weight 39}
  \end{figure}
\end{minipage}

\section*{Rooted Trees}

A tree is a rooted tree if one of it's vertices is specified as the \textit{root}. Each vertex in a tree has zero or more 
 \textit{children} (adjacent vertices moving away from the root), and each vertex (other than the root) has a
 \textit{parent}, which is the vertex above it, in relation to the root. If two vertices have the same parent, they are
 known as \textit{siblings}. In the case of $T_2$, we could say that $a$ is the root, $c$ and $g$ share the parent, $e$,
 and as such are siblings.