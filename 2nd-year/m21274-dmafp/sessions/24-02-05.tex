\lecture{Pattern Matching and Recursion}{12:00}{05/02/24}{Matthew Poole}

\section*{Importing Libraries}

As with any other language, you can import files in Haskell. There are some standard libraries included in GHC, such as
 the \verb`Data.Char` library, which includes functions for manipulating strings, such as \verb`toUpper` and
 \verb`toLower`. (Astounding features, I know). To import an entire module, you use \verb`import Data.Char`, but to
 import only specific functions you can use \verb`import Data.Char (toUpper, toLower)`. There is also a ``standard
 prelude'' which is imported automatically by the interpreter. It includes the definitions of standard functions, such
 as \verb`mod` as well as commonly used types.

Haskell includes functions, which are used as with prefix notation, e.g. \verb`mod n 2` and operators which are used
 with infix notation, e.g. \verb`2 - 1`. There is an operator which uses prefix notation, the unary minus which is used
 to represent a negative number. You can use any binary function (one with two arguments) as an infix operator, by
 surrounding it with backticks, e.g. \verb`mod n 2` could also be written as \verb+n `mod` 2+. You can also use an
 operator as a function by surrounding it with brackets, e.g. \verb`1 + x` could also be written as \verb`(+) 1 x`.

\section*{Pattern Matching}

There are two ways of defining functions - using single equations and using guards - which have already been covered,
 but there is another way, which is using pattern matching. Patterns work in a similar way to guards, and one example
 is the function \verb`not` which is defined in the prelude as 
\begin{verbatim}
not :: Bool -> Bool
not True    = False
not False   = True
\end{verbatim}
This definition is a sequence of equations. For each pattern (on the left) there is a result (on the right). When the
 function is called, the input is checked against each pattern, and if it matches, that pattern's output is returned.

You can also use the wildcard pattern \verb`_`, which matches any value. This is often useful for simplifying complex
 patterns. For example, if we wanted to redefine the boolean or operator, \verb`||`, we could define it as
\begin{verbatim}
(||) :: Bool -> Bool -> Bool
True || True    = True
True || False   = True
False || True   = True
False || False  = False
\end{verbatim}
but this is very complex, and defines 3 redundant patterns. We could instead use the wildcard, and define it as
\begin{verbatim}
(||) :: Bool -> Bool -> Bool
False || False  = False
_ || _          = True
\end{verbatim}
I would argue this is not necessarily more readable, but it is more compact and technically more efficient

\section*{Recursion}

For and while loops are very much imperative constructs, as they operate on the state of the program. This means that
 they cannot exist in pure functional programming. Therefore recursion is a fundamental concept in the functional
 paradigm. Recursion is used heavily throughout functional programming, but especially when a list or other iterable
 data type is involved.

One common example of iteration is to calculate the factorial of a number. Since the factorial of a number, $n$ is
 defined as $n \times (n - 1) \times (n - 2) \times \dots \times 2 \times 1$ and by convention, $n! = 1$, we can define
 the factorial of $n$ where $n > 0$ in terms of the factorial of $n - 1$. e.g. $3! = 3 \times 2!$. This gives us a very
 simple recursive algorithm, which could be defined as follows in Haskell
\begin{verbatim}
fact :: Int -> Int
fact n
  | n > 0   = n * fact (n - 1)
  | n == 0  = 1
\end{verbatim}
This definition, despite being correct, will fail for negative integers as there is no guard for that case. To fix this,
 you could add the following \verb`otherwise` guard to give an error message
\begin{verbatim}
  | otherwise = error "Undefined for negative integers"
\end{verbatim}

\section*{General Recursion}

The previous example of a recursive function was, in fact, a primitive recursive function, e.g. the base case considers
 the value of 0 and the recursive case considers how to get from $n - 1$ to $n$. Another example of a primitive
 recursive function is to perform multiplication with addition, e.g.
\begin{verbatim}
mult :: Int -> Int -> Int
mult n m
  | n == 0  = 0
  | n > 0   = m + mult (n - 1) m
\end{verbatim}
Since this function also has a base case of \verb`n == 0`, it is primitive. A general recursive function is one in which
 the base case is not checking for a value of 0. For example, if we were to implement integer division using
 subtraction, the base case would be where the divisor is greater than the dividend. e.g.
\begin{verbatim}
divide :: Int -> Int -> Int
divide n m
  | n < m       = 0
  | otherwise   = 1 + divide (n - m) m
\end{verbatim}