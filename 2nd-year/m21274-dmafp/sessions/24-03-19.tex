\lecture{Walks, Trails and Paths}{17:00}{19/03/24}{Janka Chlebikova}

\section*{Walks}

A walk is a sequence of vertices and edges, starting and ending with a vertex. There are several notations which can be
 used, a less formal one, such as $a$-$ab$-$b$-$bc$-$c$-$ca$-$a$, and the more formal one which I will stick to, being
 $(a, b, c, a)$. The length of a walk is the number of edges traversed. A walk can traverse any given edge and number of
 times, anywhere from $0$ to $\infty$. A walk can either be open or closed, depending upon if the walk ends at a
 different vertex or at the starting vertex, respectively.

\begin{minipage}[c]{0.45\linewidth}
  \begin{figure}[H]
    \centering
    \usetikzlibrary{graphs}
    \begin{tikzpicture}
      \tikzset{
        node distance=1.5cm,
      }
  
      \node[vertex] (ve) {$e$};
      \node[vertex, left of=ve, above of=ve] (vb) {$b$};
      \node[vertex, right of=ve, above of=ve] (vc) {$c$};
      \node[vertex, left of=ve, below of=ve] (vf) {$f$};
      \node[vertex, right of=ve, below of=ve] (vg) {$g$};
      \node[vertex, left of=vb, above of=vb, xshift=-2cm] (va) {$a$};
      \node[vertex, left of=vb, below of=vb, xshift=-2cm] (vd) {$d$};

      \draw[line width=0.15cm, red!50]
            (va) edge[] node[left] {$1$} (vd)
            (va) edge[] node[above] {$2$} (vb)
            (ve) edge[] node[right] {$3$} (vb)
            (ve) edge[] node[above] {$4, 8$} (vd)
            (vb) edge[] node[above] {$5$} (vd)
            (vb) edge[] node[above] {$6$} (vc)
            (ve) edge[] node[right] {$7$} (vc)
            (vd) edge[] node[below] {$9$} (vf);

      \draw (va) edge[] (vb)
            (va) edge[] (vd)
            (vb) edge[] (vd)
            (vb) edge[] (vc)
            (vd) edge[] (vf)
            (ve) edge[] (vb)
            (ve) edge[] (vc)
            (ve) edge[] (vd)
            (ve) edge[] (vf)
            (ve) edge[] (vg)
            (vf) edge[] (vg);
    \end{tikzpicture}
    \caption{An open walk of length 9 from $d$ to $f$}
    \label{fig:walk1}
  \end{figure}
\end{minipage}\hfill
\begin{minipage}{0.45\linewidth}
The walk shown in red in Figure \ref*{fig:walk1} is an open walk, and could be written either informally as
$d$-$da$-$a$-$ab$-$b$-$be$-$e$-$ed$-$d$-$db$-$b$-$bc$-$c$-$ce$-$e$-$ed$-$d$-$df$-$f$, or formally as
$(d, a, b, e, d, b, c, e, d, f)$.
\end{minipage}

\subsection*{Trails}

A trail is a type of walk in which each edge is traversed at most once, and the edges are therefore distinct.

\begin{figure}[H]
  \centering
  \usetikzlibrary{graphs}
  \begin{tikzpicture}
    \tikzset{
      node distance=1.5cm,
    }

    \node[vertex] (ve) {$e$};
    \node[vertex, left of=ve, above of=ve] (vb) {$b$};
    \node[vertex, right of=ve, above of=ve] (vc) {$c$};
    \node[vertex, left of=ve, below of=ve] (vf) {$f$};
    \node[vertex, right of=ve, below of=ve] (vg) {$g$};
    \node[vertex, left of=vb, above of=vb, xshift=-2cm] (va) {$a$};
    \node[vertex, left of=vb, below of=vb, xshift=-2cm] (vd) {$d$};

    \draw[line width=0.15cm, red!50]
          (vd) edge[] node[left] {$1$} (va)
          (va) edge[] node[above] {$2$} (vb)
          (vb) edge[] node[right] {$3$} (ve)
          (ve) edge[] node[above] {$4$} (vd)
          (vd) edge[] node[below] {$5$} (vf);

    \draw (va) edge[] (vb)
          (va) edge[] (vd)
          (vb) edge[] (vd)
          (vb) edge[] (vc)
          (vd) edge[] (vf)
          (ve) edge[] (vb)
          (ve) edge[] (vc)
          (ve) edge[] (vd)
          (ve) edge[] (vf)
          (ve) edge[] (vg)
          (vf) edge[] (vg);
  \end{tikzpicture}
  \caption{A trail of length 5 from $d$ to $f$}
  \label{fig:trail1}
\end{figure}

\subsection*{Paths}

A path is a type of walk in which each vertex is traversed at most once, and the vertices are therefore distinct.

\begin{figure}[H]
  \centering
  \usetikzlibrary{graphs}
  \begin{tikzpicture}
    \tikzset{
      node distance=1.5cm,
    }

    \node[vertex] (ve) {$e$};
    \node[vertex, left of=ve, above of=ve] (vb) {$b$};
    \node[vertex, right of=ve, above of=ve] (vc) {$c$};
    \node[vertex, left of=ve, below of=ve] (vf) {$f$};
    \node[vertex, right of=ve, below of=ve] (vg) {$g$};
    \node[vertex, left of=vb, above of=vb, xshift=-2cm] (va) {$a$};
    \node[vertex, left of=vb, below of=vb, xshift=-2cm] (vd) {$d$};

    \draw[line width=0.15cm, red!50]
          (vd) edge[] node[left] {$1$} (va)
          (va) edge[] node[above] {$2$} (vb)
          (vb) edge[] node[right] {$3$} (ve)
          (ve) edge[] node[left] {$4$} (vf)
          (vf) edge[] node[below] {$5$} (vg);

    \draw (va) edge[] (vb)
          (va) edge[] (vd)
          (vb) edge[] (vd)
          (vb) edge[] (vc)
          (vd) edge[] (vf)
          (ve) edge[] (vb)
          (ve) edge[] (vc)
          (ve) edge[] (vd)
          (ve) edge[] (vf)
          (ve) edge[] (vg)
          (vf) edge[] (vg);
  \end{tikzpicture}
  \caption{A path of length 5 from $d$ to $g$}
  \label{fig:path1}
\end{figure}

\subsection*{Circuits}

A circuit is a closed walk which is also a trail (All edges are distinct).

\begin{figure}[H]
  \centering
  \usetikzlibrary{graphs}
  \begin{tikzpicture}
    \tikzset{
      node distance=1.5cm,
    }

    \node[vertex] (ve) {$e$};
    \node[vertex, left of=ve, above of=ve] (vb) {$b$};
    \node[vertex, right of=ve, above of=ve] (vc) {$c$};
    \node[vertex, left of=ve, below of=ve] (vf) {$f$};
    \node[vertex, right of=ve, below of=ve] (vg) {$g$};
    \node[vertex, left of=vb, above of=vb, xshift=-2cm] (va) {$a$};
    \node[vertex, left of=vb, below of=vb, xshift=-2cm] (vd) {$d$};

    \draw[line width=0.15cm, red!50]
          (vd) edge[] node[left] {$1$} (va)
          (va) edge[] node[above] {$2$} (vb)
          (vb) edge[] node[right] {$3, 4$} (ve)
          (vb) edge[] node[above] {$5$} (vd);

    \draw (va) edge[] (vb)
          (va) edge[] (vd)
          (vb) edge[] (vd)
          (vb) edge[] (vc)
          (vd) edge[] (vf)
          (ve) edge[] (vb)
          (ve) edge[] (vc)
          (ve) edge[] (vd)
          (ve) edge[] (vf)
          (ve) edge[] (vg)
          (vf) edge[] (vg);
  \end{tikzpicture}
  \caption{A circuit of length 5 from $d$ to $d$}
  \label{fig:circuit1}
\end{figure}

\subsection*{Cycles}

A cycle is a closed walk which is also a path (All vertices are distinct)

\begin{figure}[H]
  \centering
  \usetikzlibrary{graphs}
  \begin{tikzpicture}
    \tikzset{
      node distance=1.5cm,
    }

    \node[vertex] (ve) {$e$};
    \node[vertex, left of=ve, above of=ve] (vb) {$b$};
    \node[vertex, right of=ve, above of=ve] (vc) {$c$};
    \node[vertex, left of=ve, below of=ve] (vf) {$f$};
    \node[vertex, right of=ve, below of=ve] (vg) {$g$};
    \node[vertex, left of=vb, above of=vb, xshift=-2cm] (va) {$a$};
    \node[vertex, left of=vb, below of=vb, xshift=-2cm] (vd) {$d$};

    \draw[line width=0.15cm, red!50]
          (vd) edge[] node[left] {$1$} (va)
          (va) edge[] node[above] {$2$} (vb)
          (vb) edge[] node[right] {$3$} (ve)
          (ve) edge[] node[left] {$4$} (vf)
          (vf) edge[] node[below] {$5$} (vd);

    \draw (va) edge[] (vb)
          (va) edge[] (vd)
          (vb) edge[] (vd)
          (vb) edge[] (vc)
          (vd) edge[] (vf)
          (ve) edge[] (vb)
          (ve) edge[] (vc)
          (ve) edge[] (vd)
          (ve) edge[] (vf)
          (ve) edge[] (vg)
          (vf) edge[] (vg);
  \end{tikzpicture}
  \caption{A cycle of length 5 from $d$ to $d$}
  \label{fig:cycle1}
\end{figure}

\subsection*{Connected Graphs \& Bridges}

A graph is connected if there is a path between any given pair of vertices, otherwise it is disconnected.

\begin{minipage}[c]{0.45\linewidth}
  \begin{figure}[H]
    \centering
    \usetikzlibrary{graphs}
    \begin{tikzpicture}
      \tikzset{
        node distance=2cm,
      }
  
      \node[vertex] (v1) {$v_1$};
      \node[vertex, right of=v1] (v2) {$v_2$};
      \node[vertex, right of=v2] (v3) {$v_3$};
      \node[vertex, below of=v3] (v4) {$v_4$};
      \node[vertex, below of=v2] (v5) {$v_5$};
      \node[vertex, below of=v1] (v6) {$v_6$};

      \draw (v1) edge[] (v2)
            (v1) edge[] (v6)
            (v2) edge[] (v3)
            (v2) edge[] (v5)
            (v3) edge[] (v4)
            (v4) edge[] (v5)
            (v5) edge[] (v6);
    \end{tikzpicture}
    \caption{A connected graph with 6 vertices}
    \label{fig:congraph1}
  \end{figure}
\end{minipage}\hfill
\begin{minipage}[c]{0.45\linewidth}
  \begin{figure}[H]
    \centering
    \usetikzlibrary{graphs}
    \begin{tikzpicture}
      \tikzset{
        node distance=2cm,
      }
  
      \node[vertex] (v1) {$v_1$};
      \node[vertex, right of=v1] (v2) {$v_2$};
      \node[vertex, right of=v2] (v3) {$v_3$};
      \node[vertex, below of=v3] (v4) {$v_4$};
      \node[vertex, below of=v2] (v5) {$v_5$};
      \node[vertex, below of=v1] (v6) {$v_6$};

      \draw (v1) edge[] (v2)
            (v1) edge[] (v6)
            (v2) edge[] (v5)
            (v3) edge[] (v4)
            (v5) edge[] (v6);
    \end{tikzpicture}
    \caption{A disconnected graph with 6 vertices}
    \label{fig:discongraph1}
  \end{figure}
\end{minipage}

A bridge is any edge within a connected graph which, if removed, would case the graph to become disconnected.

\begin{minipage}[c]{0.45\linewidth}
  \begin{figure}[H]
    \centering
    \usetikzlibrary{graphs}
    \begin{tikzpicture}
      \tikzset{
        node distance=2cm,
      }
  
      \node[vertex] (v1) {$v_1$};
      \node[vertex, right of=v1] (v2) {$v_2$};
      \node[vertex, right of=v2] (v3) {$v_3$};
      \node[vertex, below of=v3] (v4) {$v_4$};
      \node[vertex, below of=v2] (v5) {$v_5$};
      \node[vertex, below of=v1] (v6) {$v_6$};

      \draw (v1) edge[] (v2)
            (v1) edge[] (v6)
            (v2) edge[] (v5)
            (v2) edge[line width=0.15cm, red!50] (v3)
            (v2) edge[] (v3)
            (v3) edge[] (v4)
            (v5) edge[] (v6);
    \end{tikzpicture}
    \caption{A connected graph with 6 vertices and 2 bridges}
    \label{fig:bridgegraph1}
  \end{figure}
\end{minipage}\hfill
\begin{minipage}[c]{0.45\linewidth}
The graph shown in Figure \ref*{fig:bridgegraph1} is a connected graph with a bridge. The edge $v_2v_3$ (highlighted in
 red) is a bridge as if it were removed there would be no path between the vertices $v_1, v_2, v_5$ and $v_6$ and the
 vertices $v_3$ and $v_4$. The edge between $v_3$ and $v_4$ is also an edge, as if it were removed there would be no
 path to $v_4$.
\end{minipage}

A graph can have multiple bridges, as there could be several sections which are only connected by a single edge, or as
 in the example in Figure \ref*{fig:bridgegraph1}, there could be a single node which is only connected to the rest of
 the graph by a single edge.

\section*{Eulerian Graphs}

A Eulerian graph is one in which there is a circuit which contains every edge. This would be a closed walk where every
 edge is traversed exactly once. An easy way to tell if a graph is Eulerian, is if you are able to draw it without
 lifting your pencil off the paper. It also means that the degree of every vertex must be even. In this case, it is
 necessary and sufficient, so all you need to prove that a graph is Eulerian is to show that the degrees of all vertices
 is even.

\subsection*{Fleury's Algorithm}

There is an efficient algorithm known as Fleury's Algorithm, which allows you to find an Eulerian circuit within an
 Eulerian graph. It is as follows:
\begin{enumerate}[start=0]
  \item Select a starting vertex
  \item\label{item:fle1} Select an edge to traverse, only select a bridge if there are no other edges
  \item\label{item:fle2} Remove the traversed edge, and any vertices with a degree of 0
  \item Repeat steps \ref*{item:fle1}-\ref*{item:fle2} until all edges have been traversed, and you should end at the starting vertex
\end{enumerate}

Since it does not matter which vertex you start from, or the order in which you traverse edges, there can be several
 Eulerian circuits in each Eulerian graph.

\subsection*{Semi-Eulerian Graphs}

A connected graph with exactly two vertices of an odd degree is known as a semi-Eulerian graph and contains at least one
 open Eulerian trail, which includes every edge. These trails necessarily start and end at each of the vertices with an
 odd degree.

\section*{Hamiltonian Graphs}

A graph is Hamiltonian if it has a cycle which contains every vertex exactly once, and is a closed path. Unlike Eulerian
 graphs, there is no known generalisation to determine if a graph is Hamiltonian or not. There are several algorithms
 which are able to construct a Hamiltonian cycle on a Hamiltonian graph, but unfortunately none of them run in a reasonable
 length of time, and most run in exponential time, with some even running in factorial time.

\section*{Adjacency Matrices}

Another method for representing a graph is with an \textit{Adjacency Matrix}. For a graph $G$ with $n$ vertices, $v_1, v_2, \dots, v_n$,
 the adjacency matrix of $G$ is the matrix $A = n \times n$, whose $(i,j)$ entry is $a_{i j}$ where $1 \leq i \leq n$. For
 each $a_{i j}$ in $A$,
 $a_{i j} = \begin{cases}0 & \mathrm{\ if\ } v_iv_j \mathrm{\ is\ not\ an\ edge}\\ 1 & \mathrm{\ if\ } v_iv_j \mathrm{\ is\ an\ edge}\end{cases}$

\begin{minipage}[c]{0.35\linewidth}
  \begin{figure}[H]
    \centering
    \usetikzlibrary{graphs}
    \begin{tikzpicture}
      \tikzset{
        node distance=2cm,
      }
  
      \node[vertex] (v1) {$v_1$};
      \node[vertex, right of=v1] (v2) {$v_2$};
      \node[vertex, below of=v1] (v3) {$v_3$};
      \node[vertex, below of=v2] (v4) {$v_4$};

      \draw (v1) edge[] (v2)
            (v1) edge[] (v3)
            (v2) edge[] (v4)
            (v3) edge[] (v4);
    \end{tikzpicture}
    \caption{The graph, $G$, with 4 vertices and edges.}
    \label{fig:adjgraph1}
  \end{figure}
\end{minipage}\hfill
\begin{minipage}[c]{0.55\linewidth}
For the graph $G$, as in Figure \ref*{fig:adjgraph1}, the adjacency matrix would be $A = \begin{bmatrix}
0 & 1 & 1 & 0\\
1 & 0 & 0 & 1\\
1 & 0 & 0 & 1\\
0 & 1 & 1 & 0
\end{bmatrix}$
\end{minipage}

The diagonal entries for a graph, e.g. $a_{i i}$, are always $0$ for a proper graph, since a vertex is never connected
 to itself, which may not be the case for a pesudograph. The adjacency matrix is also always symmetrical along this line
 for an undirected graph, since the edge $v_iv_j$ is the same as the edge $v_jv_i$. The degree of $v_i$ is the number of
 $1$'s in column $i$ of $A$.

If you take the square of $A$, $A^2$, the entry $(i, i)$ is the degree of $v_i$. $(i, j)$ is the number of different
 walks of length $2$ between $v_i$ and $v_j$. This is the case for any power of $A$, and so it can be generalised as
 for any $k \geq 1$, $(i, j)$ of $A^k$ is the number of walks of length $k$ between $v_i$ and $v_j$.