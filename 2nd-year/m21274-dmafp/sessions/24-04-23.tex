\lecture{Networks and Digraphs}{17:00}{23/04/24}{Janka Chlebikova}

A directed graph (digraph) is a subtype of graph in which each edge has a specified direction. In a digraph, edges are
 known as arcs and must have a direction, and optionally a weight. Whereas an edge in an undirected graph can be
 represented in one of two ways (either $v_1v_2$ or $v_2v_1$), each arc in a digraph can only be represented by one pair
 since it represents the direction of the graph as well. As a regular graph, a digraph is represented by two sets, $V$
 and $E$. An example of a digraph is below.

\begin{minipage}[c]{0.45\linewidth}
  \begin{figure}[H]
    \centering
    \usetikzlibrary{graphs}
    \begin{tikzpicture}
      \tikzset{
        node distance=1.5cm,
      }

      \node[vertex] (v1) {$v_1$};
      \node[vertex, right of=v1] (v2) {$v_2$};
      \node[vertex, below of=v1] (v3) {$v_3$};

      \path[->] (v1) edge[] (v2)
            (v1) edge[] (v3)
            (v2) edge[] (v3);
    \end{tikzpicture}
    \caption{A digraph, $G_d$}
    \label{fig:digraph1}
  \end{figure}
\end{minipage}\hfill
\begin{minipage}[c]{0.45\linewidth}
  For the digraph $G_d$, as in Figure \ref*{fig:digraph1},\\
  $V = \{v_1, v_2, v_3\}$, and\\
  $E = \{v_1v_2, v_1v_3, v_2v_3\}$.
\end{minipage}

In this module, we will continue to assume that digraphs with loops are impossible.

Each vertex in a digraph has an \textit{indegree} and \textit{outdegree}, being the number of arcs directed into and out
 of the vertex, respectively. The adjacency matrix for a digraph is similar to that of an undirected graph, but is not
 necessarily symmetrical. This means that an arc exists between $v_1$ and $v_2$ if and only if the value in the adjacency
 matrix is one at the position $a_{1,2}$. This also means that there may not be an arc connecting $v_2$ and $v_1$. The
 outdegree of a vertex is represented by the sum of the values in the row $i$ and the indegree is the sum of values in
 the column $i$. $(i, j)$ of $A^k$ is the number of walks of length $k$ from $i$ to $j$, respecting the direction of the
 arcs.

\section*{Networks}

Digraphs are useful for representing \textit{networks}, which can be a model for any sort of physical network. This
 includes transport networks, gas networks, computer networks, etc. In each case, the objective is to find a
 \textit{maximum flow}.

The arcs of a digraph can represent anything, but in this case, they will represent the pipes in an oil pipeline, and
 the direction of each arc shows the direction which oil can flow.

\begin{minipage}[c]{0.45\linewidth}
  \begin{figure}[H]
    \centering
    \usetikzlibrary{graphs}
    \begin{tikzpicture}
      \tikzset{
        node distance=2cm,
      }

      \node[vertex] (vs) {$S$};
      \node[vertex, above of=vs, right of=vs] (va) {$A$};
      \node[vertex, below of=vs, right of=vs] (vb) {$B$};
      \node[vertex, right of=va] (vc) {$C$};
      \node[vertex, right of=vb] (vd) {$D$};
      \node[vertex, right of=vd, above of=vd] (vt) {$T$};

      \path[->] (vs) edge[] node[left] {$10$} (va)
                (vs) edge[] node[left] {$11$} (vb)
                (va) edge[] node[above] {$12$} (vc)
                (va) edge[] node[left, pos=0.3] {$4$} (vd)
                (vb) edge[] node[right, pos=0.7] {$8$} (vc)
                (vb) edge[] node[below] {$2$} (vd)
                (vc) edge[] node[right] {$11$} (vt)
                (vd) edge[] node[right] {$14$} (vt);
    \end{tikzpicture}
    \caption{The oil pipeline}
    \label{fig:network1}
  \end{figure}
\end{minipage}\hfill
\begin{minipage}[c]{0.45\linewidth}
  Oil is unloaded at $S$, the dock, and is pumped through the network to storage tanks $T$. $S$ is known as the \textit{source}
   and $T$ the \textit{sink}. The weight on each arc is its \textit{capacity}.
\end{minipage}

\subsection*{Flows}

A flow in a network is the amount of `commodity' which flows through the network in a unit time. The flow enters the
 network at the source, and flows towards the sink. No arc can receive more flow than its capacity, and no commodity can
 be lost along the way, so the flow into each vertex is the same as the flow out. Each arc, $e$ is assigned a flow $f(e)$,
 and for each arc, $f(e) \leq c(e)$.

Formally, a source is any vertex with no incoming arcs, and a sink is any vertex with no outgoing arcs. And a flow
 assigns each arc, $e$ a number which satisfies the following conditions
\begin{itemize}
  \item Feasibility condition -- $0 \leq f(e) \leq c(e)$ for all $e \in E$
  \item Conservation of flow -- $\sum_{v \in V}^{} f(uv) = \sum_{v \in V}^{} f(vu) \mathrm{\ for\ all\ } u \in V - \{S, T\}$
\end{itemize}

In any given network, the maximum flow is any flow which has the maximum value ($\sum_{v \in V} f(Sv)$, the sum of all
 outgoing arcs from $S$).

\begin{figure}[H]
  \centering
  \usetikzlibrary{graphs}
  \begin{tikzpicture}
    \tikzset{
      node distance=2cm,
    }

    \node[vertex] (vs) {$S$};
    \node[vertex, above of=vs, right of=vs] (va) {$A$};
    \node[vertex, below of=vs, right of=vs] (vb) {$B$};
    \node[vertex, right of=va] (vc) {$C$};
    \node[vertex, right of=vb] (vd) {$D$};
    \node[vertex, right of=vd, above of=vd] (vt) {$T$};

    \draw[line width=0.15cm, red!50]
              (vs) edge[] (va)
              (va) edge[] (vd)
              (vd) edge[] (vt);

    \path[->] (vs) edge[] node[left] {$2, 10$} (va)
              (vs) edge[] node[left] {$0, 11$} (vb)
              (va) edge[] node[above] {$0, 12$} (vc)
              (va) edge[] node[left, pos=0.3] {$2, 4$} (vd)
              (vb) edge[] node[right, pos=0.7] {$0, 8$} (vc)
              (vb) edge[] node[below] {$0, 2$} (vd)
              (vc) edge[] node[right] {$0, 11$} (vt)
              (vd) edge[] node[right] {$2, 14$} (vt);
  \end{tikzpicture}
  \caption{A flow of value 2 through the pipeline}
  \label{fig:flow1}
\end{figure}

\subsection*{Finding a Maximum Flow}

There is a simple method to find a maximum flow--
\begin{enumerate}
  \item Locate a path, $P$ from the source $S$ to the sink $T$
  \item Define the flow for each vertex as $f(e) = \begin{cases}0 & \mathrm{\ if\ } e \in P\\ 1 & \mathrm{\ if\ } e \notin P\end{cases}$
  \item Increment the flow of each arc by one until the capacity of any edge is reached
  \item Find another path from $S$ to $T$ and attempt to increment the edges again
\end{enumerate}

This method will result in \textit{a} maximum flow, but not necessarily \textit{the} maximum flow, so there are possible
 improvements to be made.

\begin{figure}[H]
  \centering
  \usetikzlibrary{graphs}
  \begin{tikzpicture}
    \tikzset{
      node distance=2cm,
    }

    \node[vertex] (vs) {$S$};
    \node[vertex, above of=vs, right of=vs] (va) {$A$};
    \node[vertex, below of=vs, right of=vs] (vb) {$B$};
    \node[vertex, right of=va] (vc) {$C$};
    \node[vertex, right of=vb] (vd) {$D$};
    \node[vertex, right of=vd, above of=vd] (vt) {$T$};

    \draw[line width=0.15cm, red!50]
              (vs) edge[] (va)
              (vs) edge[] (vb)
              (va) edge[] (vc)
              (vb) edge[] (vc)
              (vb) edge[] (vd)
              (vc) edge[] (vt)
              (vd) edge[] (vt);

    \path[->] (vs) edge[] node[left] {$10, 10$} (va)
              (vs) edge[] node[left] {$3, 11$} (vb)
              (va) edge[] node[above] {$10, 12$} (vc)
              (va) edge[] node[left, pos=0.3] {$0, 4$} (vd)
              (vb) edge[] node[right, pos=0.7] {$1, 8$} (vc)
              (vb) edge[] node[below] {$2, 2$} (vd)
              (vc) edge[] node[right] {$11, 11$} (vt)
              (vd) edge[] node[right] {$2, 14$} (vt);
  \end{tikzpicture}
  \caption{A flow of value 13 through the pipeline, created with the above method}
  \label{fig:flow2}
\end{figure}

Consider a path, $P$ from $S$ to $T$, in the underlying undirected graph, then each edge is either
\begin{itemize}
  \item A forward arc, which follows the direction of $P$, otherwise
  \item A backward arc
\end{itemize}
Now consider a path from $S$ to $T$ in which
\begin{itemize}
  \item All forward arcs are unsaturated
  \item All backward arcs are carrying a non-zero flow
\end{itemize}

To improve this flow, we can try to use a \textit{flow augmenting path}. To do this--
\begin{enumerate}
  \item Start with any flow
  \item\label{item:fa1} Search for a flow flow augmenting path, $P$. If no path exists, this is already the maximum flow
  \item Change the flow through the path by $\Delta$, where $\Delta$ is the minimum over the values
  \begin{itemize}
    \item $\{f(e) \mid e \mathrm{\ is\ a\ backward\ arc\ in\ } P\}$, and
    \item $\{c(e) - f(e) \mid e \mathrm{\ is\ a\ forward\ arc\ in\ } P\}$
  \end{itemize}
  \item Specifically, change the flow of each arc by
  \begin{itemize}
    \item $+\Delta$ for each forward arc, $e$
    \item $-\Delta$ for each backward arc, $e$
  \end{itemize}
  \item Go back to step \ref*{item:fa1}
\end{enumerate}

\begin{minipage}[c]{0.45\linewidth}
  \begin{figure}[H]
    \centering
    \usetikzlibrary{graphs}
    \begin{tikzpicture}
      \tikzset{
        node distance=2cm,
      }
  
      \node[vertex] (vs) {$S$};
      \node[vertex, above of=vs, right of=vs] (va) {$A$};
      \node[vertex, below of=vs, right of=vs] (vb) {$B$};
      \node[vertex, right of=va] (vc) {$C$};
      \node[vertex, right of=vb] (vd) {$D$};
      \node[vertex, right of=vd, above of=vd] (vt) {$T$};
  
      \draw[line width=0.15cm, red!50]
                (vs) edge[] (va)
                (vb) edge[] (vd)
                (vc) edge[] (vt);

      \draw[line width=0.15cm, myPurple!50]
                (vs) edge[] (vb)
                (vb) edge[] (vc)
                (vc) edge[] (va)
                (va) edge[] (vd)
                (vd) edge[] (vt);
  
      \path[->] (vs) edge[] node[left] {$10, 10$} (va)
                (vs) edge[] node[left] {$3 + \Delta, 11$} (vb)
                (va) edge[] node[above] {$10 - \Delta, 12$} (vc)
                (va) edge[] node[left, pos=0.3] {$0 + \Delta, 4$} (vd)
                (vb) edge[] node[right, pos=0.7] {$1 + \Delta, 8$} (vc)
                (vb) edge[] node[below] {$2, 2$} (vd)
                (vc) edge[] node[right] {$11, 11$} (vt)
                (vd) edge[] node[right] {$2 + \Delta, 14$} (vt);
    \end{tikzpicture}
    \caption{An augmenting flow, shown with $\Delta$ where the value needs to be changed}
    \label{fig:flow3}
  \end{figure}
\end{minipage}\hfill
\begin{minipage}[c]{0.45\linewidth}
  In the network in figure \ref*{fig:flow3}, it is possible to improve the flow using a flow-augmenting path. To improve
  the flow, we must use the above method, and find the value of $\Delta$. The values which we must take a minimum over
  are $\{8, 7, 10, 4, 12\}$, so the value of $\Delta$ in this case is $4$. Below is a graph with the flows adjusted by
  $\Delta$.
\end{minipage}

\begin{figure}[H]
  \centering
  \usetikzlibrary{graphs}
  \begin{tikzpicture}
    \tikzset{
      node distance=2cm,
    }

    \node[vertex] (vs) {$S$};
    \node[vertex, above of=vs, right of=vs] (va) {$A$};
    \node[vertex, below of=vs, right of=vs] (vb) {$B$};
    \node[vertex, right of=va] (vc) {$C$};
    \node[vertex, right of=vb] (vd) {$D$};
    \node[vertex, right of=vd, above of=vd] (vt) {$T$};

    \draw[line width=0.15cm, red!50]
              (vs) edge[] (va)
              (vs) edge[] (vb)
              (va) edge[] (vc)
              (va) edge[] (vd)
              (vb) edge[] (vc)
              (vb) edge[] (vd)
              (vc) edge[] (vt)
              (vd) edge[] (vt);

    \path[->] (vs) edge[] node[left] {$10, 10$} (va)
              (vs) edge[] node[left] {$7, 11$} (vb)
              (va) edge[] node[above] {$6, 12$} (vc)
              (va) edge[] node[left, pos=0.3] {$4, 4$} (vd)
              (vb) edge[] node[right, pos=0.7] {$5, 8$} (vc)
              (vb) edge[] node[below] {$2, 2$} (vd)
              (vc) edge[] node[right] {$11, 11$} (vt)
              (vd) edge[] node[right] {$6, 14$} (vt);
  \end{tikzpicture}
  \caption{The adjusted flow, with new value of 17}
  \label{fig:flow4}
\end{figure}

To find a flow-augmenting path, you must perform an exhaustive search of all paths from $S$ to $T$. If there is more
 than one flow-augmenting path, it is irrelevant which order they are applied in, as they will always produce the same
 result.

In a directed graph, the maximum flow is equal to the value of a minimum cut.

\section*{(S, T)-cut}

An $(S, T)$-cut is a partition, $\{S_c, T_c\}$, of $V$ such that $S \in S_c$ and $T \in T_c$. The capacity of a cut,
 $\{S_c, T_c\}$, is the sum of all capacities of all arcs from $S_c$ to $T_c$. A minimum cut is the cut of the smallest
 possible capacity.

\begin{minipage}[c]{0.45\linewidth}
  \begin{figure}[H]
    \centering
    \usetikzlibrary{graphs}
    \begin{tikzpicture}
      \tikzset{
        node distance=2cm,
      }
  
      \node[vertex] (vs) {$S$};
      \node[vertex, above of=vs, right of=vs] (va) {$A$};
      \node[vertex, below of=vs, right of=vs] (vb) {$B$};
      \node[vertex, right of=va] (vc) {$C$};
      \node[vertex, right of=vb] (vd) {$D$};
      \node[vertex, right of=vd, above of=vd] (vt) {$T$};
  
      \path[->] (vs) edge[] node[left] {$10, 10$} (va)
                (vs) edge[] node[left] {$3, 11$} (vb)
                (va) edge[] node[above] {$10, 12$} (vc)
                (va) edge[] node[left, pos=0.3] {$0, 4$} (vd)
                (vb) edge[] node[right, pos=0.7] {$1, 8$} (vc)
                (vb) edge[] node[below] {$2, 2$} (vd)
                (vc) edge[] node[right] {$11, 11$} (vt)
                (vd) edge[] node[right] {$2, 14$} (vt);
    \end{tikzpicture}
    \caption{A network, $G_1$}
  \end{figure}
\end{minipage}\hfill
\begin{minipage}[c]{0.45\linewidth}
For the network $G_1$, there are many possible cuts, for example:\\
  $S_c = \{S, A, B\}$, $T_c = \{C, D, T\}$,\\
  $S_c = \{S, A\}$, $T_c = \{B, C, D, T\}$,\\
  $S_c = \{S\}$, $T_c = \{A, B, C, D, T\}$, etc.
\end{minipage}

\begin{minipage}[c]{0.45\linewidth}
  \begin{figure}[H]
    \centering
    \usetikzlibrary{graphs}
    \begin{tikzpicture}
      \tikzset{
        node distance=2cm,
      }
  
      \node[vertex, red] (vs) {$S$};
      \node[vertex, above of=vs, right of=vs, red] (va) {$A$};
      \node[vertex, below of=vs, right of=vs, myPurple] (vb) {$B$};
      \node[vertex, right of=va, red] (vc) {$C$};
      \node[vertex, right of=vb, myPurple] (vd) {$D$};
      \node[vertex, right of=vd, above of=vd, myPurple] (vt) {$T$};
  
      \draw[line width=0.15cm, red!50]
                (vs) edge[] (vb)
                (va) edge[] (vd)
                (vc) edge[] (vt);

      \draw[line width=0.15cm, myPurple!50]
                (vb) edge[] (vc);

      \path[->] (vs) edge[] node[left] {$?, 10$} (va)
                (vs) edge[] node[left] {$?, 11$} (vb)
                (va) edge[] node[above] {$?, 12$} (vc)
                (va) edge[] node[left, pos=0.3] {$?, 4$} (vd)
                (vb) edge[] node[right, pos=0.7] {$?, 8$} (vc)
                (vb) edge[] node[below] {$?, 2$} (vd)
                (vc) edge[] node[right] {$?, 11$} (vt)
                (vd) edge[] node[right] {$?, 14$} (vt);
    \end{tikzpicture}
    \caption{A network, $G_2$}
  \end{figure}
\end{minipage}\hfill
\begin{minipage}[c]{0.45\linewidth}
  In the network $G_2$, there is an $(S, T)$-cut where \textcolor{red}{$S_c = \{S, A, C\}$} and
   \textcolor{myPurple}{$T_c = \{B, D, T\}$}. The capacity of this cut can be calculated as such--
   $\mathrm{cap}(S_c, T_c) = c(SB) + c(AD) + c(CT) = 11 + 4 + 11 = 26$. The arc $BC$ is ignored, as it points from $T_c$
   to $S_c$.
\end{minipage}

In a directed graph, the value of a maximum flow is equal to the value of a minimum $(S, T)$-cut.
\begin{itemize}
  \item Value of \textit{any} flow $\leq$ capacity of \textit{any} $(S, T)$-cut
  \item Value of any \textit{maximum} flow $\leq$ capacity of \textit{any} $(S, T)$-cut
  \item Value of any \textit{maximum} flow $\leq$ capacity of any \textit{minimum} $(S, T)$-cut
\end{itemize}