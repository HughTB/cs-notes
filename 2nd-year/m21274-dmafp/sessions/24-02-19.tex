\lecture{List Patterns and Recursion}{12:00}{19/02/24}{Matthew Poole}

\section*{Patterns}

Previously we have looked at patterns for defining functions, but they can also be used for pattern matching with lists.
Patterns can include literal values, variables or wildcards, but they can also be tuples. For example, the pattern
 \verb`(x, _) = x` would extract the first value from a tuple and discard the other value. This is exactly how the
 \verb`fst` and \verb`snd` functions are defined. These functions are projection functions and are polymorphic as they
 word for tuples of any type.

\subsection*{List Patterns}

\verb`[]` and \verb`:` are both constructors which can be used to construct a list. For example, the list \verb`[1, 2]`
 can be constructed using \verb`[1, 2]`, \verb`1:[2]` or even \verb`1:2:[]`. The last example is how lists are
 represented internally within Haskell, and the other constructor is just syntactic sugar to make it easier to define
 lists.

You can also use these constructors within patterns, for example if you want to only deal with the first element of
 the list. For example, the \verb`head` function in the prelude is defined as \verb`head (x:xs) = x` where \verb`x`
 matches the first element of the list, and \verb`xs` matches the rest of the list. The naming convention of \verb`x`
 and \verb`xs` is standard, and means "x and exes". In this case, since we don't care about the rest of the list,
 \verb`xs` could be replaced with a wildcard. Similarly, the \verb`tail` function uses the same trick but discarding
 the first element and returning the rest of the list. When using list patterns, if you want to match an empty list,
 that pattern must appear first as otherwise it will still match any other list pattern.

\section*{Recursion Over Lists}

As with any other recursive functions, to define a recursive function over a list, you need a base case and recursive
 case. However, unlike other recursive functions the base case always considers the empty list, and the recursive case
 always considers a non-empty list, matched by \verb`(x:xs)` and passes the tail of the list into the function call.
 If we want to recurse over a list and perform a calculation on each element but leave them in a list, you can prepend
 the element back into the list, within the recursive case. For example, if you wanted to double every item in the list
 you would use a base case which returns \verb`0`, and a recursive case like \verb`func (x:xs) = x * 2 : func xs`. This
 function is an example of primitive recursion, but there is also general recursion.

\subsection*{General Recursion (Over Lists)}

An example of general recursion is the function \verb`zip` from the prelude. This function joins two lists into a
 single list of tuples. It's type definition is \verb`zip :: [a] -> [b] -> [(a, b)]`. In this case, the lists don't
 have to be lists of the same type, and if the second list is longer than the first, the remaining elements are just
 sent into the void. The easiest way to define this function would be to use a recursive function where the recursive
 case takes two non-empty lists \verb`x:xs` and \verb`y:ys`, and places \verb`x` and \verb`y` into a tuple, before
 recursing on the tail of the two lists. The base case would need to return an empty list (\verb`[]`) if either of the
 input lists are empty. This could be written as three separate patterns, e.g.
\begin{verbatim}
zip (x:xs) [] = []
zip [] (y:ys) = []
zip [] []     = []
\end{verbatim}
or using a single base case with wildcards, e.g.
\begin{verbatim}
zip _ _ = []
\end{verbatim}