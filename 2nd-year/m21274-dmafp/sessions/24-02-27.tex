\lecture{Methods of Proof}{17:00}{27/02/24}{Janka Chlebikova}

A mathematical proof is a reasoned argument to convince someone of the truth of a hypothesis. More accurately, an
 theorem, or argument, is a collection of hypotheses, or premises, followed by a statement known as the conclusion.
 $(p_1 \wedge p_2 \wedge \dots \wedge p_n) \Rightarrow q$.

There are several methods of proof which are used depending upon the hypothesis. These include direct proof, proof by
 contradiction, proof by contrapositive and mathematical induction. There are many more methods of proof out there, but
 these are the only methods that are covered by this module.

\section*{Direct Proof}

Theorems are often in the form $p \Rightarrow q$, and can be easily proved by supposing that $p$ is true, and working
 to prove it as such. You start with one of the hypotheses and make a series of deductions until you reach a conclusion.
 For example, to prove the theorem below, you start by assuming that the hypotheses are true, and then work from there
 to reach a conclusion. If the conclusion is that the theorem is true, you have successfully proved the theorem.
\begin{theorem}
For all integers $n$ and $m$, if $m$ is odd and $n$ is even, then $n+m$ is odd.
\end{theorem}

\begin{proof}
An integer, $n$, is even if and only if there exists an integer, $k$, such that $n = 2k$. Another integer, $m$ is odd if
 and only if there exists an integer, $l$, such that $m = 2l + 1$.

As such,
\begin{align*}
  n + m & = 2k + (2l + 1) \\
  & = 2(k + l) + 1 
\end{align*}
$\therefore$ $n + m$ is odd.
\end{proof}

(QED stands for ``Quod Erat Demonstrandum'', or ``What was to be demonstrated'')

\section*{Proof by Contradiction}

Since the conclusion can only be either true or false, if you can prove that the theorem cannot possibly be false, you
 have arrived at a contradiction, and therefore proved that it is true. For the theorem below, you start by taking the
 inverse of the hypothesis, and attempt to prove it. Once you reach your contradiction, you can say that the theorem
 must be true.
\begin{theorem}
For all $n \in \mathbb{N}$, if $n^2$ is even, then $n$ is even.
\end{theorem}

\begin{proof}
Assume that $n$ is odd, but $n^2$ is even. Therefore, there exists some $k \in \mathbb{N}$ such that $n = 2k + 1$.

As such,
\begin{align*}
  n^2 & = (2k + 1)^2 \\
  & = 4k^2 + 2k + 2k + 1 \\
  & = 4k^2 + 4k + 1 \\
  & = 2(2k^2 + 2k) + 1
\end{align*}
This results in a contradiction, as $n^2$ must be odd since it is in the form $2k + 1$.\\
$\therefore$ if $n^2$ is even then $n$ must be even.
\end{proof}

\section*{Proof by Contrapositive}

Rather than proving $p \Rightarrow q$, you can instead attempt to prove the logically equivalent proposition,
 $\neg q \Rightarrow \neg p$. You essentially use a direct proof of the contrapositive, and since it is logically
 equivalent to the original hypothesis, you have also proved the theorem. The example for this proof is the same as the
 proof by contradiction, e.g.
\begin{theorem}
For all $n \in \mathbb{N}$, if $n^2$ is even, then $n$ is even.
\end{theorem}

\begin{proof}
Assume that $n$ is odd, and $n^2$ is also odd. Therefore, there exists some $k \in \mathbb{N}$ such that $n = 2k + 1$
 and some $l \in \mathbb{N}$ such that $n^2 = 2l + 1$.

As such,
\begin{align*}
  n^2 & = (2k + 1)^2 \\
  & = 4k^2 + 2k + 2k + 1 \\
  & = 4k^2 + 4k + 1 \\
  & = 2(2k^2 + 2k) + 1
\end{align*}
Thus if $n$ is odd, $n^2$ is odd, and therefore the contrapositive theorem is true.\\
$\therefore$ if $n^2$ is even, then $n$ must be even.
\end{proof}

\section*{Proof by Mathematical Induction}

This is one of the most basic methods of proof, and is very useful for proving properties of natural numbers, or
 proving that an equation holds for an infinite set, such at $\mathbb{N}$ or $\mathbb{Z}$. To prove a theorem, you need
 the \textit{Basis Step} and the \textit{Inductive Step}. The basis step is a predicate that is defined for some integer
 $n$, e.g. ``$P(a)$ us true for some integer, $a \in \mathbb{N}$'' and the inductive step is a statement, e.g. ``For all
 integers $k \geq a$, if $P(a)$ is true, then $P(k + 1)$ is also true''. This suggests that $P(n)$ is true for all
 $n \in \mathbb{N}$ where $n \geq a$.
\begin{theorem}
For any integer, $n \in \mathbb{N}$ where $n \geq 1$, the sum of the first $n$ natural numbers is
 $S(n) = \frac{n(n+1)}{2}$.
\end{theorem}
In this case, the basis step would be ``For $n = 1$, $S(1) = 1 = \frac{1(1+1)}{2} = 1$'', and the inductive step would
 be ``Assume that the sum, $S(n)$, of the first $n$ natural numbers is $\frac{n(n+1)}{2}$. We need to prove that this is
 also true for the first $n+1$ natural numbers''

\begin{proof}
Assuming that
\begin{equation*}
S(n) = 1 + 2 + \dots + n = \frac{n(n+1)}{2}
\end{equation*}
\begin{align*}
S(n+1) & = 1 + 2 + \dots + n + (n+1) \\
& = S(n) + (n+1) \\
& = \frac{n(n+1)}{2} + (n+1) \\
& = \frac{n(n+1)}{2} + \frac{2(n+1)}{2} \\
& = \frac{(n+1)(n+2)}{2}
\end{align*}
$\therefore$ assuming that the formula is true for $n$, the formula is also true for $n+1$.
\end{proof}

\section*{(Dis)proving a Universally Quantified Statement}

To disprove a universally quantified statement, you need to find a counterexample for $\forall x \in D: P(x)$ such that
 $\exists x \in D: \neg P(x)$. For example, the statement
\begin{equation*}
\forall n \in \mathbb{N}: (2^n + 1 \mathrm{\ is\ prime})
\end{equation*}
is obviously false. A counterexample for this would be $n = 3$, since $2^3 + 1 = 9$ which is not a prime number. Only a
 single counterexample is required to disprove a universally quantified statement, which also means that you can prove
 a universally quantified statement by contradiction or counterexample.