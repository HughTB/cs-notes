\lecture{Input/ Output}{12:00}{18/03/24}{Matthew Poole}

Real-time input and output is important for any program which requires user-interaction. This is an issue for functional
 programming languages, since they are typically immutable and therefore don't have great support for using unpredictable
 values. Another problem is that functions are meant to have a property known as referential transparency. This means
 that, given the same set of arguments, a function should always return the same value. This is a required property of
 any pure functional language, and therefore and language with support for real-time input is an impure functional
 language.

A common approach used by functional languages, is to provide a set of `functions', along the lines of \verb`inputInt :: Int`,
 which reads an integer from the standard input and returns the value. This approach breaks the rule of referential
 transparency since the function returns a different value each time it is called, which is a side effect of reading
 from the standard input.

\section*{Haskell's I/O Approach}

Haskell uses a method of I/O known as the \textit{monadic approach}, as it is based upon the mathematical concept of a
 `monad'. The input/ output is viewed as a series of actions which happen in sequence. Haskell provides the generic type
 \verb`IO a`, which represent I/O actions of type \verb`a`.

A value of type \verb`IO a` is an action which, when executed, performs some input/ output and then returns a value of
 type \verb`a`. It also provides a method for sequencing these actions, which allows them to be executed one after the
 other in a specific order. Effectively, there is another very simple imperative programming language within Haskell
 which is used for writing I/O programs. As such, a typical Haskell program consists of a set of function definitions
 which abide by referential transparency and are purely functional, and a set of I/O programs. The I/O programs are able
 to call the pure functions, as well as each other, but pure functions are only able to call other pure functions, and
 never an impure I/O program. These programs are an even higher level of abstraction, as they are syntactic sugar for a
 set of purely functional expressions.

\subsection*{Prelude Functions}

There are two built-in functions defined in the prelude for reading input. Namely: \verb`getLine :: IO String`, which
 reads a line from the standard input, and returns a string; and \verb`getChar :: IO Char`, which reads a single character
 from the standard input.

The prelude also defines a method of writing to the standard output. There are three functions which are able to write
 to the standard output: \verb`putStr` and \verb`putStrLn` which write a string and a string with a newline to the
 standard output, respectively; and \verb`print`, which is a polymorphic function, which writes the value of any type
 which extends \verb`Show` to the standard output.

Since every function has to have a return type, Haskell provides the \textit{one-element} type called \verb`()`, which
 contains a single value, also \verb`()`. All this really means is that the I/O program returns nothing of interest.
 This is similar to the \verb`void` type that most imperative languages use to represent a function with no return value.
 Haskell effectively gets this value for free, as it is simply a tuple with no elements.

\subsection*{`do' Notation}

As previously stated, an I/O program is a set of actions with a specific order. To write a program, you have to use the
 \verb`do` keyword to specify that a function is an I/O program. For example, if you wanted to write a program which
 displays a message asking the user to `Enter a string', gets a string of user input (but does nothing with it), then
 displays a `done' message, you would define it as below:
\begin{verbatim}
readALine :: IO ()
readALine = do
  putStrLn "Enter a string"
  getLine
  putStrLn "Done"
\end{verbatim}
Notice that every line consists of an action of type \verb`IO a` for some common \verb`a`.

Usually, you'll actually want to do something with the user input, and so you `capture' it into a named value (effectively
 a variable) using \verb`<-`. This appears to be like an assignment operator as in imperative languages, but since this
 is still Haskell, a named value is immutable, as everything else is. If we wanted to write a similar I/O program as
 before, but this time, output the value the user entered, we can capture the value they enter, e.g.
\begin{verbatim}
readALine :: IO ()
readALine = do
  putStrLn "Enter a string"
  line <- getLine
  putStrLn ("You entered " ++ line)
\end{verbatim}

If you wanted to get an integer from user input, you would read a value and capture it, and then pass it into the
 \verb`read` function for an integer, e.g. \verb`do str <- getLine` and then \verb`read str :: Int`. Now you have a
 problem, since \verb`read` is a pure function, it cannot be called directly from an I/O program. To get around this,
 Haskell defines a function \verb`return :: a -> IO a`, which does no I/O operations, but simply returns the value of
 the argument you give it. Putting this all together, you could create an I/O program \verb`getInt` which reads an
 integer from the standard input, as bellow
\begin{verbatim}
getInt :: IO Int
getInt = do
  str <- getLine
  return (read str :: Int)
\end{verbatim}

\subsection*{File I/O}

The prelude also defines the following functions which allow you to read from or write to files.
\begin{itemize}
  \item \verb`readFile :: String -> IO String`
  \item \verb`writeFile :: String -> String -> IO ()`
  \item \verb`appendFile :: String -> String -> IO ()`
\end{itemize}
For each function, the first argument is the path to the file, and for write and append, the second argument is the
 string which should be written or appended to the file, respectively.

\subsection*{Conditional Code}

The conditionals \verb`if`, \verb`then` and \verb`else` can be used within a \verb`do` construct to allow for more
 complex functions with flow control. Alternatively, you can move more of the logic to pure functions, allowing you to
 use guards and pattern matching for more efficient and readable functions, and simply read in the value and then call
 the pure function with that value.

\subsection*{Local Definitions}

As with pure functions, you can use local definitions (with the keyword \verb`let`) in I/O programs to reduce the number
 of IO functions which need to be used. This is because you can replace an assignment with a local definition, e.g. replace
 \verb`response <- return (even 2)` with \verb`let response = even 2`.

\subsection*{Recursion}

IO programs, like pure functions, can be recursive. The syntax is exactly the same with both, since you are simply
 calling a function within a function, it just happens to be the same function you are already in.

\section*{Haskell Programs}

To create an actual program with Haskell, you have to define a main function, the entry point for the program. This is
 similar to most imperative languages, but typically just immediately calls a pure function. The main function is defined
 as
\begin{verbatim}
main :: IO [type]
main = do
  [do stuff]
\end{verbatim}
This allows you to run your code as an actual program, using the command \verb`runhaskell <filepath>`