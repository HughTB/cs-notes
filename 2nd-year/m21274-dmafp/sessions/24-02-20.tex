\lecture{Logic II - Quantified Statements}{17:00}{20/02/24}{Janka Chlebikova}

\section*{Propositional Logic}

Propositional logic only applies to simple statements where the only possible states are either true or false. A
 statement such as ``Mr Bean is a Mathematical Professor'' (while obviously false) could conceivable be either true or
 false. However, a statement such as ``They are a Mathematical Professor'' is not a proposition, as its value depends
 upon the value of \textit{They}, making it a predicate.

\section*{Predicate Logic}

An atomic proposition is one where there is only one propositional variable, which can be either true or false. All
 predicates are statements containing one or more variables. If the values from a given domain are assigned to all the
 variables, the resulting statement is a proposition. To make a predicate into a proposition, you must replace all the
 variables with a value.

For example, the predicate $p : x \mathrm{\ is\ an\ integer\ less\ than\ } 80$ is not a proposition as the value of $x$
 is unknown. If we replace $x$ with a value, for example an integer from the domain $\mathbb{Z}$ such as $10$ then we
 can determine that $10$ is less than $80$ and therefore $p$ is a true proposition.

A more formal example of this would be the predicate $T$ where $T(x,y,z) : x < y + z$ is not a proposition until you add
 the qualifier that $T(x,y,z) : x < y + z \mathrm{\ where\ } x,y,z \in \mathbb{Z}$

\section*{Quantifiers}

A quantifier refers to quantities such as ``some'' or ``all'', which are used in statements such as ``$P$ is true
 \textit{for some} value of $x$'' or ``$T$ is true \textit{for all} values of $y$''. There are two main quantifiers,
 the ``universal quantifier'' (represented by $\forall$) and the ``existential quantifier'' (represented by $\exists$)

\subsection*{Universal Quantifier ($\forall$)}

For a predicate $P$ such as ``For every $x$ in the domain $\mathbb{Z}$, $P(x)$'', you could rewrite it using the
 universal quantifier as $\forall x \in \mathbb{Z}, P(x)$. This means that for the predicate to be true, every value
 in the domain, $\mathbb{Z}$ must have a corresponding, true proposition, $P(x)$. That also means that you only need one
 counter-example to prove that the predicate is false.

For example, the predicate $\forall x \in S, x^2 > x$ is true for the set $S = \{2,3,4,5,6\}$ because we can
 exhaustively prove that for every value of $x$, $x^2$ is greater than or equal to $x$. $2^2 > 2$, $3^2 > 3$, $4^2 > 4$,
 $5^2 > 5$, $6^2 > 6$. If the set $S$ was expanded to include $1$, then the predicate would be
 false, as $(1)^2 = 1, 1 \ngtr 1$.

\subsection*{Existential Quantifier ($\exists$)}

For a predicate $P$ such as ``There exists an $x$ in the domain $\mathbb{Z}$ such that $P(x)$ is true'', you could
 rewrite it using the existential quantifier as $\exists x \in \mathbb{Z}, P(x)$. This means that for the predicate to
 be true, there needs to be at least one value of $x$ for which there is a corresponding, true proposition, $P(x)$. It
 is much harder to disprove such a predicate, especially if the domain is infinite as in this case, and so it is usually
 necessary to disprove it logically.

For example, the predicate $\exists x \in \mathbb{Z}, x^2 > x$ is very easy to prove true, since we can find a single
 value of $x$, lets say $2$, where $2^2 > 2$ is obviously true. It is also easy to disprove a statement such as
 $\exists x \in \mathbb{Z} \mathrm{\ such\ that\ } x^2 < -10$, since we can logically say that there is no number for
 which it's square is negative.

\subsection*{Negations of Quantified Statements}

The negation of a quantified statement seems obvious at first glance, but the answer is not just to negate the
 proposition (e.g. ``All mathematicians wear glasses'' $\rightarrow$ ``No mathematicians wear glasses''), but instead to
 give the statement matching the counter-example to that statement (e.g. ``All mathematicians wear glasses''
 $\rightarrow$ ``There is at least one mathematician who does not wear glasses''). More formally, the negation of a
 statement in the form ``$\forall x \in S, P(x)$'' is logically equivalent to a statement in the form 
 ``$\exists x \in S, \neg P(x)$'', and the negation of a statement in the form ``$\exists x \in S, P(x)$'' is logically
 equivalent to a statement in the form ``$\forall x \in S, \neg P(x)$''.

\subsection*{Nested Quantifiers}

Quantifiers can be nested, e.g. $\forall x \exists y$ or $\exists x \forall y$, which can be useful for reducing the
 ambiguity of a predicate. For example, the predicate ``There is a student solving every exercise of the tutorials''
 could either mean ``There is one student who solves every exercise of the tutorials'' or ``For any given exercise, 
 there is a student who solves the exercise''. Obviously the person who wrote the statement had only one of these in
 mind, and so it would be nice to explicitly define which of the two were intended.

If we let $P(x, y)$ be the proposition ``Student $x$ solves the exercise $y$'', $S$ be the set of students and $E$ be
 the set of all exercises, there are two possible options -
\begin{itemize}
  \item $\exists x \in S\ \forall y \in E$ such that $P(x, y)$ is true. This is like the previously mentioned possible
   meaning, ``There is one student who solves every exercise of the tutorials''
  \item $\forall y \in E\ \exists x \in S$ such that $P(x, y)$ is true. This is like the other possible meaning,
   ``For any given exercise, there is a student who solves the exercise''
\end{itemize}