\lecture{Functional Programming in Python}{12:00}{22/04/24}{Matthew Poole}

Most modern imperative programming languages include a subset of functional programming concepts, but these are usually
 limited to certain environments. For example, Python includes list comprehensions, lambda expressions and higher-order
 functions, some very important functional programming concepts.

\section*{Functions are Data}

In Python, any function definition creates a new variable, which acts as a pointer to the function. This is useful, as
 it allows you to define new functions based upon existing ones, but it also means that it is possible to accidentally
 cause variable hiding by using the same name for a function and variable. It is possible to assign variables to the
 value of a function-valued variable, which then acts as a pointer to that function, but it is also possible to assign
 a different value to the function-valued variable, causing that function to be lost.

\subsection*{List Comprehensions}

List comprehension in Python is very similar to Haskell, as they create new lists from the values of an old list, and a
 function. For example, \verb`[i*2 for i in list1]` would return a new list containing each value from \verb`list1`
 multiplied by 2. You can also add a test at the end, using the \verb`if` keyword, e.g.\\\verb`[i*2 for i in list1 if i>2]`.

\subsection*{...Using Tuples}

In Python, a tuple is effectively an immutable list, which typically contains only a few values. This also works in a
 similar way to Haskell, in that you must define the `pattern' which a tuple follows, before you can take any values
 from it.

\subsection*{Lambda Expressions}

As in Haskell, Python includes anonymous functions known as \textit{lambda expressions}. A simple example of this would
 be \verb`lambda x : x * 10`, which represents a function which multiplies its argument by 10. Higher-order functions
 can also be used as an argument for a lambda expression, allowing you to return a function from a function.

\subsection*{Higher-Order Functions}

Python includes built-in functions such as \verb`map`, \verb`filter` and \verb`reduce`, which take a function and an
 list and apply the function to each element of the list. The \verb`map` and \verb`filter` functions are the same as
 those in Haskell, and the \verb`reduce` function acts very similarly to the \verb`foldr` function. The \verb`reduce`
 function can also take a third argument, which becomes the starting value for the application of the function to the
 first element in the list.

\subsection*{Other Benefits}

Another benefit of higher-order functions is that any local variables used in a returned function continue to exist,
 even if the function they were defined in has already exited. This allows you to define static variables for a function
 without having to make them global. A returned function which uses this is known as a \textit{closure}.

\section*{Recursion}

It is usually said that a recursive function is easier to read than an equivalent iterative function, but the argument
 could be made that recursive functions are almost always slower than their iterative equivalent. This is often caused
 by repeatedly computing intermediate results, which can be solved using higher-order functions and a dictionary. This
 is often called \textit{Memorization}.

\subsection*{Dictionaries}

In Python, a dictionary is an unordered collection of data, whose values are accessed via a key. Dictionary literals are
 defined as a sequence of \verb`key:value` pairs between braces, e.g. \verb`{"eggs": 2, "cheese: 1}`. The \verb`in`
 operator gives an easy way to test if a dictionary contains any given key.

\subsection*{Memorization}

A general purpose memorization function is used, which is passed the function to be optimised, and returns a new
 function, which we can use to overwrite the old function-valued variable, creating a runtime-optimised function. An
 example of this optimisation function is below--
\begin{verbatim}
def memorize(f):
  cache = {}
  def g(arg):
    if arg in cache:
      return cache[arg]
    else:
      cache[arg] = f(arg)
      return cache[arg]
  
  return g
\end{verbatim}
You would then define a function, e.g. \verb`def fibonacci`, and then assign the function-valued variable to the optimised
 function, e.g. \verb`fibonacci = memorize(fibonacci)`.