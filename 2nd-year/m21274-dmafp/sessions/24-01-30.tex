\lecture{Relations}{17:00}{30/01/24}{Janka Chlebikova}

\section*{Ordered Pairs}

A set is an unordered collection of members, but sometimes it is useful to consider the order of members. In this
 case, you can use an ordered pair, which is two members written as $(a, b)$. Since they are ordered pairs, $(a, b)$ is
 distinct from $(b, a)$

\section*{Cartesian Product}

With the sets $A$ and $B$, $A \times B$ is the Cartesian product, where $A \times B = {(a, b) \mid a \in A \mathrm{\ and\ } b \in B}$.
This is the set of all ordered pairs, in which the first item is from the first set, and the second item from the second
 set. For example, if $X = {1, 2, 3}$ and $Y = {a, b}$, then
 $A \times B = {(1, a), (1, b), (2, a), (2, b), (3, a), (3, b)}$

\section*{Relations}

If $A$ is the set of all students taking DMaFP and $B$ is the set of all modules offered by the School of Computing,
 then the relation $T$ can be defined between $A$ and $B$ as ``If the student, $x \in A$ is registered on the module, 
 $y \in B$ then $x$ is related to $y$ by the relation $T$'', e.g.
 $(\mathrm{Hugh\ Baldwin}, \mathrm{Ethical\ Hacking}) \in T$. The order matters, as $T$ is a relation from the set $A$ to
 the set $B$

To put it another way, if a set is a subset of the Cartesian product of $A$ and $B$, then it is a relation between $A$
 and $B$. If $T \subseteq A \times B$ and $(a, b) \in T$, we can say that $a$ is related to $b$ by $T$, and therefore
 $aTb$.

Relations can also be described ``by the characteristics of their members''. For example, if $A = {1, 2}$ and
 $B = {1, 2, 3}$, we can define a relation from $A$ to $B$ as follows:
 $x \in A \mathrm{\ is\ related\ to\ } y \in B \mathrm{\ if\ and\ only\ if\ } x \leq y$. With this definition, we can
 see that $(1, 3) \in R$ since $1 < 3$, but $(2, 1) \notin R$ since $2 > 1$. The full list of members is
 $R = {(1, 1), (1, 2), (1, 3), (2, 2), (2, 3)}$.

If $A = B$, then a relation \textbf{on} $A$ is a relation from $A$ to $A$, and so is a subset of $A \times A$.

\section*{Basic Properties of Relations}

\subsection*{Reflexivity}

A relation is reflexive, if and only if $(x, x) \in R$ for all $x \in A$. For example, the relation
 $R = {(1, 1), (1, 2), (1, 3), (2, 2), (3, 3)}$ on the set $A = {1, 2, 3}$ is reflexive. Another example is that the
 relation $S = {(x, y) \mid x, y \in A \mathrm{\ and\ } x \leq y}$ is reflexive on the set $A = {1, 2, 3}$ since
 $(1, 1), (2, 2), (3, 3) \in S$

\subsection*{Symmetry}

A relation is symmetric, if and only if for all $x, y \in A$, if $(x, y) \in R$ then $(y, x) \in R$

\subsection*{Transitivity}

A relation is transitive, if and only if for all $x, y, z \in A$, if $(x, y) \in R$ and $(y, z) \in R$ then
 $(x, z) \in R$

\subsection*{Equivalence}

A relation is an equivalence relation if and only if it is Reflexive, Symmetric and Transitive. Suppose that $A$ is a
 set and $R$ is an equivalence relation on $A$, for each element $a \in A$, the equivalence class of $a$, $[a]$ is the
 set of all elements in $A$ such that $x$ is related to $a$ by $R$: $[a] = {x \mid x \in A \mathrm{\ and\ } (x, a) \in R}$.
 Since $R$ must be a symmetric relation, we can also write $(a, x) \in R$. For example, for the set $A = {0, 1, 2, 3}$
 and relation $R = {(0, 0), (1, 1), (1, 3), (2, 2), (3, 3), (3, 1)}$ the equivalence class for $1$ is
 $[1] = {x \mid x \in A \mathrm{\ and\ } (x, 1) \in R} = {1, 3}$. A set of the equivalence classes is also a partition
 of the set.