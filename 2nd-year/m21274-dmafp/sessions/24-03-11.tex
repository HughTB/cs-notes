\lecture{Algebraic Types}{12:00}{11/03/24}{Matthew Poole}

As with most programming languages, it is possible to define your own data types in Haskell, known as algebraic types.
 As shown previously, you can define simple datatypes using the \verb`type` keyword, e.g. \verb`type HouseNumber = Int`.
 However, it would be hard to represent more complex structures, such as binary trees or graphs, using only simple data
 types. Algebraic types allow you to represent arbitrary types. These are defined using the \verb`data` keyword, which
 allows you to define a type, along with all of the values which are valid for that type. For example, if you wanted to
 define a datatype to store every day of the week, you could do so with the following code\\
 \verb `data Day = Mon | Tue | Wed | Thur | Fri | Sat | Sun`. In this case, the constructors are the data values, or
 members of the type. This is known as an enumerated type. The boolean type could be defined as
 \verb`data Bool = True | False`.

\section*{Functions with Enumerated Types}

The simplest way to define a function on an enumerated type is by using pattern matching. By default, there are no
 actual operators, such as \verb`==` for the type. It is possible to define them manually, but this would be very
 tedious, and there are easier ways of doing so.

You can get Haskell to automatically define the \verb`==` operator by declaring that the type is a member of the
 \verb`Eq` \textbf{type class}. We do this by adding \verb`deriving (Eq)` to the type declaration, such as
\begin{verbatim}
data Day =  Mon | Tue | Wed | Thur | Fri | Sat | Sun
            deriving (Eq)
\end{verbatim}
This adds obvious implementations for \verb`==` and \verb`/=`. There are other type classes which automatically add
 other useful operators, such as
\begin{itemize}
  \item \verb`Ord`, which defines the operators \verb`<`, \verb`<=`, \verb`>` and \verb`>=`
  \item \verb`Show`, which defines a function \verb`show`, that converts values to strings
  \item \verb`Read`, which defines a function \verb`read`, that converts strings back into values (e.g. parsing)
\end{itemize}

\section*{Product Types}

A product type is an algebraic type with a constructor with two arguments. This was previously done by giving a type
 name to a tuple, for example
\begin{verbatim}
type StudentMark = (String, Int)
\end{verbatim}
could be redefined as
\begin{verbatim}
data StudentMark = Student String Int
\end{verbatim}
where \verb`Student` is the name of the constructor. This allows you to define types in essentially the same way, but
 with labels which allow you to tell more easily which type a given item is.

\section*{Sum Types}

You can combine enumerated and product types to create more complex types where each member can have different variables
 of different types. For example, a shape type could be defined as
\begin{verbatim}
data Shape = Circle Float | Rectangle Float Float
\end{verbatim}
This would allow you to create two instance of a shape, such as \verb`Circle 9.0` and \verb`Rectangle 4.5 6.0`, both of
 which could be used as the input to a function with the input type \verb`Shape`. For example, you could use pattern
 matching to create an \verb`area` function which has a different definition for each variety of shape, i.e.
\begin{verbatim}
area :: Shape -> Float
area (Circle r)       = pi * r * r
area (Rectangle w h)  = w * h
\end{verbatim}

\section*{Recursive Types}

Algebraic types can also be defined in terms of themselves. For example, if you wanted to represent a binary tree, you
 could define each node as a type which is either null, or a node with a value, and left and right child. In the case of
 Haskell, you would define it as a tree, which can either be null or a value and left and right subtree, i.e.
\begin{verbatim}
data Tree = Null | Node Int Tree Tree
\end{verbatim}

Most functions using a recursive type would use null as the base case, and a non-null type for the recursive case.