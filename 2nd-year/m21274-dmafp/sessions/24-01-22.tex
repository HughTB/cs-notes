\lecture{Intro to Functional Programming}{12:00}{22/01/24}{Matthew Poole}

\begin{itemize}
  \item For this module, we will be using the GHC (Glasgow Haskell Compiler), or more specifically it's interactive shell, GHCi
\end{itemize}

\section*{Imperative VS Functional Programming}

\begin{itemize}
  \item Most programming languages are imperative
  \begin{itemize}
    \item Such as Python, JavaScript, C, etc
  \end{itemize}
  \item Functional programming is another programming paradigm, which is based upon the mathematical concept of a function
  \item Imperative programming has state, statements (or commands) and side effects
  \item Pure Functional programming has no state, statements, or side effects
  \item A side effect is the change of state caused by calling a functionl assigning a variable, etc
  \begin{itemize}
    \item This means that it is not always possible to predict the result of running a program, even with access to it's
    source code
  \end{itemize}
\end{itemize}

\begin{itemize}
  \item Since most programs need to cause a side effect (usually outputting data), most functional programming languages are
  not purely functional, but tend to organise the code such that only one part causes side effects
\end{itemize}

\section*{Functional Programming Languages}

\begin{itemize}
  \item There are two types of functional programming languages
  \item Pure
  \begin{itemize}
    \item Languages such as Haskell
    \item Has absolutely no state or side effects
  \end{itemize}
  \item Impure
  \begin{itemize}
    \item Languages such as ML, Clojure, Lisp, Scheme, OCaml, F\#
    \item Has some state or side effects, either everywhere or in a specific part of code
  \end{itemize}
  \item There are also some functional constructs in major imperative langages such as Python, JavaScript, and more
\end{itemize}

\section*{FP Basics}

\subsection*{Expressions}

\begin{itemize}
  \item An expression is a piece of text which has a value
  \item To get the value from the expression, you evaluate it
  \item This gives you the value of the expression
  \item e.g.
  \item \begin{verbatim}
Expression -> evaluate -> Value
2 * 3 + 1  -------------> 7
  \end{verbatim}
\end{itemize}

\subsection*{Functions}

\begin{itemize}
  \item A function whose output relies only upon the values that are input into it
  \item The result will always be the same, given the same values
  \item This is the same as a mathematical function, which is where the name Functional Programming comes from
\end{itemize}

\section*{Haskell Basics}

\begin{itemize}
  \item In Haskell, all functions have higher precidence than operators
  \item This means that you have to explicity use brackets to ensure the correct order of operations
\end{itemize}