\lecture{Connectivity and Cuts}{17:00}{30/04/24}{Janka Chlebikova}

A graph, $G$, is connected if there is a path in $G$ between any given pair of vertices, and disconnected otherwise.
 An edge in a connected graph is a bridge if deleting it would create a disconnected graph. In a connected graph without
 a bridge, deleting any edge will create a connected subgraph. This is a useful property, as it means that at least one
 edge in the graph can disappear, while the graph remains connected. It is especially useful for practical applications,
 such as a telecom network, where you want the network to be reliable, and for all customers to remain connected, even
 if a link goes down.

\begin{minipage}[c]{0.45\linewidth}
  \begin{figure}[H]
    \centering
    \usetikzlibrary{graphs}
    \begin{tikzpicture}
      \tikzset{
        node distance=3cm,
      }

      \node[vertex] (va) {$v_a$};
      \node[vertex, right of=va] (vb) {$v_b$};
      \node[vertex, below of=vb] (vc) {$v_c$};
      \node[vertex, below of=va] (vd) {$v_d$};
      \node[vertex, node distance=1.5cm, below of=va, xshift=0.75cm] (ve) {$v_e$};
      \node[vertex, node distance=1.5cm, below of=vb, xshift=-0.75cm] (vf) {$v_f$};

      \draw (va) edge[] (vb)
            (vb) edge[] (vc)
            (vc) edge[] (vd)
            (vd) edge[] (va)
            (va) edge[] (ve)
            (ve) edge[] (vf)
            (vf) edge[] (vc);
    \end{tikzpicture}
    \caption{A connected graph, $G_1$}
    \label{fig:telecom1}
  \end{figure}
\end{minipage}\hfill
\begin{minipage}[c]{0.45\linewidth}
  \begin{figure}[H]
    \centering
    \usetikzlibrary{graphs}
    \begin{tikzpicture}
      \tikzset{
        node distance=3cm,
      }

      \node[vertex] (va) {$v_a$};
      \node[vertex, right of=va] (vb) {$v_b$};
      \node[vertex, below of=vb] (vc) {$v_c$};
      \node[vertex, below of=va] (vd) {$v_d$};
      \node[vertex, node distance=1.5cm, below of=va, xshift=0.75cm] (ve) {$v_e$};
      \node[vertex, node distance=1.5cm, below of=vb, xshift=-0.75cm] (vf) {$v_f$};

      \draw (va) edge[] (vb)
            (vb) edge[] (vc)
            (vc) edge[] (vd)
            (vd) edge[] (va)
            (va) edge[] (ve)
            (ve) edge[] (vf);
    \end{tikzpicture}
    \caption{A connected subgraph of $G_1$}
    \label{fig:telecom2}
  \end{figure}
\end{minipage}

If we were to consider each edge as a telephone line, and each vertex as an exchange, we could see that deleting one
 vertex would also create a connected subgraph, but any more could be an issue.

\begin{minipage}[c]{0.45\linewidth}
  \begin{figure}[H]
    \centering
    \usetikzlibrary{graphs}
    \begin{tikzpicture}
      \tikzset{
        node distance=3cm,
      }

      \node[vertex] (va) {$v_a$};
      \node[vertex, right of=va] (vb) {$v_b$};
      \node[vertex, below of=vb] (vc) {$v_c$};
      \node[vertex, below of=va] (vd) {$v_d$};
      \node[vertex, node distance=1.5cm, below of=va, xshift=0.75cm] (ve) {$v_e$};
      \node[vertex, node distance=1.5cm, below of=vb, xshift=-0.75cm] (vf) {$v_f$};

      \draw (va) edge[] (vb)
            (vb) edge[] (vc)
            (vc) edge[] (vd)
            (vd) edge[] (va)
            (va) edge[] (ve)
            (ve) edge[] (vf)
            (vf) edge[] (vc);
    \end{tikzpicture}
    \caption{A connected graph, $G_2$}
    \label{fig:telecom3}
  \end{figure}
\end{minipage}\hfill
\begin{minipage}[c]{0.45\linewidth}
  \begin{figure}[H]
    \centering
    \usetikzlibrary{graphs}
    \begin{tikzpicture}
      \tikzset{
        node distance=3cm,
      }

      \node[vertex] (va) {$v_a$};
      \node[vertex, right of=va] (vb) {$v_b$};
      \node[vertex, below of=vb] (vc) {$v_c$};
      \node[vertex, below of=va] (vd) {$v_d$};
      \node[vertex, node distance=1.5cm, below of=va, xshift=0.75cm] (ve) {$v_e$};

      \draw (va) edge[] (vb)
            (vb) edge[] (vc)
            (vc) edge[] (vd)
            (vd) edge[] (va)
            (va) edge[] (ve);
    \end{tikzpicture}
    \caption{A connected subgraph of $G_2$}
    \label{fig:telecom4}
  \end{figure}
\end{minipage}

\section*{Types of Connectivity}

As you can see, there are two different types of connectivity, demonstrated by the two examples above. These are
 edge-connectivity (links fail) and vertex-connectivity (exchanges fail).

\subsection*{Edge-Connectivity}

The edge-connectivity, $\lambda(G)$ of a
 connected graph $G$, is the smallest number of edges whose removal could cause $G$ to become unconnected. For example,
 the graph in Figure \ref*{fig:telecom1} has an edge-connectivity ($\lambda(G_1)$) of $2$. If you remove any single edge,
 the graph remains connected, but there exist two edges such that if they are removed, $G_1$ would become disconnected.
 Therefore, $\lambda(G_1) = 2$.

Formally, a graph $G$ has an edge connectivity of $\lambda(G)$ if
\begin{itemize}
  \item There exist $\lambda(G)$ edges such that their removal disconnects $G$, and
  \item Removal of any $\lambda(G) - 1$ edges does not disconnect $G$
\end{itemize}

\subsubsection*{Edge Cuts}

An edge cut of a graph is a set of edges whose removal would cause the graph to become unconnected. There are many edge
 cuts for any given graph, and not all of the edge cuts have the same number of edges. The edge-connectivity of a graph
 is simply the size of the smallest edge cut of the graph.

A graph, $G$, is $k$-edge-connected when $\lambda(G) \geq k$. This means it remains connected when fewer than $k$ edges
 are removed.

\subsection*{Vertex-Connectivity}

The vertex-connectivity, $\kappa(G)$ of a connected graph $G$, is the smallest number of vertices whose removal could
 cause $G$ to become unconnected. When a vertex is removed, any adjacent edges are also removed. For example, the graph
 in Figure \ref*{fig:telecom3} has a vertex-connectivity ($\kappa(G_2)$) of $2$. If you remove any single vertex, the
 graph remains connected, but there exists two vertices such that if they are removed, $G_2$ would become disconnected.
 Therefore, $\kappa(G_2) = 2$.

\subsubsection*{Vertex Cuts}

A vertex cut of a graph is a set of vertices whose removal would cause the graph to become unconnected. There are many
 vertex cuts for any given graph, and not all of the vertex cuts have the same number of vertices. The vertex-connectivity
 of a graph is simply the size of the smallest vertex cut of the graph. This breaks for a complete graph, since the graph
 cannot be disconnected by removing vertices. Technically, the vertex-connectivity of a complete graph $K_n$ is $n - 1$.

A graph, $G$, is $k$-vertex-connected when $\kappa(G) \geq k$. This means it remains connected when fewer than $k$ vertices
 are removed.

\subsection*{Connectivity Upper Bounds}

There are upper bounds on both $\lambda(G)$ and $\kappa(G)$. To calculate them, we can use the following theorem--
\begin{equation*}
  \kappa(G) \leq \lambda(G) \leq \delta(G) \leq \frac{2 \lvert E \rvert}{\lvert V \rvert}
\end{equation*}
Given that $2 \lvert V \rvert$ is the sum of all vertex degrees, $\frac{2 \lvert E \rvert}{\lvert V \rvert}$ is the
 mean of all vertex degrees.

A graph for which $\kappa(G) = \lambda(G) = \delta(G) = \frac{2 \lvert E \rvert}{\lvert V \rvert}$ has the max vertex-
 and edge-connectivity possible for any graph with $\lvert V \rvert$ vertices and $\lvert E \rvert$ edges. Such a graph
 is said to have optimal connectivity. All graphs with optimal connectivity are \textit{regular graphs} (a graph in
 which every vertex has the same degree) but not all regular graphs have optimal connectivity.