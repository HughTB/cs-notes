\lecture{Logic I}{17:00}{13/02/24}{Janka Chlebikova}

\section*{Propositions}

A proposition is a statement that is either true or false, but not both. The letters $p, q, r, s$, etc. denote
 propositional variables, each of which has one of two truth values, true or false. Statements can be combined using
 logical connectives to create compound statements.

\section*{Logical Connectives}

The three main logical connectives are not ($\neg$), and ($\wedge$), or ($\vee$). These work in the same way as they do in
 other places, such as boolean algebra, but use different and objectively the correct symbols.

\subsection*{Not (Negation)}

The negation of a statement $p$ is the statement not $p$, or $\neg p$

\subsection*{And (Conjunction)}

The conjunction of two statements, $p$ and $q$, is the compound statement $p$ and $q$, or $p \wedge q$.

\subsection*{Or (Disjunction)}

The (inclusive) disjunction of two statements, $p$ and $q$, is the compound statement $p$ or $q$, or $p \vee q$. There
 is also an exclusive disjunction, which means that only one of the two statements can be true, and so is exclusive.

\section*{Conditional Propositions}

\subsection*{Implication}

An implication compound proposition ($\Rightarrow$) means an `if-then' relation, e.g. if it is raining ($p$), then I get
 wet ($q$). $p \Rightarrow q$. Since it is an implication, $q$ can be true regardless of the value of $p$, but if $p$ is
 true then $q$ must also be true. This can be represented using the following truth table:
\begin{table}[h]
  \centering
  \begin{tabular}{ c c c }
    $p$ & $q$ & $p \Rightarrow q$ \\
    \hline
    T & T & T \\
    T & F & F \\
    F & T & T \\
    F & F & T
  \end{tabular}
\end{table}

As suggested, if $p$ is false, then $p \Rightarrow q$ is true, since there is no implication as to the value of $q$

It is also useful to know that $p \Rightarrow q$ is equivalent to $\neg p \vee q$

\subsection*{Bi-conditional}

A bi-conditional compound proposition ($\Leftrightarrow$) means an `if and only if' relation, and as such $q$ can be
 true if and only if $p$ is true. This can be represented using the following truth table:
\begin{table}[h]
  \centering
  \begin{tabular}{ c c c }
    $p$ & $q$ & $p \Rightarrow q$ \\
    \hline
    T & T & T \\
    T & F & F \\
    F & T & F \\
    F & F & T
  \end{tabular}
\end{table}

\subsection*{Negation of Conditional Propositions}

The negation of a conditional proposition, $p \Rightarrow q$ would be $p \wedge \neg q$ since
\begin{center}
  $\neg (p \Rightarrow q) \equiv \neg (\neg p \vee q) \equiv \neg (\neg p) \wedge \neg q \equiv p \wedge \neg q$
\end{center}

\subsection*{Contrapositives}

The contrapositive of a conditional proposition, $p \Rightarrow q$ would be $\neg q \Rightarrow \neg p$.

\section*{Truth of Compound Propositions}

It is possible to construct truth tables for more complex compound statements, as we did for boolean algebra. The order
 of precedence for connectives is as follows:
\begin{itemize}
  \item Brackets
  \item Not ($\neg$)
  \item And ($\wedge$)
  \item Or ($\vee$)
  \item Implication ($\Rightarrow$)
  \item Bi-conditional ($\Leftrightarrow$)
\end{itemize}
For multiple of the same precedence, you can calculate them in any order (left to right or right to left).

\section*{Statement Properties}

\subsection*{Tautology}

A statement is a tautology if it is true for all possible values of it's propositional variables, e.g. $p \vee \neg q$
 is a tautology since it is always true, as demonstrated by the below truth table:
\begin{table}[h]
  \centering
  \begin{tabular}{ c c c }
    $p$ & $\neg p$ & $p \vee \neg p$ \\
    \hline
    T & F & T \\
    F & T & T \\
  \end{tabular}
\end{table}

\subsection*{Contradiction}

A statement is a contradiction if it is false for all possible values of it's propositional variables, e.g.
 $p \wedge \mathrm{p}$ is a contradiction since it is always false, as demonstrated by the below truth table:
\begin{table}[h]
  \centering
  \begin{tabular}{ c c c }
    $p$ & $\neg p$ & $p \wedge \neg p$ \\
    \hline
    T & F & F \\
    F & T & F \\
  \end{tabular}
\end{table}

\subsection*{Contingency}

A statement is a contingency if it can be either true or false depending upon the values of the propositional variables.

\section*{Logical Equivalence}

Two or more propositions with the same truth values are said to be logically equivalent, since either one will produce
 the same output given the same propositional variables. For example, $p \Rightarrow q \equiv \neg p \vee q$ since
\begin{table}[h]
  \centering
  \begin{tabular}{ c c c c c }
    $p$ & $q$ & $p \Rightarrow q$ & $\neg p$ & $\neg p \vee q$ \\
    \hline
    T & T & T & F & T \\
    T & F & F & F & F \\
    F & T & T & T & T \\
    F & F & T & T & T \\
  \end{tabular}
\end{table}

Like boolean algebra, propositions follow a few basic rules,
\begin{itemize}
  \item Commutative - $p \wedge q \equiv q \wedge p$, $p \vee q \equiv q \vee p$
  \item Distributive - $(p \vee (q \wedge r)) \equiv ((p \vee q) \wedge (p \vee r))$
  \item De Morgan's - $\neg (p \vee q) \equiv \neg p \wedge \neg q$, $\neg (p \wedge q) \equiv \neg p \vee \neg q$
\end{itemize}

You can determine if two propositions are logically equivalent in one of two ways, either by using truth tables or by
 using basic logical equivalences (the above rules) to simplify the propositions and make them easier to discuss. Using
 truth tables can be complex and cumbersome since if there are more than two propositional variables, there will be many
 rows in each table.

\section*{Necessary and Sufficient Conditions}

The sufficient condition of an `if-then' statement is $r$, for $s$ where `if $r$ then $s$'

The necessary condition of an `if-then' statement is $r$, for $s$ where `if $s$ then $r$' or `if not $r$ then not $s$'