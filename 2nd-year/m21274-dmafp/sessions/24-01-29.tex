\lecture{Intro to Functional Programming II}{12:00}{22/01/24}{Matthew Poole}

\section*{Tracing a Functional Program}

In an imperative program, it is obvious that you trace the program by determining the effect of each statement on the
 overall state of the program. However, with a functional program, each step of the tracing process is evaluating an
 expression, known as calculation. For example, we can trace the following program
\begin{verbatim}
twiceSum :: Int -> Int -> Int
twiceSum x y = 2 * (x + y)

twiceSum 4 (2 + 6)
\end{verbatim}
by replacing each of the parameters of \verb_twiceSum_ as below
\begin{verbatim}
twiceSum 4 (2 + 6)
~> 2 * (4 + (2 + 6))
~> 2 * (4 + 8)
~> 2 * 12
~> 24
\end{verbatim}

As above, in Haskell, the arguments are passed into the function verbatim, so the first step of executing a function is
 usually evaluating the arguments. This means that Haskell uses Non-Strict Computation - The arguments are passed
 into the function before being evaluated. Some functional programming languages use Strict Computation, meaning that
 the arguments are evaluated before being passed into the function. It doesn't really make a difference, other than that
 you may be asked to use a specific method in questions on exams, etc.

\section*{Guards}

Guards are boolean expressions which can be used when defining a function to give different results, depending upon the
 input or a property thereof. This is especially useful for \textbf{Guarding} against invalid inputs. The syntax in
 Haskell is as below
\begin{verbatim}
maxVal :: Int -> Int -> Int
maxVal x y
  | x >= y    = x
  | otherwise = y
\end{verbatim}
If the first guard is true, then the corresponding result is returned. If it's false, the next guard is evaluated, and
 corresponding value returned. You can also create a ``default'' case which is used if none of the other guards are
 true, which uses the keyword \verb_otherwise_. Guards can also be used instead of a chain of if statements, which is
 easier to understand, and simpler to create in the first place.

\section*{Local Definitions}

If your function uses a very complex mathematical calculation, you may want to break the calculation into several steps.
 In Haskell, you can do this using Local Definitions. Using the \verb_where_ keyword, you can define what are
 effectively local variables. This is useful as you can break down complex calculations into multiple, less complex and
 easier to understand ones. For example, the following function,
\begin{verbatim}
distance :: Float -> Float -> Float -> Float -> Float
distance x1 y1 x2 y2 = sqrt ((x1 - x2)^2 + (y1 - y2)^2)
\end{verbatim}
could also be written as
\begin{verbatim}
distance x1 y1 x2 y2 = sqrt (dxSq + dySq)
  where
  dxSq = dx^2
  dySq = dy^2
  dx = x1 - x2
  dy = y1 - y2
\end{verbatim}
The order of local definitions is irrelevant, and will still work if you reference a definition before it is actually
 defined. Local definitions are only usable within the function they are defined, hence ``local'', but are able to
 reference each other, and the parameters of the function. As well as ``variables'', you can also define ``functions''
 as local definitions. This is useful to simplify repetitive code, which is used only within the function itself. Local
 definitions can also be used in conjunction with guards, in which case they are defined after the guards.