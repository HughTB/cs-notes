% Setup basic page geometry
\usepackage{geometry}

\geometry{
  a4paper,
  total={180mm, 257mm},
  left=15mm,
  top=20mm
}

% Command to insert the cover page for each module
\newcommand{\coverpage}[4]{
  \begin{titlepage}
    \pagecolor{myPurple}
    \color{white}
    \vspace*{35mm}

    \noindent \Huge{BSc (Hons) Computer Science}\\[1mm]
    \Large{University of Portsmouth}\\
    \Large{#1 Year}

    \vfill{}

    \begin{flushright}
      \noindent \Huge{\textbf{#2}}\\[2mm]
      \Large{#3}\\
      \Large{Semester #4}
    \end{flushright}

    \vfill{}

    \noindent \Huge{Hugh Baldwin}\\[1mm]
    \Large{hugh.baldwin@myport.ac.uk}

    \vspace*{35mm}
  \end{titlepage}
  \nopagecolor{}
}

% Setup hyperlink colours
\usepackage[]{hyperref}
\hypersetup{
  colorlinks=true,
  linkcolor=myPurple,
  urlcolor=myPurple
}

% Set headers and footers, as well as PDF metadata
\usepackage{lastpage}
\usepackage{fancyhdr}
\usepackage{hyperref}

\renewcommand{\chaptermark}[1]{ \markboth{#1}{} }
\renewcommand{\sectionmark}[1]{ \markright{#1}{} }

% Module-Code Module-Name
\newcommand{\setupdocument}[2]{
  % Override the styling on a plain page
  \fancypagestyle{plain}{
    \renewcommand{\headrulewidth}{0pt}

    \fancyfoot[L]{Hugh Baldwin \\ \small{\url{https://github.com/HughTB/cs-notes}}}
    \fancyfoot[C]{\textbf{\thepage} of \pageref*{LastPage}}
    \fancyfoot[R]{#1}

    \fancyhead[L]{\leftmark}
    \fancyhead[R]{}
  }

  \pagestyle{fancy}

  \renewcommand{\headrulewidth}{0pt}

  \fancyfoot[L]{Hugh Baldwin \\ \small{\url{https://github.com/HughTB/cs-notes}}}
  \fancyfoot[C]{\textbf{\thepage} of \pageref*{LastPage}}
  \fancyfoot[R]{#1}

  \fancyhead[L]{\leftmark}
  \fancyhead[R]{}

  % Fancyhdr complains without changing the header height
  \setlength{\headheight}{14.5pt}
  \addtolength{\topmargin}{-2.5pt}

  \hypersetup{
    pdftitle={#1 - #2 (BSc Computer Science)},
    pdfauthor={Hugh Baldwin}
  }
}

% Set the format of chapter headers using titlesec
\usepackage{titlesec}

\titleformat{\chapter}[display]
  {\normalfont\bfseries}{}{0pt}{\Huge}

% Command to create a chapter for each type of session
% Title Time Date Lecturer Session-Type
\newcommand{\session}[5]{
  \chapter{#5 - #1}

  \vspace{-45pt}

  \hrule

  \vspace{-10pt}

  \begin{center}
    \begin{tabular}{ p{0.3\linewidth} p{0.3\linewidth} p{0.3\linewidth} }
      \begin{flushleft}#2\end{flushleft} & \begin{center}#3\end{center} & \begin{flushright}#4\end{flushright} \\
    \end{tabular}
  \end{center}

  \vspace{-20pt}
}

\newcommand{\lecture}[4]{
  \session{#1}{#2}{#3}{#4}{Lecture}
}

\newcommand{\practical}[4]{
  \session{#1}{#2}{#3}{#4}{Practical}
}

\newcommand{\prereading}[1]{
  \session{#1}{}{}{}{Prereading}
}

% Use the xcolor package, and add my custom purple
\usepackage[dvipsnames]{xcolor}

\definecolor{myPurple}{RGB}{134, 53, 227}

% Use the amsmath package so normal text can be used within math blocks
\usepackage{amsmath}

% Use the amsfonts package for blackboard bold fonts
\usepackage{amsfonts}
\usepackage{amssymb}

% Use the mathspec package to set main and math fonts
\usepackage{mathspec}

\setmainfont[Ligatures=TeX]{IBM Plex Sans}
\setsansfont[Ligatures=TeX]{IBM Plex Sans}
\setmonofont[Ligatures=TeX]{IBM Plex Mono}
\setmathfont(Digits,Latin){IBM Plex Math Beta 240212}

% Use the tikz package for diagrams
\usepackage{tikz}
\usetikzlibrary{arrows}

\tikzset{
  vertex/.style={circle,thick,draw,inner sep=0pt,minimum size=6.5mm}
}

% Use the amsthm package for mathematical proofs
\usepackage{amsthm}
\renewcommand\qedsymbol{QED.}

% Use the float package for nicer graphs
\usepackage{float}

% Use the enumitem package for numbered lists not starting at 1
\usepackage{enumitem}