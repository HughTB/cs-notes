\lecture{A9: Turing Machines}{13:00}{07/11/24}{Janka Chlebikova}

The main difference between finite and pushdown automata is that PDA have an infinite stack of memory. This does make
 them more powerful than FA, but since they still can only move the reading head to the right at each step, there is a
 class of languages they cannot parse. For example, NDPAs can parse the language $\{{a^n}{b^n}, n \geq 0\}$, but not
 $\{{a^n}{b^n}{c^n}, n \geq 0\}$.

If the main differentiating factor is the type of storage available, then we can say an automaton with no storage is an
 FA, if the storage is a stack, it's a PDA, but what if we had 2 or 3 stacks? They would certainly be a more powerful
 form of automaton, and theoretically able to parse more complex languages.

\section*{Turing Machines}

Turing machines are the most powerful class of automaton, since they are essentially the same model of computing as
 modern computers. The main improvement over PDAs is that the memory is modelled as an infinite `tape' of data, which
 can be read from and written to, but all of it is accessible at once. The tape `head' can move left or right, or not
 change position on each read or write operation.

The tape is divided into cells, with an infinite number in each direction from the starting cell. Each cell contains
 either a symbol from the alphabet, or a blank symbol. Since this is based more in reality than previous automatons,
 the tape must have a finite number of non-blank symbols written on it. The head can only ever read or write to the
 current cell.

Based upon the current state and symbol in the current cell, the machine may
\begin{itemize}
  \item Change the state
  \item Move the head in either direction (or not at all)
  \item Rewrite the current symbol, or leave it unchanged
\end{itemize}

Again, since these are more based in reality, the `standard' type of Turing machine is deterministic, since it would
 have to be to exist in the physical world.

\subsection*{Instructions}

Each movement step is represented by a letter-- move to the left is $L$, move to the right is $R$, and stay still/ do
 nothing is $S$.

\begin{definition*}{}{}
  Each instruction for a Turing machine consists of five parts--
  \begin{itemize}
    \item The current state of the machine (over the set of all states, $Q$)
    \item The symbol read from the current cell of the tape (from $\Gamma$, or the blank symbol)
    \item A symbol to write to the tape (from $\Gamma$, or the blank symbol)
    \item The state to transition to (from $Q$)
    \item The direction for the tape head to move
  \end{itemize}

  Each instruction is of the form
  \begin{equation*}
    T : Q \times \Gamma \rightarrow \Gamma \times Q \times \{L, R, S\}
  \end{equation*}

  And the instruction can be written as an instruction such as
  \begin{equation*}
    (i, a, b, L, j)
  \end{equation*}

  Or in a diagram in the form
  \begin{equation*}
    \frac{a}{b, L}
  \end{equation*}
\end{definition*}

\subsection*{Values on the Tape}

An input string is represented by placing the letters from the string into adjacent cells on the tape. All other cells
 in the tape initially contain the blank symbol, which is denoted by $\square$. Typically, the head starts over the
 leftmost cell of the input string (the leftmost non-blank cell).

\subsection*{Starting and Stopping}

As with FAs and PDAs, there must be a single start state, which must be specified. In the case of a Turing machine,
 there is also usually only one final state, usually known as `Halt'. A Turing machine will halt when it either enters
 the halt state, or enters a state from which there is no valid move.

\subsection*{Recognising Languages}

A Turing machine $T$ recognises a string over $\Sigma$ if and only when
\begin{itemize}
  \item $T$ starts in the initial position, with the string $x$ written on the tape
  \item $T$ halts in a final state
\end{itemize}

$T$ is said to recognise the language $A$ if $x$ is recognised by $T$, if and only if $x$ belongs to $A$. A string is
 rejected by $T$ if it never halts, or it halts in a state which is not final.

\subsection*{Instantaneous Description}

To describe a Turing machine at any instant in time, we need to know what's on the tape, where the tape head is located,
 and what state the control automaton is in. We can represent this as follows:
\begin{equation*}
  \mathrm{State\ } i: \square a a \textbf{b} a b \square
\end{equation*}
where the current position of the tape head is in bold.

{\Huge Add example 2 here}



\lecture{A10: Computing with TMs \& Alternative Definitions}{13:00}{07/11/24}{Janka Chlebikova}

