\lecture{A5: Finite Automata \& Regular Languages}{13:00}{17/10/24}{Janka Chlebikova}

\section*{Automata and Regular Languages}

How can we prove that the set of languages accepted by automata is exactly the set of regular languages? Well it takes
 multiple steps.

\textbf{Step 1}-- We need to show that for any regular expression, we can find an NFA that recognises it, thus proving
\begin{equation*}
  L(\mathrm{regular\ expressions}) \subseteq L(\mathrm{NFA})
\end{equation*}
\textbf{Step 2}-- Then we take a finite automaton and find a regular expression which describes the language of the
 automaton, thus proving that
 \begin{equation*}
  L(\mathrm{NFA}) \subseteq L(\mathrm{regular\ expressions})
\end{equation*}
\textbf{Step 3}-- We then combine the previous results to get that
\begin{equation*}
  L(\mathrm{NFA}) = L(\mathrm{regular\ expressions})
\end{equation*}

\section*{Regular Expressions to Finite Automata}

Given any regular expression, we need to be able to construct a finite automaton (either NFA or DFA) which recognises
 its language. Given that all regular expressions are built up using union, product and closure operations, we can use
 a set of rules to construct an FA for a given regular expression.

\textbf{Rule 0}-- Start the algorithm with a simple FA that has a start state, a single accepting state and an edge
 labelled with the given regular expression.

\textbf{Rule 1}-- If an edge is labelled with $\emptyset$, erase the edge.

\textbf{Rule 2}-- Transform any edge of the form $R + S$ into two edges, labelled $R$ and $S$.

\textbf{Rule 3}-- Transform any edge of the form $R \cdot S$ into a new intermediary state, with an edge labelled $R$
 going into it from the previous state, and an edge labelled $S$ going into the next state.

\textbf{Rule 4}-- Transform any edge of the form $R^*$ into a new intermediary state with a looping edge labelled $R$,
 and two edges labelled $\Lambda$, one entering and one exiting the new state.

You must apply these rules until no more of the edges can be broken up any further using the rules.

\begin{example*}{}{}
  For the regular expression $a^* + ab$, we would follow the steps below--\\
  Start with rule 0--
  \begin{center}
    \begin{tikzpicture}
      \tikzset{
        ->, % makes the edges directed
        >=stealth, % makes the arrow heads bold
        node distance=2.5cm, % specifies the minimum distance between two nodes
        every state/.style={thick, fill=gray!10}, % sets the properties for each ’state’ node
        initial text=$ $, % sets the text that appears on the start arrow
      }

      \node[state, initial] (ss) {$s$};
      \node[state, right of=ss, accepting] (sa) {$f$};

      \draw (ss) edge[above] node{$a^* + ab$} (sa);
    \end{tikzpicture}
  \end{center}

  Apply rule 2--
  \begin{center}
    \begin{tikzpicture}
      \tikzset{
        ->, % makes the edges directed
        >=stealth, % makes the arrow heads bold
        node distance=2.5cm, % specifies the minimum distance between two nodes
        every state/.style={thick, fill=gray!10}, % sets the properties for each ’state’ node
        initial text=$ $, % sets the text that appears on the start arrow
      }

      \node[state, initial] (ss) {$s$};
      \node[state, right of=ss, accepting] (sa) {$f$};

      \draw (ss) edge[above, bend left=10] node{$a^*$} (sa)
            (ss) edge[below, bend right=10] node{$ab$} (sa);
    \end{tikzpicture}
  \end{center}

  Apply rule 4 to $a^*$--
  \begin{center}
    \begin{tikzpicture}
      \tikzset{
        ->, % makes the edges directed
        >=stealth, % makes the arrow heads bold
        node distance=2.5cm, % specifies the minimum distance between two nodes
        every state/.style={thick, fill=gray!10}, % sets the properties for each ’state’ node
        initial text=$ $, % sets the text that appears on the start arrow
      }

      \node[state, initial] (ss) {$s$};
      \node[state, right of=ss] (s1) {$1$};
      \node[state, right of=s1, accepting] (sa) {$f$};

      \draw (ss) edge[above] node{$\Lambda$} (s1)
            (ss) edge[below, bend right=30] node{$ab$} (sa)
            (s1) edge[loop above] node{$a$} (s1)
            (s1) edge[above] node{$\Lambda$} (sa);
    \end{tikzpicture}
  \end{center}

  Apply rule 3 to $ab$--
  \begin{center}
    \begin{tikzpicture}
      \tikzset{
        ->, % makes the edges directed
        >=stealth, % makes the arrow heads bold
        node distance=2.5cm, % specifies the minimum distance between two nodes
        every state/.style={thick, fill=gray!10}, % sets the properties for each ’state’ node
        initial text=$ $, % sets the text that appears on the start arrow
      }

      \node[state, initial] (ss) {$s$};
      \node[state, right of=ss] (s1) {$1$};
      \node[state, below of=s1] (s2) {$2$};
      \node[state, right of=s1, accepting] (sa) {$f$};

      \draw (ss) edge[above] node{$\Lambda$} (s1)
            (ss) edge[below] node{$a$} (s2)
            (s1) edge[loop above] node{$a$} (s1)
            (s1) edge[above] node{$\Lambda$} (sa)
            (s2) edge[below] node{$b$} (sa);
    \end{tikzpicture}
  \end{center}
\end{example*}

Thus, we have proven that we can produce a finite automata for any regular expression, and given a set of rules to do so.

\section*{Finite Automata to Regular Expressions}

Now, to do the same thing in reverse, we need a new algorithm, as follows--

\textbf{Step 1}-- Create a new start state $s$, and draw an edge labelled $\Lambda$ to the original start state.

\textbf{Step 2}-- Create a new accepting state $f$, and draw edges labelled $\Lambda$ from all original accepting states.

\textbf{Step 3}-- For each pair of states $i$ and $j$ with more than one edge from $i$ to $j$, replace all edges with a
 single edge labelled with the regular expression formed by the sum of labels on each edge from $i$ to $j$.

\textbf{Step 4}-- Step-by-step eliminate any states, changing the labels on corresponding edges until the only remaining
 states are $s$ and $f$. When deleting a state, you must replace any possible transitions with a regular expression
 which matches all of the possible transitions.

You should end up with an FA with only two states, and a single edge connecting them which is labelled with the desired
 regular expression.

\begin{example*}{}{}
  Initial DFA--
  \begin{center}
    \begin{tikzpicture}
      \tikzset{
        ->, % makes the edges directed
        >=stealth, % makes the arrow heads bold
        node distance=2.5cm, % specifies the minimum distance between two nodes
        every state/.style={thick, fill=gray!10}, % sets the properties for each ’state’ node
        initial text=$ $, % sets the text that appears on the start arrow
      }

      \node[state, initial] (s0) {$0$};
      \node[state, right of=s0, accepting] (s1) {$1$};
      \node[state, below of=s1] (s2) {$2$};

      \draw (s0) edge[above] node{$a$} (s1)
            (s0) edge[below] node{$b$} (s2)
            (s1) edge[loop above] node{$a,b$} (s1)
            (s2) edge[loop below] node{$a,b$} (s2);
    \end{tikzpicture}
  \end{center}

  Steps 1 and 2: add new initial and final states--
  \begin{center}
    \begin{tikzpicture}
      \tikzset{
        ->, % makes the edges directed
        >=stealth, % makes the arrow heads bold
        node distance=2.5cm, % specifies the minimum distance between two nodes
        every state/.style={thick, fill=gray!10}, % sets the properties for each ’state’ node
        initial text=$ $, % sets the text that appears on the start arrow
      }

      \node[state, left of=s0, initial] (ss) {$s$};
      \node[state] (s0) {$0$};
      \node[state, right of=s0] (s1) {$1$};
      \node[state, below of=s1] (s2) {$2$};
      \node[state, right of=s1, accepting] (sf) {$f$};

      \draw (ss) edge[above] node{$\Lambda$} (s0)
            (s0) edge[above] node{$a$} (s1)
            (s0) edge[below] node{$b$} (s2)
            (s1) edge[loop above] node{$a,b$} (s1)
            (s1) edge[above] node{$\Lambda$} (sf)
            (s2) edge[loop below] node{$a,b$} (s2);
    \end{tikzpicture}
  \end{center}

  Apply step 3--
  \begin{center}
    \begin{tikzpicture}
      \tikzset{
        ->, % makes the edges directed
        >=stealth, % makes the arrow heads bold
        node distance=2.5cm, % specifies the minimum distance between two nodes
        every state/.style={thick, fill=gray!10}, % sets the properties for each ’state’ node
        initial text=$ $, % sets the text that appears on the start arrow
      }

      \node[state, left of=s0, initial] (ss) {$s$};
      \node[state] (s0) {$0$};
      \node[state, right of=s0] (s1) {$1$};
      \node[state, below of=s1] (s2) {$2$};
      \node[state, right of=s1, accepting] (sf) {$f$};

      \draw (ss) edge[above] node{$\Lambda$} (s0)
            (s0) edge[above] node{$a$} (s1)
            (s0) edge[below] node{$b$} (s2)
            (s1) edge[loop above] node{$a+b$} (s1)
            (s1) edge[above] node{$\Lambda$} (sf)
            (s2) edge[loop below] node{$a+b$} (s2);
    \end{tikzpicture}
  \end{center}
\end{example*}

\begin{example*}{}{}
  We can then eliminate state 2, since there is no way for a string to be accepted from it--
  \begin{center}
    \begin{tikzpicture}
      \tikzset{
        ->, % makes the edges directed
        >=stealth, % makes the arrow heads bold
        node distance=2.5cm, % specifies the minimum distance between two nodes
        every state/.style={thick, fill=gray!10}, % sets the properties for each ’state’ node
        initial text=$ $, % sets the text that appears on the start arrow
      }

      \node[state, left of=s0, initial] (ss) {$s$};
      \node[state] (s0) {$0$};
      \node[state, right of=s0] (s1) {$1$};
      \node[state, right of=s1, accepting] (sf) {$f$};

      \draw (ss) edge[above] node{$\Lambda$} (s0)
            (s0) edge[above] node{$a$} (s1)
            (s1) edge[loop above] node{$a+b$} (s1)
            (s1) edge[above] node{$\Lambda$} (sf);
    \end{tikzpicture}
  \end{center}

  Then we can start to eliminate states to get our final regular expression--
  \begin{center}
    \begin{tikzpicture}
      \tikzset{
        ->, % makes the edges directed
        >=stealth, % makes the arrow heads bold
        node distance=2.5cm, % specifies the minimum distance between two nodes
        every state/.style={thick, fill=gray!10}, % sets the properties for each ’state’ node
        initial text=$ $, % sets the text that appears on the start arrow
      }

      \node[state, initial] (ss) {$s$};
      \node[state, right of=ss] (s1) {$1$};
      \node[state, right of=s1, accepting] (sf) {$f$};

      \draw (ss) edge[above] node{$a$} (s1)
            (s1) edge[loop above] node{$a+b$} (s1)
            (s1) edge[above] node{$\Lambda$} (sf);
    \end{tikzpicture}\\
    \begin{tikzpicture}
      \tikzset{
        ->, % makes the edges directed
        >=stealth, % makes the arrow heads bold
        node distance=2.5cm, % specifies the minimum distance between two nodes
        every state/.style={thick, fill=gray!10}, % sets the properties for each ’state’ node
        initial text=$ $, % sets the text that appears on the start arrow
      }

      \node[state, initial] (ss) {$s$};
      \node[state, right of=ss, accepting] (sf) {$f$};

      \draw (ss) edge[above] node{$a{(a+b)}^*$} (sf);
    \end{tikzpicture}
  \end{center}

  And this then leaves us with the final regular expression $a{(a+b)}^*$.
\end{example*}

And now, we have proven that we can transform any regular expression into an NFA, and vice versa, and that we can
 transform any NFA into a DFA, and vice versa.

\begin{equation*}
  \mathrm{regular\ expressions} \Leftrightarrow \mathrm{NFA} \Leftrightarrow \mathrm{DFA}
\end{equation*}

\section*{Simplifying DFAs}

A lot of the time, we will end up with a complicated DFA, with far more states than is absolutely necessary. To get a
 simplified DFA, we need to transform it into a unique DFA with the minimum number of states which recognises the same
 language.

We can find this DFA in two parts-- Finding all pairs of equivalent states, and combining the equivalent states into a
 single state, with modified transition functions.

\subsection*{Equivalent States}

We can say that two states $s$ and $t$ are equivalent if for all possible strings $w$ left to consume (including $\Lambda$),
 the DFA will finish in the same \textbf{type} of state (i.e.\ accepting or non-accepting) after consuming $w$. Therefore,
 once you arrive at $s$ or $t$, the same string will always produce the same result (accepting or rejecting the string).

Two states are not equivalent if there exists a string $w$ such that, after consuming $w$, we would end up accepting the
 string if we started from $s$ and rejecting it if we started from $t$, or vice versa.



\lecture{A6: Beyond Regular Languages}{13:00}{17/10/24}{Janka Chlebikova}

\section*{Non-deterministic Finite Automata to Regular Grammars}

Every NFA can be converted into a corresponding regular grammar. Each state of the NFA is associated with a non-terminal
 symbol of the grammar, and the initial state is associated with the start symbol. Each transition is associated with
 a grammar production, and every final state has the additional production which produces $\Lambda$.

\begin{example*}{}{}
  For the NFA--
  \begin{center}
    \begin{tikzpicture}
      \tikzset{
        ->, % makes the edges directed
        >=stealth, % makes the arrow heads bold
        node distance=2cm, % specifies the minimum distance between two nodes
        every state/.style={thick, fill=gray!10}, % sets the properties for each ’state’ node
        initial text=$ $, % sets the text that appears on the start arrow
      }

      \node[state, initial] (s0) {$0$};
      \node[state, right of=s0] (s1) {$1$};
      \node[state, right of=s1, accepting] (s2) {$2$};
      \node[state, below of=s1, accepting] (s3) {$3$};

      \draw (s0) edge[above] node{$a$} (s1)
            (s0) edge[below] node{$b$} (s3)
            (s1) edge[above] node{$a$} (s2)
            (s3) edge[loop below] node{$a$} (s3);
    \end{tikzpicture}
  \end{center}

  --we can apply the rules defined above to get the following productions, supposing that $S=0$, $A=1$, $B=2$ and $C=3$
  \begin{align*}
    S &\rightarrow aA \mid aC\\
    A &\rightarrow bB\\
    B &\rightarrow \Lambda\\
    C &\rightarrow aC \mid \Lambda
  \end{align*}

  This may not be the simplest set of productions, but it does work.
\end{example*}

\section*{Testing for Regular Languages}

Given a language, we need to be able to determine if it is regular. This would also allow us to prove that a given
 language cannot be recognised by any finite automaton.

One possible method is by using the \textbf{pumping lemma}, which applies for infinite languages. Since all finite
 languages are regular, this is enough to determine if a language is regular or not.

If the input string is long enough (e.g.\ greater than the number of states in the minimum state DFA for the language),
 then there must be at least one state $Q$ which is visited more than once. Therefore, there must be at least one closed
 loop, which begins and ends at the state $Q$, and a particular string $y$, which corresponds to this loop.

We can represent this situation as the FA--
\begin{center}
  \begin{tikzpicture}
    \tikzset{
      ->, % makes the edges directed
      >=stealth, % makes the arrow heads bold
      node distance=2.5cm, % specifies the minimum distance between two nodes
      every state/.style={thick, fill=gray!10}, % sets the properties for each ’state’ node
      initial text=$ $, % sets the text that appears on the start arrow
    }

    \node[state, initial] (ss) {};
    \node[state, right of=ss] (sq) {$Q$};
    \node[state, right of=sq] (sa) {};

    \draw (ss) edge[above, dotted] node{$x$} (sq)
          (sq) edge[above, dotted] node{$z$} (sa)
          (sq) edge[loop above, dotted] node{$y$} (sq);
  \end{tikzpicture}
\end{center}
Where each dotted edge represents a path that may contain other states and transitions of the DFA.
\begin{itemize}
  \item $x$ is a string of symbols which the automaton consumes to transition from the start state, to get to the state
   $Q$
  \item $y$ is the string of symbols to transition around the closed loop
  \item $z$ is the string of symbols to transition from $Q$ to an accepting state
\end{itemize}

Thus, we know that the string $xyz$ is accepted, but also that the DFA must accept the strings
 $xy, xyz, xyyz, \ldots, x{y^k}z$, and that the central string $y$ is `pumped'.

Formally, the pumping lemma is defined as follows
\begin{definition*}{}{}
  Let $L$ be an infinite regular language accepted by a DFA with $m$ states. Then any string $w$ in $L$ with at least $m$
   symbols can be decomposed as $w = xyz$ with $|xy| \leq m$ and $|y| \geq 1$ such that
  \begin{equation*}
    w_i = x{y^i}z
  \end{equation*}
  is also in $L$ for all $i = 0, 1, 2, 3, \ldots$
\end{definition*}

This is necessary, but not sufficient for a language to be regular. If a language does not satisfy the pumping lemma,
 then it cannot be regular. But a language that satisfies the pumping lemma may not be regular. Therefore, you can use
 the lemma to prove that a language is \textbf{not} regular, but not that it is regular.