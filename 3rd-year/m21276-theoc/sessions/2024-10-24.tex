\lecture{A7: Pushdown Automata}{13:00}{24/10/24}{Janka Chlebikova}

There are many context-free languages that cannot be recognised by a finite automaton, especially non-regular languages.
 This is because they typically require an infinite memory to be recognised, such as the language $\{a^n b^n | n \geq 0\}$.
 A more powerful model of computing is known as Non-deterministic Pushdown Automata, or NDPA.

\section*{Non-Deterministic Pushdown Automata}

An NDPA is like a finite automata, except that they also have memory in the form of a stack, where they can store an
 infinite amount of information. They are made up of the input tape, an infinite sequence of symbols from the input
 alphabet, a finite control automaton which is similar to an NFA, and the memory. Since the memory is modelled as a
 stack, it works in a Last In, First Out way, and there are 3 stack operations which can be performed at each step.

\begin{itemize}
  \item pop-- Reads the top symbol and removes it from the stack
  \item push-- Writes a new symbol on to the top of the stack
  \item nop-- No operation, has no effect on the stack
\end{itemize}

The symbols contained in the stack are different to the alphabet of the language the NDPA recognises. The initial state
 of the automaton is that the stack only contains the initial stack symbol, and that the control automaton is in it's
 initial state.

\section*{Transition Functions}

At each step, the current state, input symbol and symbol at the top of the stack determine the transition to the next
 state. Each transition must change the state, may read a symbol from and advance the input tape, and change the stack
 using one of the above stack operations.

Each transition function has three inputs
\begin{itemize}
  \item The current state, $p$
  \item The input character, $a$ or $\Lambda$
  \item The stack character, $A$
\end{itemize}
and two outputs
\begin{itemize}
  \item The new state
  \item A stack operation
\end{itemize}

\section*{NDPA Formally}

\begin{definition*}{}{}
  An NDPA can be formally described by
  \begin{itemize}
    \item A finite set $Q$ of states, the initial state and the set of accepting states
    \item A finite set $\Sigma$ of symbols making up the input alphabet
    \item A finite set $\Gamma$ of symbols making up the stack alphabet, and the initial stack symbol, often $\$$
    \item A finite set of transition instructions, or a transition function $T$ of the form 
     $T : Q \times \Sigma \cup \{\Lambda\} \times \Gamma \rightarrow \Gamma^* \times Q$
  \end{itemize}
\end{definition*}

\begin{example*}{}{}
  The transition
  \begin{center}
    \begin{tikzpicture}
      \tikzset{
        ->, % makes the edges directed
        >=stealth, % makes the arrow heads bold
        node distance=4cm, % specifies the minimum distance between two nodes
        every state/.style={thick, fill=gray!10}, % sets the properties for each ’state’ node
        initial text=$ $, % sets the text that appears on the start arrow
      }

      \node[state] (sp) {$p$};
      \node[state, right of=sp] (sq) {$q$};

      \draw (sp) edge[above] node{$a, A\ |\ \mathrm{push}(B)$} (sq);
    \end{tikzpicture}
  \end{center}
  can also be written as the function
  \begin{equation*}
    T(p, a, A) = (\mathrm{push}(B), q)
  \end{equation*}
  or as the transition instruction
  \begin{equation*}
    (p, a, A, \mathrm{push}(B), q)
  \end{equation*}
\end{example*}

\section*{Accepted Languages}

A string is accepted by an NDPA if there is some path from the initial state to an accepting state which consumes the
 entire input string. Otherwise, the string is rejected by the NPDA. The language is then the set of all strings that
 it accepts.

Specifically, an input string is rejected if it finishes reading the string before reaching an accepting state, if the
 current state has no transition for the current input or stack symbol, or finally if it attempts to pop an empty stack.

\begin{example*}{}{}
  If we wanted to build an NDPA that recognises the language $\{a^n b^n | n \geq 0\}$, we could use the following plan.
  \begin{itemize}
    \item Read the string, and for each $a$ that's read, push $Y$ onto the stack
    \item Once the first $b$ is read, change state and start removing one $Y$ from the stack for each $b$ read
    \item If the input string ends and the stack has just emptied, the string is accepted
    \item Otherwise, the string should be rejected
  \end{itemize}

  We can create this NDPA as the automaton with
  \begin{itemize}
    \item $Q = \{0, 1, 2\}$ where 0 is the initial state, and 2 accepting
    \item $\Sigma = \{a, b\}$
    \item $\Gamma = \{Y, \$\}$
  \end{itemize}
  and drawn as
  \begin{center}
    \begin{tikzpicture}
      \tikzset{
        ->, % makes the edges directed
        >=stealth, % makes the arrow heads bold
        node distance=4cm, % specifies the minimum distance between two nodes
        every state/.style={thick, fill=gray!10}, % sets the properties for each ’state’ node
        initial text=$ $, % sets the text that appears on the start arrow
      }

      \node[state, initial] (s0) {$0$};
      \node[state, right of=s0] (s1) {$1$};
      \node[state, right of=s1, accepting] (s2) {$2$};

      \draw (s0) edge[loop above] node{$a, Y\ |\ \mathrm{push}(Y); a, \$\ |\ \mathrm{push}(Y)$} (s0)
            (s0) edge[above] node{$b, Y\ |\ \mathrm{pop}$} (s1)
            (s0) edge[bend right=15, below] node{$\Lambda, \$\ |\ \mathrm{nop}$} (s2)
            (s1) edge[loop above] node{$b, Y\ |\ \mathrm{pop}$} (s1)
            (s1) edge[above] node{$\Lambda, \$\ |\ \mathrm{nop}$} (s2);
    \end{tikzpicture}
  \end{center}
\end{example*}

{\Huge instantaneous description}

\section*{NDPAs and Context-Free Languages}

The class of all languages accepted by NDPAs is exactly the same as the class of context-free languages.

{\Huge Was on slide 17}