\lecture{B3: Diagonalisation and the Halting Problem}{13:00}{28/11/24}{Janka Chlebikova}

There are limits to computation, and not all problems are algorithmically solvable. This also supports the Church-Turing
 thesis, since it shows that if a problem cannot be solved by a turing machine, it likely cannot be solved by any form
 of computer.

These problems which cannot be solved by a computer are known as `undecidable problems'.

\section*{Diagonalisation}

The initial proofs for undecidability used a technique known as self-reference or diagonalisation. Any computational
 model that is powerful enough to allow self-reference will cause problems. To demonstrate this, we will use the Barber
 paradox.

\subsection*{Barber Paradox}

Suppose there is a small town in which
\begin{itemize}
  \item There is a single barber
  \item Every man is clean-shaven, either by shaving themselves or by going to the barber
  \item The barber obeys the rule `Shave all and only those men in town who do not shave themselves'
\end{itemize}
This then presents a problem if we need to know if the barber shaves himself.
\begin{itemize}
  \item If the barber shaves himself, then according to the rule he does not shave himself
  \item If the barber does not shave himself, then according to the rule he does shave himself
\end{itemize}
There is no solution to this problem, and so it is a paradox.

\subsection*{Countable Sets}

How can we prove that two sets are the same size, without counting the elements? If the sets are both finite, we can
 pair one item from each set, and if we can pair each item then they are the same size. If a set is infinite, does the
 set have the same size as $\mathbb{N}$, or is it larger?

\begin{example*}{}{}
  The set of all even numbers is the same size as $\mathbb{N}$, since they can be paired as
  \begin{gather*}
    1 - 2\\
    2 - 4\\
    3 - 6\\
    4 - 8\\
    \ldots
  \end{gather*}
\end{example*}

A set is known as \textbf{countably infinite} if it is the same size as $\mathbb{N}$. A set is countable if it is finite
 or countably infinite, meaning that you can construct a numbered list of all it's elements. The set of integers, the
 set of odd numbers and the set of rational numbers are all countable.

However, the size of the power set of $\mathbb{N}$ is not countable, which we can prove using a method known as
 diagonalisation.

\begin{example*}{}{}
  To prove that the set $P(\mathbb{N})$ is not countable, we can use proof by contradiction.

  Suppose that the set is countable, so we can write down a list of all the subsets of $\mathbb{N}$. It might look like
  \begin{equation*}
    1 - \{2, 3\}, 2 - \{4, 7\}, 3 - \{2, 4, 6, 8\}, \ldots
  \end{equation*}

  So, we have a function $f : \mathbb{N} \rightarrow P(\mathbb{N})$ that maps the numbers to sets such that every set
   appears in the list.

  If we now define a set $T$ such as: add $i$ to the set $T$ when $i \notin f(i)$, e.g. $1 \in T$ because
   $1 \notin f(1)$, $3 \in T$ because $3 \notin f(3)$, etc.

  $T$ cannot be in $P(\mathbb{N})$ because it is different from every set, but $T$ is a subset of $\mathbb{N}$. This is
   a contradiction, and so $P(\mathbb{N})$ is not countable.
\end{example*}

\section*{Decision Problems}

A decision problem is any which asks a question with a yes or no answer. {Examples}

A decision problem can be viewed as a language where the members of the language are instances whose answer is yes, and
 non-members are instances whose output is no. A decision problem is decidable if there is an algorithm which for every
 input instance of the problem, halts with a correct answer, either yes or no. If no such algorithm exists, then the
 problem is undecidable.

Decidable problems correspond to recursive languages. They can be recognised by Turing machines which halt for every
 input.

\subsection*{Partial Decision Problems}

\begin{definition*}{}{}
  An undecidable problem is partially decidable if there is an algorithm which
  \begin{itemize}
    \item Halts with the answer yes for instances that have the answer yes, but
    \item May run forever for instances which have the answer no
  \end{itemize}
  (Or the other way around in some cases)
\end{definition*}

Partial decidable problems correspond to the recursive enumerable languages, which can be recognised by Turing machines.

\section*{Existence of an undecidable problem}

How many Turing machines exist? If we let $S$ be a countable set of symbols, then any TM can be coded as a finite string
 of symbols over $S$, with all transition functions one after the other. This means there are a countable number of
 finite strings over $S$, and therefore a function which maps a unique number to each TM. This means that the set of all
 TMs is countable.

How many languages exist? A language is a subset of a countable set of strings. We have shown that the size of the set
 of all subsets of $\mathbb{N}$ is not countable. This means that the set of language is uncountable!

So, we've shown that the set of all TMs is countable, and sot the set of languages accepted by TMs is also countable,
 but that the set of all languages is uncountable. Therefore, there is a language which cannot be recognised by any
 TM, and therefore is an undecidable problem.

The Halting problem is famous because it was one of the first problems to be proven to be algorithmically undecidable.



\lecture{B4: Undecidable Problems}{13:00}{28/11/24}{Janka Chlebikova}

\section*{The Halting Problem}

Algorithms may contain loops which may be finite or infinite. The amount of work done by an algorithm usually depends
 on the input data.

The halting problem asks the question `Is there an algorithm which can decide whether the execution of an arbitrary
 program halts on an arbitrary input?'

You might initially think that you can simply run the program with the given input. This works if the program stops,
 since it then clearly halted. But if the program doesn't stop within a reasonable length of time, we have no way to
 know if it will halt at some point, and if we've not waited long enough, or if it will simply run forever.

The halting problem is undecidable, or more accurately it is partially decidable, since it will always eventually halt
 if the answer is yes and we just run the program. Therefore, there is no algorithm which could solve it.

\subsection*{Acceptance Formulation}

Define the set $A = \{<M, w> : M \mathrm{\ is\ a\ TM\ that\ accepts\ } w$, where $<M, w>$ is a unique coding. This is
 another formulation as the halting problem; is there a TM which will recognise the set $A$? No.

This proof is once again done by contradiction.

Suppose there is a machine \textbf{Solver} for which on every input $w$ and every TM $M$, would tell us if $M$ accepts
 $w$. If we build another TM \textbf{Opposer} which does the following
\begin{itemize}
  \item Take the input $w$ and determines the TM $<w>$ which $w$ encodes
  \item Ask Solver for the answer, e.g. `does the TM $<w>$ accept $w$'
  \item If Solver accepts, reject
  \item If Solver rejects, accept
\end{itemize}
Opposer must be a valid TM, since Solver always halts.

But what if the input to Opposer is the encoding of Opposer itself?
\begin{itemize}
  \item Opposer asks Solver for an answer for itself
  \item If Solver claims that Opposer accepts, then Opposer rejects
  \item If Solver claims that Opposer rejects, then Opposer accepts
\end{itemize}

This is a paradox, assuming that Solver exists, and so Solver cannot exist!