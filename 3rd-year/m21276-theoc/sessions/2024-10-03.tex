\lecture{A1: Introduction to Languages}{13:00}{03/10/24}{Janka Chlebikova}

\section*{Languages}

In this context, a \textbf{language} is a set of symbols which can be combined to create a list of acceptable \textbf{strings}.
 There are then also rules which tell us how to combine these strings together, known as \textbf{grammars}. The 
 combination of an alphabet, list of valid strings and grammars is known as a language. There will be a formal definition
 later on.

\begin{definition*}{}{}
  An \textbf{Alphabet} is a finite, non-empty set of symbols.
\end{definition*}

For example, in the English language we would define the alphabet, $A$, as $A = \{a, b, c, d, \ldots, x, y, z\}$. These
 symbols can then be combined to create a \textbf{string}.

\begin{definition*}{}{}
  A \textbf{String} is a finite sequence of symbols from the alphabet of a language.
\end{definition*}

With the alphabet $A$, we could have strings such as `$cat$', `$dog$', `$antidisestablishmentarianism$', etc \textit{over} the
 alphabet. A string with no symbols, and therefore a length of zero, is known as the \textbf{empty string}, and is
 denoted by $\Lambda$ (capital lambda).

\begin{definition*}{}{}
  A \textbf{Language} over an alphabet (e.g. English over $A$) is a set of strings -- including $\Lambda$ -- made up of
   symbols from $A$ which are considered `valid'. They could be valid as per a set of rules, or could be arbitrary as
   in most spoken languages.
\end{definition*}

If we have an alphabet, $\Sigma$, then $\Sigma^*$ denotes the infinite set of all strings made up of symbols in $\Sigma$
 -- including $\Lambda$. Therefore, a language over $\Sigma$ is any subset of $\Sigma^*$.

\begin{example*}{}{}
  If $\Sigma = \{a, b\}$, then $\Sigma^* = \{\Lambda, a, b, aa, ab, ba, bb, \ldots\}$\\
  There are many languages which could be defined over this alphabet, but a few simple one are
  \begin{itemize}
    \item $\emptyset$ (The empty set)
    \item $\{\Lambda\}$ (The set containing an empty string)
    \item $\{a\}$ (The set containing $a$)
    \item The infinite set $\Sigma^* = \{\Lambda, a, aa, aaa, \ldots\}$
  \end{itemize}
\end{example*}

\section*{Combining Languages}

There are three common ways of combining two languages to create a new language -- union, intersection and product.

\subsection*{Union and Intersection}

Since languages are sets of strings, they can be combined as you usually would for any other set with the usual operations,
 union and intersection.

\begin{example*}{}{}
  If $L = \{aa, bb, cc\}$ and $M = \{cc, dd, ee\}$, then\\
  $L \cap M = \{cc\}$, and\\
  $L \cup M = \{aa, bb, cc, dd, ee\}$
\end{example*}

\subsection*{Product}

To combine two languages $L$ and $M$, we form the set of all \textbf{concatenations} of strings in $L$ with strings In
 $M$. This is known as the \textbf{product} of the two languages.

\begin{definition*}{}{}
  To \textbf{Concatenate} two strings is to juxtapose them such that a new string is made by appending the second string
   to the end of the first.
\end{definition*}

\begin{example*}{}{}
  If we concatenate the strings $ab$ and $ba$, the result would be $abba$.\\
  This can be represented using the function $\mathrm{cat}$, such as $\mathrm{cat}(ab, ba) = abba$
\end{example*}

With this definition of concatenation, we can say that the product of two languages, $L$ and $M$ would be $L \cdot M$,
 where $L \cdot M = \{\mathrm{cat}(s, t) : s \in L\ \mathrm{and}\ t \in M\}$

\begin{example*}{}{}
  If $L = \{a, b, c\}$ and $M = \{c, d, e\}$, then the product of the two languages, $L \cdot M$, is the language\\
  $L \cdot M = \{ac, ad, ae, bc, bd, be, cc, cd, ce\}$
\end{example*}

\section*{Further on Products}

It is simple to see that for any language, $L$, the following simple properties are true:
\begin{gather*}
  L \cdot \{\Lambda\} = \{\Lambda\} \cdot L = L\\
  L \cdot \emptyset = \emptyset \cdot L = \emptyset
\end{gather*}

\subsection*{Commutativity}

Aside from the above properties, the product of any two languages is \textbf{not} commutative, and therefore
\begin{equation*}
  L \cdot M \neq M \cdot L
\end{equation*}

\begin{example*}{}{}
  If $L = \{a, b\}$ and $M = \{b, c\}$, then the product $L \cdot M$ is the language
  \begin{equation*}
    L \cdot M = \{ab, ac, bb, bc\}
  \end{equation*}
  but the product $M \cdot L$ is the language
  \begin{equation*}
    M \cdot L = \{ba, bb, ca, cb\}
  \end{equation*}
  which are clearly not the same language, as the only common string is $bb$
\end{example*}

\subsection*{Associativity}

However, the product of any two languages is associative. This means that if we had any three languages, $L$, $M$ and
 $N$, then
\begin{equation*}
  L \cdot (M \cdot N) = (L \cdot M) \cdot N
\end{equation*}

\begin{example*}{}{}
  If we have the three languages $L = \{a, b\}$, $M = \{b, c\}$ and $N = \{c, d\}$, then
  \begin{align*}
    L \cdot (M \cdot N) &= L \cdot \{bc, bd, cc, cd\}\\
    &= \{abc, abd, acc, acd, bbc, bbd, bcc, bcd\}
  \end{align*}
  which is the same as
  \begin{align*}
    (L \cdot M) \cdot N &= \{ab, ac, bb, bc\} \cdot N\\
    &= \{abc, abd, acc, acd, bbc, bbd, bcc, bcd\}
  \end{align*}
\end{example*}

\section*{Powers of Languages}

If we have the language $L$, then the product $L \cdot L$ of the language is denoted by $L^2$. The product $L^n$ for
 every $n \in \mathbb{N}$ is defined as
\begin{align*}
  L^0 &= \{\Lambda\}\\
  L^n &= L \cdot L^{n-1}, \mathrm{if} n > 0
\end{align*}

\begin{example*}{}{}
  If $L = \{a, b\}$ then the first few powers of L are
  \begin{align*}
    L^0 &= \{\Lambda\}\\
    L^1 &= L = \{a, b\}\\
    L^2 &= L \cdot L = \{aa, ab, ba, bb\}\\
    L^3 &= L \cdot L^2 = \{aaa, aab, aba, abb, baa, bab, bba, bbb\}
  \end{align*}
\end{example*}

\section*{The Closure of a Language}

If $L$ is a language over $\Sigma$, then the \textbf{closure} of $L$ is the language denoted by $L^*$, and the
 \textbf{positive closure} is language denoted by $L^+$.

\begin{definition*}{}{}
  The \textbf{Closure} of the language $L$, $L^*$ is defined as
  \begin{equation*}
    L^* = L^0 \cup L^1 \cup L^2 \cup L^3 \cup \ldots
  \end{equation*}
  and so, if $L = \{a\}$ then
  \begin{align*}
    L^* &= \{\Lambda\} \cup \{a\} \cup \{aa\} \cup \{aaa\} \cup \ldots\\
    &= \{\Lambda, a, aa, aaa, \ldots\}
  \end{align*}
\end{definition*}

\begin{definition*}{}{}
  The \textbf{Positive Closure} the language $L$, $L^+$ is defined as
  \begin{equation*}
    L^+ = L^1 \cup L^2 \cup L^3 \cup L^4 \cup \ldots
  \end{equation*}
  and so, if $L = \{a\}$ then
  \begin{align*}
    L^+ &= \{a\} \cup \{aa\} \cup \{aaa\} \cup \{aaaa\} \cup \ldots\\
    &= \{a, aa, aaa, aaaa, \ldots\}
  \end{align*}
\end{definition*}

It then follows that $L^* = L^+ \cup \{\Lambda\}$, but it's not necessarily true that $L^+ = L^* - \{\Lambda\}$.

\begin{example*}{}{}
  If our alphabet is $\Sigma = \{a\}$ and our language is $L = \{\Lambda, a\}$, then
  \begin{equation*}
    L^+ = L^*
  \end{equation*}
\end{example*}

\subsection*{Properties of Closures}

Based upon these definitions, you can derive some interesting properties of closures.

\begin{example*}{}{}
  If $L$ and $M$ are languages over the alphabet $\Sigma$, then
  \begin{gather*}
    \{\Lambda\}^* = \emptyset^* = \{\Lambda\}\\
    L^* = L^* \cdot L^* = (L^*)^*\\
    \Lambda \in L \mathrm{\ if\ and\ only\ if\ } L^+ = L^*\\
    (L^* \cdot M^*)^* = (L^* \cup M^*)^* = (L \cup M)^*\\
    L \cdot (M \cdot L)^* = (L \cdot M)^* \cdot L
  \end{gather*}

  \textit{These will be explored more during the tutorial session for this week}
\end{example*}

\section*{The Closure of an Alphabet}

Going back to the definition of $\Sigma^*$ of the alphabet $\Sigma$, it lines up perfectly with the definition of a
 closure such that $\Sigma^*$ is the set of all strings over $\Sigma$. This means that there is a nice way to represent
 $\Sigma^*$ as follows
\begin{equation*}
  \Sigma^* = \Sigma^0 \cup \Sigma^1 \cup \Sigma^2 \cup \Sigma^3 \cup \ldots
\end{equation*}
From this, we can also see that $\Sigma^k$ denotes the set of all strings over $\Sigma$ whose length is $k$.