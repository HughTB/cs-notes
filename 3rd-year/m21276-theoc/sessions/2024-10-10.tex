\lecture{A3: Regular Languages}{13:00}{10/10/24}{Janka Chlebikova}

A regular language is a simple sort of language that meets a few basic requirements. They are often used for pattern
 matching and in lexical analysers.

\begin{definition*}{}{}
  There are several ways of describing a regular language
  \begin{itemize}
    \item Languages inductively formed by combining simple languages
    \item Languages described by a regular expression
    \item Languages produced by a grammar with a special and restricted form
    \item Languages that can be accepted by a finite automata
  \end{itemize}
\end{definition*}

\section*{Inductively Formed Regular Languages}

We start with a very simple language or set thereof and build more complex ones by combining them in particular ways--

\begin{itemize}
  \item \textbf{Basis} - $\emptyset$, $\{\Lambda\}$ and $\{a\}$ are all regular languages for all $a \in \Sigma$
  \item \textbf{Induction} - If $L$ and $M$ are regular languages, then $L \cup M$, $L \cdot M$ and $L^*$ are also
   regular languages
\end{itemize}

The basis of this definition gives us the following four regular languages over the alphabet $\Sigma = \{a, b\}$:
\begin{equation*}
  \emptyset, \{\Lambda\}, \{a\}, \{b\}
\end{equation*}

\textbf{All} regular languages over $\Sigma$ can be constructed by combining these four languages in various ways by
 recursively applying the union, product and closure operations.

\begin{example*}{}{}
  Is the language $\{a, ab\}$ regular?

  Yes, since it can be constructed from the four basic regular languages using a product and union operation--
  \begin{align*}
    \{a, ab\} &= \{a\} \cdot \{\Lambda, b\}\\
    &= \{a\} \cdot (\{\Lambda\} \cup \{b\})
  \end{align*}
\end{example*}

We can also see from this that it is possible to construct any finite language in this way, and therefore \textbf{all
 finite languages are regular}. There are also many infinite languages which are regular.

\section*{Regular Expressions}

If you wanted to find an algorithm which can determine if a string belongs to a particular regular language, you would
 most likely use a \textbf{regular expression}

\begin{definition*}{}{}
  A \textbf{Regular Expression} is basically a short-hand way of showing how a regular language is constructed from a
   base set of regular languages.

  For each regular expression $E$, there is a regular language $L(E)$
\end{definition*}

Like the regular languages discussed previously, regular expressions can be manipulated inductively to form new regular
 expressions.

\begin{itemize}
  \item \textbf{Basis} - $\Lambda$, $\emptyset$ and $a$ are regular expressions for all $a \in \Sigma$
  \item \textbf{Induction} - If $R$ and $E$ are also regular expressions, then $(R)$, $R + E$, $R \cdot E$ and $R^*$ are
   also regular expressions
\end{itemize}

\begin{example*}{}{}
  A few of the regular expressions over the alphabet $\Sigma = \{a, b\}$ are
  \begin{gather*}
    \Lambda, \emptyset, a, b\\
    \Lambda + b, b^*, a + (b \cdot a)\\
    (a + b) \cdot a, a \cdot b^*, a^* \cdot b^*
  \end{gather*}
\end{example*}

So that we don't have to include lots of parentheses, the operations have the following hierarchy--
\begin{itemize}
  \item $*$ (Highest)
  \item $\cdot$
  \item $+$ (Lowest)
\end{itemize}

\begin{example*}{}{}
  The regular expression
  \begin{equation*}
    a + b \cdot a^*
  \end{equation*}
  is the same as
  \begin{equation*}
    (a + (b \cdot (a^*)))
  \end{equation*}
\end{example*}

We can also juxtapose languages rather than using $\cdot$ where there is only one possible interpretation

\begin{example*}{}{}
  The regular expression
  \begin{equation*}
    a + b \cdot a^*
  \end{equation*}
  is the same as
  \begin{equation*}
    a + ba^*
  \end{equation*}
\end{example*}

\subsection*{Operators}

In regular expressions, there are two binary operations ($+$ and $\cdot$), and one unary operation ($*$). These are
 closely related to the union, product, and closure operations over corresponding languages.

\begin{example*}{}{}
  The regular expression
  \begin{equation*}
    a + bc^*
  \end{equation*}
  is more-or-less shorthand for the regular language
  \begin{equation*}
    \{a\} \cup (\{b\} \cdot ({\{c\}}^*))
  \end{equation*}
\end{example*}

\subsection*{From Expressions to Languages}

There is a very simple substitution method to find the language described by a regular expression.

\begin{example*}{}{}
  Find the language described by the regular expression $a + bc^*$--

  \begin{align*}
    L(a + bc^*) &= L(a) \cup L(bc^*)\\
    &= L(a) \cup (L(b) \cdot L(c^*))\\
    &= L(a) \cup (L(b) \cdot {L(c)}^*)\\
    &= \{a\} \cup (\{b\} \cdot {\{c\}}^*)\\
    &= \{a\} \cup (\{b\} \cdot \{\Lambda, c, c^2, c^3, \ldots\})\\
    &= \{a\} \cup \{b, bc, bc^2, bc^3, \ldots\}\\
    &= \{a, b, bc, bc^2, bc^3, \ldots\}
  \end{align*}
\end{example*}

This approach can also be used to prove that a given language is regular.

Many infinite languages can easily be proven to be regular, by finding a regular expression that describes it. Not all
 infinite languages are regular however.

Regular expressions are also not necessarily unique, as they may represent the same language as another regular expression,
 such as $a+b$ and $b+a$ being different regular expressions, but both represent the language $\{a, b\}$

\begin{definition*}{}{}
  Regular expressions are said to be \textbf{Equal} if their languages are the same. If $R$ and $E$ are regular expressions
   where $L(R) = L(E)$, then they are equal and can be written as $R = E$.
\end{definition*}

\subsection*{Properties of Regular Expressions}

There are many general equalities for regular expressions, which hold for any regular expressions $R$, $E$, and $F$, and
 can be proven by using the basic properties of languages and sets.

\begin{itemize}
  \item Additive Properties
  \begin{align*}
    R + E &= E + R\\
    R + \emptyset &= \emptyset + R = R\\
    R + R &= R\\
    (R + E) + F &= R + (E + F)
  \end{align*}
  \item Product Properties
  \begin{align*}
    R \emptyset &= \emptyset R = \emptyset\\
    R \Lambda &= \Lambda R = R\\
    (RE)F &= R(EF)
  \end{align*}
  \item Distributive Properties
  \begin{align*}
    R(E + F) &= RE + RF\\
    (R + E)F &= RF + EF
  \end{align*}
  \item Closure Properties
  \begin{align*}
    \emptyset^* &= \Lambda^* = \Lambda\\
    R^* &= R^*R^* = {(R^*)}^* = R + R^*\\
    R^* &= \Lambda + RR^* = (\Lambda + R)R^*\\
    RR^* &= R^*R\\
    R{(ER)}^* &= {(RE)}^*R\\
    {(R + E)}^* &= {(R^*E^*)}^* = {(R^* + E^*)}^* = R^*{(ER^*)}^*
  \end{align*}
\end{itemize}

\section*{Regular Grammars}

There are several ways to characterize grammars which describe regular languages.

\begin{definition*}{}{}
  A \textbf{Regular Grammar} is one where all of the productions take one of the following forms
  \begin{align*}
    B &\rightarrow \Lambda\\
    B &\rightarrow w\\
    B &\rightarrow A\\
    B &\rightarrow wA
  \end{align*}
  Where $A$ and $B$ are non-terminals and $w$ is a non-empty string of terminals
\end{definition*}

Only one non-terminal can appear on the right hand side of any production. Non-terminals must appear on the right end
 of the right hand side. Therefore the productions $A \rightarrow aBc$ and $S \rightarrow TU$ are not productions in a
 regular grammar, but $A \rightarrow abcA$ is.

For any regular language, we can find a regular grammar which describes it, but there may also be non-regular grammars
 which produce it.



\lecture{A4: Finite Automata}{13:00}{10/10/24}{Janka Chlebikova}

There are several different `models of computation', such as finite automata, push-down automata and turing machines.
 They are all abstract models of computers which are capable of different things. They all have an input tape with a
 single string of an infinite length and can accept or reject the input string.

\section*{Finite Automata}

Finite automata are the most basic models of a computer.

\begin{definition*}{}{}
  A \textbf{Finite Automaton} has three components--
  \begin{itemize}
    \item An input tape, which contains an input string over the alphabet $\Sigma$
    \item A head, which reads the input string one symbol at a time
    \item Memory, which is realised as a finite set $Q$ of states, of which the automaton can only be in one at any time
  \end{itemize}

  The `program' of the automaton defines how the read symbol changes the current state.
\end{definition*}

To better understand an automaton, they are typically represented as a transition graph, which is a form of directed
 graph.

\begin{example*}{}{}
  \centering
  \begin{tikzpicture}
    \tikzset{
      ->, % makes the edges directed
      >=stealth, % makes the arrow heads bold
      node distance=3cm, % specifies the minimum distance between two nodes
      every state/.style={thick, fill=gray!10}, % sets the properties for each ’state’ node
      initial text=$ $, % sets the text that appears on the start arrow
    }

    \node[state] (s2) {$s_2$};
    \node[state, initial, left of=s2] (s1) {$s_1$};
    \node[state, below of=s2] (s3) {$s_3$};
    \node[state, accepting, right of=s3] (s4) {$s_4$};

    \draw (s1) edge[above] node{a} (s2)
          (s1) edge[above] node{b} (s3)
          (s2) edge[loop above] node{b} (s2)
          (s2) edge[above] node{a} (s4)
          (s3) edge[loop above] node{a} (s3)
          (s3) edge[above] node{b} (s4)
          (s4) edge[loop above] node{a,b} (s4);
  \end{tikzpicture}

  This is a finite automata which takes the alphabet $\Sigma = \{a, b\}$, and accepts strings such as $abba$, $baab$, etc.
\end{example*}

Each automaton has one initial state, denoted by the arrow entering $s_1$, and at least one final or accepting state,
 denoted by the double circle such as $s_4$. Once the whole input string is read into the automaton, if the current
 state is a final or accepting state, then the string is accepted. Otherwise, the string is rejected. The language of
 an automaton is the set of strings it accepts.

\subsection*{A Formal Definition}

\begin{definition*}{}{}
  A finite automaton is defined by
  \begin{itemize}
    \item A set of states, $Q$
    \item A start state, $s \in Q$
    \item A set of accepting states, $Q_a \subset Q$
    \item A set of transition functions $T : Q \times \Sigma \rightarrow Q$
  \end{itemize}
\end{definition*}

\begin{example*}{}{}
  A very simple finite automaton could be defined by
  \begin{gather*}
    Q = \{0, 1, 2\}\\
    s = 0\\
    Q_a = \{1\}\\
    T(0,a) = 1, T(0,b) = 2\\
    T(1, a) = T(1, b) = 1\\
    T(2, a) = T(2, b) = 2
  \end{gather*}
  And as a transition graph,\\
  \begin{tikzpicture}
    \tikzset{
      ->, % makes the edges directed
      >=stealth, % makes the arrow heads bold
      node distance=3cm, % specifies the minimum distance between two nodes
      every state/.style={thick, fill=gray!10}, % sets the properties for each ’state’ node
      initial text=$ $, % sets the text that appears on the start arrow
    }

    \node[state, initial] (s0) {$0$};
    \node[state, accepting, right of=s0] (s1) {$1$};
    \node[state, below of=s1] (s2) {$2$};

    \draw (s0) edge[above] node{a} (s1)
          (s0) edge[above] node{b} (s2)
          (s1) edge[loop above] node{a,b} (s1)
          (s2) edge[loop above] node{a,b} (s2);
  \end{tikzpicture}
\end{example*}

\subsection*{State Transition Functions}

A state transition function, $T : Q \times \Sigma \rightarrow Q$, of the form
\begin{figure}[H]
  \begin{tikzpicture}
    \tikzset{
      ->, % makes the edges directed
      >=stealth, % makes the arrow heads bold
      node distance=3cm, % specifies the minimum distance between two nodes
      every state/.style={thick, fill=gray!10}, % sets the properties for each ’state’ node
      initial text=$ $, % sets the text that appears on the start arrow
    }

    \node[state] (s0) {$0$};
    \node[state, right of=s0] (s1) {$1$};

    \draw (s0) edge[above] node{a} (s1);
  \end{tikzpicture}
\end{figure}
is represented by $T(0, a) = 1$, where $0, 1 \in Q$ and $a \in \Sigma$.

\section*{Deterministic Finite Automata}

In the previous examples, there is always exactly one transition function for every state and symbol -- every node has
 exactly one edge for each possible input symbol. Such automata are known as \textbf{deterministic} finite automata.
 That means that for any input string, we always know which unique state we are in at any symbol in the string.

\begin{definition*}{}{}
  A \textbf{DFA} accepts a string $w$ over $\Sigma^*$ if and only if there is a path from the initial state to an
   accepting state, such that $w$ is the concatenation of the labels on the edges of the path. Otherwise, the DFA
   rejects $w$. For a FA to be deterministic, there must also be \textbf{exactly} one edge from each state for each
   symbol in $\Sigma$.

  The set of all strings over $\Sigma$ accepted by the DFA $M$ is known as the language of $M$, and is represented by
   $L(M)$.
\end{definition*}

It has been proven that ``the class of regular languages is exactly the same as the class of languages accepted by DFAs''.
 This means that for any given regular language, we can find at least one DFA which recognises it, and vice versa.

\section*{Non-Deterministic Finite Automata}

With a DFA, you always know exactly which state it will be in after a given input string. This is not the case with an
 NFA, since in any given state there may be zero or more transitions for each symbol in the alphabet. Since more than
 one transition is possible for each state, it is impossible to know which state it will be in given an input string.
 If the NFA reaches an accepting state after reading the string, it is accepted. Otherwise it is rejected. Since there
 could be multiple paths for any given strings, as long as one or more path reaches an accepting state, it is accepted.
 Also, if there are no edges that can be traversed from a state using the current input symbol, the input string is
 immediately rejected.

Additionally, an edge may be labelled with $\Lambda$, which means that one can make the transition without consuming any
 symbols from the input string.

The non-determinism of these FA also means that they are impossible to run on a traditional computer, at least without
 using tricks that make them much less efficient. With the advent of quantum computers, it may become possible to use
 NFAs for much more efficient parsing and pattern matching.

\begin{example*}{}{}
  \centering
  \begin{tikzpicture}
    \tikzset{
      ->, % makes the edges directed
      >=stealth, % makes the arrow heads bold
      node distance=2cm, % specifies the minimum distance between two nodes
      every state/.style={thick, fill=gray!10}, % sets the properties for each ’state’ node
      initial text=$ $, % sets the text that appears on the start arrow
    }

    \node[state, initial] (s0) {$0$};
    \node[state, right of=s0] (s1) {$1$};
    \node[state, below of=s1] (s2) {$2$};
    \node[state, right of=s1, accepting] (s3) {$3$};

    \draw (s0) edge[above] node{$a$} (s1)
          (s1) edge[above] node{$b$} (s3)
          (s0) edge[below] node{$\Lambda$} (s2)
          (s2) edge[loop below] node{$a$} (s2)
          (s2) edge[below] node{$a$} (s3);
  \end{tikzpicture}

  This is an NFA which corresponds to the regular expression $ab + a^*a$.
\end{example*}

Since we are able to have multiple edges with the same symbol from a node, we could also use this different NFA which
 recognises the language of the same regular expression.

\begin{example*}{}{}
  \centering
  \begin{tikzpicture}
    \tikzset{
      ->, % makes the edges directed
      >=stealth, % makes the arrow heads bold
      node distance=2cm, % specifies the minimum distance between two nodes
      every state/.style={thick, fill=gray!10}, % sets the properties for each ’state’ node
      initial text=$ $, % sets the text that appears on the start arrow
    }

    \node[state, initial] (s0) {$0$};
    \node[state, right of=s0] (s1) {$1$};
    \node[state, right of=s1, accepting] (s2) {$2$};
    \node[state, below of=s1, accepting] (s3) {$3$};

    \draw (s0) edge[above] node{$a$} (s1)
          (s0) edge[below] node{$a$} (s3)
          (s1) edge[above] node{$b$} (s2)
          (s3) edge[loop below] node{$a$} (s3);
  \end{tikzpicture}
\end{example*}

\subsection*{NFA Transition Functions}

Since NFAs are non-deterministic, their transition functions instead take the form $T : Q \times \Sigma \rightarrow P(Q)$.
 This allows for the result of any given transition function to be a set of states which could be transitioned to.

\begin{example*}{}{}
  Taking the first example of an NFA, the transition function might look as below--
  \begin{gather*}
    T : Q \times \Sigma \rightarrow P(Q)\\
    T(0, a) = \{1\}, T(0, \Lambda) = \{2\}\\
    T(1, b) = \{3\}\\
    T(2, a) = \{2, 3\}
  \end{gather*}
  and for the second example--
  \begin{gather*}
    T : Q \times \Sigma \rightarrow P(Q)\\
    T(0, a) = \{1, 3\}\\
    T(1, b) = \{2\}\\
    T(3, a) = \{3\}
  \end{gather*}
\end{example*}

\section*{DFAs vs NFAs}

On the surface it may seem like there aren't many differences between NFAs and DFAs, and in fact, DFAs are a subset of
 NFAs. It has also been proven that ``the class of languages accepted by NFAs is also exactly the class of regular
 languages'', just as for DFAs. This also means that for every NFA there must be an equivalent DFA, or at least one that
 accepts the same language.

Generally, NFAs are easier to construct for any given regular expression, and tend to be more simple, with fewer states
 and transitions. However, DFAs are much easier to execute, especially on the deterministic computers we currently use.
 Since they recognise the same set of languages, it is always possible to find equivalents, though the DFA will usually
 be much larger and more complex.

\subsection*{Finding an Equivalent DFA}

Generally, the DFA will have more states than the equivalent NFA. If the DFA has $n$ states, the DFA may have as many
 as $2^n$ states. To actually do this conversion, you can follow the algorithm below (shown here converting the NFA in
 the figure below)

\begin{figure}[H]
  \centering
  \begin{tikzpicture}
    \tikzset{
      ->, % makes the edges directed
      >=stealth, % makes the arrow heads bold
      node distance=2cm, % specifies the minimum distance between two nodes
      every state/.style={thick, fill=gray!10}, % sets the properties for each ’state’ node
      initial text=$ $, % sets the text that appears on the start arrow
    }

    \node[state, initial] (s0) {$s_0$};
    \node[state, right of=s0] (s1) {$s_1$};
    \node[state, right of=s1, accepting] (s2) {$s_2$};

    \draw (s0) edge[loop below] node{$a,b$} (s0)
          (s0) edge[above] node{$a$} (s1)
          (s1) edge[above] node{$b$} (s2);
  \end{tikzpicture}
  \caption{The NFA which accepts the language for the regular expression ${(a + b)}^*ab$}
\end{figure}

\textbf{Step 1} -- Begin with the NFA start state. If it's connected to any other states by $\Lambda$, make the initial
 state of the DFA a set containing the NFA's initial state and those others.

\begin{figure}[H]
  \centering
  \begin{tikzpicture}
    \tikzset{
      ->, % makes the edges directed
      >=stealth, % makes the arrow heads bold
      node distance=3cm, % specifies the minimum distance between two nodes
      every state/.style={thick, fill=gray!10}, % sets the properties for each ’state’ node
      initial text=$ $, % sets the text that appears on the start arrow
    }

    \node[state, initial] (s0) {$\{s_0\}$};
  \end{tikzpicture}
  \caption{The resulting DFA after step 1}
\end{figure}

\textbf{Step 2} -- For each symbol, determine the set of possible NFA states you may be in after reading it. This set
 becomes the label of the state, and is connected to the previous state by that symbol. Only create a new state if one
 with that label does not already exist.

\begin{figure}[H]
  \centering
  \begin{tikzpicture}
    \tikzset{
      ->, % makes the edges directed
      >=stealth, % makes the arrow heads bold
      node distance=3cm, % specifies the minimum distance between two nodes
      every state/.style={thick, fill=gray!10}, % sets the properties for each ’state’ node
      initial text=$ $, % sets the text that appears on the start arrow
    }

    \node[state, initial] (s0) {$\{s_0\}$};
    \node[state, right of=s0] (s1) {$\{s_0, s_1\}$};

    \draw (s0) edge[loop below] node{$b$} (s0)
          (s0) edge[above] node{$a$} (s1);
  \end{tikzpicture}
  \caption{The resulting DFA after step 2}
\end{figure}

\textbf{Step 3} -- Repeat Step 2 for each new DFA state, until the system is closed. DFA accepting states are those that
 include one or more NFA accepting states. If there is no transition for a symbol in the NFA, create a new state in the
 DFA with the label $\emptyset$, and add loops for all symbols (known as a non-final trap state).

\begin{figure}[H]
  \centering
  \begin{tikzpicture}
    \tikzset{
      ->, % makes the edges directed
      >=stealth, % makes the arrow heads bold
      node distance=3cm, % specifies the minimum distance between two nodes
      every state/.style={thick, fill=gray!10}, % sets the properties for each ’state’ node
      initial text=$ $, % sets the text that appears on the start arrow
    }

    \node[state, initial] (s0) {$\{s_0\}$};
    \node[state, right of=s0] (s1) {$\{s_0, s_1\}$};
    \node[state, right of=s1, accepting] (s2) {$\{s_0, s_2\}$};

    \draw (s0) edge[loop below] node{$b$} (s0)
          (s0) edge[above] node{$a$} (s1)
          (s1) edge[loop below] node{$a$} (s1)
          (s1) edge[above, bend left=10] node{$b$} (s2)
          (s2) edge[below, bend left=10] node{$a$} (s1)
          (s2) edge[above, bend right=30] node{$b$} (s0);
  \end{tikzpicture}
  \caption{The final DFA}
\end{figure}

\begin{figure}[H]
  \centering
  \begin{tikzpicture}
    \tikzset{
      ->, % makes the edges directed
      >=stealth, % makes the arrow heads bold
      node distance=3cm, % specifies the minimum distance between two nodes
      every state/.style={thick, fill=gray!10}, % sets the properties for each ’state’ node
      initial text=$ $, % sets the text that appears on the start arrow
    }

    \node[state, initial] (s0) {$\{s_0\}$};
    \node[state, right of=s0] (s1) {$\{s_0, s_1\}$};
    \node[state, right of=s1, accepting] (s2) {$\{s_0, s_2\}$};
    \node[state, below of=s1] (st) {$\emptyset$};

    \draw (s0) edge[loop below] node{$b$} (s0)
          (s0) edge[above] node{$a$} (s1)
          (s1) edge[loop below] node{$a$} (s1)
          (s1) edge[above, bend left=10] node{$b$} (s2)
          (s2) edge[below, bend left=10] node{$a$} (s1)
          (s2) edge[above, bend right=30] node{$b$} (s0);

    \draw (s0) edge[above] node{$c$} (st)
          (s1) edge[left, bend right=30] node{$c$} (st)
          (s2) edge[above] node{$c$} (st)
          (st) edge[loop below] node{$a,b,c$} (st);
  \end{tikzpicture}
  \caption{The same DFA, assuming that the language was $\{a,b,c\}$ rather than $\{a,b\}$}
\end{figure}