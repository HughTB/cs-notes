\lecture{Local Area Networks}{11:00}{04/11/24}{Asim Ali}

\section*{Data Link Layer}

Within the Data Link layer of the OSI reference model, there are two control systems-- the Logical Link Control and
 Media Access Control. The logical link control is responsible for interfacing with the layers above it, and performing
 flow and error control.

Media Access Control is responsible for more of the addressing and actual communications, and handles assembling and
 disassembling the transmission frame, which consists of
\begin{itemize}
  \item Receiver's address
  \item Error detection data, e.g.\ checksum or crc
\end{itemize}

\subsection*{Media Access Protocol}

The original ethernet protocol was designed for use in bus or star networks, and so includes error and collision
 handling. Since only one station is able to transmit at a time in a bus or (hub) star network, there needs to be a way
 to determine if a collision has occurred and how it should be resolved. This is made up of a set of rules--

\begin{itemize}
  \item If there is already a station transmitting, do not transmit.
  \item If another station transmits while you are transmitting, jam the medium and wait for a random length of time,
   using the back-off algorithm. (Select a random length of time from $0$ to $2^n - 1$ slots, where n is the number of
   collisions).
  \item If there have been an excessive number of collisions, cancel the transmission and send an error up through the
   layers.
  \item After waiting, attempt to transmit again.
\end{itemize}

The length of one transmission slot depends upon the speed of the medium, and the minimum size of packets. If the slot
 size is $512$bits, then at $10$Mbps, the slot time is $\frac{512}{10000000}$ or $0.0000512$ seconds (or
 $51.2$~{\textmu}sec)

To determine if a collision has occurred, each station is connected to the medium with two wires, one for transmitting
 and one for receiving. During transmission, if the signal sent out on the transmission line is not the same as what is
 received, then another station is interfering with the transmission, and a collision has occurred.

\section*{Switching}

\subsection*{Store-and-Forward}

When the switch receives a packet, it waits for the entire packet to arrive and stores it in memory before forwarding it
 to the intended recipient. This mode of operation works for both half and full-duplex transmission.

The latency in this mode can be calculated by sender-switch transmission time (packet size / transmission speed) +
 switch latency + switch-recipient transmission time + propagation delay of the medium (how long the data takes to get
 from one end of the medium to the other).

\subsection*{Virtual-cut Through}

When the switch receives a packet, it only buffers enough of the packet to allow it time to process the MAC header, and
 determine where the packet should be sent. It then just forwards the buffer and the data as it flows into the switch.

The latency here can be calculated by sender-switch transmission time + switch latency + propagation delay of the
 medium.