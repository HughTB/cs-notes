\lecture{Wireless Local Area Networks}{11:00}{11/11/24}{Asim Ali}

A single cell network is one which has a single Access Point (or Control Module), using a single frequency, but may have
 any number of clients (User Modules). A multiple-cell network may have multiple access points using different
 frequencies, each with multiple clients, but they must be connected to the same wired network for back-haul.

Each cell acts like a star network, since client can only communicate with the access point, and not directly with any
 of the other clients on the same network. In a multiple-cell network, they also cannot communicate directly between
 cells without a wired back-haul, which is usually part of an existing wired network.

Before a client can communicate in a cell, it needs to associate with the access point. In a single cell network, this
 is simple since there is only ever a single access point that needs to be associated with. In a multi-cell network,
 the client must disassociate from one access point, and associate to the new one. This may be done manually, if each
 access point is broadcasting a different SSID, or it may be done automatically if it's a cohesive network where the
 access points communicate to hand-off clients.

There are also ad-hoc WLAN networks, in which the devices are all simultaneously clients and access points, and
 communicate directly with each other.

\section*{IEEE 802.11}

\subsection*{Terms}

\begin{itemize}
  \item Associated Stations-- Devices connected to a WLAN
  \item Access Point-- Provides access to the network for associated devices
  \item Basic Service Set (BSS)-- A set of stations controlled by one coordinator
  \item Extended Service Set (ESS)-- A set of one or more connected BSSs
  \item Distribution System (DS)-- A system used to connect a set of BSSs and LANs to create an ESS
  \item Frame-- A unit of data in the MAC protocol
\end{itemize}

\subsection*{Services}

\begin{table}
  \centering
  \begin{tabular}[H]{ lll }
    \hline Service & Provider & Used for\\
    \hline Association & Distribution System & MSDU\\
    Authentication & Stations & Access \& Security\\
    De-authentication &  Stations & Access \& Security\\
    Disassociation & Distribution System & MSDU\\
    Distribution & Distribution System & MSDU\\
    Integration & Distribution System & MSDU\\
    MAC Service Data Unit (MSDU) & Stations & \\
    Privacy & Stations & Access \& Security\\
    Re-association & Distribution System & MSDU\\
    \hline
  \end{tabular}
\end{table}

The MAC Service Data Unit is effectively the mechanism by which packets are prepared and sent over the medium.

\section*{Access Control (CSMA/CA)}

A very similar protocol to CSMA/CD, but designed specifically for wireless networks rather than wired bus networks.
 The difference is that the nodes might not know that they are causing a collision, since they may be in range of the
 access point, but not the other transmitting node.

There is an optional centralised controller known as the Point Coordination Function (PCF), which provides collision
 avoidance by providing permission to each node before it is allowed to send a message. There is also a distributed
 version known as the distributed coordination function (DCF), which is used in ad-hoc networks or in networks where
 throughput is more important than latency.

\subsection*{Point Coordination Function}

The controller (usually the access point) offers a transmission slot to each known client, which the client then replies
 to, saying that it either does or does not wish to transmit. If the client does not wish to transmit, the slot is
 offered to the next client, and so on. It uses a round-robin method to give each client an opportunity to transmit, and
 guarantees there are no collisions. This does however waste a lot of capacity, since nodes often do not wish to
 transmit, and if there are some devices with lots of data, and some with no data, they are still given the same share
 of the throughput.

The access point also sends out a `beacon frame' $10-100$ times per second, which advertises the network to any stations
 within the coverage area. The frame also contains information such as hopping frequencies, dwell times, clock
 synchronisation and more.

\subsection*{Distributed Coordination Function}

DCF uses CSMA/CA to provide collision avoidance, as well as various levels of Inter-Frame Spacing (IFS). There are two
 methods which a network can use to confirm that a packet has been received correctly.
\begin{itemize}
  \item Frame Exchange Protocol
  \begin{itemize}
    \item Sender transmits frame immediately
    \item Receiver response with an acknowledgement
    \item If sender doesn't receive acknowledgement, it retransmits the frame
  \end{itemize}
  \item Four Frame Exchange
  \begin{itemize}
    \item Sender sends request-to-send (RTS)
    \item Receiver responds with a clear-to-send (CTS)
    \item Sender transmits frame
    \item Receiver responds with an acknowledgement
  \end{itemize}
\end{itemize}

If any station receives an RTS or CTS message not addressed to them, they immediately put themselves into a Non-Active
 Mode (NAV) until the medium is again idle. That way, even if the station is not in range of the requesting station, it
 should always be in range of the clearing station, as it will most likely be the access point. If the sending station
 does not receive an acknowledgement within a short time, the transmission is cancelled and it must send a new RTS
 before it can transmit.

The Distributed Inter-Frame Spacing (DIFS) is the length of time that a node must wait between transmissions, and the
 length of time that is multiplied to get the exponential back-off delays.

\subsection*{Priority Traffic}

Some traffic needs to be given priority over others, such as acknowledgements, CTS messages and poll responses. These
 types of traffic use the Short Inter-Frame Spacing (SIFS) so they are more likely to transmit immediately. The access
 point also has priority over all other stations, and uses the Point-Coordinator Inter-Frame Spacing (PIFS), which is
 shorter than the DIFS, but longer than the SIFS.

\section*{Super Frames}

The bandwidth of the frequency is split into so-called \textbf{Super Frames}, which are a certain length of time.

Each `Super Frame' has two sections. The first part of the frame uses PCF to poll each station for time-sensitive frames
 which need to be sent before anything else. For the second part, the point-coordinator goes idle and allows time for
 DCF frames to be sent asynchronously. This allows both time-sensitive and throughput-sensitive traffic on the same
 network.

If the end of one transmission overlaps the start of a super frame, the length of the super frame is reduced and the
 PCF section is delayed as to not collide with the previous transmission. This typically means that the DCF time in this
 super frame is reduced.