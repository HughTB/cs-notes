\lecture{Satellite Networks}{11:00}{09/12/24}{Asim Ali}

Most communication satellites effectively act as a reflector which sends signals between two stations on the ground.
 They typically have several transponders, each of which is able to listen to several portions of the spectrum, and
 amplify the incoming (uplink) signal, change it's frequency and then broadcasts it to the receiving station (downlink).
 Each ground station has it's own transmitters and receivers, with antennae pointed towards the satellite. They allow
 communication between two points on the ground which do not have line-of-sight, typically due to their distance and
 the curvature of the Earth.

Satellites can either be point-to-point, or have a broadcast link, where there is one transmitter, and the signal is
 amplified and broadcasted to multiple receivers on the ground.

\section*{Satellite Orbits}

There are several important factors which determine which orbit a satellite will use, such as where it needs to have
 line-of-sight, and when. All orbits have an apogee (the furthest point from Earth) and a perigee (closest point to
 Earth). If the orbit is elliptical, then it is usually off-centre, and so the satellite gets much closer and further
 away along it's orbit. There are also circular orbits, which are more useful for Geostationary orbits, or for
 satellites which need to be a constant distance from the ground.

\subsection*{Orbital Periods}

The orbital period is the time it takes for the satellite to make one complete orbit around the planet, and it can be
 calculated as
\begin{equation*}
  T = \frac{{(R + A)}^{1.5}}{100}
\end{equation*}
 where $R$ is the radius of the earth (6378Km), $A$ is the altitude of the satellite, and $T$ is the orbital period in
 seconds.

\subsection*{Different Orbits}

There are three main types of orbits which are used for communication satellites--
\begin{itemize}
  \item Low-Earth Orbit (LEO)-- 500 to 1500Km altitude, moving very quickly. Much lower latency than other orbits
  \item Medium-Earth Orbit (MEO)-- 5000 to 18000Km altitude, moving slower than LEO. Higher latency than LEO but lower
   than geostationary
  \item Geostationary-Earth Orbit (GEO)-- Roughly 35786Km altitude, moving much slower than MEO. Much higher latency
   than LEO or MEO, but always above the same point on Earth, so useful for static communication links
\end{itemize}

There are also two layers of the Van Allen belt, which are layers of charged particles, which very quickly destroy any
 satellites which enter them. They are from 1500 to 5000 and 15000 to 20000Km in altitude.

Using a higher orbit increases the coverage area of each satellite, which means that less are needed for full coverage.
 However, it also greatly increases latency and so is not suitable for realtime communications.

\subsection*{Geostationary Satellites}

Since the orbital period of a geostationary satellite is 24 hours, they always stay above the same point on the equator,
 which means that antennae don't need to constantly adjust to follow the satellite as they move across the sky. They
 provide great coverage, theoretically covering 120 degrees of the surface, everywhere except the poles, meaning that
 only 3 satellites are needed for total coverage. There are at most 180 satellites in a perfect geostationary orbit
 at any time, as they need 2 degrees of rotational separation to ensure they don't collide. The allocation of `slots'
 is managed by the ITU.

\subsection*{LEO Satellites}

Typically have a circular or slightly elliptical orbit, with an altitude lower than 2000Km. The orbital period is
 typically anywhere from 1.5 to 2 hours, and they typically have a coverage area with a diameter of roughly 8000Km.
 They are typically only visible for up to 20 minutes at any given point, and so the system must be able to aim the
 antennae to maintain line-of-sight. There is also a large amount of doppler shift in the signal due to the very high
 relative speed of the satellites. Because they orbit so low, they experience atmospheric drag and must boost their
 altitude often to avoid orbital deterioration and then falling back to Earth and burning up.

\subsection*{MEO Satellites}

Can have either circular or elliptical orbit, with an altitude from 5000 to 18000Km. The orbital period is typically
 roughly 6 hours. They do not experience atmospheric drag, and so typically have a longer lifespan than LEO satellites.
 GPS satellites are typically MEO, and orbit at around 18000Km. Their orbits are usually inclined with respect to the
 equator, which means that they cover a larger portion of the surface over time.

\section*{Frequency Bands}

{\Huge add table here}

\subsection*{Transmission Performance}

There are several factors which determine the performance of a satellite communication link--
\begin{itemize}
  \item The distance between the station antennae and the satellite antennae, due to signal dissipation and latency
  \item The ground distance between the station antennae and the main focus point of the satellite
  \item Atmospheric attenuation due to varying weather and atmospheric pressure
\end{itemize}

\section*{Satellite Capacity Allocation}

Since there is a limited number of satellites in each orbit, and each satellite has a limited number of transponders,
 to serve a large number of users it is necessary to use multiplexing to increase capacity. Methods discussed before
 are used.

\subsection*{Frequency Division Multiple Access (FDMA)}

The bandwidth assigned to the satellite is divided into channels, for example a 500MHz bandwidth divided into 24 40MHz
 channels, with a guard band of 4MHz built in. Each of the channels could be used for any purpose, and they can all
 transmit simultaneously without colliding. There are only 12 40MHz channels in a 500MHz band, but satellites often use
 horizontal and vertical polarisation to double the bandwidth available. Within each channel, there can be multiple
 stations, each of which uses a different amount of bandwidth and different frequencies.

\subsection*{Time Division Multiple Access (TDMA)}

There are a set number of slots in each second, which are divided up so that each station has the same share of time.
 The uplink is divided up and each of the stations transmits at a different time, only to the satellite. The downlink
 is divided up and the satellite broadcasts all of the data to all of the stations, and they ignore any data destined
 for the other stations. There is also a guard time between each slot, to make sure that there are no collisions. At the
 start of each frame, there is a so-called `reference burst', which are used to denote the start of a new frame, and to
 synchronise the clocks in the devices.

