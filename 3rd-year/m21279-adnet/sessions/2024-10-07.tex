\lecture{Signal Encoding II}{11:00}{07/10/24}{Asim Ali}

\section*{Transmission Media}

There are many different mediums which can be used to transmit information. In the case of networks, there are two main
 types, guided and unguided media.

\begin{itemize}
  \item Guided Media
  \begin{itemize}
    \item The signal is contained within a medium, such as
    \item Copper cable (twisted pair, coaxial, etc)
    \item Fibre optics
  \end{itemize}
  \item Unguided Media
  \begin{itemize}
    \item The signal is sent out without containment, through the air
    \item This does not mean that the signal is not directional, microwave networks are often highly directional to
     reduce signal drop-off and interference
    \item Microwave, Radio, Cellular and Satellite
  \end{itemize}
\end{itemize}

\section*{Signal Properties}

\begin{itemize}
  \item \textbf{Data Rate} -- The speed in bits per second (bps or b/s) at which data can be communicated
  \item \textbf{Error} -- The reception of a 1 when 0 was transmitted, or vice versa
  \item \textbf{Error Rate} -- The rate at which errors occur, usually as a ratio or percentage
  \item \textbf{Frequency Bandwidth} -- The difference between the upper and lower frequency in a continuous frequency
   band
  \item \textbf{Channel Capacity} -- The maximum rate at which data can be transmitted through a channel 
  \item \textbf{Signal-to-Noise Ratio} -- The ratio of the signal power to noise power in decibels (dB)
\end{itemize}

\section*{Interference and Noise}

\begin{definition*}{}{}
  \textbf{Interference} is when two signals are added together. There are two main types of interference -- constructive
   and destructive. When two signals combine to create a signal with a higher amplitude, this is constructive interference.
   If the amplitude is reduced, this is destructive interference.
\end{definition*}

\begin{definition*}{}{}
  \textbf{Noise} is any other signal which interferes with the desired signal along the medium of transmission. Typically,
   any noise in a signal will make it more difficult to decode, since the signal may be reduced in amplitude or shifted
   in phase.  
\end{definition*}

\begin{minipage}[c]{0.47\linewidth}
  \begin{figure}[H]
    \centering
    \begin{tikzpicture}[scale=0.9]
      \node at (-2,-1.5) {Interfering Waves};
      \node at (-2,-6) {Resulting Wave};

      \draw (0,0) -- (6,0);
      \draw (0,-1) -- (0,1);
      \draw[thick,myPurple] plot[domain=0:12*pi, samples=180] ({(\x/pi)/2},{(sin(\x r)/2)});

      \draw (0,-3) -- (6,-3);
      \draw (0,-4) -- (0,-2);
      \draw[thick,complGreen] plot[domain=0:12*pi, samples=180] ({(\x/pi)/2},{-3 + (sin(\x r)/2)});
      
      \draw (0,-6) -- (6,-6);
      \draw (0,-7) -- (0,-5);
      \draw[thick] plot[domain=0:12*pi, samples=180] ({(\x/pi)/2},{-6 + sin(\x r)});
    \end{tikzpicture}
    \caption{Constructive interference}
  \end{figure}
\end{minipage}\hfill
\begin{minipage}{0.47\linewidth}
  \begin{figure}[H]
    \centering
    \begin{tikzpicture}[scale=0.9]
      \node at (-2,-1.5) {Interfering Waves};
      \node at (-2,-6) {Resulting Wave};

      \draw (0,0) -- (6,0);
      \draw (0,-1) -- (0,1);
      \draw[thick,myPurple] plot[domain=0:12*pi, samples=180] ({(\x/pi)/2},{(sin(\x r)/2)});

      \draw (0,-3) -- (6,-3);
      \draw (0,-4) -- (0,-2);
      \draw[thick,complGreen] plot[domain=0:12*pi, samples=180] ({(\x/pi)/2},{-3 - (sin(\x r)/2)});
      
      \draw (0,-6) -- (6,-6);
      \draw (0,-7) -- (0,-5);
      \draw[thick] plot[domain=0:12*pi, samples=180] ({(\x/pi)/2},{-6});
    \end{tikzpicture}
    \caption{Destructive interference}
  \end{figure}
\end{minipage}

\section*{Securing a Signal}

When a signal is sent through any media, if a third party has access to the same media, they can intercept, block and
 even fake signals. This is because there is no inherent security in an electromagnetic wave, so we need to find ways of
 securing it.

\subsection*{Spread Spectrum}

If we were to transmit a signal using a wide range of the spectrum, rather than a single frequency, it makes it harder
 for a third party to entirely block the signal. This method is often used in military applications, as well as wired and
 wireless networks. Three common techniques are--

\begin{itemize}
  \item Frequency Hopping Spread Spectrum (FHSS)
  \begin{itemize}
    \item Data is sent using one of the previously discussed analogue transmission techniques, but the carrier frequency
     is changed rapidly across a wide spectrum, using a code or pattern known by both the transmitter and receiver
    \item This makes it harder to intercept or block transmissions, as long as the pattern of frequency hopping is not
     known
    \item It is also very resilient to narrowband interference (interference over a small band of frequencies)
  \end{itemize}
  \item Direct Sequence Spread Spectrum (DSSS)
  \begin{itemize}
    \item Each bit of the data is represented by multiple bits of the transmitted signal, using a spreading
     code
    \item The bits in the PN sequence are known as chips, and the sequence of them the chip sequence
    \item One technique is to combine the data with the spreading code bit stream using an exclusive-or
  \end{itemize}
  \item Code Division Multiple Access (CDMA)
\end{itemize}

\section*{Multiplexing}

\begin{definition*}{}{}
  \textbf{Multiplexing} is sending multiple data streams or signals over a single transmission medium.
\end{definition*}

\subsection*{Frequency Division}

With frequency division multiplexing, you have a transmission medium which can send data over a wide spectrum of frequencies,
 and multiple signals you need to send at the same time. If you modulate the different signals using a different carrier
 frequency for each, and leave a suitable gap between them, you can then combine the waves together to create a single
 composite wave which encodes the data of all the streams. The receiver then needs to know only the frequency of the
 signal they need to receive, and pass the composite wave through a filter for that specific frequency to recover the
 signal intended for them.

\subsection*{Time Division}

With time division multiplexing, each of the signals gets a certain share of time in which it can be transmitted. This
 often means that the data needs to be buffered on both ends, so may be more costly to implement. This also greatly
 limits the bandwidth available to each signal, as it would be split between all of the signals on the same channel.

\subsection*{Code Division Multiple Access (CDMA)}

CDMA is a multiplexing technique which can be used with a spread-spectrum signal. Given a data stream of bit rate $R$,
 we assign each bit a unique user code of $n$ according to a Walsh matrix. If a user, $k$, sends a 1, the transmitter
 sends the chip code $ck$, and if they send a 0, the transmitter sends the inverse of that chip code, $c'k$. The chip
 codes of all users will add up to a bipolar signal, $D$. The receiver decodes the signal for a specific user with a
 function, by taking the cartesian product of $D$ and the specific chip code $ck$, $D \times ck$. If $D \times ck = n$
 then a 1 is received, and if $D \times ck = -n$ then a 0 is received.

\begin{example*}{}{}
  We have 4 users, with chip codes as follows
  \begin{itemize}
    \item A:\ $(-1, -1, -1, -1)$
    \item B:\ $(-1, +1, -1, +1)$
    \item C:\ $(-1, -1, +1, +1)$
    \item D:\ $(-1, +1, +1, -1)$
  \end{itemize}

  If the users were to send $1, 1, 1, 0$ respectively, then we would add up the signals such that
   $A+B+C+D' = (-1, -1, -1, -1) + (-1, +1, -1, +1) + (-1, -1, +1, +1) + (+1, -1, -1, +1) = (-2, -2, -2, +2)$\\
  
  Then, the receivers filter out each individual signal by multiplying the received signal by the chip code of the signal
   they're interested in. To get back the signal for A, you would do the following
   $(-2, -2, -2, +2) \times (-1, -1, -1, -1) = (-2 \times -1) + (-2 \times -1) + (-2 \times -1) + (+2 \times -1) = 2+2+2-2 = 4$,
   which shows that A was sending a $1$.
\end{example*}