\lecture{Software-Defined Networks (SDN)}{11:00}{02/12/24}{Asim Ali}

\section*{VLANs}

VLANs are virtual networks made up of one or more physical LANs, combined together using switches to create a virtual
 LAN. This is useful for separating different types of devices on the same network, and making sure they don't conflict
 with each other.

\section*{Management, Control and Data Planes}

The Data plane is responsible for transmitting and receiving packets, as well as forwarding them from one device to
 another. It is also responsible for enforcing the routing rules decided upon by the Control plane. Uses common
 protocols such as TCP/IP and UDP. All functions relating to data reside within the Data plane.

The Control plane is responsible for routing packets between networks, naming devices, policy declaration (such as
 bandwidth and speeds, encryption and authentication), security and routing protocols such as OSPF, BGP and MPLS. It
 also decides upon the route packets should take from the source to the destination, which is enforced by the Data
 plane.

The Management plane is responsible for configuration and monitoring of network devices. Uses protocols such as SSH,
 Telnet and SNMP.

\section*{SDN Architecture}

The network is logically divided into three planes-- the Data, Control and Application planes. The data plane contains 
 the physical switches, routers and devices in the network. The control plane is made up of software which manages
 routing, traffic engineering and mobility, typically running on a physical machine somewhere on the network. The
 application plane contains the SDN applications, business applications and any cloud orchestration software.

The OpenFlow API is used to communicate between the SDN software and physical network devices, which performs the
 configuration and monitoring of devices, to form the network how the control plane decides. Because the control layer
 is decoupled from the physical devices, it has only a single interface, allowing administrators to easily and
 dynamically adjust the network to meet the needs of applications.

\subsection*{OpenFlow Switches}

The first packet of each new device is encapsulated and forwarded to the SDN controller, so routing decisions can be
 made. The switch forwards all incoming packets through the appropriate port using a flow table. Each flow can
 temporarily or permanently drop packets, as decided by the controller.

\subsection*{Message Types}

\begin{itemize}
  \item Controller-to-Switch-- Sent from the controller to one or all switches to configure the switch, or retrieve
   information from it
  \item Asynchronous (Switch-to-Controller)-- Sent from the switch to the controller to report packet information,
   state changes, error rates, and so on
  \item Symmetric (Switch-to-Controller or Controller-to-Switch)-- Messages sent without solicitation from either side,
   typically used for health check messages or to check that channels are available
\end{itemize}

\subsection*{Controller Actions}

\begin{itemize}
  \item Reactive Mode
  \begin{itemize}
    \item The controller is queried when a packet arrives that has no flow entry or an expired timer
    \item The controller creates and installs a rule to the flow table of switches on the path with rules for that
     specific packet
    \item The first switch then forwards the packet according to the new rule(s)
  \end{itemize}
  \item Proactive Mode
  \begin{itemize}
    \item The controller creates flow rules for all possible types of traffic
    \item These rules are installed statically ahead of time
    \item The controller is not queried since all incoming packets should find a flow entry
    \item This means that packets are switched at line-rate with no delay, but implementing the tables requires lots
     of expensive memory for moderately large tables
  \end{itemize}
\end{itemize}