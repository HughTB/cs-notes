\lecture{Bluetooth Networks}{11:00}{18/11/24}{Asim Ali}

The original purpose was to create a universal interface for using small, local ad-hoc networks. It was also intended as
 a replacement for irDA-- an infrared communication protocol.

The design specifications required a maximum range of $10 - 100m$, low power consumption, and a license free frequency
 to keep costs low. The specific frequency it uses is roughly $2.45\mathrm{GHz}$, which is within the license free
 frequency band also occupied by $2.4\mathrm{GHz}$ Wi-Fi.
 
\section*{Bluetooth Protocols}

Ordered from lowest to highest in the protocol stack, these are some of the main protocols used to control and
 communicate between bluetooth devices.

\subsection*{Radio}

Responsible for frequency hopping, modulation and demodulation, and managing transmission power.

\subsection*{Baseband}

Responsible for establishing connections, addressing different devices, formatting packets, timing messages and
 controlling transmission power.

\subsection*{Link Manager Protocol}

The Link Manager Protocol (LMP) is responsible for initiating and maintaining connections between devices,
 authentication and encryption, and is in the lower layers of the bluetooth protocol.

\subsection*{Logical Link Control and Adaption}

Adapts existing protocols to work over a baseband network, such as TCP/IP and telephony. It offers connectionless and
 connection-oriented services, similar to TCP and UDP.

\subsection*{Service Discovery Protocol}

Provides device information and service advertisement, which can be queried by other devices to initiate establishing a
 connection.

\section*{Piconet}

A \textbf{Piconet} is a collection of bluetooth devices connected in an ad-hoc manner. They form a star network with
 one device acting as the master, and up to 255 devices connected to it as clients. All devices in a piconet are
 synchronised to the same hopping sequence, and each piconet has it's own hopping pattern to avoid interfering with each
 other. Before a device is able to join a network, it has to synchronise with the hopping pattern.

While up to 255 devices can be in a piconet, only 7 devices can be actively communicating with the master at any time.
 All other devices are placed into a parked state, where they master is still aware they exist and may want to connect,
 but they cannot actively communicate.

Active clients can be in one of three states-- Active (Transmits and receives), Sniffing (Reduced number of slots at a
 reduced power) or Holding (Further reduced power, only supporting SCO links).

There is also a standby state, in which the device is in range of the piconet, but is not participating.

\section*{Radio Specifications}

{\Huge Include table from slide 9}

\section*{Baseband Specification}

Bluetooth baseband uses Frequency Hopping Spread Spectrum (FHSS) to make it less likely for interference to occur, and
 to make it harder to eavesdrop on communications. The hopping frequency for baseband is 1600 hops/sec, with a dwell
 time of $625$~{\textmu}sec, which is also the slot time. There are 79 carrier frequencies within the band, each of
 which is hopped to in a pre-determined but pseudorandom sequence, which further minimises both the likelihood and
 impact of interference.

Bluetooth devices use Time Division Duplex, where the same frequency is used for sending and receiving data, but not at
 the same time. Each slot is labelled from $0$ to $2^{27} - 1$, and loops. The master uses even slots, and clients share
 the odd slots using Time Division Multiple Access (TDMA). In general, Piconets use FH-TDD-TDMA (Frequency Hopping Time
 Division Duplex Time Division Multiple Access).

\section*{Scatternet}

Multiple piconets may exist in one area, and may form a scatternet. A bridge device connects two piconets together, and
 may be a client in both networks, or a master in one and client in another. The bridge acts as a relay between the two
 networks, and forwards messages between them. Clients transmit only with the permission of the master, using Time
 Division Multiplexing to access the medium.

\section*{Links Between Master and Clients}

\subsection*{Bluetooth Packet Format}

\begin{itemize}
  \item Access Code-- $72$ bits used for timing synchronisation, offset compensation, paging and inquiry
  \item Header-- $54$ bits used to identify packet type and contain protocol control information
  \item Payload-- $0-2745$ bits of actual data
\end{itemize}

\subsection*{Synchronous Connection Oriented (SCO)}

A fixed number of slots are assigned between point-to-point and master-client communication. The master maintains
 connections with all devices using reserved slots at a regular interval. Slots must be reserved, and the smallest unit
 of reservation is 2 consecutive slots-- one in each direction. The payload of each slot may be 80, 160 or 240 bits.
 The most reliable variant is 80 bits, at 800 slots/sec as the packets are replicated 3 times. The master can support
 three simultaneous links, if each of them uses the 80 bit variant. SCO communications are never re-transmitted, and
 error correction techniques are used at each end instead. Each variant uses a different type of Forward Error
 Correction (FEC), 80 using 1/3 FEC, 160 using 2/3 FEC and 240 using no FEC, where the fraction determines how much
 of the bandwidth is used for actual data. This is typically used for realtime data, video and audio.

\subsection*{Asynchronous Connectionless (ACL)}

There can be only a single ACL link at a time, and transmits in slots not already reserved for SCO communications. ACL
 links use 1, 3 or 5 slot packets, which are assigned by the master, depending upon the requirements of devices on the
 network. The first 259~{\textmu}sec of the slot are used as settling time, then 126 bits of header, and finally
 240, 1490 or 2740 bits of data, depending on the number of slots used. There is no bandwidth guaranteed, and
 retransmission may be required. A client may only reply to the master with an ACL slot, and may never initiate
 communication without the master having assigned slots. After every ACL slot, a single SCO slot is reserved for the
 client to reply to the master.