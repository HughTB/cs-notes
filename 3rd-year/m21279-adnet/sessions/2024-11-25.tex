\lecture{Cellular Networks}{11:00}{25/11/24}{Asim Ali}

In a cellular network, each cell contains a base station, a number of relay stations and a number of client devices,
 called user equipment. Each cell has a set bandwidth and number of available channels, each of which can be used to
 communicate with one device at a time. For example, if a base station has 10 channels, it can communicate directly with
 10 devices at once. In this context, a base station is very similar to that in a WLAN network, as it has a controller
 which handles communication within the cell's channels and a set of transceivers, one for each channel.

A relay station is a smaller station which has equipment to relay communications from devices to the base station. These
 are usually used near the edges of the cell, or in places with particularly poor reception, such as big concrete
 warehouses. This is especially needed in cells that span a larger area, since they can be anywhere from a few hundred
 to 30,000 metres in size.

A cellular network is made up of many cells in close proximity, which form an area of continuous coverage. Cells
 typically have omnidirectional transceivers, meaning they cover a circular area around them, and usually overlap with
 other cells around the edges. The space each cell actually covers is usually a hexagon, which is the most effective way
 to pack cells together, and the overlaps are ignored.

\subsection*{Cell Geometry}

The area of each cell is $2.6 \times R \times R$ or $2.6{R}^2$, and the distance between the centre of each cell is
 $d = \sqrt{3}R$. In this case, $R$ is the radius and $d$ the diameter of the circular coverage range of the cell, and
 the given area is the area of the theoretical hexagonal area it covers.

\section*{Frequency Reuse}

It is not possible to use the same frequency in directly adjacent cells, since the signals would interfere with each
 other. After a certain distance, $D$, it becomes possible to reuse frequencies since the cells do not have overlapping
 coverage areas. This distance can be calculated by $\frac{D}{d} = \sqrt{N}$ or replacing $d$, $\frac{D}{R} = \sqrt{SN}$
 where $N$ is the number of cells in a repeating pattern.

Two cells using the same frequency cause co-channel interference (CCI). The frequency reuse distance can be evaluated
 using the following method with two variables, $i$ and $j$.
\begin{itemize}
  \item Move $i$ steps from the reference cell in any direction
  \item Turn counter-clockwise by 60 degrees
  \item Move $j$ steps in that direction
  \item The number of cells in a cluster, $N = i^2 + j^2 + ij$
\end{itemize}

\section*{Calculating Network Capacity}

In a network with 32 cells, each of which have a radius of 1.6km, a total frequency bandwidth of 336 channels, and
 reuse factor of 7, how would you find the total coverage area and total channel capacity?

\begin{itemize}
  \item Area of each cell is $2.6 \times 1.6 \times 1.6 = 6.7{km}^2$ 
  \item Area of all cells is $6.7 \times 32 = 212.9{km}^2$
  \item Number of channels per cell is $\frac{336}{7} = 48$
  \item Total number of channels is $48 \times 32 = 1536$
\end{itemize}

\section*{Increasing Cell Capacity}

There are several ways to increase the capacity of an individual cell within a cellular network. One option is to simply
 add more channels, but this is not always possible. To increase capacity without increasing the number of channels,
 you can use
\begin{itemize}
  \item \textbf{Frequency Borrowing}-- Cells with lower usage can have their channels `borrowed' by other cells which
   are more congested.
  \item \textbf{Cell Splitting}-- Congested cells can be split into smaller cells, reducing the radius of a cell by a
   factor of $p$, and increasing the number of base stations by $p^2$, which increases the number of channels in the
   same area.
  \item \textbf{Cell Sectioning}-- Congested cells can be divided into a number of wedge-shaped sectors, each of which
   have their own set of channels transmitted using directional antennae such that they don't overlap.
  \item \textbf{MicroCells}-- Antennae are moved to buildings, lamp posts, etc, in cities and have their transmission
   power greatly reduced, such that they only cover a small area, but with far more cells to increase the capacity.
\end{itemize}

\section*{Cellular Network Operations}

Out of every 66 total channels, 11 are control channels and 55 are traffic channels. The client device listens to the
 control channels, and selects a base station to connect to, based on the relative strength of the control channels it
 can receive. Base stations are connected to a Mobile Telecommunications Switching Office (MTSO), which connects the
 cellular network to the main telephone system. The MTSO also assigns channels, performs hand-offs and monitors calls
 to meter bills. Each MTSO is usually responsible for more than one cell, and often one cluster of cells as defined by
 the frequency reuse pattern.

\section*{GSM Networks}

{\Huge diagram of GSM network?}

\subsection*{Channels}

Each GSM provider is typically assigned two 25MHz chunks of the spectrum, one for uplink channels and one for downlink
 channels. Within this, each channel is 200KHz, meaning that there can be up to 125 channels for uplink and downlink.
 This means that there are 125 full-duplex channels, each of which can be split up using 8-way TDMA, to get a total user
 count of 1000 in each cell.

\section*{2nd Generation, CDMA}

Up to 35 users per channel, using CDMA with a $64 \times 64$ Walsh Matrix.