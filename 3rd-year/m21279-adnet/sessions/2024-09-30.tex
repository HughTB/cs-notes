\lecture{Introduction}{11:00}{30/09/24}{Asim Ali}

\section*{Admin}

\begin{itemize}
  \item Lectures given by Dr Asim Ali
  \item Office - BK 2.20
  \begin{itemize}
    \item Monday 1300-1400
    \item Thursday 1100-1200
  \end{itemize}
  \item 2 hour lecture every week (PK 2.01)
  \item 1 hour seminar every week (AG 1.03)
\end{itemize}

The module information is available at the URL below:\\
\url{https://course-module-catalog.port.ac.uk/#/moduleDetail/M21279/2024\%2F25}

\section*{Assessments}

This module is assessed by one exam and one piece of coursework as below
\begin{itemize}
  \item 60\%, 90 minute Exam covering LO1,2,3\&4 - TB1 assessment period (January)
  \item 40\%, Group coursework covering LO5 - 2-4 members \textit{or} individual (but not recommended)
  \begin{itemize}
    \item 5 pages maximum
    \item Template provided
    \item Previous samples provided
  \end{itemize}
\end{itemize}



\lecture{Signal Encoding}{11:30}{30/09/24}{Asim Ali}

\section*{Signals}

(In this context) Signals are typically electromagnetic waves that carry information using some method of encoding.
 This means that the `shape' of the signal is changed to represent information. There are three main types of signal--

\begin{itemize}
  \item Analogue Signals, which vary continuously and smoothly over time
  \item Digital Signals, which have only 2 levels and change near instantly
  \item Discrete Signals, which have 2 or more levels and change near instantly
\end{itemize}

When information is transmitted, it is often converted between many different signals before it reaches the destination.
 For example, if you have ADSL internet, your computer sends the data as a digital signal to the modem, which is then
 converted to an analogue signal before it is transmitted across a POTS network (Plain Old Telephone System), then the
 receiver's modem converts it back to a digital signal before it is sent on to the receiver.

\section*{Digital Signal Encoding}

There are many methods of encoding a digital signal, but some of the most common are--

\begin{itemize}
  \item Non-return to Zero Level (NRZ-L)
  \begin{itemize}
    \item 0 -- High voltage level
    \item 1 -- Low voltage level
  \end{itemize}
  \item Non-return to Zero Inverted (NRZ-I)
  \begin{itemize}
    \item 0 -- No transition (high-low or low-high voltage level) on clock pulse
    \item 1 -- Transition on clock pulse
  \end{itemize}
  \item Bipolar-AMI
  \begin{itemize}
    \item 0 -- Zero voltage
    \item 1 -- Alternating positive or negative voltage level 
  \end{itemize}
  \item MLT-3 (Multi-Level Transmit 3)
  \begin{itemize}
    \item 0 -- Remain at the same voltage level
    \item 1 -- Transition to the next voltage level
    \item Uses 3 voltage levels, named $+1$,$0$\&$-1$ but can be any arbitrary voltages
  \end{itemize}
  \item Manchester
  \begin{itemize}
    \item 0 -- Transition from high-low voltage level in the middle of the clock interval
    \item 1 -- Transition from low-high voltage level
  \end{itemize}
  \item Differential Manchester
  \begin{itemize}
    \item 0 -- Transition (high-low or low-high voltage level) on clock pulse
    \item 1 -- No transition on clock pulse
    \item Always transitions in the middle of the clock interval  
  \end{itemize}
\end{itemize}

In this context, the high and low voltage levels can be whatever you like, but are typically either a positive and
 negative voltage of the same magnitude, or are a positive voltage and zero. The most common voltages used for
 signalling are $3.3$, $5$ and $12$ volts. In the case of Bipolar-AMI, the 3 voltage levels can be arbitrary, and don't
 have to be, for example $+5$, $0$ and $-5$ volts.

\subsection*{Diagrams}

Below is the binary string $01100110$ encoded using all of the above encoding schemes. This assumes that the previous
 state before the transmission starts is always low, and that each vertical dotted line represents a clock pulse.

\begin{minipage}[c]{0.47\linewidth}
  \begin{figure}[H]
    \centering
    \begin{tikzpicture}[scale=0.9]
      \draw[step=1cm,gray,very thin,dashed] (0.01,-0.99) grid (7.99,0.99);
      \draw (0,0) -- (8,0);
      \node at (0.5,1.25) {$0$}; \node at (1.5,1.25) {$1$}; \node at (2.5,1.25) {$1$}; \node at (3.5,1.25) {$0$};
      \node at (4.5,1.25) {$0$}; \node at (5.5,1.25) {$1$}; \node at (6.5,1.25) {$1$}; \node at (7.5,1.25) {$0$};
      \node at (-0.5,1) {$\mathrm{High}$}; \node at (-0.5,0) {$\mathrm{Zero}$}; \node at (-0.5,-1) {$\mathrm{Low}$};
    
      \draw[thick,myPurple] (0,-1) -- (1,-1) -- (1,1) -- (2,1) -- (3,1) -- (3,-1) -- (4,-1) -- (5,-1) -- (5,1) -- (6,1) -- (7,1) -- (7,-1) -- (8,-1);
    \end{tikzpicture}
    \caption{Non-return to Zero Level (NRZ-L)}
  \end{figure}
\end{minipage}\hfill
\begin{minipage}{0.47\linewidth}
  \begin{figure}[H]
    \centering
    \begin{tikzpicture}[scale=0.9]
      \draw[step=1cm,gray,very thin,dashed] (0.01,-0.99) grid (7.99,0.99);
      \draw (0,0) -- (8,0);
      \node at (0.5,1.25) {$0$}; \node at (1.5,1.25) {$1$}; \node at (2.5,1.25) {$1$}; \node at (3.5,1.25) {$0$};
      \node at (4.5,1.25) {$0$}; \node at (5.5,1.25) {$1$}; \node at (6.5,1.25) {$1$}; \node at (7.5,1.25) {$0$};
      \node at (-0.5,1) {$\mathrm{High}$}; \node at (-0.5,0) {$\mathrm{Zero}$}; \node at (-0.5,-1) {$\mathrm{Low}$};
    
      \draw[thick,myPurple] (0,-1) -- (1,-1) -- (1,1) -- (2,1) -- (2,-1) -- (5,-1) -- (5,1) -- (6,1) -- (6,-1) -- (8,-1);
    \end{tikzpicture}
    \caption{Non-return to Zero Inverted (NRZ-I)}
  \end{figure}
\end{minipage}

\begin{minipage}[c]{0.47\linewidth}
  \begin{figure}[H]
    \centering
    \begin{tikzpicture}[scale=0.9]
      \draw[step=1cm,gray,very thin,dashed] (0.01,-0.99) grid (7.99,0.99);
      \draw (0,0) -- (8,0);
      \node at (0.5,1.25) {$0$}; \node at (1.5,1.25) {$1$}; \node at (2.5,1.25) {$1$}; \node at (3.5,1.25) {$0$};
      \node at (4.5,1.25) {$0$}; \node at (5.5,1.25) {$1$}; \node at (6.5,1.25) {$1$}; \node at (7.5,1.25) {$0$};
      \node at (-0.5,1) {$\mathrm{High}$}; \node at (-0.5,0) {$\mathrm{Zero}$}; \node at (-0.5,-1) {$\mathrm{Low}$};
    
      \draw[thick,myPurple] (0,-1) -- (0,0) -- (1,0) -- (1,1) -- (2,1) -- (2,-1) -- (3,-1) -- (3,0) -- (5,0) -- (5,1) -- (6,1) -- (6,-1) -- (7,-1) -- (7,0) -- (8,0);
    \end{tikzpicture}
    \caption{Bipolar-AMI}
  \end{figure}
\end{minipage}\hfill
\begin{minipage}{0.47\linewidth}
  \begin{figure}[H]
    \centering
    \begin{tikzpicture}[scale=0.9]
      \draw[step=1cm,gray,very thin,dashed] (0.01,-0.99) grid (7.99,0.99);
      \draw (0,0) -- (8,0);
      \node at (0.5,1.25) {$0$}; \node at (1.5,1.25) {$1$}; \node at (2.5,1.25) {$1$}; \node at (3.5,1.25) {$0$};
      \node at (4.5,1.25) {$0$}; \node at (5.5,1.25) {$1$}; \node at (6.5,1.25) {$1$}; \node at (7.5,1.25) {$0$};
      \node at (-0.5,1) {$\mathrm{High}$}; \node at (-0.5,0) {$\mathrm{Zero}$}; \node at (-0.5,-1) {$\mathrm{Low}$};
    
      \draw[thick,myPurple] (0,-1) -- (1,-1) -- (1,0) -- (2,0) -- (2,1) -- (5,1) -- (5,0) -- (6,0) -- (6,-1) -- (8,-1);
    \end{tikzpicture}
    \caption{Multi-Level Transmit 3 (MLT-3)}
  \end{figure}
\end{minipage}

\begin{minipage}[c]{0.47\linewidth}
  \begin{figure}[H]
    \centering
    \begin{tikzpicture}[scale=0.9]
      \draw[step=1cm,gray,very thin,dashed] (0.01,-0.99) grid (7.99,0.99);
      \draw (0,0) -- (8,0);
      \node at (0.5,1.25) {$0$}; \node at (1.5,1.25) {$1$}; \node at (2.5,1.25) {$1$}; \node at (3.5,1.25) {$0$};
      \node at (4.5,1.25) {$0$}; \node at (5.5,1.25) {$1$}; \node at (6.5,1.25) {$1$}; \node at (7.5,1.25) {$0$};
      \node at (-0.5,1) {$\mathrm{High}$}; \node at (-0.5,0) {$\mathrm{Zero}$}; \node at (-0.5,-1) {$\mathrm{Low}$};
    
      \draw[thick,myPurple] (0,-1) -- (0,1) -- (0.5,1) -- (0.5,-1) -- (1.5,-1) -- (1.5,1) -- (2,1) -- (2,-1) -- (2.5,-1) -- (2.5,1) -- (3.5,1) -- (3.5,-1) -- (4,-1) -- (4,1) -- (4.5,1) -- (4.5,-1) -- (5.5,-1) -- (5.5,1) -- (6,1) -- (6,-1) -- (6.5,-1) -- (6.5,1) -- (7.5,1) -- (7.5,-1) -- (8,-1);
    \end{tikzpicture}
    \caption{Manchester}
  \end{figure}
\end{minipage}\hfill
\begin{minipage}{0.47\linewidth}
  \begin{figure}[H]
    \centering
    \begin{tikzpicture}[scale=0.9]
      \draw[step=1cm,gray,very thin,dashed] (0.01,-0.99) grid (7.99,0.99);
      \draw (0,0) -- (8,0);
      \node at (0.5,1.25) {$0$}; \node at (1.5,1.25) {$1$}; \node at (2.5,1.25) {$1$}; \node at (3.5,1.25) {$0$};
      \node at (4.5,1.25) {$0$}; \node at (5.5,1.25) {$1$}; \node at (6.5,1.25) {$1$}; \node at (7.5,1.25) {$0$};
      \node at (-0.5,1) {$\mathrm{High}$}; \node at (-0.5,0) {$\mathrm{Zero}$}; \node at (-0.5,-1) {$\mathrm{Low}$};
    
      \draw[thick,myPurple] (0,-1) -- (0,1) -- (0.5,1) -- (0.5,-1) -- (1.5,-1) -- (1.5,1) -- (2.5,1) -- (2.5,-1) -- (3,-1) -- (3,1) -- (3.5,1) -- (3.5,-1) -- (4,-1) -- (4,1) -- (4.5,1) -- (4.5,-1) -- (5.5,-1) -- (5.5,1) -- (6.5,1) -- (6.5,-1) -- (7,-1) -- (7,1) -- (7.5,1) -- (7.5,-1) -- (8,-1);
    \end{tikzpicture}
    \caption{Differential Manchester}
  \end{figure}
\end{minipage}

\section*{Carrier Waves and Modulation}

A carrier wave is a continuous waveform that, on it's own, carries no information. This carrier wave is then modified
 by another signal to convey information. This modification can be of it's amplitude, frequency, phase, or a combination
 of all 3. This is known as modulation, hence the names AM (Amplitude Modulation) and FM (Frequency Modulation) for the
 two types of analogue radio.

\subsection*{Digital-to-Analogue Encoding}

There are several methods of encoding a digital signal on an analogue transmission medium. In the case of sending
 digital data over a POTS network, the signal is encoded onto a carrier wave in the range of $300$-$3400$Hz using a
 MoDem (Modulator-Demodulator). The main methods of encoding are--

\begin{itemize}
  \item Amplitude-Shift keying (ASK)
  \item Frequency-Shift keying (FSK)
  \begin{itemize}
    \item Binary FSK (BFSK)
    \item Multiple FSK (MFSK)
  \end{itemize}
  \item Phase-Shift keying (PSK)
  \begin{itemize}
    \item Binary PSK -- 1 = A sine wave, 0 = The same sine wave shifted by $180^{\circ}$
    \item Differential PSK (DPSK) -- 1 = The previous sine wave shifted by $180^{\circ}$, 0 = The previous sine wave
    \item Multiple-level PSK
  \end{itemize}
  \item Quadratic AM (Combination of ASK and PSK)
\end{itemize}

The sender and receiver may use different frequencies as to allow full-duplex data transmission on the same physical
 medium or radio channel. Full-duplex meaning full-speed transmission in both directions simultaneously.

\subsection*{Diagrams}      

\begin{minipage}[c]{0.47\linewidth}
  \begin{figure}[H]
    \centering
    \begin{tikzpicture}[scale=0.9]
      \draw[step=1cm,gray,very thin,dashed] (0.01,-0.99) grid (7.99,0.99);
      \draw (0,0) -- (8,0);
      \draw (0,-1) -- (0,1);
      \node at (0.5,1.25) {$0$}; \node at (1.5,1.25) {$1$}; \node at (2.5,1.25) {$1$}; \node at (3.5,1.25) {$0$};
      \node at (4.5,1.25) {$0$}; \node at (5.5,1.25) {$1$}; \node at (6.5,1.25) {$1$}; \node at (7.5,1.25) {$0$};
    
      \draw[thick,myPurple] plot[domain=0:2*pi, samples=30] ({(\x/pi)/2},{sin(\x r)});
      \draw[thick,myPurple] plot[domain=2*pi:6*pi, samples=60] ({(\x/pi)/2},{sin(\x r)/2});
      \draw[thick,myPurple] plot[domain=6*pi:10*pi, samples=60] ({(\x/pi)/2},{sin(\x r)});
      \draw[thick,myPurple] plot[domain=10*pi:14*pi, samples=60] ({(\x/pi)/2},{sin(\x r)/2});
      \draw[thick,myPurple] plot[domain=14*pi:16*pi, samples=30] ({(\x/pi)/2},{sin(\x r)});
    \end{tikzpicture}
    \caption{Amplitude-Shift Keying}
  \end{figure}
\end{minipage}\hfill
\begin{minipage}{0.47\linewidth}
  \begin{figure}[H]
    \centering
    \begin{tikzpicture}[scale=0.9]
      \draw[step=1cm,gray,very thin,dashed] (0.01,-0.99) grid (7.99,0.99);
      \draw (0,0) -- (8,0);
      \draw (0,-1) -- (0,1);
      \node at (0.5,1.25) {$0$}; \node at (1.5,1.25) {$1$}; \node at (2.5,1.25) {$1$}; \node at (3.5,1.25) {$0$};
      \node at (4.5,1.25) {$0$}; \node at (5.5,1.25) {$1$}; \node at (6.5,1.25) {$1$}; \node at (7.5,1.25) {$0$};
    
      \draw[thick,myPurple] plot[domain=0:2*pi, samples=30] ({(\x/pi)/2},{sin(\x r)});
      \draw[thick,myPurple] plot[domain=4*pi:12*pi, samples=60] ({(\x/pi/4)},{sin(\x r)});
      \draw[thick,myPurple] plot[domain=6*pi:10*pi, samples=60] ({(\x/pi)/2},{sin(\x r)});
      \draw[thick,myPurple] plot[domain=20*pi:28*pi, samples=60] ({(\x/pi/4)},{sin(\x r)});
      \draw[thick,myPurple] plot[domain=14*pi:16*pi, samples=30] ({(\x/pi)/2},{sin(\x r)});
    \end{tikzpicture}
    \caption{Binary Frequency-Shift Keying}
  \end{figure}
\end{minipage}

\begin{minipage}[c]{0.47\linewidth}
  \begin{figure}[H]
    \centering
    \begin{tikzpicture}[scale=0.9]
      \draw[step=1cm,gray,very thin,dashed] (0.01,-0.99) grid (7.99,0.99);
      \draw (0,0) -- (8,0);
      \draw (0,-1) -- (0,1);
      \node at (0.5,1.25) {$0$}; \node at (1.5,1.25) {$1$}; \node at (2.5,1.25) {$1$}; \node at (3.5,1.25) {$0$};
      \node at (4.5,1.25) {$0$}; \node at (5.5,1.25) {$1$}; \node at (6.5,1.25) {$1$}; \node at (7.5,1.25) {$0$};
    
      \draw[thick,myPurple] plot[domain=0:2*pi, samples=30] ({(\x/pi)/2},{0 - sin(\x r)});
      \draw[thick,myPurple] plot[domain=2*pi:6*pi, samples=60] ({(\x/pi)/2},{sin(\x r)});
      \draw[thick,myPurple] plot[domain=6*pi:10*pi, samples=60] ({(\x/pi)/2},{0 - sin(\x r)});
      \draw[thick,myPurple] plot[domain=10*pi:14*pi, samples=60] ({(\x/pi)/2},{sin(\x r)});
      \draw[thick,myPurple] plot[domain=14*pi:16*pi, samples=60] ({(\x/pi)/2},{0 - sin(\x r)});
    \end{tikzpicture}
    \caption{Binary Phase-Shift Keying}
  \end{figure}
\end{minipage}\hfill
\begin{minipage}{0.47\linewidth}
  \begin{figure}[H]
    \centering
    \begin{tikzpicture}[scale=0.9]
      \draw[step=1cm,gray,very thin,dashed] (0.01,-0.99) grid (7.99,0.99);
      \draw (0,0) -- (8,0);
      \draw (0,-1) -- (0,1);
      \node at (0.5,1.25) {$0$}; \node at (1.5,1.25) {$1$}; \node at (2.5,1.25) {$1$}; \node at (3.5,1.25) {$0$};
      \node at (4.5,1.25) {$0$}; \node at (5.5,1.25) {$1$}; \node at (6.5,1.25) {$1$}; \node at (7.5,1.25) {$0$};
    
      \draw[thick,myPurple] plot[domain=0:2*pi, samples=30] ({(\x/pi)/2},{0 - sin(\x r)});
      \draw[thick,myPurple] plot[domain=2*pi:4*pi, samples=30] ({(\x/pi)/2},{sin(\x r)});
      \draw[thick,myPurple] plot[domain=4*pi:10*pi, samples=90] ({(\x/pi)/2},{0 - sin(\x r)});
      \draw[thick,myPurple] plot[domain=10*pi:12*pi, samples=30] ({(\x/pi)/2},{sin(\x r)});
      \draw[thick,myPurple] plot[domain=12*pi:16*pi, samples=60] ({(\x/pi)/2},{0 - sin(\x r)});
    \end{tikzpicture}
    \caption{Differential Phase-Shift Keying}
  \end{figure}
\end{minipage}