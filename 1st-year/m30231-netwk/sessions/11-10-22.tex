\lecture{Network Protocols}{09:00}{11/10/22}{Amanda Peart}

Protocols are the rules for communication.
They define the rules that are used to communicate between devices, applications or components of an application 

What if conditions:
\begin{itemize}
  \item Networks protocols define the behavior for a "what if" condition. e.g. missing packets, bit flips, receiver dropping packets due to limited processing power, etc
  \item This behavior could be anything from ignoring it and continuing, or resending the entire message, depending upon the protocol
\end{itemize}

\subsection*{An example}

\begin{itemize}
  \item Consider the problems that early telegraph operators would have faced
  \item 2 train stations have a telegraph line between them
  \item There are 10 telegraphs to send in the morning
\end{itemize}

The first problem:
\begin{itemize}
  \item Should you just send a random telegram at any time?
  \item Should you send the shortest telegram first, or send them in a specific order?
  \item What if there's no one at the other end? Should there be a special "are you there" message before the actual telegram?
\end{itemize}

The second problem:
\begin{itemize}
  \item Should you send the telegrams immediately after each other?
  \item Should you receive an acknowledgement from the other end after each?
  \item Should there be a break between telegrams?
\end{itemize}

The third problem:
\begin{itemize}
  \item What if A is sending faster than B can receive?
  \item What if B has to stop receiving telegrams to do something else?
  \item What if you finish your shift but there are still telegrams to be sent or received?
  \item What if both A and B send at the same time?
\end{itemize}

These are all problems that are faced in a modern network, and are the reason that standardised protocols are so important

\section*{Connection-oriented protocols}

\begin{itemize}
  \item TCP is an example of a connection-oriented protocol
  \item A connection-oriented protocol is any protocol where there is a "private network" that directly links the sender and receiver
  \begin{itemize}
    \item They work similarly to a phone call as there is a "virtual cable" directly connecting between the sender and receiver
    \item Connection established
    \item Data sent between devices
    \item Connection closed
  \end{itemize}
\end{itemize}

\section*{Connectionless protocols}

\begin{itemize}
  \item Less assurance that the message got to the receiver
  \item No connection established and therefore no disconnection
  \item IP is an example of a connectionless protocol
\end{itemize}

Tradeoffs of connectionless vs connection-oriented protocols:
\begin{itemize}
  \item Connectionless protocols don't need to establish or clear a connection
  \item Packets in connectionless protocols are more wasteful of bandwidth, as they need to have additional addressing information in the metadata of every packet, which adds up quickly, whereas a connection-oriented protocol only needs the virtual cable id added to each message
  \item Packets can be easily discarded if the network is too busy, whereas a virtual cable must be carefully managed
\end{itemize}

\subsection*{Why do we need TCP/IP?}

\begin{itemize}
  \item If we just sent out packets without the protocol, they would just get lost on the network
  \item For example, if we wanted to send a packet across the internet between LANs, each router along the "journey" would read the desired IP address from the packet and relay the packet to the next router, getting closer to the receiver with each hop
  \item IP addressing is needed to route the packets across the internet
  \item TCP is needed to assure that the packets are all received, uncorrupted and in the correct order to ensure that all of the data is correct
  \item Every message sent across the internet uses at least these two protocols (TCP \& IP) but usually also use other protocols within the message itself so the receiver knows how to interpret the message, for example an email may use SMTP or POP as well as TCP/IP
\end{itemize}