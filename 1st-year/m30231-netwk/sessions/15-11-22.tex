\lecture{Standards and the OSI Model}{09:00}{15/11/22}{Amanda Peart}

\begin{itemize}
  \item Development of the OSI model started in 1977, with a draft published in 1979 and finalised in 1984 as an international standard
  \item OSI stands for Open Systems Interconnection
  \item The OSI model provides common terminology as well as a framework for networking
  \item The standard is still used today, and is the standard model for inter-computer communication
  \item It describes how data is sent from an application, through a network medium, and into another application, on a different computer or network
  \item This data transmission is split into the 7 layers of the OSI model
  \item Each layer has a specific function that it performs before sending the data to the next layer
  \item The upper 3 layers provide services to the application, while the lower 4 deal with the actual transmission from one device to another
  \item There are 7 layers on the way down, and 7 on the way up
\end{itemize}

\begin{center}
  \begin{tabular}{ |p{0.075\linewidth}|p{0.2\linewidth}|p{0.725\linewidth}| }
    \hline
    Layer & Name & Purpose \\
    \hline
    7 & Application Layer & Provides support for email, file transfer and other protocols \\
    6 & Presentation Layer & Ensures that the data is in the correct format, and is where any encryption will occur \\
    5 & Session Layer & Maintains the connection and controls ports and sessions \\
    4 & Transport Layer & Transmits data using TCP and / or UDP \\
    3 & Network Layer & Provides IP addressing, routing and segmentation \\
    2 & Data link Layer & Defines how the data is formatted when it is sent over the network \\
    1 & Physical Layer & Adapts the data to be sent over the medium (Fibre transceivers, etc) \\
    \hline
  \end{tabular}
\end{center}
  
\section*{Layers in more detail}

\begin{itemize}
  \item Layer 1 - Physical Link
  \begin{itemize}
    \item Deals with the physical communication over the medium
    \item It defines the specification of communication between the physical link on the sender and receiver
    \item Defines characteristics such as 
    \begin{itemize}
      \item Voltage levels
      \item Timing of voltage changes
      \item Physical data rates
      \item Maximum transmission distance
      \item Physical connectors (e.g. RJ45, TIA-232 aka RS-232)
    \end{itemize}
  \end{itemize}
  \item Layer 2 - Data Link
  \begin{itemize}
    \item Deals with transmission across the medium
    \item Provides the location of the intended destination on the network
    \item Can provide reliable transmission using MAC (Media Access Control) addresses
    \item Uses MAC addresses to differentiate between the different nodes connected to the same physical medium
    \item This layer deals with network topology and access, error handling, ordered delivery of frames, and flow control
    \item Standardised protocols such as Ethernet, Frame Relay and FDDI
  \end{itemize}
  \item Layer 3 - Network
  \begin{itemize}
    \item Defines the logical addressing
    \item Sets how routing works and how routes are learned or discovered so that packets can be delivered
    \item Also defines how packets could be split into smaller packets to be delivered more efficiently on different media
    \item Routers operate on this layer
  \end{itemize}
  \item Layer 4 - Transport
  \begin{itemize}
    \item Regulates the flow of data to ensure end-to-end connectivity
    \item Segments the data into packets on the sending host, and reassembles them on the receiving host
    \item Protocols on this layer include TCP and UDP
  \end{itemize}
  \item Layer 5 - Session
  \begin{itemize}
    \item Defines how to start, control and end connections (or sessions) between applications
    \item Uses "dialogue control" for management of bi-directional communication
    \item Synchronises dialogue between the presentation layers and manages their data exchange
    \item Allows for efficient data transfer
  \end{itemize}
  \item Layer 6 - Presentation
  \begin{itemize}
    \item Ensures that the data sent by the application is readable by the application layer on the receiving device
    \item Translates between different data formats using a common format
    \item Provides encryption and compression of data
  \end{itemize}
  \item Layer 7 - Application
  \begin{itemize}
    \item This layer is closest to the user
    \item Provides network services to the user's applications
    \item Does not provide services to any other OSI layer, only to applications
    \item Checks if the receiver is available to receive data
    \item Synchronises and agrees upon procedures or protocols for error handling and control of data integrity
  \end{itemize}
\end{itemize}

\section*{Connection and connectionless transport}

\begin{itemize}
  \item Connection-oriented transport such as TCP is used when the data needs to arrive intact and in the right order
  \item Connectionless transport such as UDP is used when the application is capable of data integrity control
  \begin{itemize}
    \item They can do this by repeating the request after a timeout
    \item This can sometimes cause duplicate operations if the response was delayed or just not received
    \item Common uses for UDP are Broadcasting and real-time VoIP applications
  \end{itemize}
\end{itemize}

\section*{The importance of standards}

\begin{itemize}
  \item The use of open standards is fundamental to Open Systems
  \item Needed to maintain interoperability between devices made by different vendors
  \item Standards should be internationally recognised
  \item It's important to track new standards in order to know when it is "safe" to use a new standard
  \item However, the creation of standards can take many years, and by the time the standards are released, the device that would've used it would be obsolete
  \item 'Fast tracking' can be used to develop the devices and standards in parallel
  \item When using fast tracking, vendors often end up releasing products before the standards are released
\end{itemize}

\section*{Important standards organisations}

\begin{itemize}
  \item ISO - International Standardisation Organisation
  \item ETSI - European Telecommunications Standards Institute
  \item IETF - The TCP/IP Internet Engineering task force
  \item IEEE - Institute of Electrical and Electronics Engineers
  \item ANSI - American National Standards Institute
\end{itemize}