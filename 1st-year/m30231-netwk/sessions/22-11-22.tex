\lecture{Communication Media}{09:00}{22/11/22}{Amanda Peart}

\begin{itemize}
  \item Different types of media can all transmit the same data
  \item Cables are in layer 1 of the OSI model
  \item There are 3 main types of cabling:
  \begin{itemize}
    \item Twisted-pair cabling
    \item Coaxial cabling
    \item Fibre-optic cabling
  \end{itemize}
  \item In a wired network, there is usually a networking rack on each floor, that contains a patch panel or switch, which all of the network ports on that floor are connected to
  \item The network designer has to decide which sort of media to use for each connection, depending upon
  \begin{itemize}
    \item The required bandwidth (including future growth)
    \item The level of electrical interference
    \item The maximum length of cabling that will be needed
    \item Cost of the media
  \end{itemize}
  \item Unshielded twisted-pair cables
  \begin{itemize}
    \item The least expensive type of media
    \item Can be used up to 100m
    \item Data capacity defined by EIA/TIA 568
    \item Cat3 supports up to 10Mbps
    \item Cat4 20Mbps
    \item Cat5 100Mbps
    \item Cat5e, Cat6, Cat6a and above are used for 1000Mbps and above
    \item A UTP cable consists of 8 conductors twisted into 4 pairs
    \item They are terminated in an RJ-45 connector
  \end{itemize}
  \item Multiplexing can be used to combine multiple signals to be sent across one physical medium
  \item This can be used to reduce the number of cables needed
  \item Cat6 cabling
  \begin{itemize}
    \item The newest type of UTP cabling
    \item There are few differences between Cat6 and Cat5e, mostly increasing the quality of signal
    \item There are 2 forms of Cat6
    \begin{itemize}
      \item UTP or ScTP (Screened Twisted Pair)
      \item ScTP has an additional layer of metallic foil to improve its resilience to interference
      \item Cat7 and Cat8 are SSTP (Screen Shielded Twisted Pair) or SFTP (Screened Foiled Twisted Pair)
    \end{itemize}
  \end{itemize}
  \item Coaxial cabling
  \begin{itemize}
    \item Low noise (low error rate)
    \item Used to be used in a variety of applications
    \begin{itemize}
      \item In IBM networks
      \item In early ethernet (limited to 10Mbps)
    \end{itemize}
    \item The shielding may include multiple layers of foil or braid
  \end{itemize}
  \item Fibre-Optic cabling
  \begin{itemize}
    \item Used for extremely high bandwidths
    \begin{itemize}
      \item Up to multiple Terabits per second, if using high-grade fibre
      \item Many times more bandwidth than typical twisted-pair cabling
    \end{itemize}
    \item Typically formed of two individual fibres, each of which transmits only in one direction
    \item They need to convert the electrical signals to optical signals and back
    \item Not susceptible to electromagnetic interference as signals are sent as light, not electrical signals
    \item Very thin glass strands
    \begin{itemize}
      \item Multimode fibre is on the order of 50 microns
      \item Singlemode fibre is on the order of 10 microns
    \end{itemize}
    \item The actual fibre cabling costs roughly the same as high-grade twisted pair cabling, but the connectors needed on either end are rather expensive, depending upon the type of connector, and the device it's connecting to
    \item Media converters are needed on each end of the fibre cable, and depending upon the device it's connecting to, and the bandwidth needed, they can be rather expensive
  \end{itemize}
  \item Additionally, wireless communication may be used, such as
  \begin{itemize}
    \item Television and Radio
    \item Satellite Comms
    \item Radar
    \item Mobile Telephone System (Cellular Communication)
    \item GPS (Global Positioning System)
    \item Infrared Communication (needs line of sight and has low bandwidth)
    \item WLAN (Wi-Fi) - IEEE 802.11
    \item Bluetooth
    \item Cordless landline phones
    \item RFID (Radio Frequency Identification)
    \item NFC (Near Field Communication)
  \end{itemize}
\end{itemize}