\practical{Riverbed Simulation Results}{12:00}{16/11/22}{Athanasios Paraskelidis}

\section*{Understanding the results}

\begin{itemize}
  \item The traffic sent was the same, regardless of whether a Switch or Hub was used
  \item The traffic received in the Hub scenario was less than in the Switch scenario
  \item This is known as packet loss (The data received is less than what was sent)
  \item The average delay in the Switch scenario was almost (but not quite) 0, whereas the delay in the Hub scenario was roughly 14 times higher
\end{itemize}

\subsection*{Traffic generation}

\begin{itemize}
  \item How much data is being generated by each node?
  \begin{itemize}
    \item The Interarrival time is 0.02 seconds - so 50 packets are sent each second
    \item The packet size is 1500 bytes
    \item So each packet is 1500 * 8 = 12000 bits
    \item At 50 packets per second, and each packet being 12000 bits, there is 50 * 12000 = 600000 bps
    \item This is the same as 600000 / 1000 = 600 Kbps
  \end{itemize}
  \item Therefore, with 16 nodes transmitting at 600 kbps, the total network traffic is 16 * 600 = 9600 Kbps
  \begin{itemize}
    \item This is the same as 9600 / 1000 = 9.6 Mbps
  \end{itemize}
  \item Clearly, this is too much traffic for the Hub to handle, but the Switch is able to handle it with ease
  \item The capacity of the network is usually limited by the medium being used to connect the devices on the network
  \begin{itemize}
    \item e.g. Cat-5 has a capacity of 100 Mbps, Cat-5E or Cat-6 has a capacity of 1 Gbps
  \end{itemize}
  \item The cabling used in networking can vary massively depending upon the requirements
  \item Ethernet standards are named as such:
  \begin{itemize}
    \item 10BaseT
    \begin{itemize}
      \item The 10 means 10 Mbps
      \item Base means Baseband communication (This is outside the scope of the module)
      \item T means twisted pair cabling, meaning that there are 8 conductors, twisted into 4 pairs
    \end{itemize}
    \item 100BaseT is the same concept, but 100 Mbps
    \item 1000BaseT is the same again, but 1000 Mbps or 1 Gbps
    \item 10GBaseT is the similar, but in this case it means 10000 Mbps or 10 Gbps
  \end{itemize}
\end{itemize}

\begin{center}
  \begin{tabular}{ |p{0.3\linewidth}|p{0.3\linewidth}|p{0.3\linewidth}| }
    \hline
     & Hub & Switch \\
    \hline
    How are packets sent to their intended destination? & The packets are broadcast to all nodes connected to the hub, and only the node that needed the packet responds to it & The packets are sent to only the node that needs it, all other nodes do not see the packet at all \\
    \hline
    Does the device support half or full-duplex? & Half-Duplex & Full-Duplex \\
    \hline
    What are the advantages of using it? &   & Each node has it's own collision domain that contains only itself and the switch, so collisions are very rare \\
    \hline
    What are the disadvantages of using it? & All of the nodes connected to the hub are in one single collision domain, so collisions effect all nodes &   \\
    \hline
  \end{tabular}
\end{center}