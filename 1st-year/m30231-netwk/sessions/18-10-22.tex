\lecture{More Protocols}{09:00}{18/10/22}{Amanda Peart}

\begin{itemize}
  \item Rules for sending and receiving data across a network
  \item Provides addressing
  \item Management and verification of transmission
  \item Often used in conjunction with other protocols, such as TCP and IP
\end{itemize}

\section*{Connection Orientated}

\begin{itemize}
  \item Similar to phoning someone on a landline
  \item 3 phases
  \begin{itemize}
    \item Connection setup
    \item Open connection
    \item (Send data)
    \item Close connection
  \end{itemize}
  \item Quality of Service
  \begin{itemize}
    \item High quality of service
    \item Low fixed delay between sender and receiver
    \item Limited packet loss
  \end{itemize}
\end{itemize}

\section*{Connectionless}

\begin{itemize}
  \item Similar to sending a letter in the post
  \item Each packet (letter) has an address attached before it is sent over the network (put in the post box)
  \item Once it is sent, you just have to assume that it was received
  \item Quality of Service
  \begin{itemize}
    \item Variable delay between sender and receiver
    \item Packets can and will be lost
    \item Issues with packets arriving in the wrong order
  \end{itemize}
\end{itemize}

\section*{Packets}

\begin{itemize}
  \item A packet is a single unit of data that is sent across a network - The size of the packets is determined by the sender
  \item Data is broken down into packets before it is sent across the network
  \item Examples of data that is sent across the Internet using packets:
  \begin{itemize}
    \item Emails - SMTP (Simple Mail Transfer Protocol) or POP3 (Post Office Protocol 3)
    \item Files - FTP (File Transfer Protocol)
    \item Web pages and images - HTTP (Hyper Text Transfer Protocol)
  \end{itemize}
  \item Each packet also contains header information - This could be compared to the address written on the front of the envelope in the postal analogy
  \begin{itemize}
    \item This includes the IP address of both the sender and receiver
    \item It also includes information on how to handle transmission errors
    \item Header information is used by routers and switches to determine where the packet should be sent next
  \end{itemize}
  \item Routers are devices dedicated to reading header information and relaying packets to the next router
  \item Packets move from router to router until they reach their final destination - Similar to sorting offices in the postal analogy
  \item Each packet of a communication may not necessarily follow the same route to their destination
  \item The route is determined by the router, and which path is the fastest or least congested at the time, which can change between packets
\end{itemize}

\section*{TCP/IP}

\begin{itemize}
  \item TCP/IP is a connectionless protocol, which is actually made up of 2 protocols, and is used almost everywhere on the internet
  \item TCP = Transmission Control Protocol
  \begin{itemize}
    \item Breaks up the data into packets that are easier for the network to handle
    \item Verifies that all of the packets arrive at the destination
    \item Re-orders the packets into the correct sequence to get the data back out again
    \item If any packets are damaged, TCP will request them to be resent
    \item It also acknowledges that all of the packets have been received successfully
  \end{itemize}
  \item IP = Internet Protocol
  \begin{itemize}
    \item Breaks the data into packets
    \item Adds the header information into each packet
    \item Determines how much data should be put into each packet
  \end{itemize}
\end{itemize}
  
For example, sending an email:
\begin{itemize}
  \item The data that makes up the email message is split up into packets by the IP (Internet Protocol)
  \begin{itemize}
    \item Header data is also added to each packet
  \end{itemize}
  \item Using the header information in each packet, the routers and switches that make up the Internet determine the best path for each packet to take to their final destination
  \item TCP (Transmission Control Protocol) then reassembles the packets into the correct order and ensures that all packets were received undamaged, then extracts the email message data from the packets
\end{itemize}