\lecture{Network Security Cont.d}{09:00}{21/02/23}{Amanda Peart}

\subsection*{Vulnerabilities}

\begin{itemize}
  \item Remote attacks
  \item Software developed with "back doors"
  \item Insecure configuration
  \item Internal attacks (Disgruntled employees)
  \item Access control
  \item Connecting compromised personal devices to secure networks
\end{itemize}

\subsection*{Security Management}

\begin{itemize}
  \item Control and Distribution
  \begin{itemize}
    \item Control who can and cannot access files
    \item Control where files can be downloaded to (e.g. not personal devices or storage)
  \end{itemize}
  \item Event Logging
  \item Monitoring
  \item Parameter Management
\end{itemize}

\subsection*{Security Services}

\begin{itemize}
  \item Denial of Service prevention
  \begin{itemize}
    \item Have a device outside of the network that is capable of absorbing the traffic of a DoS attack
  \end{itemize}
  \item Access Control
  \item User Authentication
  \begin{itemize}
    \item Multi-Factor
    \item 2FA
  \end{itemize}
  \item Data Confidentiality
  \item Accountability
\end{itemize}

\subsection*{Security Mechanisms}

\begin{itemize}
  \item Encryption and Decryption
  \item Message Authentication
  \item Password Policy
  \item Digital Signatures
  \item Access Control
\end{itemize}

\subsection*{Secure Communications over Insecure Networks}

\begin{itemize}
  \item Use encryption to ensure that even if a man-in-the-middle attack takes place, the message remains unreadable for the attacker
  \item Asymmetric encryption is less secure than symmetric encryption, but it is usually necessary to use both for better security
  \item One issue with this is that it's impossible to determine who sent the encrypted message(s), so another method of security must be used to certify authenticity
  \item To do this, we can combine cryptography and digital signatures to ensure that the message is from a trusted source, and that no one else is able to read the message
\end{itemize}

\subsection*{Secure Sockets Layer / Transport Layer Security (SSL / TLS)}

\begin{itemize}
  \item These protocols are used when communicating securely using HTTPS
  \item The encryption key may be anywhere from 40 to 128 bits
  \begin{itemize}
    \item 40 bit keys are obviously less secure than 128 bit keys, but both are susceptible to issues with key generations, such as poor Random Number Generation on specific platforms
    \item 256 bit keys can be used for applications needing higher security, but this is not a standard feature
  \end{itemize}
  \item Trusted certificates contain the owner's public key and is cryptographically signed by a trusted certificate agency
  \item Types of Encryption
  \begin{itemize}
    \item The Data Encryption Standard (DES) dates all the way back to the 70s, and uses a 56-bit key, which could be broken in a matter of hours on a modern computer
    \item Triple DES has a much longer effective key length but is still inadequate for our current needs
    \item The more recent Advanced Encryption Standard (AES) is much more secure, as it uses keys between 128 and 256 bits and a specifically designed algorithm
  \end{itemize}
\end{itemize}

\subsection*{Virtual Private Networks (VPNs)}

\begin{itemize}
  \item A private network that uses a public network (usually the Internet) to connect remote sites or users together
  \item Rather than using dedicated or rented physical lines, it uses a virtual connection
  \item While it is often touted that they're secure, the security is not inherent to the VPN's operation
  \begin{itemize}
    \item VPNs may use encryption to send all traffic, but this is not a given
    \item The encrypted packets are sent in "Routable IP Packets"
  \end{itemize}
  \item An outsider may be able to intercept packets in-flight, but if they're encrypted it is practically impossible for them to read or modify the packets
  \item VPNs give no QoS assurance, and so packets are delivered on a best-effort basis
\end{itemize}

\subsection*{Remote Authentication Dial-In User Service (RADIUS)}

\begin{itemize}
  \item Provides Authentication, Authorisation Checking and Accounting
  \item Uses a Point-to-Point protocol
  \item Operates on port 1812
  \item Commonly used to facilitate roaming
  \item Authentication and Authorisation Flow
  \begin{itemize}
    \item Client sends Access-Request to the Server
    \item Server responds with one of the following, depending upon if the user is authorised
    \begin{itemize}
      \item Access-Accept
      \item Access-Reject
      \item Access-Challenge
    \end{itemize}
  \end{itemize}
\end{itemize}

\subsection*{Uncontrolled Connections to the Internet}

\begin{itemize}
  \item It is very easy to connect to the Internet
  \begin{itemize}
    \item All you need is a router and approval
  \end{itemize}
  \item However, this is often not a good idea as it would make it very easy to steal data sent over the Internet
  \item There are dangers present if uncontrolled connections to the Internet are allowed
  \item There are a few ways of preventing this, but the most common one is Firewalls
\end{itemize}

\subsection*{Firewalls}

\begin{itemize}
  \item Routers which connect to the Internet typically have a firewall
  \item Firewalls filter out requests which are unwanted
  \item This may consist of adult content filters, or might filter what can be connected on the network, e.g. preventing VPN connections
  \item There can also be firewalls on individual devices
  \item These are known as Host-Based firewalls, and are useful for devices exposed to the Internet, as it allows them to block unwanted connections
  \item Host-Based firewalls are typically more secure, but are more expensive to setup and maintain
  \item Dedicated Firewall devices may have other functions such as
  \begin{itemize}
    \item Intrusion Detection (Signature verification, etc)
    \item Network Address Translation (NAT)
    \item URL and Content filtering
  \end{itemize}
\end{itemize}

\subsection*{Common Criteria Evaluation Assurance Levels (EAL)}

\begin{itemize}
  \item An internationally recognised method of comparing security of different network-enabled devices
  \item These levels range from 1 to 7
  \item EAL 2 is the minimum level to be accepted
  \item EAL 4 is the highest attainable level for a retrofitted product
  \item EALs 5-7 are extremely expensive to obtain and are typically limited to applications such as Governments, Militaries and Healthcare
\end{itemize}