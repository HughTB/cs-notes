\lecture{Network Policies and Standards}{09:00}{28/03/23}{Amanda Peart}

\subsection*{The Internet Society}

\begin{itemize}
  \item Aims to oversee the standardisation of protocols, improving global interoperability
  \item Coordinates Internet
  \begin{itemize}
    \item Design
    \item Engineering
    \item Management
  \end{itemize}
  \item The society has 3 main sections
  \begin{itemize}
    \item Internet Architecture Board (IAB)
    \begin{itemize}
      \item Defines the architecture of the internet
    \end{itemize}
    \item Internet Engineering Task Force (IETF)
    \begin{itemize}
      \item Defines the protocols of the internet, based upon the architecture given by the IAB
    \end{itemize}
    \item Internet Engineering Steering Group (IESG)
    \begin{itemize}
      \item Technical management
      \item Defines the "internet standard"
    \end{itemize}
  \end{itemize}
  \item Working groups investigate the actual details of any and all proposed standards and protocols
  \begin{itemize}
    \item Draft version is created
    \item Given out for consultation
    \item The IESG gives final approval
    \item Published as a Request for Comments (RFC)
    \begin{itemize}
      \item If the draft hasn't progressed in 6 months, it is withdrawn
    \end{itemize}
  \end{itemize}
  \item The criteria for all standards
  \begin{itemize}
    \item Be clear
    \item Be technically competent
    \item Have multiple independent but interoperable implementations
    \item Gain significant public support (both users and manufacturers)
    \item Be useful
  \end{itemize}
  \item The process
  \begin{itemize}
    \item Draft made of the standard
    \item The standard is proposed, and left open for 6 months for comments
    \item If the standard is relevant and correct, it becomes a draft standard and left open for 4 months
    \item If the standard is still relevant and no errors are found, it becomes an internet standard
    \item When the standard becomes irrelevant, it becomes a historical standard
  \end{itemize}
  \item For a standard to be ratified, it needs to have two independently developed implementations that are fully interoperable
  \begin{itemize}
    \item Once it is ratified, it is given a STD number and RFC number
  \end{itemize}
\end{itemize}

\subsection*{International Standards Organisation (ISO)}

\begin{itemize}
  \item The ISO aims to promote standardisation and related activities to facilitate international exchange of information and services
  \begin{itemize}
    \item Provides standards for all sorts of things, including networks and other electronics
  \end{itemize}
  \item 6 step development process
  \begin{itemize}
    \item Proposal Stage
    \begin{itemize}
      \item A new proposal is assigned to a working group, who create a proposal for the standard
    \end{itemize}
    \item Prepatory Stage
    \begin{itemize}
      \item A working group creates the draft of the standard
      \item Once created, it is passed to the committee for the public consensus phase
    \end{itemize}
    \item Committee Stage
    \begin{itemize}
      \item Registered at the ISO central Secretariat
      \item Distributed for balloting and comments
      \item Once a census is achieved, it becomes a Draft International Standard (DIS)
    \end{itemize}
    \item Enquiry Stage
    \begin{itemize}
      \item DIS is sent to all ISO bodies
      \item Time limit of 5 months for voting and approval
      \item It becomes a Final Draft International Standard
      \item If not approved it is returned to the working group to be reworked
    \end{itemize}
    \item Approval Stage
    \begin{itemize}
      \item Redistributed for final acceptance
      \item 2 month time limit
      \item Technical comments are no longer considered
      \item If not approved it is returned to the working group to be reworked
    \end{itemize}
    \item Final Stage
    \begin{itemize}
      \item Once approved, it becomes an International Standard
      \item Some minor editorial changes are allowed before final publication
    \end{itemize}
  \end{itemize}
\end{itemize}

\subsection*{Telecommunications Standardisation Sector (ITU)}

\begin{itemize}
  \item UN specialised agency
  \item Members are governments
  \item Responsible for broadband standards based upon ATM technology
  \item Processes
  \begin{itemize}
    \item Works in 4 year cycles
    \item Meetings of world telecommunication standardisation conference
    \item Work program for the next cycle established
    \item Study groups are created and abolished
    \item There is an accelerated procedure allowing recommendations to be approved when they are ready
  \end{itemize}
  \item Ballot Process
  \begin{itemize}
    \item Wide participation by Governments, Users and Industrial representatives
    \item This causes significant delays
    \item Because of this, it is now being streamlined by majority rule ballot
  \end{itemize}
\end{itemize}

\subsection*{Institute of Electrical and Electronic Engineers (IEEE)}

\begin{itemize}
  \item Initiating a project
  \begin{itemize}
    \item Sponsors input to create a collaborative group
    \item A Project Authorisation Request (PAR) is produced
  \end{itemize}
  \item Once a PAR is approved,
  \begin{itemize}
    \item A working group is created
    \item They work to write standards
    \item Anyone can be included in the group
  \end{itemize}
  \item Drafting the standard
  \begin{itemize}
    \item The first draft is written
    \item The Mandatory Editorial Coordinator checks the draft
    \item Then it goes to ballot
  \end{itemize}
  \item Balloting process
  \begin{itemize}
    \item Once the standard is stable
    \item The sponsor creates balloting groups
    \begin{itemize}
      \item Anyone is able to comment upon the standard
    \end{itemize}
    \item Balloting group consists of
    \begin{itemize}
      \item Producers
      \item Users
      \item Government
      \item General internet users
    \end{itemize}
    \item Balloting lasts 30-60 days
    \item Decisions
    \begin{itemize}
      \item Approval
      \item Disapproval
      \item Abstain
      \item Consensus is 75\% respondents with 75\% group approval
    \end{itemize}
  \end{itemize}
  \item Gaining approval
  \begin{itemize}
    \item The IEEE-SA board approves or denies the final standard
    \item The board's decision is based upon recommendations by the committee
    \item Standards are valid for 5 years, after which they can be
    \begin{itemize}
      \item Reaffirmed
      \item Revised
      \item Withdrawn
    \end{itemize}
  \end{itemize}
\end{itemize}