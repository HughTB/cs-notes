\lecture{Application Support Protocols Cont.d}{09:00}{21/03/23}{Amanda Peart}

\subsection*{Hyper Text Transfer Protocol (HTTP)}

\begin{itemize}
  \item Standard protocol used on the world wide web
  \item Requests one file at a time, typically starting with an HTML web page
  \item The web page may point to other files that need to be loaded
  \item Mainly consist of `GET` and `POST` requests
  \item Typically uses port 80
  \item Some websites are interactive, and as such the HTML changes
  \item Some of these websites use server-side rendered HTML to keep the website faster for low performance clients
  \item How do we send data for an interactive web page?
  \begin{itemize}
    \item Using a GET request
    \begin{itemize}
      \item GET \verb|http://{website}/{file}?information={data} HTTP/1.1|
    \end{itemize}
    \item Using POST
    \begin{itemize}
      \item POST \verb|http://{website}/{file} HTTP/1.1 {data}|
    \end{itemize}
    \item Why do the two different methods exist?
    \begin{itemize}
      \item GET exists to allow links to dynamically generated content
      \item POST keeps data out of the URL, which is useful for information such as passwords
    \end{itemize}
  \end{itemize}
\end{itemize}


\subsection*{Email (SMTP, POP3, etc)}

\begin{itemize}
  \item Before we can send an email, we need to find out the IP address of the mail server
  \item To do this, the application requests the MX record for the domain in question, e.g. `port.ac.uk` if trying to send an email to `amanda.peart@port.ac.uk`
  \item Once we know the IP address, we use SMTP (Simple Mail Transfer Protocol) on port 25
  \item Anti-Spam measures
  \begin{itemize}
    \item Standard practice is to use closed relays
    \begin{itemize}
      \item Only accept mail destined for that mail server
      \item Only send mail out from the mail server when the client requesting mail to be sent has an IP within a certain range, or is authenticated with a certificate or password
    \end{itemize}
    \item Real-time blackhole lists (RBL)
    \begin{itemize}
      \item Automatically blocks reported email addresses which are then blocked
      \item Can be annoying for people who have been inadvertently placed in the list
    \end{itemize}
    \item Content scanners can be employed to block spam messages, however these are often seen as a breach of privacy
  \end{itemize}
  \item Can we impersonate someone else?
  \begin{itemize}
    \item You can replace the sender and recipient addresses in the SMTP messages to be whatever you wish
    \item However, there are multiple mechanisms that block this
    \begin{itemize}
      \item Anti-spam measures often block these emails anyway
      \item Most mail servers reject messages sent from an IP address other than the one specified in the MX record for the domain
      \item e.g. you change the sender to `boris@gov.uk`, most recipients will block the message as your IP address does not match that belonging to the MX record of `gov.uk`
    \end{itemize}
  \end{itemize}
\end{itemize}

\subsection*{How do we get our emails? (POP3)}

\begin{itemize}
  \item To get our messages from the mail server, we need another protocol known as POP3 or Post Office Protocol v3
  \item Mail delivered by SMTP is stored on the mail server
  \item Our email client then requests the mail from the server using POP3
  \item POP3 is another simple text-based protocol
  \begin{itemize}
    \item Authenticate with the server using a username and password
    \item Request the number of messages waiting on the server
    \item If more than 0, request each email one by one
    \item Optionally: delete the messages from the server after they've been downloaded
  \end{itemize}
\end{itemize}

\subsection*{The real world}

\begin{itemize}
  \item How would a company set up their internal intranet with a website and internal mail server?
  \begin{itemize}
    \item Typically one or more internal DNS server on the network which handles an internal domain, or subdomain
    \item HTTP server (Apache, IIS, Nginx, etc)
    \item Mail server (Exchange, sendmail, etc)
  \end{itemize}
  \item In a smaller business, all of these services could be running on one machine, but it is often better to split them up to improve
  \begin{itemize}
    \item Redundancy
    \item Security
    \item Performance
    \item Network bandwidth
  \end{itemize}
\end{itemize}

\subsection*{HTTP over SSL (HTTPS)}

\begin{itemize}
  \item Used for the vast majority of the world wide web at this point
  \begin{itemize}
    \item Originally only used for high security applications such as banking
  \end{itemize}
  \item Server certificates are used to verify the authenticity of the server
  \item There are a few problems with HTTPS
  \begin{itemize}
    \item Server certificates (used) to be expensive
    \item It is possible to impersonate servers if they forget to renew their certificates, or use self-signed certificates
  \end{itemize}
\end{itemize}