\practical{Signals Cont.d}{12:00}{30/11/22}{Athanasios Paraskelidis}

\begin{itemize}
  \item If you have the following equation, $s(t) = 10.5\sin(100\pi t) + 8.25\sin(200\pi t) + 6.5\sin(300\pi t)$, you can get the following information just by reading the components:
  \begin{itemize}
    \item The amplitude of the signal is the greatest multiple, in this case, $A = 10.5$
    \item The frequency of each component is derived by dividing the value in the brackets by $2\pi$, e.g. $f_1 = \frac{100\pi}{2\pi} = 50\mathrm{Hz}$
    \item The period of each component is $T = \frac{1}{f}$, using the previously calculated value for $f$
    \item The bandwidth is the difference between the highest and lowest frequency, e.g. $Bandwidth = f_{min} - f_{max}$
    \item Attenuation can be calculated using $10^{\frac{db}{20}}$ where $db$ is the power attenuation
  \end{itemize}
\end{itemize}