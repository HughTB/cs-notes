\lecture{Network Management}{09:00}{28/02/23}{Amanda Peart}

\begin{itemize}
  \item Computer networks have become mission critical for many businesses
  \item Network downtime causes problems for communication and individual work
\end{itemize}

\subsection*{Goals for Network Management}

\begin{itemize}
  \item Responsive network management is needed, for example
  \begin{itemize}
    \item Help desk
    \item Network support technicians
    \item Network system management
    \begin{itemize}
      \item Monitor the network
      \item Have the ability to diagnose and control the network
    \end{itemize}
  \end{itemize}
\end{itemize}

\subsection*{Network Systems Management}

\begin{itemize}
  \item Monitors the network
  \item Displays the current status of the network, often using a "traffic light" system
  \begin{itemize}
    \item Red = Active outage
    \item Amber = Problems that may cause an outage
    \item Green = No problems detected
  \end{itemize}
  \item Provides notifications as problems arise
\end{itemize}

\subsection*{TCP/IP Network Management}

\begin{itemize}
  \item Network management involves 3 distinct needs
  \begin{itemize}
    \item A protocol for creating network management data, such as event reports
    \item A database of information, that contains data about queue length, throughput, etc over time
    \item A computer that is able to run independently of the network so that if the network goes down, the data is not compromised
  \end{itemize}
  \item These needs are met by
  \begin{itemize}
    \item SNMP - Simple Network Management Protocols
    \begin{itemize}
      \item Read/Write between network management devices and network client devices
    \end{itemize}
    \item MIB - Management Information Bases
    \begin{itemize}
      \item The database of information pertaining to the network
    \end{itemize}
    \item SMI - Structure of Management Information
    \begin{itemize}
      \item Device independent notation of device information
    \end{itemize}
  \end{itemize}
\end{itemize}

\subsection*{Simple Network Management Protocol (SNMP)}

\begin{itemize}
  \item The manager needs to be able to monitor and control the devices on the network
  \item It needs to be able to
  \begin{itemize}
    \item Read the value of parameters (SNMP Get)
    \item Read sequences of table entries (SNMP Get\_Next)
    \item Write into parameter values (SNMP Set)
    \item Receive unsolicited event reports (SNMP Trap)
  \end{itemize}
\end{itemize}

\subsection*{Remote MONitor (RMON)}

\begin{itemize}
  \item SNMP Management Information Bases (MIB) include remote monitoring capabilities
  \item RMON can be implemented in different ways
  \begin{itemize}
    \item As an independent probe device attached to each LAN segment
    \item Integrated into network devices, such as switches or routers
  \end{itemize}
  \item RMON is available in two forms
  \begin{itemize}
    \item RMON 1 monitors OSI layers 1 and 2, collecting collision and error statistics
    \item RMON 2 monitors higher levels of the OSI model, collecting information about application traffic
  \end{itemize}
  \item Can be cost effective, and control network traffic
  \item Increases the effectiveness of network management personnel, as they can remotely diagnose and fix issues
\end{itemize}

\subsection*{Network Management Areas}

The OSI identifies five areas on network management:
\begin{itemize}
  \item Configuration
  \item Fault management
  \item Performance management
  \item Accounting management
  \item Security management
\end{itemize}

\subsection*{Configuration}

\begin{itemize}
  \item A wide range of issues can fall under configuration issues
  \begin{itemize}
    \item Faulty IP address assignment
    \item Hardware or Software updates to switches, routers, etc
    \item Software license control
  \end{itemize}
  \item There are several parameters that fall under configuration
  \begin{itemize}
    \item Configure switches and routers to filter out certain traffic
    \item Multi-protocol routers can be configured to run selected protocols
    \item Configuration of bit rate, parity, etc
  \end{itemize}
\end{itemize}

\subsection*{Fault Management}

\begin{itemize}
  \item Provides identification and isolation of detected faults
  \item Tools and techniques include
  \begin{itemize}
    \item Bit-Error Rate Tests (BERT)
    \item Time Domain Reflectometer (TDR)
    \item Optical TDR (OTDR)
    \item Protocol Analyser (for data links and LANs)
    \item Loopback Tests
    \item Ping Tests
    \item Artificial Traffic Generation (Usually done out-of-hours to avoid creating further issues for the network)
  \end{itemize}
\end{itemize}

\subsection*{Fault Isolation (LANs)}

\begin{itemize}
  \item Limiting faults is possible by isolating the fault using switches and router configuration
  \item All traffic across the LAN can be monitored
  \item All exceptional conditions can be detected, e.g. collisions, viruses infecting computers across the network, etc
  \item Devices called LAN analysers or LAN protocol analysers can be attached to the network
  \begin{itemize}
    \item These devices record selective information about packets that may be of interest
    \item May be set up to filter based on address, protocol, or other fields of interest
  \end{itemize}
\end{itemize}

\subsection*{Performance Management}

\begin{itemize}
  \item Network Performance Management
  \begin{itemize}
    \item Concerned mainly with statistical data
    \item Round trip delays
    \item Throughput
  \end{itemize}
  \item May require prioritisation of traffic
  \begin{itemize}
    \item May include QoS capabilities
  \end{itemize}
  \item Tuning of performance (eliminating bottlenecks)
  \begin{itemize}
    \item Buffer size adjustment
    \item Setting timer values
  \end{itemize}
  \item Establish a baseline
  \begin{itemize}
    \item Set a minimum level of performance that is needed for the business
  \end{itemize}
  \item Performance management also involves finding bottlenecks
  \begin{itemize}
    \item WAN links between remote switches and routers
    \item Access to server resources, e.g. storage
    \item Parts of the network which are near or at capacity
  \end{itemize}
  \item Many fault-management tools are useful for performance management
\end{itemize}

\subsection*{Accounting Management}

\begin{itemize}
  \item Can be the billing for network usage
  \item Accounting Parameters usually include
  \begin{itemize}
    \item Number of connections made
    \item Duration of each connection
    \item Number of email packets sent and received
    \item Number of packets generally sent and received
    \item Systems that are accessed across the network
    \item Internet usage
  \end{itemize}
  \item Accounting management may be broadened to include other network attached resources
  \begin{itemize}
    \item Server usage (connection times and storage used)
    \item Traffic using expensive dedicated WAN circuits
    \item Data storage
  \end{itemize}
  \item Accounting management may also be used to cap the use of network resources or storage space
\end{itemize}

\subsection*{Security Management}

\begin{itemize}
  \item Confidentiality
  \item Integrity
  \item Authentication
  \item Access Control
  \item Nonrepudiation
  \item Vulnerabilities
  \begin{itemize}
    \item Wiretaps placed on cables
    \item Third parties intercepting remote logins
    \item Viruses and other Malware
  \end{itemize}
  \item Protection Mechanisms Include
  \begin{itemize}
    \item Encryption
    \item Physical Access Control
    \item Access-Control lists
    \item Audit Data Collection
  \end{itemize}
\end{itemize}