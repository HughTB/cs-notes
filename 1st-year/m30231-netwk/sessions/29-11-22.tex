\lecture{Communication Circuits}{09:00}{29/11/22}{Amanda Peart}

Things to consider when designing a network:
\begin{itemize}
  \item What is the maximum data rate needed?
  \item What is the maximum length a single run of cable is required to be?
  \item Could there be an issue with electrical interference?
  \item What constraints are there with the cable runs? e.g.
  \begin{itemize}
    \item Listed buildings
    \item Client requirements
    \item etc
  \end{itemize}
  \item What are the major costs associated with the selected medium?
  \item What medium is the external connection? e.g.
  \begin{itemize}
    \item Broadband
    \item Fibre
    \item Dial-up
    \item Satellite
  \end{itemize}
\end{itemize}

\subsection*{Types of communication circuit}

\begin{itemize}
  \item Dial-up
  \begin{itemize}
    \item Dial-up is available in two forms
    \begin{itemize}
      \item Analogue (Mostly legacy, not everywhere)
      \item Digital
    \end{itemize}
    \item Analogue links require a digital-to-analogue modem - this converts the digital signals from your computer to analogue signals
    \item Digital links require a digital-to-digital modem - this converts the digital signals from your computer to a different digital signal to be sent over the phone line
  \end{itemize}
  \item Modulation
  \begin{itemize}
    \item Modulation converts a digital signal into an analogue signal, which can be sent across an analogue connection
    \item Demodulation converts the analogue signal back into a digital signal, which can be used by a computer
    \item Modem stands for \textbf{Mo}dulation \textbf{dem}odulation
    \item Standard modem speeds are as such:
    \begin{itemize}
      \item V.34 at 28.8 or 33.6 kbps
      \item V.90 at 56 kbps
      \item V.92 allows higher-speed connections and the ability to accept an incoming call
    \end{itemize}
  \end{itemize}
  \item Reasons to switch to digital
  \begin{itemize}
    \item Computers are inherently digital, so easier to convert to a digital communication standard
    \item Higher data rates are available
    \item Easier to switch
    \item Better (lower) error rate
    \begin{itemize}
      \item Noise is not amplified along the line
    \end{itemize}
  \end{itemize}
  \item Digital telephone communication channels are available
  \begin{itemize}
    \item Each channel communicates at either 56 or 64 kbps
    \item These channels are then multiplexed (combined together) to create higher data rate connections
  \end{itemize}
  \item Problems with E1/T1 and T3/E3 systems
  \begin{itemize}
    \item T1 is used in the US and Japan but is incompatible with E1 which is used in Europe and the rest of the world
    \item It is complicated to add or remove a channel to convert between the two
    \item There is a need for higher bandwidth
    \item A new standard is needed
  \end{itemize}
  \item SONET/SDH
  \begin{itemize}
    \item Synchronous Optical Network (SONET) is a North American standard
    \begin{itemize}
      \item Works in multiples of 51.84 Mbps
      \item STS-3 supports triple the bandwidth (155.52 Mbps)
      \item Multiples of 4 up to 40 Gbps
    \end{itemize}
    \item Synchronous Digital Hierarchy (SDH) is an international standard
    \begin{itemize}
      \item Works in multiples of 155 Mbps
      \item A very resilient form of SONET and SDH is the dual ring, where there is a ring in both directions
    \end{itemize}
    \item If a cable is cut or goes down, the nodes at the ends reroute the data back along the ring in the other direction
    \item This recovery happens in ~50 milliseconds
  \end{itemize}
\end{itemize}