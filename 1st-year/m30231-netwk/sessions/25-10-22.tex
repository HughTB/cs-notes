\lecture{NICs and Ethernet}{09:00}{25/10/22}{Amanda Peart}

\section*{NICs (Network Interface Cards)}

\begin{itemize}
  \item There are many different types of aNICs, which are used to communicate using different mediums, e.g. WiFi, Ethernet, etc
  \item Each NIC has a 48-bit unique identifier known as a MAC address (Media Access Control address)
  \item The MAC address allows you to determine both which NIC communicated, but also which manufacturer made the card, and theoretically when the NIC was produced
  \item NICs read all broadcast addresses and
  \item All multicast messages with addresses it's been programmed to read
  \item The hardware will simply ignore all other messages
\end{itemize}

\section*{Ethernet LAN access devices}

\begin{itemize}
  \item Client devices have a cable between them and an interconnection device, usually in a network rack
  \item An "interconnection device" could be:
  \begin{itemize}
    \item A hub (Legacy)
    \item A Switch
    \item A Router (To access a different network, such as the internet)
  \end{itemize}
\end{itemize}
  
Access rules for ethernet hubs:
\begin{itemize}
  \item Listen before sending
  \item Stop if multiple users start at the same time
\end{itemize}
  
Distribution rules for ethernet hubs:
\begin{itemize}
  \item All traffic is sent everywhere
  \item One packet is sent at a time
\end{itemize}
  
Access and distribution rules for Ethernet LANs:
\begin{itemize}
  \item Send whenever you want to
  \item No collisions
  \item Traffic is only sent where it needs to be
  \item Multiple packets can be flowing at the same time
\end{itemize}

\section*{Switched Ethernet}
  
Characteristics of a switch:
\begin{itemize}
  \item Automatically learns the addresses of all connected devices
  \item Forwards only to the destination
  \item Supports many ports per switch
  \item Supports full duplex on dedicated ports (Can send at full speed in both directions at the same time)
  \item Supports different data rates on different ports
  \item Ethernet switches usually operate in store-and-forward mode
  \begin{itemize}
    \item Temporarily holds the packet while deciding which port the packet needs to be sent through
  \end{itemize}
  \item Some switches may also support cut-through operation
\end{itemize}

\subsection*{Unicast vs Multicast}

\begin{itemize}
  \item Unicast sends a packet in one direction to a single node on the network
  \item Multicast sends a packet to multiple nodes on the network in a target group (not all nodes on the network)
  \item Broadcast sends a packet to all nodes on the network
\end{itemize}

\subsection*{The LAN networking model}

\begin{itemize}
  \item The data link layer is split into two sub-layers
  \begin{itemize}
    \item LLC (Logical Link Control)
    \item MAC (Media Access Control)
  \end{itemize}
  \item Common aspects of LAN standards
  \begin{itemize}
    \item All use the same MAC addresses
    \item Supports broadcast and multicast addressing
    \item All have 32-bit error checking
  \end{itemize}
  \item Different aspects
  \begin{itemize}
    \item Access methods (CSMA/CD vs Token)
    \item Maximum frame size
    \item Support for features such as priority
    \item Specific data rates
  \end{itemize}
\end{itemize}

\subsection*{Virtual LANs (VLANs)}

\begin{itemize}
  \item Software emulates a physical LAN
  \item The purpose of VLANs is to limit broadcast traffic to a set group
  \item The group is set by network management
  \item VLANs are enforced by
  \begin{itemize}
    \item Selecting a set of ports on a switch
    \item Selecting a set of MAC addresses
  \end{itemize}
  \item VLANs are more convenient than re-wiring the entire network
\end{itemize}

\subsection*{Ethernet standards}

\begin{itemize}
  \item PoE (Power over Ethernet)
  \begin{itemize}
    \item Provides power through the ethernet cabling, reducing the number of cables and ports needed for low power devices such as access points
    \item Defined in 802.3af
  \end{itemize}
  \item 10 Base5 (10mbps, 500m max)
  \item 10 Base2 (10mbps, 185m max)
  \item 10 Base-T (Unshielded twisted pair (UTP) 10mbps, 100m max)
  \item 10 Base-F (Fibre optic ethernet 10mbps, theoretically unlimited range)
\end{itemize}