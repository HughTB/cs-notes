\lecture{Network Security}{09:00}{14/02/23}{Amanda Peart}

\subsection*{Key Elements}

\begin{itemize}
  \item Security Attack - Any action that compromises the security of information
  \item Security Mechanism - A mechanism that is designed to detect, prevent or recover from an attack
  \item Security Service - A service that enhances the security of data processing systems and information transfers, usually making use of more than one security mechanism
\end{itemize}

\subsection*{Security Goals}

\begin{itemize}
  \item Confidentiality - Keeping the information private
  \item Integrity - Preventing the information from changing
  \item Authentication - Ensuring the information is from a known source
\end{itemize}

\subsection*{Security Attacks}

\begin{itemize}
  \item Interruption - Data does not flow to the destination
  \item Interception - Data flows to the destination, as well as a 3rd party
  \item Modification - Data is intercepted and changed before reaching it's destination
  \item Fabrication - Data is fabricated and sent to the destination, without any interaction from the source
\end{itemize}

\subsection*{Passive Threats}

\begin{itemize}
  \item Passive attacks are listening in on transmissions
  \item The goal of the attacker is to obtain information that is being transmitted
\end{itemize}

\subsection*{Active Threats}

\begin{itemize}
  \item Attempt to actively cause harm, often through system faults or errors, or using brute force attacks
  \item May attempt to overload victim's computers to the point of unusability or crashes (known as Denial of Service (DoS) attacks)
\end{itemize}

\subsection*{Security Services}

\begin{itemize}
  \item Access control
  \begin{itemize}
    \item Levels of access
    \item Some people may need write access, others may not
  \end{itemize}
  \item Availability
  \begin{itemize}
    \item Denial of Service Attacks
    \item Viruses that delete files
  \end{itemize}
\end{itemize}

\subsection*{Methods of Defence}

\begin{itemize}
  \item Encryption
  \begin{itemize}
    \item Transforming the data in such a way that only people who have been given a piece of information are able to read it
  \end{itemize}
  \item Software Controls
  \begin{itemize}
    \item Access control used to limit user access to a database
    \item Operating system controls used to limit user access to other users and their information
  \end{itemize}
  \item Hardware Controls
  \begin{itemize}
    \item Smartcards used to access data
    \item Biometrics such as finger prints or retinal scans required to access data
  \end{itemize}
  \item Policies and Procedures
  \begin{itemize}
    \item Frequent password changes
    \item Strong password requirements
  \end{itemize}
  \item Physical Controls
  \begin{itemize}
    \item Limit physical access to computers that can access information, or the servers the information is stored on
  \end{itemize}
\end{itemize}
\subsection*{Security Vulnerabilities}

\begin{itemize}
  \item Securing network communication has always been a problem
  \item It is hard to secure the initial requests
  \item The data needs to be protected at all times when it is in transit
  \item Users need to be trusted before they are given access
\end{itemize}