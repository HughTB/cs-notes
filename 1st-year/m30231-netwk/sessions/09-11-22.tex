\practical{Switches and Hubs}{12:00}{09/11/22}{Athanasios Paraskelidis}

\section*{Riverbed Simulation}

\begin{itemize}
  \item You can create multiple scenarios within one riverbed project
  \item You can then switch between them at any time
  \item Additionally, if you go to Manage Scenarios, you can simulate any scenario in the project easily to collect data
  \item If you then go into the DES menu, you can go to Results -> Compare Results, which allows you to easily compare the data from the two scenarios
\end{itemize}

\section*{Switches vs Hubs}

\begin{itemize}
  \item When using a Switch instead of a Hub, the average delay in the network is reduced drastically, in this simulation it decreases from ~0.14 to ~0.01
  \begin{itemize}
    \item A hub broadcasts all incoming traffic to all interfaces, and therefor to every node connected to it
    \item On the other hand, a switch reads the header in each packet and relays it only to the node that needs it
    \item The result of this is that all devices connected to a hub are part of one "collision domain", and therefore only one node can communicate at a time, meaning that all other nodes have to wait until the network is not being used
    \item This causes the greatly increased delay when using a hub rather than a switch
  \end{itemize}
  \item Switches use a learning process to discover which nodes are connected to which interfaces (ports)
  \begin{itemize}
    \item They have a register which relates the interfaces (ports)
    \item Each time a node sends a packet, the switch reads the header and finds the IP address of the node that sent the packet, which it can then store in the register for future use
    \item If the register does not contain the IP address the packet is destined for, it broadcasts the packet on all interfaces
    \item Usually this learning process is quite fast, because any time a TCP transmission is sent, the receiving node will respond with a confirmation, which the switch can also use to learn the IP address of the node that responded
    \item Once the learning process is complete, the nodes connected to a switch are each part of their own collision domain, containing only the switch and the node, making collisions essentially impossible
  \end{itemize}
\end{itemize}