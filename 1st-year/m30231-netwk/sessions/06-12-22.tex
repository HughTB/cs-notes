\lecture{Wide Area Networks}{09:00}{06/12/22}{Amanda Peart}

\begin{itemize}
  \item Unlimited distance (Interconnections provided by ISPs (Internet Service Providers))
  \item High speed
  \item Relatively expensive given the complex design
  \item Only the interface to the WAN and services hosted on it are of concern to the user (The actual design is irrelevant)
  \item "Value added" WANs add more uses for dedicated point-to-point links
  \item Transparent LAN Services (TLS) hide the complexities of the WAN from the network administrator
\end{itemize}

\subsection*{Packet / Frame / Cell-Switched WAN links}

\begin{itemize}
  \item Individual units of data may be called:
  \begin{itemize}
    \item Packets
    \item Frames
    \item Cells
  \end{itemize}
  \item The difference between the 3 is that Packets and Frames can be of variable length
  \begin{itemize}
    \item This allows the network to be more efficient but requires more processing in software on either end
    \item The speed of processing in software limits the speed of communication over the link
  \end{itemize}
  \item Cells are of fixed length
  \begin{itemize}
    \item 5 bytes of header data and 48 bytes of payload data
    \item Because they are all the same length they can be processed in hardware, which is usually much faster
    \item This allows for much higher data rates
  \end{itemize}
\end{itemize}

\subsection*{Switched and Permanent Virtual Circuits}

\begin{itemize}
  \item Some alternatives to packets, frames or cells can come in one of two forms:
  \begin{itemize}
    \item Switched Virtual Circuits (SVC)
    \item Permanent Virtual Circuits (PVC)
  \end{itemize}
  \item SVC is like a dial-up connection
  \item PVC connections are always connected and leased out
  \item Not all WAN technologies support both SVCs and PVCs
  \begin{itemize}
    \item X.25 Virtual Circuits (VCs) are usually SVCs
    \item Frame Relay VCs are usually PVCs
    \item ATM VCs can be either SVCs or PVCs
    \item TLS is like a "best effort" service
  \end{itemize}
\end{itemize}

\subsection*{Services provided by WANs}

\begin{itemize}
  \item Provide a route across the network, between the source and destination
  \item Divide and reassemble data as required to be sent across the network
  \item Limit the network traffic to a level that can be effectively handled (congestion control)
\end{itemize}

\subsection*{X.25 Interface}

\begin{itemize}
  \item X.25 is a WAN interface ITU-T (International Telecommunication Union Telecommunication Standardization Sector) standard
  \item Connected to a public packet-switching network
  \item Covers the physical, data-link and network layers from the OSI reference model
  \item Last used over "the WAN" in 2015 by financial and debit/credit card companies
  \item Still used in the Aviation industry
\end{itemize}

\subsection*{Frame Relay}

\begin{itemize}
  \item Connection-oriented, public switched service
  \item A Layer 2 protocol defined in 1984 by the ITU-T and ANSI
  \item A much higher performance alternative to X.25
  \begin{itemize}
    \item Needed for high-performance applications, such as graphics and image transfers
    \item Very useful for LAN-to-LAN communications which need high throughput
  \end{itemize}
  \item It provides higher throughput by using
  \begin{itemize}
    \item Larger frame sizes (1500+ bytes)
    \item Higher interface data rates
    \item Reduced processing requirements
  \end{itemize}
  \item A variation of High-Level Data Link Control (HDLC)
  \begin{itemize}
    \item Detects and discards frames with errors, and does not retransmit them
    \item You must use another protocol on top of Frame Relay, e.g. TCP
  \end{itemize}
  \item Builds on the fibre-optic network
  \item A good alternative to E and T carriers
  \item Supports two levels of traffic
  \begin{itemize}
    \item Committed Information Rate (CIR)
    \begin{itemize}
      \item Traffic up to this rate will be accepted, anything above will be rejected
    \end{itemize}
    \item Excess Information Rate (EIR)
    \begin{itemize}
        \item Traffic between the CIR and EIR can still be sent but will be marked as "Eligible for Discard", so they can be discarded if congestion is too great
    \end{itemize}
  \end{itemize}
  \item It conveys congestion information, which can be controlled by users
  \item Advantages:
  \begin{itemize}
    \item International protocol
    \item Available in many (but not all) countries
    \item Available from all major vendors
    \item Takes advantage of modern fibre-optic infrastructure
    \item Good LAN-to-LAN support
    \item T and E carrier throughput capabilities
    \item Less expensive than fully meshed E1/T1 lines for bursty traffic
  \end{itemize}
  \item Disadvantages:
  \begin{itemize}
    \item Poor support for SVCs
    \item Does not provide any built-in fault tolerance, other protocols such as TCP are needed for error handling
    \item Not suitable for latency-sensitive data such as real-time audio or video conferencing
    \item Data overhead and processing overhead for every packet sent
    \item More expensive compared to internet service
  \end{itemize}
\end{itemize}

\subsection*{X.25 vs Frame Relay}

\begin{center}
\begin{tabular}{ |p{0.3\textwidth}|p{0.3\textwidth}|p{0.3\textwidth}| }
  \hline
  & X.25 & Frame Relay \\
  \hline
  Development Date & Mid 70s-Early 80s & Late 80s-Mid 90s \\
  Underlying Infrastructure & Low data rate, error prone copper circuits & High-speed, highly reliable fibre-optics \\
  Original Design Objectives & Support terminal to host & Support LAN-to-LAN \\
  Design Approach & 3 layers of the OSI model (Network, Data link, Physical) & 2 layers (Data link, Physical) \\
  Typical Data Rate & 9.6-64 kbps & Fractional or full T1/E1 \\
  Error Recovery & Error detection and transmission & Error detection with discard, no recovery other than when using TCP \\
  Max Packet/Frame Size & 128-4096 bytes & 1500 bytes (Full ethernet frame) \\
  Processing per Packet/Frame & Two dozen processing steps & Half-dozen processing steps \\
  Availability & Worldwide & Only in countries with fibre infrastructure \\
  Applications & Good for terminal-to-host but not for LAN-to-LAN, used for credit/debit card verification & Good for LAN-to-LAN, could be used for credit/debit card verification \\
  \hline
\end{tabular}
\end{center}