\lecture{Application Support Protocols}{09:00}{14/03/23}{Amanda Peart}

\subsection*{TCP vs UDP}

\begin{itemize}
  \item User Datagram Protocol
  \begin{itemize}
    \item No guarantees of delivery
    \item Acts as an interface to IP-Level for the Application layer
    \item Just sends data directly, without checking if the receiver actually exists or that a path exists
    \item Packets that are lost are just lost
    \item There is no checking of dropped, corrupted or incorrectly ordered packets
    \item As it's unreliable, there are only a few applications that it should be used for, e.g. voice or video streaming
    \begin{itemize}
      \item Low delay is essential in these applications
      \item The lost data is most likely irrelevant, so does not need to be re-requested
    \end{itemize}
  \end{itemize}
  \item Transmission Control Protocol
  \begin{itemize}
    \item Reliable data transmission
    \item Performs integrity checking, retransmission of dropped packets, reordering of packets, etc
    \item Connection oriented - a connection must be established before data can be sent to verify that a path exists and that the receiver is willing to receive the data
    \item The connections are setup as a virtual channel between the sender and receiver
    \item The 3-way handshake is made before the connection is established
    \item TCP should be used almost anywhere else, such as file transfers, webpages, email, etc
    \begin{itemize}
      \item Delay is not as sensitive
      \item Any data that may be lost needs to be recovered, otherwise the data would become corrupted and useless
    \end{itemize}
  \end{itemize}
\end{itemize}

\subsection*{TCP/UDP Multiplexing}

\begin{itemize}
  \item Since multiple applications on a computer may be using TCP or UDP at once, there must be a way of differentiating the incoming transmissions
  \item This is done using **ports** which are like a virtual connection to the computer
  \item Each device has 65535 ports
  \item Each application uses it's own port, and some have their own specific port, such as HTTP which uses port 80, HTTPS on 443 or SSH on 21
  \item The incoming packets enter the transport layer, where they are split up and sent into the application layer of the application using that port
\end{itemize}

\subsection*{Layer 6 - Presentation}

\begin{itemize}
  \item This layer interprets the data before the application receives it
  \item Where the Secure Sockets Layer (SSL) resides
  \item Not always used in protocols
  \item Data abstraction
  \begin{itemize}
    \item All protocols have their own format for the data they're sending
    \item The application does not need to see this data directly, just what is encoded in that data
    \item This could be something such as translating between different character sets, e.g. between ASCII and UTF-8
  \end{itemize}
\end{itemize}

\subsection*{Secure Sockets Layer (SSL)}

\begin{itemize}
  \item A socket is a method of making connections
  \item You can open a socket to connect to a remote host, or you can open a local socket to listen to a port on the computer
  \item SSL
  \begin{itemize}
    \item Provides end-to-end encryption
    \item Provides the same abstraction as other protocols and can usually be used as a slot-in replacement for traditional sockets
  \end{itemize}
  \item When is SSL used?
  \begin{itemize}
    \item Almost any time that secure transmission is needed
    \item HTTPS (Banking websites, most websites at this point... 2006 powerpoint smh)
    \item SFTP (Secure File Transfer Protocol)
    \item SSL Email (More secure emails)
  \end{itemize}
\end{itemize}

\subsection*{Layer 7 - Application}

\begin{itemize}
  \item Client - Server model
  \begin{itemize}
    \item The server is usually a more powerful computer which responds to many clients at the same time
    \item This is good as it reduces the number of machines needed, but it reduces the resilience of the application as there is a single point of failure
  \end{itemize}
  \item Peer-to-Peer
  \begin{itemize}
    \item Each peer acts as both a client and server
    \item Broadcast searches for services
    \item Each peer has a smaller internet connection, so multiple peers are used at the same time to increase speeds
    \item Disadvantage of this is that other people are responsible for hosting the data, and could modify the data
  \end{itemize}
\end{itemize}

\subsection*{Domain Name System}

\begin{itemize}
  \item A register of which Domain Names point to which IP addresses
  \begin{itemize}
    \item e.g. \verb|google.com| is currently linked to \verb|142.250.187.206|
  \end{itemize}
  \item There are multiple layers in a DNS, starting at the root server for the TLD of the domain
  \item DNS typically uses port 53
  \item Uses UDP for queries and TCP for transfers
  \item There are multiple types of records
  \begin{itemize}
    \item A record (Maps subdomain to an IPv4 address)
    \item AAAA record (Maps a subdomain to an IPv6 address)
    \item MX record (Maps to the IP address of a mail server)
    \item PTR records (Reverse lookups)
  \end{itemize}
\end{itemize}