\lecture{Asynchronous Transfer Mode (ATM)}{09:00}{31/01/23}{Amanda Peart}

\begin{itemize}
  \item Telecommunications standard defined by ANSI and ITU-T
  \item ATM is a data link layer
  \item Used in WANs
  \item Supports the transfer of data using a wide range of QoS assurance methods
  \item Core protocol used in SONET \& SDH
  \item A form of "cell relay"
  \item Relatively large packets which are segmented into 48-octet chunks for transmission (with 5-octet headers)
  \item These packets are switched across the network and than reassembled at the destination
  \item There is an unpredictable amount of time between the arrival of packets
  \item The cells are multiplexed with others when transmitted through the system
  \item Designed to provide virtual circuits across highly reliable media
  \item Optimised for the connectionless style of IP
\end{itemize}

\subsection*{Traffic Engineering / QoS}

\begin{itemize}
  \item QoS can be configured for every ATM interface
  \item Constant Bit Rate (CBR)
  \begin{itemize}
    \item Can transmit at a Peak Cell Rate (PCR) for a maximum interval before it becomes problematic and throttles back to the CBR
  \end{itemize}
  \item Variable Bit Rate (VBR)
  \begin{itemize}
    \item Can transmit at a Peak Cell Rate (PCR) for a certain time until it has to drop back to the Sustainable Cell Rate (SCR)
  \end{itemize}
  \item Available Bit Rate (ABR)
  \begin{itemize}
    \item A minimum bit rate is guaranteed
  \end{itemize}
  \item Unspecified Bit Rate (UBR)
  \begin{itemize}
    \item Traffic is allocated until the maximum bitrate is reached, any further cells have to find another route or wait for bandwidth
  \end{itemize}
\end{itemize}
\section*{Uses of ATM}

\begin{itemize}
  \item ATM is usually limited to the backbone of Wide Area Networks as it is not cost effective to run to edge nodes (e.g. homes or businesses)
  \item Since it has built in support for voice and video data, it is good for high-speed backbones
  \item It's QoS is very good, making it a good choice for high-speed backbones
\end{itemize}

\subsection*{Advantages}

\begin{itemize}
  \item Meets international and industry standards
  \item In use in most high-speed WANs
  \item Has built in support for voice and video, providing good QoS for these uses
  \item Cost competitive in the core of the network
\end{itemize}

\subsection*{Disadvantages}

\begin{itemize}
  \item Complex operation and configuration
  \item Somewhat inefficient (roughly 10\% overhead due to header data)
  \item Not cost competitive to the edges of networks
\end{itemize}

\section*{Transparent LAN Services (TLS)}

\begin{itemize}
  \item Transparent meaning that you don't have to deal with it manually
  \begin{itemize}
    \item No need to worry about the WAN
    \item No provision needed for frame relays, ATM, leased lines, etc
  \end{itemize}
  \item TLS bridges geographically separated LANs
  \begin{itemize}
    \item This makes them appear as one big LAN
    \item Decreases the need to manage WANs
  \end{itemize}
  \item Often use ATM circuits
  \item Supplies ATM access to Ethernet circuits
  \begin{itemize}
    \item Often known as "Metro Ethernet" or "Ethernet Transport"
    \item Available at all Ethernet data rates
  \end{itemize}
\end{itemize}

\section*{Overview of Wired WAN Technologies}

\begin{center}
\begin{tabular}{ |p{0.3\linewidth}|p{0.3\linewidth}|p{0.3\linewidth}| }
  \hline
  OSI Layer 1 & OSI Layer 2 & Medium \\
  \hline
  Dial-up over PSTN & PPP & Copper \\
  ISDN & PPP or Frame Relay & Copper \\
  DSL & PPP, Ethernet or ATM & Copper or Fibre \\
  Cable Broadband & Cable Broadband, Ethernet & Copper and Fibre \\
  T/E-Carrier & PPP, Frame Relay or ATM & Copper or Fibre \\
  SONET/SDH & PPP, Frame Relay, ATM, MPLS & Fibre \\
  \hline
\end{tabular}
\end{center}

\begin{itemize}
  \item PPP = Point-to-Point Protocol
  \item MPLS = Multiprotocol Label Switching
\end{itemize}