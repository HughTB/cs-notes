\lecture{Interconnection Protocols}{09:00}{07/02/23}{Amanda Peart}

\section*{Voice over Internet Protocol (VoIP)}

\begin{itemize}
  \item Originally, the motivations were to reduce the number of emails that need to be sent, and with VoIP phones you only need one network
  \item Current motivations are:
  \begin{itemize}
    \item Reduction of cost
    \item Single network
    \item More capable than typical phones
    \item Avoid delays
    \item Provide good QoS
  \end{itemize}
  \item Downsides are:
  \begin{itemize}
    \item The quality of the connection may not be great, depending upon the gateway between IP phones and legacy networks
    \item Wireless devices may drop connection temporarily when moving between access points
  \end{itemize}
\end{itemize}
  
\section*{Session Initiation Protocol (SIP)}

\begin{itemize}
  \item Application layer protocol
  \item Signalling protocol for real-time sessions
  \item Provides infrastructure for voice, video, instant messaging
  \item 5 Categories
  \begin{itemize}
    \item User Location - Realtime local discovery
    \item User Availability - Is the user available to communicate (online, engaged, etc)
    \item User Capability - Choice of media and encoding
    \item Session set-up - Establishing the session
    \item Session management - Transferring sessions, modifying parameters
  \end{itemize}
  \item SIP is "similar" to HTTP, as it's a request-response connection
\end{itemize}

\section*{The Internet and Network Access Points (NAPs)}

\begin{itemize}
  \item The internet consists of many ISPs which operate on different levels:
  \begin{itemize}
    \item Tier 1: International
    \item Tier 2: National
    \item Tier 3: Regional
    \item Tier 4: Local
  \end{itemize}
  \item Network Access Points (NAPs) are a type of Internet Exchange Point (IXP)
  \begin{itemize}
    \item They connect between public ISPs to exchange traffic
    \item Routing information is exchanged using BGP-4
  \end{itemize}
  \item Selective peering may be done with direct links to other ISPs
  \item NAPs are layer 2 switches
  \begin{itemize}
    \item Typically use ATM switching
    \item Support for ISO-provided routers
  \end{itemize}
  \item NAPs are connected by high-speed backbones
\end{itemize}

\subsection*{Router Capabilities}

\begin{itemize}
  \item Routers may be of several types:
  \begin{itemize}
    \item Access Routers - Edge of the internet
    \item Enterprise Routers - Organisation networks
    \item Core Routers - Handle heavy data flow
  \end{itemize}
  \item Routers may also have Layer 2 switching
  \item May have hardware or software routing capabilities
  \item Routers may be table top or rack-mount
  \item Modern "Routers" may be embedded into other multi-feature devices, such as
  \begin{itemize}
    \item Wireless Access Points
    \item Small (e.g. 4 or 5 port) ethernet switch
    \item Firewall
  \end{itemize}
\end{itemize}

\section*{Multi-Protocol Label Switching (MPLS)}

\begin{itemize}
  \item "Route at the edges, switch in the core"
  \item Provides the best parts of Layer 3 routing, and Layer 2 switching
  \item Intended for use in the core of the Internet or Intranets
  \begin{itemize}
    \item Useful for carriers, ISPs and enterprise WAN networks
    \item MPLS router in the core of the network is known as a label-switching router (LSR)
  \end{itemize}
  \item Why use MPLS?
  \begin{itemize}
    \item Specifications allow many options
    \item The first packet between two networks is routed, so that the Layer 2 switched connection can be setup
    \item Subsequent packets are handled at Layer 2, swapping the label at each LSR
  \end{itemize}
  \item Benefits:
  \begin{itemize}
    \item Traffic engineering capabilities (paths can be explicitly set without routing)
    \item MPLS-based VPNs can be setup with simpler provisioning of network infrastructure and bandwidth
    \item Good QoS
    \item Improved performance as compared to routing at each hop
    \item Much greater scalability
    \item Also has many of the benefits of connection-oriented networking
  \end{itemize}
\end{itemize}

\section*{QoS with IP}

\begin{itemize}
  \item QoS usually refers to providing support for time-sensitive delivery, such as voice or compressed video
  \item Efforts include
  \begin{itemize}
    \item Various forms of IP switching
    \item Differentiation between services (e.g. prioritise VoIP or Video packets over emails or web browsing)
    \item Multi-Protocol Label Switching (MPLS)
  \end{itemize}
\end{itemize}