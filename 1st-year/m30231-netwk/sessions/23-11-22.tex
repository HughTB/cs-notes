\practical{Signals}{12:00}{23/11/22}{Athanasios Paraskelidis}

\begin{itemize}
  \item When using a switch in the network the ethernet delay drops over time, as opposed to when using a hub, in which the delay increases over time
  \begin{itemize}
    \item This is because, at the start of the simulation, the switch does not know the IP Address of any of the nodes
    \item Over time the delay decreases, as the switch learns the IP addresses of the nodes connected to it, meaning it can use the lookup table to send to a specific node rather than broadcasting
  \end{itemize}
\end{itemize}
    
\section{Key Terminology - Signals}

\begin{itemize}
  \item Periodic signal
  \begin{itemize}
    \item A signal that repeats (every x seconds, minutes or hours)
  \end{itemize}
  \item Non-periodic signal
  \begin{itemize}
    \item A signal that does not repeat (it instantly disappears)
  \end{itemize}
  \item Amplitude (A)
  \begin{itemize}
    \item The strength of the signal (Decibels, dB)
  \end{itemize}
  \item Frequency (f)
  \begin{itemize}
    \item The number of times the signal repeats every second (Hertz, Hz)
    \item \large{$f = \frac{1}{T}$}
  \end{itemize}
  \item Period (T)
  \item The time for one cycle to be completed (Seconds, s)
  \begin{itemize}
    \item \large{$T = \frac{1}{f}$}
  \end{itemize}
  \item Bandwidth
  \begin{itemize}
    \item The range of frequencies that are used to communicate, the distance between the maximum and minimum frequency
    \item The higher the bandwidth, the greater the communication speed that can be used (Higher bps)
  \end{itemize}
\end{itemize}